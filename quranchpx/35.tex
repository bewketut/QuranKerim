%% License: BSD style (Berkley) (i.e. Put the Copyright owner's name always)
%% Writer and Copyright (to): Bewketu(Bilal) Tadilo (2016-17)
\shadowbox{\section{\LR{\textamharic{ሱራቱ ፋጢር -}  \RL{سوره  فاطر}}}}
\begin{longtable}{%
  @{}
    p{.5\textwidth}
  @{~~~~~~~~~~~~~}||
    p{.5\textwidth}
    @{}
}
\nopagebreak
\textamh{\ \ \ \ \ \  ቢስሚላሂ አራህመኒ ራሂይም } &  بِسمِ ٱللَّهِ ٱلرَّحمَـٰنِ ٱلرَّحِيمِ\\
\textamh{1.\  } &  ٱلحَمدُ لِلَّهِ فَاطِرِ ٱلسَّمَـٰوَٟتِ وَٱلأَرضِ جَاعِلِ ٱلمَلَـٰٓئِكَةِ رُسُلًا أُو۟لِىٓ أَجنِحَةٍۢ مَّثنَىٰ وَثُلَـٰثَ وَرُبَٰعَ ۚ يَزِيدُ فِى ٱلخَلقِ مَا يَشَآءُ ۚ إِنَّ ٱللَّهَ عَلَىٰ كُلِّ شَىءٍۢ قَدِيرٌۭ ﴿١﴾\\
\textamh{2.\  } & مَّا يَفتَحِ ٱللَّهُ لِلنَّاسِ مِن رَّحمَةٍۢ فَلَا مُمسِكَ لَهَا ۖ وَمَا يُمسِك فَلَا مُرسِلَ لَهُۥ مِنۢ بَعدِهِۦ ۚ وَهُوَ ٱلعَزِيزُ ٱلحَكِيمُ ﴿٢﴾\\
\textamh{3.\  } & يَـٰٓأَيُّهَا ٱلنَّاسُ ٱذكُرُوا۟ نِعمَتَ ٱللَّهِ عَلَيكُم ۚ هَل مِن خَـٰلِقٍ غَيرُ ٱللَّهِ يَرزُقُكُم مِّنَ ٱلسَّمَآءِ وَٱلأَرضِ ۚ لَآ إِلَـٰهَ إِلَّا هُوَ ۖ فَأَنَّىٰ تُؤفَكُونَ ﴿٣﴾\\
\textamh{4.\  } & وَإِن يُكَذِّبُوكَ فَقَد كُذِّبَت رُسُلٌۭ مِّن قَبلِكَ ۚ وَإِلَى ٱللَّهِ تُرجَعُ ٱلأُمُورُ ﴿٤﴾\\
\textamh{5.\  } & يَـٰٓأَيُّهَا ٱلنَّاسُ إِنَّ وَعدَ ٱللَّهِ حَقٌّۭ ۖ فَلَا تَغُرَّنَّكُمُ ٱلحَيَوٰةُ ٱلدُّنيَا ۖ وَلَا يَغُرَّنَّكُم بِٱللَّهِ ٱلغَرُورُ ﴿٥﴾\\
\textamh{6.\  } & إِنَّ ٱلشَّيطَٰنَ لَكُم عَدُوٌّۭ فَٱتَّخِذُوهُ عَدُوًّا ۚ إِنَّمَا يَدعُوا۟ حِزبَهُۥ لِيَكُونُوا۟ مِن أَصحَـٰبِ ٱلسَّعِيرِ ﴿٦﴾\\
\textamh{7.\  } & ٱلَّذِينَ كَفَرُوا۟ لَهُم عَذَابٌۭ شَدِيدٌۭ ۖ وَٱلَّذِينَ ءَامَنُوا۟ وَعَمِلُوا۟ ٱلصَّـٰلِحَـٰتِ لَهُم مَّغفِرَةٌۭ وَأَجرٌۭ كَبِيرٌ ﴿٧﴾\\
\textamh{8.\  } & أَفَمَن زُيِّنَ لَهُۥ سُوٓءُ عَمَلِهِۦ فَرَءَاهُ حَسَنًۭا ۖ فَإِنَّ ٱللَّهَ يُضِلُّ مَن يَشَآءُ وَيَهدِى مَن يَشَآءُ ۖ فَلَا تَذهَب نَفسُكَ عَلَيهِم حَسَرَٰتٍ ۚ إِنَّ ٱللَّهَ عَلِيمٌۢ بِمَا يَصنَعُونَ ﴿٨﴾\\
\textamh{9.\  } & وَٱللَّهُ ٱلَّذِىٓ أَرسَلَ ٱلرِّيَـٰحَ فَتُثِيرُ سَحَابًۭا فَسُقنَـٰهُ إِلَىٰ بَلَدٍۢ مَّيِّتٍۢ فَأَحيَينَا بِهِ ٱلأَرضَ بَعدَ مَوتِهَا ۚ كَذَٟلِكَ ٱلنُّشُورُ ﴿٩﴾\\
\textamh{10.\  } & مَن كَانَ يُرِيدُ ٱلعِزَّةَ فَلِلَّهِ ٱلعِزَّةُ جَمِيعًا ۚ إِلَيهِ يَصعَدُ ٱلكَلِمُ ٱلطَّيِّبُ وَٱلعَمَلُ ٱلصَّـٰلِحُ يَرفَعُهُۥ ۚ وَٱلَّذِينَ يَمكُرُونَ ٱلسَّيِّـَٔاتِ لَهُم عَذَابٌۭ شَدِيدٌۭ ۖ وَمَكرُ أُو۟لَـٰٓئِكَ هُوَ يَبُورُ ﴿١٠﴾\\
\textamh{11.\  } & وَٱللَّهُ خَلَقَكُم مِّن تُرَابٍۢ ثُمَّ مِن نُّطفَةٍۢ ثُمَّ جَعَلَكُم أَزوَٟجًۭا ۚ وَمَا تَحمِلُ مِن أُنثَىٰ وَلَا تَضَعُ إِلَّا بِعِلمِهِۦ ۚ وَمَا يُعَمَّرُ مِن مُّعَمَّرٍۢ وَلَا يُنقَصُ مِن عُمُرِهِۦٓ إِلَّا فِى كِتَـٰبٍ ۚ إِنَّ ذَٟلِكَ عَلَى ٱللَّهِ يَسِيرٌۭ ﴿١١﴾\\
\textamh{12.\  } & وَمَا يَستَوِى ٱلبَحرَانِ هَـٰذَا عَذبٌۭ فُرَاتٌۭ سَآئِغٌۭ شَرَابُهُۥ وَهَـٰذَا مِلحٌ أُجَاجٌۭ ۖ وَمِن كُلٍّۢ تَأكُلُونَ لَحمًۭا طَرِيًّۭا وَتَستَخرِجُونَ حِليَةًۭ تَلبَسُونَهَا ۖ وَتَرَى ٱلفُلكَ فِيهِ مَوَاخِرَ لِتَبتَغُوا۟ مِن فَضلِهِۦ وَلَعَلَّكُم تَشكُرُونَ ﴿١٢﴾\\
\textamh{13.\  } & يُولِجُ ٱلَّيلَ فِى ٱلنَّهَارِ وَيُولِجُ ٱلنَّهَارَ فِى ٱلَّيلِ وَسَخَّرَ ٱلشَّمسَ وَٱلقَمَرَ كُلٌّۭ يَجرِى لِأَجَلٍۢ مُّسَمًّۭى ۚ ذَٟلِكُمُ ٱللَّهُ رَبُّكُم لَهُ ٱلمُلكُ ۚ وَٱلَّذِينَ تَدعُونَ مِن دُونِهِۦ مَا يَملِكُونَ مِن قِطمِيرٍ ﴿١٣﴾\\
\textamh{14.\  } & إِن تَدعُوهُم لَا يَسمَعُوا۟ دُعَآءَكُم وَلَو سَمِعُوا۟ مَا ٱستَجَابُوا۟ لَكُم ۖ وَيَومَ ٱلقِيَـٰمَةِ يَكفُرُونَ بِشِركِكُم ۚ وَلَا يُنَبِّئُكَ مِثلُ خَبِيرٍۢ ﴿١٤﴾\\
\textamh{15.\  } & ۞ يَـٰٓأَيُّهَا ٱلنَّاسُ أَنتُمُ ٱلفُقَرَآءُ إِلَى ٱللَّهِ ۖ وَٱللَّهُ هُوَ ٱلغَنِىُّ ٱلحَمِيدُ ﴿١٥﴾\\
\textamh{16.\  } & إِن يَشَأ يُذهِبكُم وَيَأتِ بِخَلقٍۢ جَدِيدٍۢ ﴿١٦﴾\\
\textamh{17.\  } & وَمَا ذَٟلِكَ عَلَى ٱللَّهِ بِعَزِيزٍۢ ﴿١٧﴾\\
\textamh{18.\  } & وَلَا تَزِرُ وَازِرَةٌۭ وِزرَ أُخرَىٰ ۚ وَإِن تَدعُ مُثقَلَةٌ إِلَىٰ حِملِهَا لَا يُحمَل مِنهُ شَىءٌۭ وَلَو كَانَ ذَا قُربَىٰٓ ۗ إِنَّمَا تُنذِرُ ٱلَّذِينَ يَخشَونَ رَبَّهُم بِٱلغَيبِ وَأَقَامُوا۟ ٱلصَّلَوٰةَ ۚ وَمَن تَزَكَّىٰ فَإِنَّمَا يَتَزَكَّىٰ لِنَفسِهِۦ ۚ وَإِلَى ٱللَّهِ ٱلمَصِيرُ ﴿١٨﴾\\
\textamh{19.\  } & وَمَا يَستَوِى ٱلأَعمَىٰ وَٱلبَصِيرُ ﴿١٩﴾\\
\textamh{20.\  } & وَلَا ٱلظُّلُمَـٰتُ وَلَا ٱلنُّورُ ﴿٢٠﴾\\
\textamh{21.\  } & وَلَا ٱلظِّلُّ وَلَا ٱلحَرُورُ ﴿٢١﴾\\
\textamh{22.\  } & وَمَا يَستَوِى ٱلأَحيَآءُ وَلَا ٱلأَموَٟتُ ۚ إِنَّ ٱللَّهَ يُسمِعُ مَن يَشَآءُ ۖ وَمَآ أَنتَ بِمُسمِعٍۢ مَّن فِى ٱلقُبُورِ ﴿٢٢﴾\\
\textamh{23.\  } & إِن أَنتَ إِلَّا نَذِيرٌ ﴿٢٣﴾\\
\textamh{24.\  } & إِنَّآ أَرسَلنَـٰكَ بِٱلحَقِّ بَشِيرًۭا وَنَذِيرًۭا ۚ وَإِن مِّن أُمَّةٍ إِلَّا خَلَا فِيهَا نَذِيرٌۭ ﴿٢٤﴾\\
\textamh{25.\  } & وَإِن يُكَذِّبُوكَ فَقَد كَذَّبَ ٱلَّذِينَ مِن قَبلِهِم جَآءَتهُم رُسُلُهُم بِٱلبَيِّنَـٰتِ وَبِٱلزُّبُرِ وَبِٱلكِتَـٰبِ ٱلمُنِيرِ ﴿٢٥﴾\\
\textamh{26.\  } & ثُمَّ أَخَذتُ ٱلَّذِينَ كَفَرُوا۟ ۖ فَكَيفَ كَانَ نَكِيرِ ﴿٢٦﴾\\
\textamh{27.\  } & أَلَم تَرَ أَنَّ ٱللَّهَ أَنزَلَ مِنَ ٱلسَّمَآءِ مَآءًۭ فَأَخرَجنَا بِهِۦ ثَمَرَٰتٍۢ مُّختَلِفًا أَلوَٟنُهَا ۚ وَمِنَ ٱلجِبَالِ جُدَدٌۢ بِيضٌۭ وَحُمرٌۭ مُّختَلِفٌ أَلوَٟنُهَا وَغَرَابِيبُ سُودٌۭ ﴿٢٧﴾\\
\textamh{28.\  } & وَمِنَ ٱلنَّاسِ وَٱلدَّوَآبِّ وَٱلأَنعَـٰمِ مُختَلِفٌ أَلوَٟنُهُۥ كَذَٟلِكَ ۗ إِنَّمَا يَخشَى ٱللَّهَ مِن عِبَادِهِ ٱلعُلَمَـٰٓؤُا۟ ۗ إِنَّ ٱللَّهَ عَزِيزٌ غَفُورٌ ﴿٢٨﴾\\
\textamh{29.\  } & إِنَّ ٱلَّذِينَ يَتلُونَ كِتَـٰبَ ٱللَّهِ وَأَقَامُوا۟ ٱلصَّلَوٰةَ وَأَنفَقُوا۟ مِمَّا رَزَقنَـٰهُم سِرًّۭا وَعَلَانِيَةًۭ يَرجُونَ تِجَٰرَةًۭ لَّن تَبُورَ ﴿٢٩﴾\\
\textamh{30.\  } & لِيُوَفِّيَهُم أُجُورَهُم وَيَزِيدَهُم مِّن فَضلِهِۦٓ ۚ إِنَّهُۥ غَفُورٌۭ شَكُورٌۭ ﴿٣٠﴾\\
\textamh{31.\  } & وَٱلَّذِىٓ أَوحَينَآ إِلَيكَ مِنَ ٱلكِتَـٰبِ هُوَ ٱلحَقُّ مُصَدِّقًۭا لِّمَا بَينَ يَدَيهِ ۗ إِنَّ ٱللَّهَ بِعِبَادِهِۦ لَخَبِيرٌۢ بَصِيرٌۭ ﴿٣١﴾\\
\textamh{32.\  } & ثُمَّ أَورَثنَا ٱلكِتَـٰبَ ٱلَّذِينَ ٱصطَفَينَا مِن عِبَادِنَا ۖ فَمِنهُم ظَالِمٌۭ لِّنَفسِهِۦ وَمِنهُم مُّقتَصِدٌۭ وَمِنهُم سَابِقٌۢ بِٱلخَيرَٰتِ بِإِذنِ ٱللَّهِ ۚ ذَٟلِكَ هُوَ ٱلفَضلُ ٱلكَبِيرُ ﴿٣٢﴾\\
\textamh{33.\  } & جَنَّـٰتُ عَدنٍۢ يَدخُلُونَهَا يُحَلَّونَ فِيهَا مِن أَسَاوِرَ مِن ذَهَبٍۢ وَلُؤلُؤًۭا ۖ وَلِبَاسُهُم فِيهَا حَرِيرٌۭ ﴿٣٣﴾\\
\textamh{34.\  } & وَقَالُوا۟ ٱلحَمدُ لِلَّهِ ٱلَّذِىٓ أَذهَبَ عَنَّا ٱلحَزَنَ ۖ إِنَّ رَبَّنَا لَغَفُورٌۭ شَكُورٌ ﴿٣٤﴾\\
\textamh{35.\  } & ٱلَّذِىٓ أَحَلَّنَا دَارَ ٱلمُقَامَةِ مِن فَضلِهِۦ لَا يَمَسُّنَا فِيهَا نَصَبٌۭ وَلَا يَمَسُّنَا فِيهَا لُغُوبٌۭ ﴿٣٥﴾\\
\textamh{36.\  } & وَٱلَّذِينَ كَفَرُوا۟ لَهُم نَارُ جَهَنَّمَ لَا يُقضَىٰ عَلَيهِم فَيَمُوتُوا۟ وَلَا يُخَفَّفُ عَنهُم مِّن عَذَابِهَا ۚ كَذَٟلِكَ نَجزِى كُلَّ كَفُورٍۢ ﴿٣٦﴾\\
\textamh{37.\  } & وَهُم يَصطَرِخُونَ فِيهَا رَبَّنَآ أَخرِجنَا نَعمَل صَـٰلِحًا غَيرَ ٱلَّذِى كُنَّا نَعمَلُ ۚ أَوَلَم نُعَمِّركُم مَّا يَتَذَكَّرُ فِيهِ مَن تَذَكَّرَ وَجَآءَكُمُ ٱلنَّذِيرُ ۖ فَذُوقُوا۟ فَمَا لِلظَّـٰلِمِينَ مِن نَّصِيرٍ ﴿٣٧﴾\\
\textamh{38.\  } & إِنَّ ٱللَّهَ عَـٰلِمُ غَيبِ ٱلسَّمَـٰوَٟتِ وَٱلأَرضِ ۚ إِنَّهُۥ عَلِيمٌۢ بِذَاتِ ٱلصُّدُورِ ﴿٣٨﴾\\
\textamh{39.\  } & هُوَ ٱلَّذِى جَعَلَكُم خَلَـٰٓئِفَ فِى ٱلأَرضِ ۚ فَمَن كَفَرَ فَعَلَيهِ كُفرُهُۥ ۖ وَلَا يَزِيدُ ٱلكَـٰفِرِينَ كُفرُهُم عِندَ رَبِّهِم إِلَّا مَقتًۭا ۖ وَلَا يَزِيدُ ٱلكَـٰفِرِينَ كُفرُهُم إِلَّا خَسَارًۭا ﴿٣٩﴾\\
\textamh{40.\  } & قُل أَرَءَيتُم شُرَكَآءَكُمُ ٱلَّذِينَ تَدعُونَ مِن دُونِ ٱللَّهِ أَرُونِى مَاذَا خَلَقُوا۟ مِنَ ٱلأَرضِ أَم لَهُم شِركٌۭ فِى ٱلسَّمَـٰوَٟتِ أَم ءَاتَينَـٰهُم كِتَـٰبًۭا فَهُم عَلَىٰ بَيِّنَتٍۢ مِّنهُ ۚ بَل إِن يَعِدُ ٱلظَّـٰلِمُونَ بَعضُهُم بَعضًا إِلَّا غُرُورًا ﴿٤٠﴾\\
\textamh{41.\  } & ۞ إِنَّ ٱللَّهَ يُمسِكُ ٱلسَّمَـٰوَٟتِ وَٱلأَرضَ أَن تَزُولَا ۚ وَلَئِن زَالَتَآ إِن أَمسَكَهُمَا مِن أَحَدٍۢ مِّنۢ بَعدِهِۦٓ ۚ إِنَّهُۥ كَانَ حَلِيمًا غَفُورًۭا ﴿٤١﴾\\
\textamh{42.\  } & وَأَقسَمُوا۟ بِٱللَّهِ جَهدَ أَيمَـٰنِهِم لَئِن جَآءَهُم نَذِيرٌۭ لَّيَكُونُنَّ أَهدَىٰ مِن إِحدَى ٱلأُمَمِ ۖ فَلَمَّا جَآءَهُم نَذِيرٌۭ مَّا زَادَهُم إِلَّا نُفُورًا ﴿٤٢﴾\\
\textamh{43.\  } & ٱستِكبَارًۭا فِى ٱلأَرضِ وَمَكرَ ٱلسَّيِّئِ ۚ وَلَا يَحِيقُ ٱلمَكرُ ٱلسَّيِّئُ إِلَّا بِأَهلِهِۦ ۚ فَهَل يَنظُرُونَ إِلَّا سُنَّتَ ٱلأَوَّلِينَ ۚ فَلَن تَجِدَ لِسُنَّتِ ٱللَّهِ تَبدِيلًۭا ۖ وَلَن تَجِدَ لِسُنَّتِ ٱللَّهِ تَحوِيلًا ﴿٤٣﴾\\
\textamh{44.\  } & أَوَلَم يَسِيرُوا۟ فِى ٱلأَرضِ فَيَنظُرُوا۟ كَيفَ كَانَ عَـٰقِبَةُ ٱلَّذِينَ مِن قَبلِهِم وَكَانُوٓا۟ أَشَدَّ مِنهُم قُوَّةًۭ ۚ وَمَا كَانَ ٱللَّهُ لِيُعجِزَهُۥ مِن شَىءٍۢ فِى ٱلسَّمَـٰوَٟتِ وَلَا فِى ٱلأَرضِ ۚ إِنَّهُۥ كَانَ عَلِيمًۭا قَدِيرًۭا ﴿٤٤﴾\\
\textamh{45.\  } & وَلَو يُؤَاخِذُ ٱللَّهُ ٱلنَّاسَ بِمَا كَسَبُوا۟ مَا تَرَكَ عَلَىٰ ظَهرِهَا مِن دَآبَّةٍۢ وَلَـٰكِن يُؤَخِّرُهُم إِلَىٰٓ أَجَلٍۢ مُّسَمًّۭى ۖ فَإِذَا جَآءَ أَجَلُهُم فَإِنَّ ٱللَّهَ كَانَ بِعِبَادِهِۦ بَصِيرًۢا ﴿٤٥﴾\\
\end{longtable} \newpage
