%% License: BSD style (Berkley) (i.e. Put the Copyright owner's name always)
%% Writer and Copyright (to): Bewketu(Bilal) Tadilo (2016-17)
\shadowbox{\section{\LR{\textamharic{ሱራቱ አልሀቃ -}  \RL{سوره  الحاقة}}}}
\begin{longtable}{%
  @{}
    p{.5\textwidth}
  @{~~~~~~~~~~~~~}||
    p{.5\textwidth}
    @{}
}
\nopagebreak
\textamh{\ \ \ \ \ \  ቢስሚላሂ አራህመኒ ራሂይም } &  بِسمِ ٱللَّهِ ٱلرَّحمَـٰنِ ٱلرَّحِيمِ\\
\textamh{1.\  } &  ٱلحَآقَّةُ ﴿١﴾\\
\textamh{2.\  } & مَا ٱلحَآقَّةُ ﴿٢﴾\\
\textamh{3.\  } & وَمَآ أَدرَىٰكَ مَا ٱلحَآقَّةُ ﴿٣﴾\\
\textamh{4.\  } & كَذَّبَت ثَمُودُ وَعَادٌۢ بِٱلقَارِعَةِ ﴿٤﴾\\
\textamh{5.\  } & فَأَمَّا ثَمُودُ فَأُهلِكُوا۟ بِٱلطَّاغِيَةِ ﴿٥﴾\\
\textamh{6.\  } & وَأَمَّا عَادٌۭ فَأُهلِكُوا۟ بِرِيحٍۢ صَرصَرٍ عَاتِيَةٍۢ ﴿٦﴾\\
\textamh{7.\  } & سَخَّرَهَا عَلَيهِم سَبعَ لَيَالٍۢ وَثَمَـٰنِيَةَ أَيَّامٍ حُسُومًۭا فَتَرَى ٱلقَومَ فِيهَا صَرعَىٰ كَأَنَّهُم أَعجَازُ نَخلٍ خَاوِيَةٍۢ ﴿٧﴾\\
\textamh{8.\  } & فَهَل تَرَىٰ لَهُم مِّنۢ بَاقِيَةٍۢ ﴿٨﴾\\
\textamh{9.\  } & وَجَآءَ فِرعَونُ وَمَن قَبلَهُۥ وَٱلمُؤتَفِكَـٰتُ بِٱلخَاطِئَةِ ﴿٩﴾\\
\textamh{10.\  } & فَعَصَوا۟ رَسُولَ رَبِّهِم فَأَخَذَهُم أَخذَةًۭ رَّابِيَةً ﴿١٠﴾\\
\textamh{11.\  } & إِنَّا لَمَّا طَغَا ٱلمَآءُ حَمَلنَـٰكُم فِى ٱلجَارِيَةِ ﴿١١﴾\\
\textamh{12.\  } & لِنَجعَلَهَا لَكُم تَذكِرَةًۭ وَتَعِيَهَآ أُذُنٌۭ وَٟعِيَةٌۭ ﴿١٢﴾\\
\textamh{13.\  } & فَإِذَا نُفِخَ فِى ٱلصُّورِ نَفخَةٌۭ وَٟحِدَةٌۭ ﴿١٣﴾\\
\textamh{14.\  } & وَحُمِلَتِ ٱلأَرضُ وَٱلجِبَالُ فَدُكَّتَا دَكَّةًۭ وَٟحِدَةًۭ ﴿١٤﴾\\
\textamh{15.\  } & فَيَومَئِذٍۢ وَقَعَتِ ٱلوَاقِعَةُ ﴿١٥﴾\\
\textamh{16.\  } & وَٱنشَقَّتِ ٱلسَّمَآءُ فَهِىَ يَومَئِذٍۢ وَاهِيَةٌۭ ﴿١٦﴾\\
\textamh{17.\  } & وَٱلمَلَكُ عَلَىٰٓ أَرجَآئِهَا ۚ وَيَحمِلُ عَرشَ رَبِّكَ فَوقَهُم يَومَئِذٍۢ ثَمَـٰنِيَةٌۭ ﴿١٧﴾\\
\textamh{18.\  } & يَومَئِذٍۢ تُعرَضُونَ لَا تَخفَىٰ مِنكُم خَافِيَةٌۭ ﴿١٨﴾\\
\textamh{19.\  } & فَأَمَّا مَن أُوتِىَ كِتَـٰبَهُۥ بِيَمِينِهِۦ فَيَقُولُ هَآؤُمُ ٱقرَءُوا۟ كِتَـٰبِيَه ﴿١٩﴾\\
\textamh{20.\  } & إِنِّى ظَنَنتُ أَنِّى مُلَـٰقٍ حِسَابِيَه ﴿٢٠﴾\\
\textamh{21.\  } & فَهُوَ فِى عِيشَةٍۢ رَّاضِيَةٍۢ ﴿٢١﴾\\
\textamh{22.\  } & فِى جَنَّةٍ عَالِيَةٍۢ ﴿٢٢﴾\\
\textamh{23.\  } & قُطُوفُهَا دَانِيَةٌۭ ﴿٢٣﴾\\
\textamh{24.\  } & كُلُوا۟ وَٱشرَبُوا۟ هَنِيٓـًٔۢا بِمَآ أَسلَفتُم فِى ٱلأَيَّامِ ٱلخَالِيَةِ ﴿٢٤﴾\\
\textamh{25.\  } & وَأَمَّا مَن أُوتِىَ كِتَـٰبَهُۥ بِشِمَالِهِۦ فَيَقُولُ يَـٰلَيتَنِى لَم أُوتَ كِتَـٰبِيَه ﴿٢٥﴾\\
\textamh{26.\  } & وَلَم أَدرِ مَا حِسَابِيَه ﴿٢٦﴾\\
\textamh{27.\  } & يَـٰلَيتَهَا كَانَتِ ٱلقَاضِيَةَ ﴿٢٧﴾\\
\textamh{28.\  } & مَآ أَغنَىٰ عَنِّى مَالِيَه ۜ ﴿٢٨﴾\\
\textamh{29.\  } & هَلَكَ عَنِّى سُلطَٰنِيَه ﴿٢٩﴾\\
\textamh{30.\  } & خُذُوهُ فَغُلُّوهُ ﴿٣٠﴾\\
\textamh{31.\  } & ثُمَّ ٱلجَحِيمَ صَلُّوهُ ﴿٣١﴾\\
\textamh{32.\  } & ثُمَّ فِى سِلسِلَةٍۢ ذَرعُهَا سَبعُونَ ذِرَاعًۭا فَٱسلُكُوهُ ﴿٣٢﴾\\
\textamh{33.\  } & إِنَّهُۥ كَانَ لَا يُؤمِنُ بِٱللَّهِ ٱلعَظِيمِ ﴿٣٣﴾\\
\textamh{34.\  } & وَلَا يَحُضُّ عَلَىٰ طَعَامِ ٱلمِسكِينِ ﴿٣٤﴾\\
\textamh{35.\  } & فَلَيسَ لَهُ ٱليَومَ هَـٰهُنَا حَمِيمٌۭ ﴿٣٥﴾\\
\textamh{36.\  } & وَلَا طَعَامٌ إِلَّا مِن غِسلِينٍۢ ﴿٣٦﴾\\
\textamh{37.\  } & لَّا يَأكُلُهُۥٓ إِلَّا ٱلخَـٰطِـُٔونَ ﴿٣٧﴾\\
\textamh{38.\  } & فَلَآ أُقسِمُ بِمَا تُبصِرُونَ ﴿٣٨﴾\\
\textamh{39.\  } & وَمَا لَا تُبصِرُونَ ﴿٣٩﴾\\
\textamh{40.\  } & إِنَّهُۥ لَقَولُ رَسُولٍۢ كَرِيمٍۢ ﴿٤٠﴾\\
\textamh{41.\  } & وَمَا هُوَ بِقَولِ شَاعِرٍۢ ۚ قَلِيلًۭا مَّا تُؤمِنُونَ ﴿٤١﴾\\
\textamh{42.\  } & وَلَا بِقَولِ كَاهِنٍۢ ۚ قَلِيلًۭا مَّا تَذَكَّرُونَ ﴿٤٢﴾\\
\textamh{43.\  } & تَنزِيلٌۭ مِّن رَّبِّ ٱلعَـٰلَمِينَ ﴿٤٣﴾\\
\textamh{44.\  } & وَلَو تَقَوَّلَ عَلَينَا بَعضَ ٱلأَقَاوِيلِ ﴿٤٤﴾\\
\textamh{45.\  } & لَأَخَذنَا مِنهُ بِٱليَمِينِ ﴿٤٥﴾\\
\textamh{46.\  } & ثُمَّ لَقَطَعنَا مِنهُ ٱلوَتِينَ ﴿٤٦﴾\\
\textamh{47.\  } & فَمَا مِنكُم مِّن أَحَدٍ عَنهُ حَـٰجِزِينَ ﴿٤٧﴾\\
\textamh{48.\  } & وَإِنَّهُۥ لَتَذكِرَةٌۭ لِّلمُتَّقِينَ ﴿٤٨﴾\\
\textamh{49.\  } & وَإِنَّا لَنَعلَمُ أَنَّ مِنكُم مُّكَذِّبِينَ ﴿٤٩﴾\\
\textamh{50.\  } & وَإِنَّهُۥ لَحَسرَةٌ عَلَى ٱلكَـٰفِرِينَ ﴿٥٠﴾\\
\textamh{51.\  } & وَإِنَّهُۥ لَحَقُّ ٱليَقِينِ ﴿٥١﴾\\
\textamh{52.\  } & فَسَبِّح بِٱسمِ رَبِّكَ ٱلعَظِيمِ ﴿٥٢﴾\\
\end{longtable} \newpage
