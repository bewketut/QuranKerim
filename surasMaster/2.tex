%% License: BSD style (Berkley) (i.e. Put the Copyright owner's name always)
%% Writer and Copyright (to): Bewketu(Bilal) Tadilo (2016-17)
\shadowbox{\section{\LR{\textamharic{ሱራቱ አልበቀራ -}  \RL{سوره  البقرة}}}}
% Copyright: Bewketu/Bilal Tadilo, North Gondar Ethiopia (2016-17)
% Licence: Berkley (BSD)
\begin{longtable}{%
  @{}
    p{.5\textwidth}
  @{~~~~~~~~~~~~~}||
    p{.5\textwidth}
    @{}
}
\nopagebreak
\textamh{\ \ \ \ \ \  ቢስሚላሂ አራህመኒ ራሂይም } &  بِسمِ ٱللَّهِ ٱلرَّحمَـٰنِ ٱلرَّحِيمِ\\
\textamh{1.\  } &  الٓمٓ ﴿١﴾\\
\textamh{2.\  } & ذَٟلِكَ ٱلكِتَـٰبُ لَا رَيبَ ۛ فِيهِ ۛ هُدًۭى لِّلمُتَّقِينَ ﴿٢﴾\\
\textamh{3.\  } & ٱلَّذِينَ يُؤمِنُونَ بِٱلغَيبِ وَيُقِيمُونَ ٱلصَّلَوٰةَ وَمِمَّا رَزَقنَـٰهُم يُنفِقُونَ ﴿٣﴾\\
\textamh{4.\  } & وَٱلَّذِينَ يُؤمِنُونَ بِمَآ أُنزِلَ إِلَيكَ وَمَآ أُنزِلَ مِن قَبلِكَ وَبِٱلءَاخِرَةِ هُم يُوقِنُونَ ﴿٤﴾\\
\textamh{5.\  } & أُو۟لَـٰٓئِكَ عَلَىٰ هُدًۭى مِّن رَّبِّهِم ۖ وَأُو۟لَـٰٓئِكَ هُمُ ٱلمُفلِحُونَ ﴿٥﴾\\
\textamh{6.\  } & إِنَّ ٱلَّذِينَ كَفَرُوا۟ سَوَآءٌ عَلَيهِم ءَأَنذَرتَهُم أَم لَم تُنذِرهُم لَا يُؤمِنُونَ ﴿٦﴾\\
\textamh{7.\  } & خَتَمَ ٱللَّهُ عَلَىٰ قُلُوبِهِم وَعَلَىٰ سَمعِهِم ۖ وَعَلَىٰٓ أَبصَـٰرِهِم غِشَـٰوَةٌۭ ۖ وَلَهُم عَذَابٌ عَظِيمٌۭ ﴿٧﴾\\
\textamh{8.\  } & وَمِنَ ٱلنَّاسِ مَن يَقُولُ ءَامَنَّا بِٱللَّهِ وَبِٱليَومِ ٱلءَاخِرِ وَمَا هُم بِمُؤمِنِينَ ﴿٨﴾\\
\textamh{9.\  } & يُخَـٰدِعُونَ ٱللَّهَ وَٱلَّذِينَ ءَامَنُوا۟ وَمَا يَخدَعُونَ إِلَّآ أَنفُسَهُم وَمَا يَشعُرُونَ ﴿٩﴾\\
\textamh{10.\  } & فِى قُلُوبِهِم مَّرَضٌۭ فَزَادَهُمُ ٱللَّهُ مَرَضًۭا ۖ وَلَهُم عَذَابٌ أَلِيمٌۢ بِمَا كَانُوا۟ يَكذِبُونَ ﴿١٠﴾\\
\textamh{11.\  } & وَإِذَا قِيلَ لَهُم لَا تُفسِدُوا۟ فِى ٱلأَرضِ قَالُوٓا۟ إِنَّمَا نَحنُ مُصلِحُونَ ﴿١١﴾\\
\textamh{12.\  } & أَلَآ إِنَّهُم هُمُ ٱلمُفسِدُونَ وَلَـٰكِن لَّا يَشعُرُونَ ﴿١٢﴾\\
\textamh{13.\  } & وَإِذَا قِيلَ لَهُم ءَامِنُوا۟ كَمَآ ءَامَنَ ٱلنَّاسُ قَالُوٓا۟ أَنُؤمِنُ كَمَآ ءَامَنَ ٱلسُّفَهَآءُ ۗ أَلَآ إِنَّهُم هُمُ ٱلسُّفَهَآءُ وَلَـٰكِن لَّا يَعلَمُونَ ﴿١٣﴾\\
\textamh{14.\  } & وَإِذَا لَقُوا۟ ٱلَّذِينَ ءَامَنُوا۟ قَالُوٓا۟ ءَامَنَّا وَإِذَا خَلَوا۟ إِلَىٰ شَيَـٰطِينِهِم قَالُوٓا۟ إِنَّا مَعَكُم إِنَّمَا نَحنُ مُستَهزِءُونَ ﴿١٤﴾\\
\textamh{15.\  } & ٱللَّهُ يَستَهزِئُ بِهِم وَيَمُدُّهُم فِى طُغيَـٰنِهِم يَعمَهُونَ ﴿١٥﴾\\
\textamh{16.\  } & أُو۟لَـٰٓئِكَ ٱلَّذِينَ ٱشتَرَوُا۟ ٱلضَّلَـٰلَةَ بِٱلهُدَىٰ فَمَا رَبِحَت تِّجَٰرَتُهُم وَمَا كَانُوا۟ مُهتَدِينَ ﴿١٦﴾\\
\textamh{17.\  } & مَثَلُهُم كَمَثَلِ ٱلَّذِى ٱستَوقَدَ نَارًۭا فَلَمَّآ أَضَآءَت مَا حَولَهُۥ ذَهَبَ ٱللَّهُ بِنُورِهِم وَتَرَكَهُم فِى ظُلُمَـٰتٍۢ لَّا يُبصِرُونَ ﴿١٧﴾\\
\textamh{18.\  } & صُمٌّۢ بُكمٌ عُمىٌۭ فَهُم لَا يَرجِعُونَ ﴿١٨﴾\\
\textamh{19.\  } & أَو كَصَيِّبٍۢ مِّنَ ٱلسَّمَآءِ فِيهِ ظُلُمَـٰتٌۭ وَرَعدٌۭ وَبَرقٌۭ يَجعَلُونَ أَصَـٰبِعَهُم فِىٓ ءَاذَانِهِم مِّنَ ٱلصَّوَٟعِقِ حَذَرَ ٱلمَوتِ ۚ وَٱللَّهُ مُحِيطٌۢ بِٱلكَـٰفِرِينَ ﴿١٩﴾\\
\textamh{20.\  } & يَكَادُ ٱلبَرقُ يَخطَفُ أَبصَـٰرَهُم ۖ كُلَّمَآ أَضَآءَ لَهُم مَّشَوا۟ فِيهِ وَإِذَآ أَظلَمَ عَلَيهِم قَامُوا۟ ۚ وَلَو شَآءَ ٱللَّهُ لَذَهَبَ بِسَمعِهِم وَأَبصَـٰرِهِم ۚ إِنَّ ٱللَّهَ عَلَىٰ كُلِّ شَىءٍۢ قَدِيرٌۭ ﴿٢٠﴾\\
\textamh{21.\  } & يَـٰٓأَيُّهَا ٱلنَّاسُ ٱعبُدُوا۟ رَبَّكُمُ ٱلَّذِى خَلَقَكُم وَٱلَّذِينَ مِن قَبلِكُم لَعَلَّكُم تَتَّقُونَ ﴿٢١﴾\\
\textamh{22.\  } & ٱلَّذِى جَعَلَ لَكُمُ ٱلأَرضَ فِرَٰشًۭا وَٱلسَّمَآءَ بِنَآءًۭ وَأَنزَلَ مِنَ ٱلسَّمَآءِ مَآءًۭ فَأَخرَجَ بِهِۦ مِنَ ٱلثَّمَرَٰتِ رِزقًۭا لَّكُم ۖ فَلَا تَجعَلُوا۟ لِلَّهِ أَندَادًۭا وَأَنتُم تَعلَمُونَ ﴿٢٢﴾\\
\textamh{23.\  } & وَإِن كُنتُم فِى رَيبٍۢ مِّمَّا نَزَّلنَا عَلَىٰ عَبدِنَا فَأتُوا۟ بِسُورَةٍۢ مِّن مِّثلِهِۦ وَٱدعُوا۟ شُهَدَآءَكُم مِّن دُونِ ٱللَّهِ إِن كُنتُم صَـٰدِقِينَ ﴿٢٣﴾\\
\textamh{24.\  } & فَإِن لَّم تَفعَلُوا۟ وَلَن تَفعَلُوا۟ فَٱتَّقُوا۟ ٱلنَّارَ ٱلَّتِى وَقُودُهَا ٱلنَّاسُ وَٱلحِجَارَةُ ۖ أُعِدَّت لِلكَـٰفِرِينَ ﴿٢٤﴾\\
\textamh{25.\  } & وَبَشِّرِ ٱلَّذِينَ ءَامَنُوا۟ وَعَمِلُوا۟ ٱلصَّـٰلِحَـٰتِ أَنَّ لَهُم جَنَّـٰتٍۢ تَجرِى مِن تَحتِهَا ٱلأَنهَـٰرُ ۖ كُلَّمَا رُزِقُوا۟ مِنهَا مِن ثَمَرَةٍۢ رِّزقًۭا ۙ قَالُوا۟ هَـٰذَا ٱلَّذِى رُزِقنَا مِن قَبلُ ۖ وَأُتُوا۟ بِهِۦ مُتَشَـٰبِهًۭا ۖ وَلَهُم فِيهَآ أَزوَٟجٌۭ مُّطَهَّرَةٌۭ ۖ وَهُم فِيهَا خَـٰلِدُونَ ﴿٢٥﴾\\
\textamh{26.\  } & ۞ إِنَّ ٱللَّهَ لَا يَستَحىِۦٓ أَن يَضرِبَ مَثَلًۭا مَّا بَعُوضَةًۭ فَمَا فَوقَهَا ۚ فَأَمَّا ٱلَّذِينَ ءَامَنُوا۟ فَيَعلَمُونَ أَنَّهُ ٱلحَقُّ مِن رَّبِّهِم ۖ وَأَمَّا ٱلَّذِينَ كَفَرُوا۟ فَيَقُولُونَ مَاذَآ أَرَادَ ٱللَّهُ بِهَـٰذَا مَثَلًۭا ۘ يُضِلُّ بِهِۦ كَثِيرًۭا وَيَهدِى بِهِۦ كَثِيرًۭا ۚ وَمَا يُضِلُّ بِهِۦٓ إِلَّا ٱلفَـٰسِقِينَ ﴿٢٦﴾\\
\textamh{27.\  } & ٱلَّذِينَ يَنقُضُونَ عَهدَ ٱللَّهِ مِنۢ بَعدِ مِيثَـٰقِهِۦ وَيَقطَعُونَ مَآ أَمَرَ ٱللَّهُ بِهِۦٓ أَن يُوصَلَ وَيُفسِدُونَ فِى ٱلأَرضِ ۚ أُو۟لَـٰٓئِكَ هُمُ ٱلخَـٰسِرُونَ ﴿٢٧﴾\\
\textamh{28.\  } & كَيفَ تَكفُرُونَ بِٱللَّهِ وَكُنتُم أَموَٟتًۭا فَأَحيَـٰكُم ۖ ثُمَّ يُمِيتُكُم ثُمَّ يُحيِيكُم ثُمَّ إِلَيهِ تُرجَعُونَ ﴿٢٨﴾\\
\textamh{29.\  } & هُوَ ٱلَّذِى خَلَقَ لَكُم مَّا فِى ٱلأَرضِ جَمِيعًۭا ثُمَّ ٱستَوَىٰٓ إِلَى ٱلسَّمَآءِ فَسَوَّىٰهُنَّ سَبعَ سَمَـٰوَٟتٍۢ ۚ وَهُوَ بِكُلِّ شَىءٍ عَلِيمٌۭ ﴿٢٩﴾\\
\textamh{30.\  } & وَإِذ قَالَ رَبُّكَ لِلمَلَـٰٓئِكَةِ إِنِّى جَاعِلٌۭ فِى ٱلأَرضِ خَلِيفَةًۭ ۖ قَالُوٓا۟ أَتَجعَلُ فِيهَا مَن يُفسِدُ فِيهَا وَيَسفِكُ ٱلدِّمَآءَ وَنَحنُ نُسَبِّحُ بِحَمدِكَ وَنُقَدِّسُ لَكَ ۖ قَالَ إِنِّىٓ أَعلَمُ مَا لَا تَعلَمُونَ ﴿٣٠﴾\\
\textamh{31.\  } & وَعَلَّمَ ءَادَمَ ٱلأَسمَآءَ كُلَّهَا ثُمَّ عَرَضَهُم عَلَى ٱلمَلَـٰٓئِكَةِ فَقَالَ أَنۢبِـُٔونِى بِأَسمَآءِ هَـٰٓؤُلَآءِ إِن كُنتُم صَـٰدِقِينَ ﴿٣١﴾\\
\textamh{32.\  } & قَالُوا۟ سُبحَـٰنَكَ لَا عِلمَ لَنَآ إِلَّا مَا عَلَّمتَنَآ ۖ إِنَّكَ أَنتَ ٱلعَلِيمُ ٱلحَكِيمُ ﴿٣٢﴾\\
\textamh{33.\  } & قَالَ يَـٰٓـَٔادَمُ أَنۢبِئهُم بِأَسمَآئِهِم ۖ فَلَمَّآ أَنۢبَأَهُم بِأَسمَآئِهِم قَالَ أَلَم أَقُل لَّكُم إِنِّىٓ أَعلَمُ غَيبَ ٱلسَّمَـٰوَٟتِ وَٱلأَرضِ وَأَعلَمُ مَا تُبدُونَ وَمَا كُنتُم تَكتُمُونَ ﴿٣٣﴾\\
\textamh{34.\  } & وَإِذ قُلنَا لِلمَلَـٰٓئِكَةِ ٱسجُدُوا۟ لِءَادَمَ فَسَجَدُوٓا۟ إِلَّآ إِبلِيسَ أَبَىٰ وَٱستَكبَرَ وَكَانَ مِنَ ٱلكَـٰفِرِينَ ﴿٣٤﴾\\
\textamh{35.\  } & وَقُلنَا يَـٰٓـَٔادَمُ ٱسكُن أَنتَ وَزَوجُكَ ٱلجَنَّةَ وَكُلَا مِنهَا رَغَدًا حَيثُ شِئتُمَا وَلَا تَقرَبَا هَـٰذِهِ ٱلشَّجَرَةَ فَتَكُونَا مِنَ ٱلظَّـٰلِمِينَ ﴿٣٥﴾\\
\textamh{36.\  } & فَأَزَلَّهُمَا ٱلشَّيطَٰنُ عَنهَا فَأَخرَجَهُمَا مِمَّا كَانَا فِيهِ ۖ وَقُلنَا ٱهبِطُوا۟ بَعضُكُم لِبَعضٍ عَدُوٌّۭ ۖ وَلَكُم فِى ٱلأَرضِ مُستَقَرٌّۭ وَمَتَـٰعٌ إِلَىٰ حِينٍۢ ﴿٣٦﴾\\
\textamh{37.\  } & فَتَلَقَّىٰٓ ءَادَمُ مِن رَّبِّهِۦ كَلِمَـٰتٍۢ فَتَابَ عَلَيهِ ۚ إِنَّهُۥ هُوَ ٱلتَّوَّابُ ٱلرَّحِيمُ ﴿٣٧﴾\\
\textamh{38.\  } & قُلنَا ٱهبِطُوا۟ مِنهَا جَمِيعًۭا ۖ فَإِمَّا يَأتِيَنَّكُم مِّنِّى هُدًۭى فَمَن تَبِعَ هُدَاىَ فَلَا خَوفٌ عَلَيهِم وَلَا هُم يَحزَنُونَ ﴿٣٨﴾\\
\textamh{39.\  } & وَٱلَّذِينَ كَفَرُوا۟ وَكَذَّبُوا۟ بِـَٔايَـٰتِنَآ أُو۟لَـٰٓئِكَ أَصحَـٰبُ ٱلنَّارِ ۖ هُم فِيهَا خَـٰلِدُونَ ﴿٣٩﴾\\
\textamh{40.\  } & يَـٰبَنِىٓ إِسرَٰٓءِيلَ ٱذكُرُوا۟ نِعمَتِىَ ٱلَّتِىٓ أَنعَمتُ عَلَيكُم وَأَوفُوا۟ بِعَهدِىٓ أُوفِ بِعَهدِكُم وَإِيَّٰىَ فَٱرهَبُونِ ﴿٤٠﴾\\
\textamh{41.\  } & وَءَامِنُوا۟ بِمَآ أَنزَلتُ مُصَدِّقًۭا لِّمَا مَعَكُم وَلَا تَكُونُوٓا۟ أَوَّلَ كَافِرٍۭ بِهِۦ ۖ وَلَا تَشتَرُوا۟ بِـَٔايَـٰتِى ثَمَنًۭا قَلِيلًۭا وَإِيَّٰىَ فَٱتَّقُونِ ﴿٤١﴾\\
\textamh{42.\  } & وَلَا تَلبِسُوا۟ ٱلحَقَّ بِٱلبَٰطِلِ وَتَكتُمُوا۟ ٱلحَقَّ وَأَنتُم تَعلَمُونَ ﴿٤٢﴾\\
\textamh{43.\  } & وَأَقِيمُوا۟ ٱلصَّلَوٰةَ وَءَاتُوا۟ ٱلزَّكَوٰةَ وَٱركَعُوا۟ مَعَ ٱلرَّٟكِعِينَ ﴿٤٣﴾\\
\textamh{44.\  } & ۞ أَتَأمُرُونَ ٱلنَّاسَ بِٱلبِرِّ وَتَنسَونَ أَنفُسَكُم وَأَنتُم تَتلُونَ ٱلكِتَـٰبَ ۚ أَفَلَا تَعقِلُونَ ﴿٤٤﴾\\
\textamh{45.\  } & وَٱستَعِينُوا۟ بِٱلصَّبرِ وَٱلصَّلَوٰةِ ۚ وَإِنَّهَا لَكَبِيرَةٌ إِلَّا عَلَى ٱلخَـٰشِعِينَ ﴿٤٥﴾\\
\textamh{46.\  } & ٱلَّذِينَ يَظُنُّونَ أَنَّهُم مُّلَـٰقُوا۟ رَبِّهِم وَأَنَّهُم إِلَيهِ رَٰجِعُونَ ﴿٤٦﴾\\
\textamh{47.\  } & يَـٰبَنِىٓ إِسرَٰٓءِيلَ ٱذكُرُوا۟ نِعمَتِىَ ٱلَّتِىٓ أَنعَمتُ عَلَيكُم وَأَنِّى فَضَّلتُكُم عَلَى ٱلعَـٰلَمِينَ ﴿٤٧﴾\\
\textamh{48.\  } & وَٱتَّقُوا۟ يَومًۭا لَّا تَجزِى نَفسٌ عَن نَّفسٍۢ شَيـًۭٔا وَلَا يُقبَلُ مِنهَا شَفَـٰعَةٌۭ وَلَا يُؤخَذُ مِنهَا عَدلٌۭ وَلَا هُم يُنصَرُونَ ﴿٤٨﴾\\
\textamh{49.\  } & وَإِذ نَجَّينَـٰكُم مِّن ءَالِ فِرعَونَ يَسُومُونَكُم سُوٓءَ ٱلعَذَابِ يُذَبِّحُونَ أَبنَآءَكُم وَيَستَحيُونَ نِسَآءَكُم ۚ وَفِى ذَٟلِكُم بَلَآءٌۭ مِّن رَّبِّكُم عَظِيمٌۭ ﴿٤٩﴾\\
\textamh{50.\  } & وَإِذ فَرَقنَا بِكُمُ ٱلبَحرَ فَأَنجَينَـٰكُم وَأَغرَقنَآ ءَالَ فِرعَونَ وَأَنتُم تَنظُرُونَ ﴿٥٠﴾\\
\textamh{51.\  } & وَإِذ وَٟعَدنَا مُوسَىٰٓ أَربَعِينَ لَيلَةًۭ ثُمَّ ٱتَّخَذتُمُ ٱلعِجلَ مِنۢ بَعدِهِۦ وَأَنتُم ظَـٰلِمُونَ ﴿٥١﴾\\
\textamh{52.\  } & ثُمَّ عَفَونَا عَنكُم مِّنۢ بَعدِ ذَٟلِكَ لَعَلَّكُم تَشكُرُونَ ﴿٥٢﴾\\
\textamh{53.\  } & وَإِذ ءَاتَينَا مُوسَى ٱلكِتَـٰبَ وَٱلفُرقَانَ لَعَلَّكُم تَهتَدُونَ ﴿٥٣﴾\\
\textamh{54.\  } & وَإِذ قَالَ مُوسَىٰ لِقَومِهِۦ يَـٰقَومِ إِنَّكُم ظَلَمتُم أَنفُسَكُم بِٱتِّخَاذِكُمُ ٱلعِجلَ فَتُوبُوٓا۟ إِلَىٰ بَارِئِكُم فَٱقتُلُوٓا۟ أَنفُسَكُم ذَٟلِكُم خَيرٌۭ لَّكُم عِندَ بَارِئِكُم فَتَابَ عَلَيكُم ۚ إِنَّهُۥ هُوَ ٱلتَّوَّابُ ٱلرَّحِيمُ ﴿٥٤﴾\\
\textamh{55.\  } & وَإِذ قُلتُم يَـٰمُوسَىٰ لَن نُّؤمِنَ لَكَ حَتَّىٰ نَرَى ٱللَّهَ جَهرَةًۭ فَأَخَذَتكُمُ ٱلصَّـٰعِقَةُ وَأَنتُم تَنظُرُونَ ﴿٥٥﴾\\
\textamh{56.\  } & ثُمَّ بَعَثنَـٰكُم مِّنۢ بَعدِ مَوتِكُم لَعَلَّكُم تَشكُرُونَ ﴿٥٦﴾\\
\textamh{57.\  } & وَظَلَّلنَا عَلَيكُمُ ٱلغَمَامَ وَأَنزَلنَا عَلَيكُمُ ٱلمَنَّ وَٱلسَّلوَىٰ ۖ كُلُوا۟ مِن طَيِّبَٰتِ مَا رَزَقنَـٰكُم ۖ وَمَا ظَلَمُونَا وَلَـٰكِن كَانُوٓا۟ أَنفُسَهُم يَظلِمُونَ ﴿٥٧﴾\\
\textamh{58.\  } & وَإِذ قُلنَا ٱدخُلُوا۟ هَـٰذِهِ ٱلقَريَةَ فَكُلُوا۟ مِنهَا حَيثُ شِئتُم رَغَدًۭا وَٱدخُلُوا۟ ٱلبَابَ سُجَّدًۭا وَقُولُوا۟ حِطَّةٌۭ نَّغفِر لَكُم خَطَٰيَـٰكُم ۚ وَسَنَزِيدُ ٱلمُحسِنِينَ ﴿٥٨﴾\\
\textamh{59.\  } & فَبَدَّلَ ٱلَّذِينَ ظَلَمُوا۟ قَولًا غَيرَ ٱلَّذِى قِيلَ لَهُم فَأَنزَلنَا عَلَى ٱلَّذِينَ ظَلَمُوا۟ رِجزًۭا مِّنَ ٱلسَّمَآءِ بِمَا كَانُوا۟ يَفسُقُونَ ﴿٥٩﴾\\
\textamh{60.\  } & ۞ وَإِذِ ٱستَسقَىٰ مُوسَىٰ لِقَومِهِۦ فَقُلنَا ٱضرِب بِّعَصَاكَ ٱلحَجَرَ ۖ فَٱنفَجَرَت مِنهُ ٱثنَتَا عَشرَةَ عَينًۭا ۖ قَد عَلِمَ كُلُّ أُنَاسٍۢ مَّشرَبَهُم ۖ كُلُوا۟ وَٱشرَبُوا۟ مِن رِّزقِ ٱللَّهِ وَلَا تَعثَوا۟ فِى ٱلأَرضِ مُفسِدِينَ ﴿٦٠﴾\\
\textamh{61.\  } & وَإِذ قُلتُم يَـٰمُوسَىٰ لَن نَّصبِرَ عَلَىٰ طَعَامٍۢ وَٟحِدٍۢ فَٱدعُ لَنَا رَبَّكَ يُخرِج لَنَا مِمَّا تُنۢبِتُ ٱلأَرضُ مِنۢ بَقلِهَا وَقِثَّآئِهَا وَفُومِهَا وَعَدَسِهَا وَبَصَلِهَا ۖ قَالَ أَتَستَبدِلُونَ ٱلَّذِى هُوَ أَدنَىٰ بِٱلَّذِى هُوَ خَيرٌ ۚ ٱهبِطُوا۟ مِصرًۭا فَإِنَّ لَكُم مَّا سَأَلتُم ۗ وَضُرِبَت عَلَيهِمُ ٱلذِّلَّةُ وَٱلمَسكَنَةُ وَبَآءُو بِغَضَبٍۢ مِّنَ ٱللَّهِ ۗ ذَٟلِكَ بِأَنَّهُم كَانُوا۟ يَكفُرُونَ بِـَٔايَـٰتِ ٱللَّهِ وَيَقتُلُونَ ٱلنَّبِيِّۦنَ بِغَيرِ ٱلحَقِّ ۗ ذَٟلِكَ بِمَا عَصَوا۟ وَّكَانُوا۟ يَعتَدُونَ ﴿٦١﴾\\
\textamh{62.\  } & إِنَّ ٱلَّذِينَ ءَامَنُوا۟ وَٱلَّذِينَ هَادُوا۟ وَٱلنَّصَـٰرَىٰ وَٱلصَّـٰبِـِٔينَ مَن ءَامَنَ بِٱللَّهِ وَٱليَومِ ٱلءَاخِرِ وَعَمِلَ صَـٰلِحًۭا فَلَهُم أَجرُهُم عِندَ رَبِّهِم وَلَا خَوفٌ عَلَيهِم وَلَا هُم يَحزَنُونَ ﴿٦٢﴾\\
\textamh{63.\  } & وَإِذ أَخَذنَا مِيثَـٰقَكُم وَرَفَعنَا فَوقَكُمُ ٱلطُّورَ خُذُوا۟ مَآ ءَاتَينَـٰكُم بِقُوَّةٍۢ وَٱذكُرُوا۟ مَا فِيهِ لَعَلَّكُم تَتَّقُونَ ﴿٦٣﴾\\
\textamh{64.\  } & ثُمَّ تَوَلَّيتُم مِّنۢ بَعدِ ذَٟلِكَ ۖ فَلَولَا فَضلُ ٱللَّهِ عَلَيكُم وَرَحمَتُهُۥ لَكُنتُم مِّنَ ٱلخَـٰسِرِينَ ﴿٦٤﴾\\
\textamh{65.\  } & وَلَقَد عَلِمتُمُ ٱلَّذِينَ ٱعتَدَوا۟ مِنكُم فِى ٱلسَّبتِ فَقُلنَا لَهُم كُونُوا۟ قِرَدَةً خَـٰسِـِٔينَ ﴿٦٥﴾\\
\textamh{66.\  } & فَجَعَلنَـٰهَا نَكَـٰلًۭا لِّمَا بَينَ يَدَيهَا وَمَا خَلفَهَا وَمَوعِظَةًۭ لِّلمُتَّقِينَ ﴿٦٦﴾\\
\textamh{67.\  } & وَإِذ قَالَ مُوسَىٰ لِقَومِهِۦٓ إِنَّ ٱللَّهَ يَأمُرُكُم أَن تَذبَحُوا۟ بَقَرَةًۭ ۖ قَالُوٓا۟ أَتَتَّخِذُنَا هُزُوًۭا ۖ قَالَ أَعُوذُ بِٱللَّهِ أَن أَكُونَ مِنَ ٱلجَٰهِلِينَ ﴿٦٧﴾\\
\textamh{68.\  } & قَالُوا۟ ٱدعُ لَنَا رَبَّكَ يُبَيِّن لَّنَا مَا هِىَ ۚ قَالَ إِنَّهُۥ يَقُولُ إِنَّهَا بَقَرَةٌۭ لَّا فَارِضٌۭ وَلَا بِكرٌ عَوَانٌۢ بَينَ ذَٟلِكَ ۖ فَٱفعَلُوا۟ مَا تُؤمَرُونَ ﴿٦٨﴾\\
\textamh{69.\  } & قَالُوا۟ ٱدعُ لَنَا رَبَّكَ يُبَيِّن لَّنَا مَا لَونُهَا ۚ قَالَ إِنَّهُۥ يَقُولُ إِنَّهَا بَقَرَةٌۭ صَفرَآءُ فَاقِعٌۭ لَّونُهَا تَسُرُّ ٱلنَّـٰظِرِينَ ﴿٦٩﴾\\
\textamh{70.\  } & قَالُوا۟ ٱدعُ لَنَا رَبَّكَ يُبَيِّن لَّنَا مَا هِىَ إِنَّ ٱلبَقَرَ تَشَـٰبَهَ عَلَينَا وَإِنَّآ إِن شَآءَ ٱللَّهُ لَمُهتَدُونَ ﴿٧٠﴾\\
\textamh{71.\  } & قَالَ إِنَّهُۥ يَقُولُ إِنَّهَا بَقَرَةٌۭ لَّا ذَلُولٌۭ تُثِيرُ ٱلأَرضَ وَلَا تَسقِى ٱلحَرثَ مُسَلَّمَةٌۭ لَّا شِيَةَ فِيهَا ۚ قَالُوا۟ ٱلـَٰٔنَ جِئتَ بِٱلحَقِّ ۚ فَذَبَحُوهَا وَمَا كَادُوا۟ يَفعَلُونَ ﴿٧١﴾\\
\textamh{72.\  } & وَإِذ قَتَلتُم نَفسًۭا فَٱدَّٰرَٰٔتُم فِيهَا ۖ وَٱللَّهُ مُخرِجٌۭ مَّا كُنتُم تَكتُمُونَ ﴿٧٢﴾\\
\textamh{73.\  } & فَقُلنَا ٱضرِبُوهُ بِبَعضِهَا ۚ كَذَٟلِكَ يُحىِ ٱللَّهُ ٱلمَوتَىٰ وَيُرِيكُم ءَايَـٰتِهِۦ لَعَلَّكُم تَعقِلُونَ ﴿٧٣﴾\\
\textamh{74.\  } & ثُمَّ قَسَت قُلُوبُكُم مِّنۢ بَعدِ ذَٟلِكَ فَهِىَ كَٱلحِجَارَةِ أَو أَشَدُّ قَسوَةًۭ ۚ وَإِنَّ مِنَ ٱلحِجَارَةِ لَمَا يَتَفَجَّرُ مِنهُ ٱلأَنهَـٰرُ ۚ وَإِنَّ مِنهَا لَمَا يَشَّقَّقُ فَيَخرُجُ مِنهُ ٱلمَآءُ ۚ وَإِنَّ مِنهَا لَمَا يَهبِطُ مِن خَشيَةِ ٱللَّهِ ۗ وَمَا ٱللَّهُ بِغَٰفِلٍ عَمَّا تَعمَلُونَ ﴿٧٤﴾\\
\textamh{75.\  } & ۞ أَفَتَطمَعُونَ أَن يُؤمِنُوا۟ لَكُم وَقَد كَانَ فَرِيقٌۭ مِّنهُم يَسمَعُونَ كَلَـٰمَ ٱللَّهِ ثُمَّ يُحَرِّفُونَهُۥ مِنۢ بَعدِ مَا عَقَلُوهُ وَهُم يَعلَمُونَ ﴿٧٥﴾\\
\textamh{76.\  } & وَإِذَا لَقُوا۟ ٱلَّذِينَ ءَامَنُوا۟ قَالُوٓا۟ ءَامَنَّا وَإِذَا خَلَا بَعضُهُم إِلَىٰ بَعضٍۢ قَالُوٓا۟ أَتُحَدِّثُونَهُم بِمَا فَتَحَ ٱللَّهُ عَلَيكُم لِيُحَآجُّوكُم بِهِۦ عِندَ رَبِّكُم ۚ أَفَلَا تَعقِلُونَ ﴿٧٦﴾\\
\textamh{77.\  } & أَوَلَا يَعلَمُونَ أَنَّ ٱللَّهَ يَعلَمُ مَا يُسِرُّونَ وَمَا يُعلِنُونَ ﴿٧٧﴾\\
\textamh{78.\  } & وَمِنهُم أُمِّيُّونَ لَا يَعلَمُونَ ٱلكِتَـٰبَ إِلَّآ أَمَانِىَّ وَإِن هُم إِلَّا يَظُنُّونَ ﴿٧٨﴾\\
\textamh{79.\  } & فَوَيلٌۭ لِّلَّذِينَ يَكتُبُونَ ٱلكِتَـٰبَ بِأَيدِيهِم ثُمَّ يَقُولُونَ هَـٰذَا مِن عِندِ ٱللَّهِ لِيَشتَرُوا۟ بِهِۦ ثَمَنًۭا قَلِيلًۭا ۖ فَوَيلٌۭ لَّهُم مِّمَّا كَتَبَت أَيدِيهِم وَوَيلٌۭ لَّهُم مِّمَّا يَكسِبُونَ ﴿٧٩﴾\\
\textamh{80.\  } & وَقَالُوا۟ لَن تَمَسَّنَا ٱلنَّارُ إِلَّآ أَيَّامًۭا مَّعدُودَةًۭ ۚ قُل أَتَّخَذتُم عِندَ ٱللَّهِ عَهدًۭا فَلَن يُخلِفَ ٱللَّهُ عَهدَهُۥٓ ۖ أَم تَقُولُونَ عَلَى ٱللَّهِ مَا لَا تَعلَمُونَ ﴿٨٠﴾\\
\textamh{81.\  } & بَلَىٰ مَن كَسَبَ سَيِّئَةًۭ وَأَحَـٰطَت بِهِۦ خَطِيٓـَٔتُهُۥ فَأُو۟لَـٰٓئِكَ أَصحَـٰبُ ٱلنَّارِ ۖ هُم فِيهَا خَـٰلِدُونَ ﴿٨١﴾\\
\textamh{82.\  } & وَٱلَّذِينَ ءَامَنُوا۟ وَعَمِلُوا۟ ٱلصَّـٰلِحَـٰتِ أُو۟لَـٰٓئِكَ أَصحَـٰبُ ٱلجَنَّةِ ۖ هُم فِيهَا خَـٰلِدُونَ ﴿٨٢﴾\\
\textamh{83.\  } & وَإِذ أَخَذنَا مِيثَـٰقَ بَنِىٓ إِسرَٰٓءِيلَ لَا تَعبُدُونَ إِلَّا ٱللَّهَ وَبِٱلوَٟلِدَينِ إِحسَانًۭا وَذِى ٱلقُربَىٰ وَٱليَتَـٰمَىٰ وَٱلمَسَـٰكِينِ وَقُولُوا۟ لِلنَّاسِ حُسنًۭا وَأَقِيمُوا۟ ٱلصَّلَوٰةَ وَءَاتُوا۟ ٱلزَّكَوٰةَ ثُمَّ تَوَلَّيتُم إِلَّا قَلِيلًۭا مِّنكُم وَأَنتُم مُّعرِضُونَ ﴿٨٣﴾\\
\textamh{84.\  } & وَإِذ أَخَذنَا مِيثَـٰقَكُم لَا تَسفِكُونَ دِمَآءَكُم وَلَا تُخرِجُونَ أَنفُسَكُم مِّن دِيَـٰرِكُم ثُمَّ أَقرَرتُم وَأَنتُم تَشهَدُونَ ﴿٨٤﴾\\
\textamh{85.\  } & ثُمَّ أَنتُم هَـٰٓؤُلَآءِ تَقتُلُونَ أَنفُسَكُم وَتُخرِجُونَ فَرِيقًۭا مِّنكُم مِّن دِيَـٰرِهِم تَظَـٰهَرُونَ عَلَيهِم بِٱلإِثمِ وَٱلعُدوَٟنِ وَإِن يَأتُوكُم أُسَـٰرَىٰ تُفَـٰدُوهُم وَهُوَ مُحَرَّمٌ عَلَيكُم إِخرَاجُهُم ۚ أَفَتُؤمِنُونَ بِبَعضِ ٱلكِتَـٰبِ وَتَكفُرُونَ بِبَعضٍۢ ۚ فَمَا جَزَآءُ مَن يَفعَلُ ذَٟلِكَ مِنكُم إِلَّا خِزىٌۭ فِى ٱلحَيَوٰةِ ٱلدُّنيَا ۖ وَيَومَ ٱلقِيَـٰمَةِ يُرَدُّونَ إِلَىٰٓ أَشَدِّ ٱلعَذَابِ ۗ وَمَا ٱللَّهُ بِغَٰفِلٍ عَمَّا تَعمَلُونَ ﴿٨٥﴾\\
\textamh{86.\  } & أُو۟لَـٰٓئِكَ ٱلَّذِينَ ٱشتَرَوُا۟ ٱلحَيَوٰةَ ٱلدُّنيَا بِٱلءَاخِرَةِ ۖ فَلَا يُخَفَّفُ عَنهُمُ ٱلعَذَابُ وَلَا هُم يُنصَرُونَ ﴿٨٦﴾\\
\textamh{87.\  } & وَلَقَد ءَاتَينَا مُوسَى ٱلكِتَـٰبَ وَقَفَّينَا مِنۢ بَعدِهِۦ بِٱلرُّسُلِ ۖ وَءَاتَينَا عِيسَى ٱبنَ مَريَمَ ٱلبَيِّنَـٰتِ وَأَيَّدنَـٰهُ بِرُوحِ ٱلقُدُسِ ۗ أَفَكُلَّمَا جَآءَكُم رَسُولٌۢ بِمَا لَا تَهوَىٰٓ أَنفُسُكُمُ ٱستَكبَرتُم فَفَرِيقًۭا كَذَّبتُم وَفَرِيقًۭا تَقتُلُونَ ﴿٨٧﴾\\
\textamh{88.\  } & وَقَالُوا۟ قُلُوبُنَا غُلفٌۢ ۚ بَل لَّعَنَهُمُ ٱللَّهُ بِكُفرِهِم فَقَلِيلًۭا مَّا يُؤمِنُونَ ﴿٨٨﴾\\
\textamh{89.\  } & وَلَمَّا جَآءَهُم كِتَـٰبٌۭ مِّن عِندِ ٱللَّهِ مُصَدِّقٌۭ لِّمَا مَعَهُم وَكَانُوا۟ مِن قَبلُ يَستَفتِحُونَ عَلَى ٱلَّذِينَ كَفَرُوا۟ فَلَمَّا جَآءَهُم مَّا عَرَفُوا۟ كَفَرُوا۟ بِهِۦ ۚ فَلَعنَةُ ٱللَّهِ عَلَى ٱلكَـٰفِرِينَ ﴿٨٩﴾\\
\textamh{90.\  } & بِئسَمَا ٱشتَرَوا۟ بِهِۦٓ أَنفُسَهُم أَن يَكفُرُوا۟ بِمَآ أَنزَلَ ٱللَّهُ بَغيًا أَن يُنَزِّلَ ٱللَّهُ مِن فَضلِهِۦ عَلَىٰ مَن يَشَآءُ مِن عِبَادِهِۦ ۖ فَبَآءُو بِغَضَبٍ عَلَىٰ غَضَبٍۢ ۚ وَلِلكَـٰفِرِينَ عَذَابٌۭ مُّهِينٌۭ ﴿٩٠﴾\\
\textamh{91.\  } & وَإِذَا قِيلَ لَهُم ءَامِنُوا۟ بِمَآ أَنزَلَ ٱللَّهُ قَالُوا۟ نُؤمِنُ بِمَآ أُنزِلَ عَلَينَا وَيَكفُرُونَ بِمَا وَرَآءَهُۥ وَهُوَ ٱلحَقُّ مُصَدِّقًۭا لِّمَا مَعَهُم ۗ قُل فَلِمَ تَقتُلُونَ أَنۢبِيَآءَ ٱللَّهِ مِن قَبلُ إِن كُنتُم مُّؤمِنِينَ ﴿٩١﴾\\
\textamh{92.\  } & ۞ وَلَقَد جَآءَكُم مُّوسَىٰ بِٱلبَيِّنَـٰتِ ثُمَّ ٱتَّخَذتُمُ ٱلعِجلَ مِنۢ بَعدِهِۦ وَأَنتُم ظَـٰلِمُونَ ﴿٩٢﴾\\
\textamh{93.\  } & وَإِذ أَخَذنَا مِيثَـٰقَكُم وَرَفَعنَا فَوقَكُمُ ٱلطُّورَ خُذُوا۟ مَآ ءَاتَينَـٰكُم بِقُوَّةٍۢ وَٱسمَعُوا۟ ۖ قَالُوا۟ سَمِعنَا وَعَصَينَا وَأُشرِبُوا۟ فِى قُلُوبِهِمُ ٱلعِجلَ بِكُفرِهِم ۚ قُل بِئسَمَا يَأمُرُكُم بِهِۦٓ إِيمَـٰنُكُم إِن كُنتُم مُّؤمِنِينَ ﴿٩٣﴾\\
\textamh{94.\  } & قُل إِن كَانَت لَكُمُ ٱلدَّارُ ٱلءَاخِرَةُ عِندَ ٱللَّهِ خَالِصَةًۭ مِّن دُونِ ٱلنَّاسِ فَتَمَنَّوُا۟ ٱلمَوتَ إِن كُنتُم صَـٰدِقِينَ ﴿٩٤﴾\\
\textamh{95.\  } & وَلَن يَتَمَنَّوهُ أَبَدًۢا بِمَا قَدَّمَت أَيدِيهِم ۗ وَٱللَّهُ عَلِيمٌۢ بِٱلظَّـٰلِمِينَ ﴿٩٥﴾\\
\textamh{96.\  } & وَلَتَجِدَنَّهُم أَحرَصَ ٱلنَّاسِ عَلَىٰ حَيَوٰةٍۢ وَمِنَ ٱلَّذِينَ أَشرَكُوا۟ ۚ يَوَدُّ أَحَدُهُم لَو يُعَمَّرُ أَلفَ سَنَةٍۢ وَمَا هُوَ بِمُزَحزِحِهِۦ مِنَ ٱلعَذَابِ أَن يُعَمَّرَ ۗ وَٱللَّهُ بَصِيرٌۢ بِمَا يَعمَلُونَ ﴿٩٦﴾\\
\textamh{97.\  } & قُل مَن كَانَ عَدُوًّۭا لِّجِبرِيلَ فَإِنَّهُۥ نَزَّلَهُۥ عَلَىٰ قَلبِكَ بِإِذنِ ٱللَّهِ مُصَدِّقًۭا لِّمَا بَينَ يَدَيهِ وَهُدًۭى وَبُشرَىٰ لِلمُؤمِنِينَ ﴿٩٧﴾\\
\textamh{98.\  } & مَن كَانَ عَدُوًّۭا لِّلَّهِ وَمَلَـٰٓئِكَتِهِۦ وَرُسُلِهِۦ وَجِبرِيلَ وَمِيكَىٰلَ فَإِنَّ ٱللَّهَ عَدُوٌّۭ لِّلكَـٰفِرِينَ ﴿٩٨﴾\\
\textamh{99.\  } & وَلَقَد أَنزَلنَآ إِلَيكَ ءَايَـٰتٍۭ بَيِّنَـٰتٍۢ ۖ وَمَا يَكفُرُ بِهَآ إِلَّا ٱلفَـٰسِقُونَ ﴿٩٩﴾\\
\textamh{100.\  } & أَوَكُلَّمَا عَـٰهَدُوا۟ عَهدًۭا نَّبَذَهُۥ فَرِيقٌۭ مِّنهُم ۚ بَل أَكثَرُهُم لَا يُؤمِنُونَ ﴿١٠٠﴾\\
\textamh{101.\  } & وَلَمَّا جَآءَهُم رَسُولٌۭ مِّن عِندِ ٱللَّهِ مُصَدِّقٌۭ لِّمَا مَعَهُم نَبَذَ فَرِيقٌۭ مِّنَ ٱلَّذِينَ أُوتُوا۟ ٱلكِتَـٰبَ كِتَـٰبَ ٱللَّهِ وَرَآءَ ظُهُورِهِم كَأَنَّهُم لَا يَعلَمُونَ ﴿١٠١﴾\\
\textamh{102.\  } & وَٱتَّبَعُوا۟ مَا تَتلُوا۟ ٱلشَّيَـٰطِينُ عَلَىٰ مُلكِ سُلَيمَـٰنَ ۖ وَمَا كَفَرَ سُلَيمَـٰنُ وَلَـٰكِنَّ ٱلشَّيَـٰطِينَ كَفَرُوا۟ يُعَلِّمُونَ ٱلنَّاسَ ٱلسِّحرَ وَمَآ أُنزِلَ عَلَى ٱلمَلَكَينِ بِبَابِلَ هَـٰرُوتَ وَمَـٰرُوتَ ۚ وَمَا يُعَلِّمَانِ مِن أَحَدٍ حَتَّىٰ يَقُولَآ إِنَّمَا نَحنُ فِتنَةٌۭ فَلَا تَكفُر ۖ فَيَتَعَلَّمُونَ مِنهُمَا مَا يُفَرِّقُونَ بِهِۦ بَينَ ٱلمَرءِ وَزَوجِهِۦ ۚ وَمَا هُم بِضَآرِّينَ بِهِۦ مِن أَحَدٍ إِلَّا بِإِذنِ ٱللَّهِ ۚ وَيَتَعَلَّمُونَ مَا يَضُرُّهُم وَلَا يَنفَعُهُم ۚ وَلَقَد عَلِمُوا۟ لَمَنِ ٱشتَرَىٰهُ مَا لَهُۥ فِى ٱلءَاخِرَةِ مِن خَلَـٰقٍۢ ۚ وَلَبِئسَ مَا شَرَوا۟ بِهِۦٓ أَنفُسَهُم ۚ لَو كَانُوا۟ يَعلَمُونَ ﴿١٠٢﴾\\
\textamh{103.\  } & وَلَو أَنَّهُم ءَامَنُوا۟ وَٱتَّقَوا۟ لَمَثُوبَةٌۭ مِّن عِندِ ٱللَّهِ خَيرٌۭ ۖ لَّو كَانُوا۟ يَعلَمُونَ ﴿١٠٣﴾\\
\textamh{104.\  } & يَـٰٓأَيُّهَا ٱلَّذِينَ ءَامَنُوا۟ لَا تَقُولُوا۟ رَٰعِنَا وَقُولُوا۟ ٱنظُرنَا وَٱسمَعُوا۟ ۗ وَلِلكَـٰفِرِينَ عَذَابٌ أَلِيمٌۭ ﴿١٠٤﴾\\
\textamh{105.\  } & مَّا يَوَدُّ ٱلَّذِينَ كَفَرُوا۟ مِن أَهلِ ٱلكِتَـٰبِ وَلَا ٱلمُشرِكِينَ أَن يُنَزَّلَ عَلَيكُم مِّن خَيرٍۢ مِّن رَّبِّكُم ۗ وَٱللَّهُ يَختَصُّ بِرَحمَتِهِۦ مَن يَشَآءُ ۚ وَٱللَّهُ ذُو ٱلفَضلِ ٱلعَظِيمِ ﴿١٠٥﴾\\
\textamh{106.\  } & ۞ مَا نَنسَخ مِن ءَايَةٍ أَو نُنسِهَا نَأتِ بِخَيرٍۢ مِّنهَآ أَو مِثلِهَآ ۗ أَلَم تَعلَم أَنَّ ٱللَّهَ عَلَىٰ كُلِّ شَىءٍۢ قَدِيرٌ ﴿١٠٦﴾\\
\textamh{107.\  } & أَلَم تَعلَم أَنَّ ٱللَّهَ لَهُۥ مُلكُ ٱلسَّمَـٰوَٟتِ وَٱلأَرضِ ۗ وَمَا لَكُم مِّن دُونِ ٱللَّهِ مِن وَلِىٍّۢ وَلَا نَصِيرٍ ﴿١٠٧﴾\\
\textamh{108.\  } & أَم تُرِيدُونَ أَن تَسـَٔلُوا۟ رَسُولَكُم كَمَا سُئِلَ مُوسَىٰ مِن قَبلُ ۗ وَمَن يَتَبَدَّلِ ٱلكُفرَ بِٱلإِيمَـٰنِ فَقَد ضَلَّ سَوَآءَ ٱلسَّبِيلِ ﴿١٠٨﴾\\
\textamh{109.\  } & وَدَّ كَثِيرٌۭ مِّن أَهلِ ٱلكِتَـٰبِ لَو يَرُدُّونَكُم مِّنۢ بَعدِ إِيمَـٰنِكُم كُفَّارًا حَسَدًۭا مِّن عِندِ أَنفُسِهِم مِّنۢ بَعدِ مَا تَبَيَّنَ لَهُمُ ٱلحَقُّ ۖ فَٱعفُوا۟ وَٱصفَحُوا۟ حَتَّىٰ يَأتِىَ ٱللَّهُ بِأَمرِهِۦٓ ۗ إِنَّ ٱللَّهَ عَلَىٰ كُلِّ شَىءٍۢ قَدِيرٌۭ ﴿١٠٩﴾\\
\textamh{110.\  } & وَأَقِيمُوا۟ ٱلصَّلَوٰةَ وَءَاتُوا۟ ٱلزَّكَوٰةَ ۚ وَمَا تُقَدِّمُوا۟ لِأَنفُسِكُم مِّن خَيرٍۢ تَجِدُوهُ عِندَ ٱللَّهِ ۗ إِنَّ ٱللَّهَ بِمَا تَعمَلُونَ بَصِيرٌۭ ﴿١١٠﴾\\
\textamh{111.\  } & وَقَالُوا۟ لَن يَدخُلَ ٱلجَنَّةَ إِلَّا مَن كَانَ هُودًا أَو نَصَـٰرَىٰ ۗ تِلكَ أَمَانِيُّهُم ۗ قُل هَاتُوا۟ بُرهَـٰنَكُم إِن كُنتُم صَـٰدِقِينَ ﴿١١١﴾\\
\textamh{112.\  } & بَلَىٰ مَن أَسلَمَ وَجهَهُۥ لِلَّهِ وَهُوَ مُحسِنٌۭ فَلَهُۥٓ أَجرُهُۥ عِندَ رَبِّهِۦ وَلَا خَوفٌ عَلَيهِم وَلَا هُم يَحزَنُونَ ﴿١١٢﴾\\
\textamh{113.\  } & وَقَالَتِ ٱليَهُودُ لَيسَتِ ٱلنَّصَـٰرَىٰ عَلَىٰ شَىءٍۢ وَقَالَتِ ٱلنَّصَـٰرَىٰ لَيسَتِ ٱليَهُودُ عَلَىٰ شَىءٍۢ وَهُم يَتلُونَ ٱلكِتَـٰبَ ۗ كَذَٟلِكَ قَالَ ٱلَّذِينَ لَا يَعلَمُونَ مِثلَ قَولِهِم ۚ فَٱللَّهُ يَحكُمُ بَينَهُم يَومَ ٱلقِيَـٰمَةِ فِيمَا كَانُوا۟ فِيهِ يَختَلِفُونَ ﴿١١٣﴾\\
\textamh{114.\  } & وَمَن أَظلَمُ مِمَّن مَّنَعَ مَسَـٰجِدَ ٱللَّهِ أَن يُذكَرَ فِيهَا ٱسمُهُۥ وَسَعَىٰ فِى خَرَابِهَآ ۚ أُو۟لَـٰٓئِكَ مَا كَانَ لَهُم أَن يَدخُلُوهَآ إِلَّا خَآئِفِينَ ۚ لَهُم فِى ٱلدُّنيَا خِزىٌۭ وَلَهُم فِى ٱلءَاخِرَةِ عَذَابٌ عَظِيمٌۭ ﴿١١٤﴾\\
\textamh{115.\  } & وَلِلَّهِ ٱلمَشرِقُ وَٱلمَغرِبُ ۚ فَأَينَمَا تُوَلُّوا۟ فَثَمَّ وَجهُ ٱللَّهِ ۚ إِنَّ ٱللَّهَ وَٟسِعٌ عَلِيمٌۭ ﴿١١٥﴾\\
\textamh{116.\  } & وَقَالُوا۟ ٱتَّخَذَ ٱللَّهُ وَلَدًۭا ۗ سُبحَـٰنَهُۥ ۖ بَل لَّهُۥ مَا فِى ٱلسَّمَـٰوَٟتِ وَٱلأَرضِ ۖ كُلٌّۭ لَّهُۥ قَـٰنِتُونَ ﴿١١٦﴾\\
\textamh{117.\  } & بَدِيعُ ٱلسَّمَـٰوَٟتِ وَٱلأَرضِ ۖ وَإِذَا قَضَىٰٓ أَمرًۭا فَإِنَّمَا يَقُولُ لَهُۥ كُن فَيَكُونُ ﴿١١٧﴾\\
\textamh{118.\  } & وَقَالَ ٱلَّذِينَ لَا يَعلَمُونَ لَولَا يُكَلِّمُنَا ٱللَّهُ أَو تَأتِينَآ ءَايَةٌۭ ۗ كَذَٟلِكَ قَالَ ٱلَّذِينَ مِن قَبلِهِم مِّثلَ قَولِهِم ۘ تَشَـٰبَهَت قُلُوبُهُم ۗ قَد بَيَّنَّا ٱلءَايَـٰتِ لِقَومٍۢ يُوقِنُونَ ﴿١١٨﴾\\
\textamh{119.\  } & إِنَّآ أَرسَلنَـٰكَ بِٱلحَقِّ بَشِيرًۭا وَنَذِيرًۭا ۖ وَلَا تُسـَٔلُ عَن أَصحَـٰبِ ٱلجَحِيمِ ﴿١١٩﴾\\
\textamh{120.\  } & وَلَن تَرضَىٰ عَنكَ ٱليَهُودُ وَلَا ٱلنَّصَـٰرَىٰ حَتَّىٰ تَتَّبِعَ مِلَّتَهُم ۗ قُل إِنَّ هُدَى ٱللَّهِ هُوَ ٱلهُدَىٰ ۗ وَلَئِنِ ٱتَّبَعتَ أَهوَآءَهُم بَعدَ ٱلَّذِى جَآءَكَ مِنَ ٱلعِلمِ ۙ مَا لَكَ مِنَ ٱللَّهِ مِن وَلِىٍّۢ وَلَا نَصِيرٍ ﴿١٢٠﴾\\
\textamh{121.\  } & ٱلَّذِينَ ءَاتَينَـٰهُمُ ٱلكِتَـٰبَ يَتلُونَهُۥ حَقَّ تِلَاوَتِهِۦٓ أُو۟لَـٰٓئِكَ يُؤمِنُونَ بِهِۦ ۗ وَمَن يَكفُر بِهِۦ فَأُو۟لَـٰٓئِكَ هُمُ ٱلخَـٰسِرُونَ ﴿١٢١﴾\\
\textamh{122.\  } & يَـٰبَنِىٓ إِسرَٰٓءِيلَ ٱذكُرُوا۟ نِعمَتِىَ ٱلَّتِىٓ أَنعَمتُ عَلَيكُم وَأَنِّى فَضَّلتُكُم عَلَى ٱلعَـٰلَمِينَ ﴿١٢٢﴾\\
\textamh{123.\  } & وَٱتَّقُوا۟ يَومًۭا لَّا تَجزِى نَفسٌ عَن نَّفسٍۢ شَيـًۭٔا وَلَا يُقبَلُ مِنهَا عَدلٌۭ وَلَا تَنفَعُهَا شَفَـٰعَةٌۭ وَلَا هُم يُنصَرُونَ ﴿١٢٣﴾\\
\textamh{124.\  } & ۞ وَإِذِ ٱبتَلَىٰٓ إِبرَٰهِۦمَ رَبُّهُۥ بِكَلِمَـٰتٍۢ فَأَتَمَّهُنَّ ۖ قَالَ إِنِّى جَاعِلُكَ لِلنَّاسِ إِمَامًۭا ۖ قَالَ وَمِن ذُرِّيَّتِى ۖ قَالَ لَا يَنَالُ عَهدِى ٱلظَّـٰلِمِينَ ﴿١٢٤﴾\\
\textamh{125.\  } & وَإِذ جَعَلنَا ٱلبَيتَ مَثَابَةًۭ لِّلنَّاسِ وَأَمنًۭا وَٱتَّخِذُوا۟ مِن مَّقَامِ إِبرَٰهِۦمَ مُصَلًّۭى ۖ وَعَهِدنَآ إِلَىٰٓ إِبرَٰهِۦمَ وَإِسمَـٰعِيلَ أَن طَهِّرَا بَيتِىَ لِلطَّآئِفِينَ وَٱلعَـٰكِفِينَ وَٱلرُّكَّعِ ٱلسُّجُودِ ﴿١٢٥﴾\\
\textamh{126.\  } & وَإِذ قَالَ إِبرَٰهِۦمُ رَبِّ ٱجعَل هَـٰذَا بَلَدًا ءَامِنًۭا وَٱرزُق أَهلَهُۥ مِنَ ٱلثَّمَرَٰتِ مَن ءَامَنَ مِنهُم بِٱللَّهِ وَٱليَومِ ٱلءَاخِرِ ۖ قَالَ وَمَن كَفَرَ فَأُمَتِّعُهُۥ قَلِيلًۭا ثُمَّ أَضطَرُّهُۥٓ إِلَىٰ عَذَابِ ٱلنَّارِ ۖ وَبِئسَ ٱلمَصِيرُ ﴿١٢٦﴾\\
\textamh{127.\  } & وَإِذ يَرفَعُ إِبرَٰهِۦمُ ٱلقَوَاعِدَ مِنَ ٱلبَيتِ وَإِسمَـٰعِيلُ رَبَّنَا تَقَبَّل مِنَّآ ۖ إِنَّكَ أَنتَ ٱلسَّمِيعُ ٱلعَلِيمُ ﴿١٢٧﴾\\
\textamh{128.\  } & رَبَّنَا وَٱجعَلنَا مُسلِمَينِ لَكَ وَمِن ذُرِّيَّتِنَآ أُمَّةًۭ مُّسلِمَةًۭ لَّكَ وَأَرِنَا مَنَاسِكَنَا وَتُب عَلَينَآ ۖ إِنَّكَ أَنتَ ٱلتَّوَّابُ ٱلرَّحِيمُ ﴿١٢٨﴾\\
\textamh{129.\  } & رَبَّنَا وَٱبعَث فِيهِم رَسُولًۭا مِّنهُم يَتلُوا۟ عَلَيهِم ءَايَـٰتِكَ وَيُعَلِّمُهُمُ ٱلكِتَـٰبَ وَٱلحِكمَةَ وَيُزَكِّيهِم ۚ إِنَّكَ أَنتَ ٱلعَزِيزُ ٱلحَكِيمُ ﴿١٢٩﴾\\
\textamh{130.\  } & وَمَن يَرغَبُ عَن مِّلَّةِ إِبرَٰهِۦمَ إِلَّا مَن سَفِهَ نَفسَهُۥ ۚ وَلَقَدِ ٱصطَفَينَـٰهُ فِى ٱلدُّنيَا ۖ وَإِنَّهُۥ فِى ٱلءَاخِرَةِ لَمِنَ ٱلصَّـٰلِحِينَ ﴿١٣٠﴾\\
\textamh{131.\  } & إِذ قَالَ لَهُۥ رَبُّهُۥٓ أَسلِم ۖ قَالَ أَسلَمتُ لِرَبِّ ٱلعَـٰلَمِينَ ﴿١٣١﴾\\
\textamh{132.\  } & وَوَصَّىٰ بِهَآ إِبرَٰهِۦمُ بَنِيهِ وَيَعقُوبُ يَـٰبَنِىَّ إِنَّ ٱللَّهَ ٱصطَفَىٰ لَكُمُ ٱلدِّينَ فَلَا تَمُوتُنَّ إِلَّا وَأَنتُم مُّسلِمُونَ ﴿١٣٢﴾\\
\textamh{133.\  } & أَم كُنتُم شُهَدَآءَ إِذ حَضَرَ يَعقُوبَ ٱلمَوتُ إِذ قَالَ لِبَنِيهِ مَا تَعبُدُونَ مِنۢ بَعدِى قَالُوا۟ نَعبُدُ إِلَـٰهَكَ وَإِلَـٰهَ ءَابَآئِكَ إِبرَٰهِۦمَ وَإِسمَـٰعِيلَ وَإِسحَـٰقَ إِلَـٰهًۭا وَٟحِدًۭا وَنَحنُ لَهُۥ مُسلِمُونَ ﴿١٣٣﴾\\
\textamh{134.\  } & تِلكَ أُمَّةٌۭ قَد خَلَت ۖ لَهَا مَا كَسَبَت وَلَكُم مَّا كَسَبتُم ۖ وَلَا تُسـَٔلُونَ عَمَّا كَانُوا۟ يَعمَلُونَ ﴿١٣٤﴾\\
\textamh{135.\  } & وَقَالُوا۟ كُونُوا۟ هُودًا أَو نَصَـٰرَىٰ تَهتَدُوا۟ ۗ قُل بَل مِلَّةَ إِبرَٰهِۦمَ حَنِيفًۭا ۖ وَمَا كَانَ مِنَ ٱلمُشرِكِينَ ﴿١٣٥﴾\\
\textamh{136.\  } & قُولُوٓا۟ ءَامَنَّا بِٱللَّهِ وَمَآ أُنزِلَ إِلَينَا وَمَآ أُنزِلَ إِلَىٰٓ إِبرَٰهِۦمَ وَإِسمَـٰعِيلَ وَإِسحَـٰقَ وَيَعقُوبَ وَٱلأَسبَاطِ وَمَآ أُوتِىَ مُوسَىٰ وَعِيسَىٰ وَمَآ أُوتِىَ ٱلنَّبِيُّونَ مِن رَّبِّهِم لَا نُفَرِّقُ بَينَ أَحَدٍۢ مِّنهُم وَنَحنُ لَهُۥ مُسلِمُونَ ﴿١٣٦﴾\\
\textamh{137.\  } & فَإِن ءَامَنُوا۟ بِمِثلِ مَآ ءَامَنتُم بِهِۦ فَقَدِ ٱهتَدَوا۟ ۖ وَّإِن تَوَلَّوا۟ فَإِنَّمَا هُم فِى شِقَاقٍۢ ۖ فَسَيَكفِيكَهُمُ ٱللَّهُ ۚ وَهُوَ ٱلسَّمِيعُ ٱلعَلِيمُ ﴿١٣٧﴾\\
\textamh{138.\  } & صِبغَةَ ٱللَّهِ ۖ وَمَن أَحسَنُ مِنَ ٱللَّهِ صِبغَةًۭ ۖ وَنَحنُ لَهُۥ عَـٰبِدُونَ ﴿١٣٨﴾\\
\textamh{139.\  } & قُل أَتُحَآجُّونَنَا فِى ٱللَّهِ وَهُوَ رَبُّنَا وَرَبُّكُم وَلَنَآ أَعمَـٰلُنَا وَلَكُم أَعمَـٰلُكُم وَنَحنُ لَهُۥ مُخلِصُونَ ﴿١٣٩﴾\\
\textamh{140.\  } & أَم تَقُولُونَ إِنَّ إِبرَٰهِۦمَ وَإِسمَـٰعِيلَ وَإِسحَـٰقَ وَيَعقُوبَ وَٱلأَسبَاطَ كَانُوا۟ هُودًا أَو نَصَـٰرَىٰ ۗ قُل ءَأَنتُم أَعلَمُ أَمِ ٱللَّهُ ۗ وَمَن أَظلَمُ مِمَّن كَتَمَ شَهَـٰدَةً عِندَهُۥ مِنَ ٱللَّهِ ۗ وَمَا ٱللَّهُ بِغَٰفِلٍ عَمَّا تَعمَلُونَ ﴿١٤٠﴾\\
\textamh{141.\  } & تِلكَ أُمَّةٌۭ قَد خَلَت ۖ لَهَا مَا كَسَبَت وَلَكُم مَّا كَسَبتُم ۖ وَلَا تُسـَٔلُونَ عَمَّا كَانُوا۟ يَعمَلُونَ ﴿١٤١﴾\\
\textamh{142.\  } & ۞ سَيَقُولُ ٱلسُّفَهَآءُ مِنَ ٱلنَّاسِ مَا وَلَّىٰهُم عَن قِبلَتِهِمُ ٱلَّتِى كَانُوا۟ عَلَيهَا ۚ قُل لِّلَّهِ ٱلمَشرِقُ وَٱلمَغرِبُ ۚ يَهدِى مَن يَشَآءُ إِلَىٰ صِرَٰطٍۢ مُّستَقِيمٍۢ ﴿١٤٢﴾\\
\textamh{143.\  } & وَكَذَٟلِكَ جَعَلنَـٰكُم أُمَّةًۭ وَسَطًۭا لِّتَكُونُوا۟ شُهَدَآءَ عَلَى ٱلنَّاسِ وَيَكُونَ ٱلرَّسُولُ عَلَيكُم شَهِيدًۭا ۗ وَمَا جَعَلنَا ٱلقِبلَةَ ٱلَّتِى كُنتَ عَلَيهَآ إِلَّا لِنَعلَمَ مَن يَتَّبِعُ ٱلرَّسُولَ مِمَّن يَنقَلِبُ عَلَىٰ عَقِبَيهِ ۚ وَإِن كَانَت لَكَبِيرَةً إِلَّا عَلَى ٱلَّذِينَ هَدَى ٱللَّهُ ۗ وَمَا كَانَ ٱللَّهُ لِيُضِيعَ إِيمَـٰنَكُم ۚ إِنَّ ٱللَّهَ بِٱلنَّاسِ لَرَءُوفٌۭ رَّحِيمٌۭ ﴿١٤٣﴾\\
\textamh{144.\  } & قَد نَرَىٰ تَقَلُّبَ وَجهِكَ فِى ٱلسَّمَآءِ ۖ فَلَنُوَلِّيَنَّكَ قِبلَةًۭ تَرضَىٰهَا ۚ فَوَلِّ وَجهَكَ شَطرَ ٱلمَسجِدِ ٱلحَرَامِ ۚ وَحَيثُ مَا كُنتُم فَوَلُّوا۟ وُجُوهَكُم شَطرَهُۥ ۗ وَإِنَّ ٱلَّذِينَ أُوتُوا۟ ٱلكِتَـٰبَ لَيَعلَمُونَ أَنَّهُ ٱلحَقُّ مِن رَّبِّهِم ۗ وَمَا ٱللَّهُ بِغَٰفِلٍ عَمَّا يَعمَلُونَ ﴿١٤٤﴾\\
\textamh{145.\  } & وَلَئِن أَتَيتَ ٱلَّذِينَ أُوتُوا۟ ٱلكِتَـٰبَ بِكُلِّ ءَايَةٍۢ مَّا تَبِعُوا۟ قِبلَتَكَ ۚ وَمَآ أَنتَ بِتَابِعٍۢ قِبلَتَهُم ۚ وَمَا بَعضُهُم بِتَابِعٍۢ قِبلَةَ بَعضٍۢ ۚ وَلَئِنِ ٱتَّبَعتَ أَهوَآءَهُم مِّنۢ بَعدِ مَا جَآءَكَ مِنَ ٱلعِلمِ ۙ إِنَّكَ إِذًۭا لَّمِنَ ٱلظَّـٰلِمِينَ ﴿١٤٥﴾\\
\textamh{146.\  } & ٱلَّذِينَ ءَاتَينَـٰهُمُ ٱلكِتَـٰبَ يَعرِفُونَهُۥ كَمَا يَعرِفُونَ أَبنَآءَهُم ۖ وَإِنَّ فَرِيقًۭا مِّنهُم لَيَكتُمُونَ ٱلحَقَّ وَهُم يَعلَمُونَ ﴿١٤٦﴾\\
\textamh{147.\  } & ٱلحَقُّ مِن رَّبِّكَ ۖ فَلَا تَكُونَنَّ مِنَ ٱلمُمتَرِينَ ﴿١٤٧﴾\\
\textamh{148.\  } & وَلِكُلٍّۢ وِجهَةٌ هُوَ مُوَلِّيهَا ۖ فَٱستَبِقُوا۟ ٱلخَيرَٰتِ ۚ أَينَ مَا تَكُونُوا۟ يَأتِ بِكُمُ ٱللَّهُ جَمِيعًا ۚ إِنَّ ٱللَّهَ عَلَىٰ كُلِّ شَىءٍۢ قَدِيرٌۭ ﴿١٤٨﴾\\
\textamh{149.\  } & وَمِن حَيثُ خَرَجتَ فَوَلِّ وَجهَكَ شَطرَ ٱلمَسجِدِ ٱلحَرَامِ ۖ وَإِنَّهُۥ لَلحَقُّ مِن رَّبِّكَ ۗ وَمَا ٱللَّهُ بِغَٰفِلٍ عَمَّا تَعمَلُونَ ﴿١٤٩﴾\\
\textamh{150.\  } & وَمِن حَيثُ خَرَجتَ فَوَلِّ وَجهَكَ شَطرَ ٱلمَسجِدِ ٱلحَرَامِ ۚ وَحَيثُ مَا كُنتُم فَوَلُّوا۟ وُجُوهَكُم شَطرَهُۥ لِئَلَّا يَكُونَ لِلنَّاسِ عَلَيكُم حُجَّةٌ إِلَّا ٱلَّذِينَ ظَلَمُوا۟ مِنهُم فَلَا تَخشَوهُم وَٱخشَونِى وَلِأُتِمَّ نِعمَتِى عَلَيكُم وَلَعَلَّكُم تَهتَدُونَ ﴿١٥٠﴾\\
\textamh{151.\  } & كَمَآ أَرسَلنَا فِيكُم رَسُولًۭا مِّنكُم يَتلُوا۟ عَلَيكُم ءَايَـٰتِنَا وَيُزَكِّيكُم وَيُعَلِّمُكُمُ ٱلكِتَـٰبَ وَٱلحِكمَةَ وَيُعَلِّمُكُم مَّا لَم تَكُونُوا۟ تَعلَمُونَ ﴿١٥١﴾\\
\textamh{152.\  } & فَٱذكُرُونِىٓ أَذكُركُم وَٱشكُرُوا۟ لِى وَلَا تَكفُرُونِ ﴿١٥٢﴾\\
\textamh{153.\  } & يَـٰٓأَيُّهَا ٱلَّذِينَ ءَامَنُوا۟ ٱستَعِينُوا۟ بِٱلصَّبرِ وَٱلصَّلَوٰةِ ۚ إِنَّ ٱللَّهَ مَعَ ٱلصَّـٰبِرِينَ ﴿١٥٣﴾\\
\textamh{154.\  } & وَلَا تَقُولُوا۟ لِمَن يُقتَلُ فِى سَبِيلِ ٱللَّهِ أَموَٟتٌۢ ۚ بَل أَحيَآءٌۭ وَلَـٰكِن لَّا تَشعُرُونَ ﴿١٥٤﴾\\
\textamh{155.\  } & وَلَنَبلُوَنَّكُم بِشَىءٍۢ مِّنَ ٱلخَوفِ وَٱلجُوعِ وَنَقصٍۢ مِّنَ ٱلأَموَٟلِ وَٱلأَنفُسِ وَٱلثَّمَرَٰتِ ۗ وَبَشِّرِ ٱلصَّـٰبِرِينَ ﴿١٥٥﴾\\
\textamh{156.\  } & ٱلَّذِينَ إِذَآ أَصَـٰبَتهُم مُّصِيبَةٌۭ قَالُوٓا۟ إِنَّا لِلَّهِ وَإِنَّآ إِلَيهِ رَٰجِعُونَ ﴿١٥٦﴾\\
\textamh{157.\  } & أُو۟لَـٰٓئِكَ عَلَيهِم صَلَوَٟتٌۭ مِّن رَّبِّهِم وَرَحمَةٌۭ ۖ وَأُو۟لَـٰٓئِكَ هُمُ ٱلمُهتَدُونَ ﴿١٥٧﴾\\
\textamh{158.\  } & ۞ إِنَّ ٱلصَّفَا وَٱلمَروَةَ مِن شَعَآئِرِ ٱللَّهِ ۖ فَمَن حَجَّ ٱلبَيتَ أَوِ ٱعتَمَرَ فَلَا جُنَاحَ عَلَيهِ أَن يَطَّوَّفَ بِهِمَا ۚ وَمَن تَطَوَّعَ خَيرًۭا فَإِنَّ ٱللَّهَ شَاكِرٌ عَلِيمٌ ﴿١٥٨﴾\\
\textamh{159.\  } & إِنَّ ٱلَّذِينَ يَكتُمُونَ مَآ أَنزَلنَا مِنَ ٱلبَيِّنَـٰتِ وَٱلهُدَىٰ مِنۢ بَعدِ مَا بَيَّنَّـٰهُ لِلنَّاسِ فِى ٱلكِتَـٰبِ ۙ أُو۟لَـٰٓئِكَ يَلعَنُهُمُ ٱللَّهُ وَيَلعَنُهُمُ ٱللَّٰعِنُونَ ﴿١٥٩﴾\\
\textamh{160.\  } & إِلَّا ٱلَّذِينَ تَابُوا۟ وَأَصلَحُوا۟ وَبَيَّنُوا۟ فَأُو۟لَـٰٓئِكَ أَتُوبُ عَلَيهِم ۚ وَأَنَا ٱلتَّوَّابُ ٱلرَّحِيمُ ﴿١٦٠﴾\\
\textamh{161.\  } & إِنَّ ٱلَّذِينَ كَفَرُوا۟ وَمَاتُوا۟ وَهُم كُفَّارٌ أُو۟لَـٰٓئِكَ عَلَيهِم لَعنَةُ ٱللَّهِ وَٱلمَلَـٰٓئِكَةِ وَٱلنَّاسِ أَجمَعِينَ ﴿١٦١﴾\\
\textamh{162.\  } & خَـٰلِدِينَ فِيهَا ۖ لَا يُخَفَّفُ عَنهُمُ ٱلعَذَابُ وَلَا هُم يُنظَرُونَ ﴿١٦٢﴾\\
\textamh{163.\  } & وَإِلَـٰهُكُم إِلَـٰهٌۭ وَٟحِدٌۭ ۖ لَّآ إِلَـٰهَ إِلَّا هُوَ ٱلرَّحمَـٰنُ ٱلرَّحِيمُ ﴿١٦٣﴾\\
\textamh{164.\  } & إِنَّ فِى خَلقِ ٱلسَّمَـٰوَٟتِ وَٱلأَرضِ وَٱختِلَـٰفِ ٱلَّيلِ وَٱلنَّهَارِ وَٱلفُلكِ ٱلَّتِى تَجرِى فِى ٱلبَحرِ بِمَا يَنفَعُ ٱلنَّاسَ وَمَآ أَنزَلَ ٱللَّهُ مِنَ ٱلسَّمَآءِ مِن مَّآءٍۢ فَأَحيَا بِهِ ٱلأَرضَ بَعدَ مَوتِهَا وَبَثَّ فِيهَا مِن كُلِّ دَآبَّةٍۢ وَتَصرِيفِ ٱلرِّيَـٰحِ وَٱلسَّحَابِ ٱلمُسَخَّرِ بَينَ ٱلسَّمَآءِ وَٱلأَرضِ لَءَايَـٰتٍۢ لِّقَومٍۢ يَعقِلُونَ ﴿١٦٤﴾\\
\textamh{165.\  } & وَمِنَ ٱلنَّاسِ مَن يَتَّخِذُ مِن دُونِ ٱللَّهِ أَندَادًۭا يُحِبُّونَهُم كَحُبِّ ٱللَّهِ ۖ وَٱلَّذِينَ ءَامَنُوٓا۟ أَشَدُّ حُبًّۭا لِّلَّهِ ۗ وَلَو يَرَى ٱلَّذِينَ ظَلَمُوٓا۟ إِذ يَرَونَ ٱلعَذَابَ أَنَّ ٱلقُوَّةَ لِلَّهِ جَمِيعًۭا وَأَنَّ ٱللَّهَ شَدِيدُ ٱلعَذَابِ ﴿١٦٥﴾\\
\textamh{166.\  } & إِذ تَبَرَّأَ ٱلَّذِينَ ٱتُّبِعُوا۟ مِنَ ٱلَّذِينَ ٱتَّبَعُوا۟ وَرَأَوُا۟ ٱلعَذَابَ وَتَقَطَّعَت بِهِمُ ٱلأَسبَابُ ﴿١٦٦﴾\\
\textamh{167.\  } & وَقَالَ ٱلَّذِينَ ٱتَّبَعُوا۟ لَو أَنَّ لَنَا كَرَّةًۭ فَنَتَبَرَّأَ مِنهُم كَمَا تَبَرَّءُوا۟ مِنَّا ۗ كَذَٟلِكَ يُرِيهِمُ ٱللَّهُ أَعمَـٰلَهُم حَسَرَٰتٍ عَلَيهِم ۖ وَمَا هُم بِخَـٰرِجِينَ مِنَ ٱلنَّارِ ﴿١٦٧﴾\\
\textamh{168.\  } & يَـٰٓأَيُّهَا ٱلنَّاسُ كُلُوا۟ مِمَّا فِى ٱلأَرضِ حَلَـٰلًۭا طَيِّبًۭا وَلَا تَتَّبِعُوا۟ خُطُوَٟتِ ٱلشَّيطَٰنِ ۚ إِنَّهُۥ لَكُم عَدُوٌّۭ مُّبِينٌ ﴿١٦٨﴾\\
\textamh{169.\  } & إِنَّمَا يَأمُرُكُم بِٱلسُّوٓءِ وَٱلفَحشَآءِ وَأَن تَقُولُوا۟ عَلَى ٱللَّهِ مَا لَا تَعلَمُونَ ﴿١٦٩﴾\\
\textamh{170.\  } & وَإِذَا قِيلَ لَهُمُ ٱتَّبِعُوا۟ مَآ أَنزَلَ ٱللَّهُ قَالُوا۟ بَل نَتَّبِعُ مَآ أَلفَينَا عَلَيهِ ءَابَآءَنَآ ۗ أَوَلَو كَانَ ءَابَآؤُهُم لَا يَعقِلُونَ شَيـًۭٔا وَلَا يَهتَدُونَ ﴿١٧٠﴾\\
\textamh{171.\  } & وَمَثَلُ ٱلَّذِينَ كَفَرُوا۟ كَمَثَلِ ٱلَّذِى يَنعِقُ بِمَا لَا يَسمَعُ إِلَّا دُعَآءًۭ وَنِدَآءًۭ ۚ صُمٌّۢ بُكمٌ عُمىٌۭ فَهُم لَا يَعقِلُونَ ﴿١٧١﴾\\
\textamh{172.\  } & يَـٰٓأَيُّهَا ٱلَّذِينَ ءَامَنُوا۟ كُلُوا۟ مِن طَيِّبَٰتِ مَا رَزَقنَـٰكُم وَٱشكُرُوا۟ لِلَّهِ إِن كُنتُم إِيَّاهُ تَعبُدُونَ ﴿١٧٢﴾\\
\textamh{173.\  } & إِنَّمَا حَرَّمَ عَلَيكُمُ ٱلمَيتَةَ وَٱلدَّمَ وَلَحمَ ٱلخِنزِيرِ وَمَآ أُهِلَّ بِهِۦ لِغَيرِ ٱللَّهِ ۖ فَمَنِ ٱضطُرَّ غَيرَ بَاغٍۢ وَلَا عَادٍۢ فَلَآ إِثمَ عَلَيهِ ۚ إِنَّ ٱللَّهَ غَفُورٌۭ رَّحِيمٌ ﴿١٧٣﴾\\
\textamh{174.\  } & إِنَّ ٱلَّذِينَ يَكتُمُونَ مَآ أَنزَلَ ٱللَّهُ مِنَ ٱلكِتَـٰبِ وَيَشتَرُونَ بِهِۦ ثَمَنًۭا قَلِيلًا ۙ أُو۟لَـٰٓئِكَ مَا يَأكُلُونَ فِى بُطُونِهِم إِلَّا ٱلنَّارَ وَلَا يُكَلِّمُهُمُ ٱللَّهُ يَومَ ٱلقِيَـٰمَةِ وَلَا يُزَكِّيهِم وَلَهُم عَذَابٌ أَلِيمٌ ﴿١٧٤﴾\\
\textamh{175.\  } & أُو۟لَـٰٓئِكَ ٱلَّذِينَ ٱشتَرَوُا۟ ٱلضَّلَـٰلَةَ بِٱلهُدَىٰ وَٱلعَذَابَ بِٱلمَغفِرَةِ ۚ فَمَآ أَصبَرَهُم عَلَى ٱلنَّارِ ﴿١٧٥﴾\\
\textamh{176.\  } & ذَٟلِكَ بِأَنَّ ٱللَّهَ نَزَّلَ ٱلكِتَـٰبَ بِٱلحَقِّ ۗ وَإِنَّ ٱلَّذِينَ ٱختَلَفُوا۟ فِى ٱلكِتَـٰبِ لَفِى شِقَاقٍۭ بَعِيدٍۢ ﴿١٧٦﴾\\
\textamh{177.\  } & ۞ لَّيسَ ٱلبِرَّ أَن تُوَلُّوا۟ وُجُوهَكُم قِبَلَ ٱلمَشرِقِ وَٱلمَغرِبِ وَلَـٰكِنَّ ٱلبِرَّ مَن ءَامَنَ بِٱللَّهِ وَٱليَومِ ٱلءَاخِرِ وَٱلمَلَـٰٓئِكَةِ وَٱلكِتَـٰبِ وَٱلنَّبِيِّۦنَ وَءَاتَى ٱلمَالَ عَلَىٰ حُبِّهِۦ ذَوِى ٱلقُربَىٰ وَٱليَتَـٰمَىٰ وَٱلمَسَـٰكِينَ وَٱبنَ ٱلسَّبِيلِ وَٱلسَّآئِلِينَ وَفِى ٱلرِّقَابِ وَأَقَامَ ٱلصَّلَوٰةَ وَءَاتَى ٱلزَّكَوٰةَ وَٱلمُوفُونَ بِعَهدِهِم إِذَا عَـٰهَدُوا۟ ۖ وَٱلصَّـٰبِرِينَ فِى ٱلبَأسَآءِ وَٱلضَّرَّآءِ وَحِينَ ٱلبَأسِ ۗ أُو۟لَـٰٓئِكَ ٱلَّذِينَ صَدَقُوا۟ ۖ وَأُو۟لَـٰٓئِكَ هُمُ ٱلمُتَّقُونَ ﴿١٧٧﴾\\
\textamh{178.\  } & يَـٰٓأَيُّهَا ٱلَّذِينَ ءَامَنُوا۟ كُتِبَ عَلَيكُمُ ٱلقِصَاصُ فِى ٱلقَتلَى ۖ ٱلحُرُّ بِٱلحُرِّ وَٱلعَبدُ بِٱلعَبدِ وَٱلأُنثَىٰ بِٱلأُنثَىٰ ۚ فَمَن عُفِىَ لَهُۥ مِن أَخِيهِ شَىءٌۭ فَٱتِّبَاعٌۢ بِٱلمَعرُوفِ وَأَدَآءٌ إِلَيهِ بِإِحسَـٰنٍۢ ۗ ذَٟلِكَ تَخفِيفٌۭ مِّن رَّبِّكُم وَرَحمَةٌۭ ۗ فَمَنِ ٱعتَدَىٰ بَعدَ ذَٟلِكَ فَلَهُۥ عَذَابٌ أَلِيمٌۭ ﴿١٧٨﴾\\
\textamh{179.\  } & وَلَكُم فِى ٱلقِصَاصِ حَيَوٰةٌۭ يَـٰٓأُو۟لِى ٱلأَلبَٰبِ لَعَلَّكُم تَتَّقُونَ ﴿١٧٩﴾\\
\textamh{180.\  } & كُتِبَ عَلَيكُم إِذَا حَضَرَ أَحَدَكُمُ ٱلمَوتُ إِن تَرَكَ خَيرًا ٱلوَصِيَّةُ لِلوَٟلِدَينِ وَٱلأَقرَبِينَ بِٱلمَعرُوفِ ۖ حَقًّا عَلَى ٱلمُتَّقِينَ ﴿١٨٠﴾\\
\textamh{181.\  } & فَمَنۢ بَدَّلَهُۥ بَعدَمَا سَمِعَهُۥ فَإِنَّمَآ إِثمُهُۥ عَلَى ٱلَّذِينَ يُبَدِّلُونَهُۥٓ ۚ إِنَّ ٱللَّهَ سَمِيعٌ عَلِيمٌۭ ﴿١٨١﴾\\
\textamh{182.\  } & فَمَن خَافَ مِن مُّوصٍۢ جَنَفًا أَو إِثمًۭا فَأَصلَحَ بَينَهُم فَلَآ إِثمَ عَلَيهِ ۚ إِنَّ ٱللَّهَ غَفُورٌۭ رَّحِيمٌۭ ﴿١٨٢﴾\\
\textamh{183.\  } & يَـٰٓأَيُّهَا ٱلَّذِينَ ءَامَنُوا۟ كُتِبَ عَلَيكُمُ ٱلصِّيَامُ كَمَا كُتِبَ عَلَى ٱلَّذِينَ مِن قَبلِكُم لَعَلَّكُم تَتَّقُونَ ﴿١٨٣﴾\\
\textamh{184.\  } & أَيَّامًۭا مَّعدُودَٟتٍۢ ۚ فَمَن كَانَ مِنكُم مَّرِيضًا أَو عَلَىٰ سَفَرٍۢ فَعِدَّةٌۭ مِّن أَيَّامٍ أُخَرَ ۚ وَعَلَى ٱلَّذِينَ يُطِيقُونَهُۥ فِديَةٌۭ طَعَامُ مِسكِينٍۢ ۖ فَمَن تَطَوَّعَ خَيرًۭا فَهُوَ خَيرٌۭ لَّهُۥ ۚ وَأَن تَصُومُوا۟ خَيرٌۭ لَّكُم ۖ إِن كُنتُم تَعلَمُونَ ﴿١٨٤﴾\\
\textamh{185.\  } & شَهرُ رَمَضَانَ ٱلَّذِىٓ أُنزِلَ فِيهِ ٱلقُرءَانُ هُدًۭى لِّلنَّاسِ وَبَيِّنَـٰتٍۢ مِّنَ ٱلهُدَىٰ وَٱلفُرقَانِ ۚ فَمَن شَهِدَ مِنكُمُ ٱلشَّهرَ فَليَصُمهُ ۖ وَمَن كَانَ مَرِيضًا أَو عَلَىٰ سَفَرٍۢ فَعِدَّةٌۭ مِّن أَيَّامٍ أُخَرَ ۗ يُرِيدُ ٱللَّهُ بِكُمُ ٱليُسرَ وَلَا يُرِيدُ بِكُمُ ٱلعُسرَ وَلِتُكمِلُوا۟ ٱلعِدَّةَ وَلِتُكَبِّرُوا۟ ٱللَّهَ عَلَىٰ مَا هَدَىٰكُم وَلَعَلَّكُم تَشكُرُونَ ﴿١٨٥﴾\\
\textamh{186.\  } & وَإِذَا سَأَلَكَ عِبَادِى عَنِّى فَإِنِّى قَرِيبٌ ۖ أُجِيبُ دَعوَةَ ٱلدَّاعِ إِذَا دَعَانِ ۖ فَليَستَجِيبُوا۟ لِى وَليُؤمِنُوا۟ بِى لَعَلَّهُم يَرشُدُونَ ﴿١٨٦﴾\\
\textamh{187.\  } & أُحِلَّ لَكُم لَيلَةَ ٱلصِّيَامِ ٱلرَّفَثُ إِلَىٰ نِسَآئِكُم ۚ هُنَّ لِبَاسٌۭ لَّكُم وَأَنتُم لِبَاسٌۭ لَّهُنَّ ۗ عَلِمَ ٱللَّهُ أَنَّكُم كُنتُم تَختَانُونَ أَنفُسَكُم فَتَابَ عَلَيكُم وَعَفَا عَنكُم ۖ فَٱلـَٰٔنَ بَٰشِرُوهُنَّ وَٱبتَغُوا۟ مَا كَتَبَ ٱللَّهُ لَكُم ۚ وَكُلُوا۟ وَٱشرَبُوا۟ حَتَّىٰ يَتَبَيَّنَ لَكُمُ ٱلخَيطُ ٱلأَبيَضُ مِنَ ٱلخَيطِ ٱلأَسوَدِ مِنَ ٱلفَجرِ ۖ ثُمَّ أَتِمُّوا۟ ٱلصِّيَامَ إِلَى ٱلَّيلِ ۚ وَلَا تُبَٰشِرُوهُنَّ وَأَنتُم عَـٰكِفُونَ فِى ٱلمَسَـٰجِدِ ۗ تِلكَ حُدُودُ ٱللَّهِ فَلَا تَقرَبُوهَا ۗ كَذَٟلِكَ يُبَيِّنُ ٱللَّهُ ءَايَـٰتِهِۦ لِلنَّاسِ لَعَلَّهُم يَتَّقُونَ ﴿١٨٧﴾\\
\textamh{188.\  } & وَلَا تَأكُلُوٓا۟ أَموَٟلَكُم بَينَكُم بِٱلبَٰطِلِ وَتُدلُوا۟ بِهَآ إِلَى ٱلحُكَّامِ لِتَأكُلُوا۟ فَرِيقًۭا مِّن أَموَٟلِ ٱلنَّاسِ بِٱلإِثمِ وَأَنتُم تَعلَمُونَ ﴿١٨٨﴾\\
\textamh{189.\  } & ۞ يَسـَٔلُونَكَ عَنِ ٱلأَهِلَّةِ ۖ قُل هِىَ مَوَٟقِيتُ لِلنَّاسِ وَٱلحَجِّ ۗ وَلَيسَ ٱلبِرُّ بِأَن تَأتُوا۟ ٱلبُيُوتَ مِن ظُهُورِهَا وَلَـٰكِنَّ ٱلبِرَّ مَنِ ٱتَّقَىٰ ۗ وَأتُوا۟ ٱلبُيُوتَ مِن أَبوَٟبِهَا ۚ وَٱتَّقُوا۟ ٱللَّهَ لَعَلَّكُم تُفلِحُونَ ﴿١٨٩﴾\\
\textamh{190.\  } & وَقَـٰتِلُوا۟ فِى سَبِيلِ ٱللَّهِ ٱلَّذِينَ يُقَـٰتِلُونَكُم وَلَا تَعتَدُوٓا۟ ۚ إِنَّ ٱللَّهَ لَا يُحِبُّ ٱلمُعتَدِينَ ﴿١٩٠﴾\\
\textamh{191.\  } & وَٱقتُلُوهُم حَيثُ ثَقِفتُمُوهُم وَأَخرِجُوهُم مِّن حَيثُ أَخرَجُوكُم ۚ وَٱلفِتنَةُ أَشَدُّ مِنَ ٱلقَتلِ ۚ وَلَا تُقَـٰتِلُوهُم عِندَ ٱلمَسجِدِ ٱلحَرَامِ حَتَّىٰ يُقَـٰتِلُوكُم فِيهِ ۖ فَإِن قَـٰتَلُوكُم فَٱقتُلُوهُم ۗ كَذَٟلِكَ جَزَآءُ ٱلكَـٰفِرِينَ ﴿١٩١﴾\\
\textamh{192.\  } & فَإِنِ ٱنتَهَوا۟ فَإِنَّ ٱللَّهَ غَفُورٌۭ رَّحِيمٌۭ ﴿١٩٢﴾\\
\textamh{193.\  } & وَقَـٰتِلُوهُم حَتَّىٰ لَا تَكُونَ فِتنَةٌۭ وَيَكُونَ ٱلدِّينُ لِلَّهِ ۖ فَإِنِ ٱنتَهَوا۟ فَلَا عُدوَٟنَ إِلَّا عَلَى ٱلظَّـٰلِمِينَ ﴿١٩٣﴾\\
\textamh{194.\  } & ٱلشَّهرُ ٱلحَرَامُ بِٱلشَّهرِ ٱلحَرَامِ وَٱلحُرُمَـٰتُ قِصَاصٌۭ ۚ فَمَنِ ٱعتَدَىٰ عَلَيكُم فَٱعتَدُوا۟ عَلَيهِ بِمِثلِ مَا ٱعتَدَىٰ عَلَيكُم ۚ وَٱتَّقُوا۟ ٱللَّهَ وَٱعلَمُوٓا۟ أَنَّ ٱللَّهَ مَعَ ٱلمُتَّقِينَ ﴿١٩٤﴾\\
\textamh{195.\  } & وَأَنفِقُوا۟ فِى سَبِيلِ ٱللَّهِ وَلَا تُلقُوا۟ بِأَيدِيكُم إِلَى ٱلتَّهلُكَةِ ۛ وَأَحسِنُوٓا۟ ۛ إِنَّ ٱللَّهَ يُحِبُّ ٱلمُحسِنِينَ ﴿١٩٥﴾\\
\textamh{196.\  } & وَأَتِمُّوا۟ ٱلحَجَّ وَٱلعُمرَةَ لِلَّهِ ۚ فَإِن أُحصِرتُم فَمَا ٱستَيسَرَ مِنَ ٱلهَدىِ ۖ وَلَا تَحلِقُوا۟ رُءُوسَكُم حَتَّىٰ يَبلُغَ ٱلهَدىُ مَحِلَّهُۥ ۚ فَمَن كَانَ مِنكُم مَّرِيضًا أَو بِهِۦٓ أَذًۭى مِّن رَّأسِهِۦ فَفِديَةٌۭ مِّن صِيَامٍ أَو صَدَقَةٍ أَو نُسُكٍۢ ۚ فَإِذَآ أَمِنتُم فَمَن تَمَتَّعَ بِٱلعُمرَةِ إِلَى ٱلحَجِّ فَمَا ٱستَيسَرَ مِنَ ٱلهَدىِ ۚ فَمَن لَّم يَجِد فَصِيَامُ ثَلَـٰثَةِ أَيَّامٍۢ فِى ٱلحَجِّ وَسَبعَةٍ إِذَا رَجَعتُم ۗ تِلكَ عَشَرَةٌۭ كَامِلَةٌۭ ۗ ذَٟلِكَ لِمَن لَّم يَكُن أَهلُهُۥ حَاضِرِى ٱلمَسجِدِ ٱلحَرَامِ ۚ وَٱتَّقُوا۟ ٱللَّهَ وَٱعلَمُوٓا۟ أَنَّ ٱللَّهَ شَدِيدُ ٱلعِقَابِ ﴿١٩٦﴾\\
\textamh{197.\  } & ٱلحَجُّ أَشهُرٌۭ مَّعلُومَـٰتٌۭ ۚ فَمَن فَرَضَ فِيهِنَّ ٱلحَجَّ فَلَا رَفَثَ وَلَا فُسُوقَ وَلَا جِدَالَ فِى ٱلحَجِّ ۗ وَمَا تَفعَلُوا۟ مِن خَيرٍۢ يَعلَمهُ ٱللَّهُ ۗ وَتَزَوَّدُوا۟ فَإِنَّ خَيرَ ٱلزَّادِ ٱلتَّقوَىٰ ۚ وَٱتَّقُونِ يَـٰٓأُو۟لِى ٱلأَلبَٰبِ ﴿١٩٧﴾\\
\textamh{198.\  } & لَيسَ عَلَيكُم جُنَاحٌ أَن تَبتَغُوا۟ فَضلًۭا مِّن رَّبِّكُم ۚ فَإِذَآ أَفَضتُم مِّن عَرَفَـٰتٍۢ فَٱذكُرُوا۟ ٱللَّهَ عِندَ ٱلمَشعَرِ ٱلحَرَامِ ۖ وَٱذكُرُوهُ كَمَا هَدَىٰكُم وَإِن كُنتُم مِّن قَبلِهِۦ لَمِنَ ٱلضَّآلِّينَ ﴿١٩٨﴾\\
\textamh{199.\  } & ثُمَّ أَفِيضُوا۟ مِن حَيثُ أَفَاضَ ٱلنَّاسُ وَٱستَغفِرُوا۟ ٱللَّهَ ۚ إِنَّ ٱللَّهَ غَفُورٌۭ رَّحِيمٌۭ ﴿١٩٩﴾\\
\textamh{200.\  } & فَإِذَا قَضَيتُم مَّنَـٰسِكَكُم فَٱذكُرُوا۟ ٱللَّهَ كَذِكرِكُم ءَابَآءَكُم أَو أَشَدَّ ذِكرًۭا ۗ فَمِنَ ٱلنَّاسِ مَن يَقُولُ رَبَّنَآ ءَاتِنَا فِى ٱلدُّنيَا وَمَا لَهُۥ فِى ٱلءَاخِرَةِ مِن خَلَـٰقٍۢ ﴿٢٠٠﴾\\
\textamh{201.\  } & وَمِنهُم مَّن يَقُولُ رَبَّنَآ ءَاتِنَا فِى ٱلدُّنيَا حَسَنَةًۭ وَفِى ٱلءَاخِرَةِ حَسَنَةًۭ وَقِنَا عَذَابَ ٱلنَّارِ ﴿٢٠١﴾\\
\textamh{202.\  } & أُو۟لَـٰٓئِكَ لَهُم نَصِيبٌۭ مِّمَّا كَسَبُوا۟ ۚ وَٱللَّهُ سَرِيعُ ٱلحِسَابِ ﴿٢٠٢﴾\\
\textamh{203.\  } & ۞ وَٱذكُرُوا۟ ٱللَّهَ فِىٓ أَيَّامٍۢ مَّعدُودَٟتٍۢ ۚ فَمَن تَعَجَّلَ فِى يَومَينِ فَلَآ إِثمَ عَلَيهِ وَمَن تَأَخَّرَ فَلَآ إِثمَ عَلَيهِ ۚ لِمَنِ ٱتَّقَىٰ ۗ وَٱتَّقُوا۟ ٱللَّهَ وَٱعلَمُوٓا۟ أَنَّكُم إِلَيهِ تُحشَرُونَ ﴿٢٠٣﴾\\
\textamh{204.\  } & وَمِنَ ٱلنَّاسِ مَن يُعجِبُكَ قَولُهُۥ فِى ٱلحَيَوٰةِ ٱلدُّنيَا وَيُشهِدُ ٱللَّهَ عَلَىٰ مَا فِى قَلبِهِۦ وَهُوَ أَلَدُّ ٱلخِصَامِ ﴿٢٠٤﴾\\
\textamh{205.\  } & وَإِذَا تَوَلَّىٰ سَعَىٰ فِى ٱلأَرضِ لِيُفسِدَ فِيهَا وَيُهلِكَ ٱلحَرثَ وَٱلنَّسلَ ۗ وَٱللَّهُ لَا يُحِبُّ ٱلفَسَادَ ﴿٢٠٥﴾\\
\textamh{206.\  } & وَإِذَا قِيلَ لَهُ ٱتَّقِ ٱللَّهَ أَخَذَتهُ ٱلعِزَّةُ بِٱلإِثمِ ۚ فَحَسبُهُۥ جَهَنَّمُ ۚ وَلَبِئسَ ٱلمِهَادُ ﴿٢٠٦﴾\\
\textamh{207.\  } & وَمِنَ ٱلنَّاسِ مَن يَشرِى نَفسَهُ ٱبتِغَآءَ مَرضَاتِ ٱللَّهِ ۗ وَٱللَّهُ رَءُوفٌۢ بِٱلعِبَادِ ﴿٢٠٧﴾\\
\textamh{208.\  } & يَـٰٓأَيُّهَا ٱلَّذِينَ ءَامَنُوا۟ ٱدخُلُوا۟ فِى ٱلسِّلمِ كَآفَّةًۭ وَلَا تَتَّبِعُوا۟ خُطُوَٟتِ ٱلشَّيطَٰنِ ۚ إِنَّهُۥ لَكُم عَدُوٌّۭ مُّبِينٌۭ ﴿٢٠٨﴾\\
\textamh{209.\  } & فَإِن زَلَلتُم مِّنۢ بَعدِ مَا جَآءَتكُمُ ٱلبَيِّنَـٰتُ فَٱعلَمُوٓا۟ أَنَّ ٱللَّهَ عَزِيزٌ حَكِيمٌ ﴿٢٠٩﴾\\
\textamh{210.\  } & هَل يَنظُرُونَ إِلَّآ أَن يَأتِيَهُمُ ٱللَّهُ فِى ظُلَلٍۢ مِّنَ ٱلغَمَامِ وَٱلمَلَـٰٓئِكَةُ وَقُضِىَ ٱلأَمرُ ۚ وَإِلَى ٱللَّهِ تُرجَعُ ٱلأُمُورُ ﴿٢١٠﴾\\
\textamh{211.\  } & سَل بَنِىٓ إِسرَٰٓءِيلَ كَم ءَاتَينَـٰهُم مِّن ءَايَةٍۭ بَيِّنَةٍۢ ۗ وَمَن يُبَدِّل نِعمَةَ ٱللَّهِ مِنۢ بَعدِ مَا جَآءَتهُ فَإِنَّ ٱللَّهَ شَدِيدُ ٱلعِقَابِ ﴿٢١١﴾\\
\textamh{212.\  } & زُيِّنَ لِلَّذِينَ كَفَرُوا۟ ٱلحَيَوٰةُ ٱلدُّنيَا وَيَسخَرُونَ مِنَ ٱلَّذِينَ ءَامَنُوا۟ ۘ وَٱلَّذِينَ ٱتَّقَوا۟ فَوقَهُم يَومَ ٱلقِيَـٰمَةِ ۗ وَٱللَّهُ يَرزُقُ مَن يَشَآءُ بِغَيرِ حِسَابٍۢ ﴿٢١٢﴾\\
\textamh{213.\  } & كَانَ ٱلنَّاسُ أُمَّةًۭ وَٟحِدَةًۭ فَبَعَثَ ٱللَّهُ ٱلنَّبِيِّۦنَ مُبَشِّرِينَ وَمُنذِرِينَ وَأَنزَلَ مَعَهُمُ ٱلكِتَـٰبَ بِٱلحَقِّ لِيَحكُمَ بَينَ ٱلنَّاسِ فِيمَا ٱختَلَفُوا۟ فِيهِ ۚ وَمَا ٱختَلَفَ فِيهِ إِلَّا ٱلَّذِينَ أُوتُوهُ مِنۢ بَعدِ مَا جَآءَتهُمُ ٱلبَيِّنَـٰتُ بَغيًۢا بَينَهُم ۖ فَهَدَى ٱللَّهُ ٱلَّذِينَ ءَامَنُوا۟ لِمَا ٱختَلَفُوا۟ فِيهِ مِنَ ٱلحَقِّ بِإِذنِهِۦ ۗ وَٱللَّهُ يَهدِى مَن يَشَآءُ إِلَىٰ صِرَٰطٍۢ مُّستَقِيمٍ ﴿٢١٣﴾\\
\textamh{214.\  } & أَم حَسِبتُم أَن تَدخُلُوا۟ ٱلجَنَّةَ وَلَمَّا يَأتِكُم مَّثَلُ ٱلَّذِينَ خَلَوا۟ مِن قَبلِكُم ۖ مَّسَّتهُمُ ٱلبَأسَآءُ وَٱلضَّرَّآءُ وَزُلزِلُوا۟ حَتَّىٰ يَقُولَ ٱلرَّسُولُ وَٱلَّذِينَ ءَامَنُوا۟ مَعَهُۥ مَتَىٰ نَصرُ ٱللَّهِ ۗ أَلَآ إِنَّ نَصرَ ٱللَّهِ قَرِيبٌۭ ﴿٢١٤﴾\\
\textamh{215.\  } & يَسـَٔلُونَكَ مَاذَا يُنفِقُونَ ۖ قُل مَآ أَنفَقتُم مِّن خَيرٍۢ فَلِلوَٟلِدَينِ وَٱلأَقرَبِينَ وَٱليَتَـٰمَىٰ وَٱلمَسَـٰكِينِ وَٱبنِ ٱلسَّبِيلِ ۗ وَمَا تَفعَلُوا۟ مِن خَيرٍۢ فَإِنَّ ٱللَّهَ بِهِۦ عَلِيمٌۭ ﴿٢١٥﴾\\
\textamh{216.\  } & كُتِبَ عَلَيكُمُ ٱلقِتَالُ وَهُوَ كُرهٌۭ لَّكُم ۖ وَعَسَىٰٓ أَن تَكرَهُوا۟ شَيـًۭٔا وَهُوَ خَيرٌۭ لَّكُم ۖ وَعَسَىٰٓ أَن تُحِبُّوا۟ شَيـًۭٔا وَهُوَ شَرٌّۭ لَّكُم ۗ وَٱللَّهُ يَعلَمُ وَأَنتُم لَا تَعلَمُونَ ﴿٢١٦﴾\\
\textamh{217.\  } & يَسـَٔلُونَكَ عَنِ ٱلشَّهرِ ٱلحَرَامِ قِتَالٍۢ فِيهِ ۖ قُل قِتَالٌۭ فِيهِ كَبِيرٌۭ ۖ وَصَدٌّ عَن سَبِيلِ ٱللَّهِ وَكُفرٌۢ بِهِۦ وَٱلمَسجِدِ ٱلحَرَامِ وَإِخرَاجُ أَهلِهِۦ مِنهُ أَكبَرُ عِندَ ٱللَّهِ ۚ وَٱلفِتنَةُ أَكبَرُ مِنَ ٱلقَتلِ ۗ وَلَا يَزَالُونَ يُقَـٰتِلُونَكُم حَتَّىٰ يَرُدُّوكُم عَن دِينِكُم إِنِ ٱستَطَٰعُوا۟ ۚ وَمَن يَرتَدِد مِنكُم عَن دِينِهِۦ فَيَمُت وَهُوَ كَافِرٌۭ فَأُو۟لَـٰٓئِكَ حَبِطَت أَعمَـٰلُهُم فِى ٱلدُّنيَا وَٱلءَاخِرَةِ ۖ وَأُو۟لَـٰٓئِكَ أَصحَـٰبُ ٱلنَّارِ ۖ هُم فِيهَا خَـٰلِدُونَ ﴿٢١٧﴾\\
\textamh{218.\  } & إِنَّ ٱلَّذِينَ ءَامَنُوا۟ وَٱلَّذِينَ هَاجَرُوا۟ وَجَٰهَدُوا۟ فِى سَبِيلِ ٱللَّهِ أُو۟لَـٰٓئِكَ يَرجُونَ رَحمَتَ ٱللَّهِ ۚ وَٱللَّهُ غَفُورٌۭ رَّحِيمٌۭ ﴿٢١٨﴾\\
\textamh{219.\  } & ۞ يَسـَٔلُونَكَ عَنِ ٱلخَمرِ وَٱلمَيسِرِ ۖ قُل فِيهِمَآ إِثمٌۭ كَبِيرٌۭ وَمَنَـٰفِعُ لِلنَّاسِ وَإِثمُهُمَآ أَكبَرُ مِن نَّفعِهِمَا ۗ وَيَسـَٔلُونَكَ مَاذَا يُنفِقُونَ قُلِ ٱلعَفوَ ۗ كَذَٟلِكَ يُبَيِّنُ ٱللَّهُ لَكُمُ ٱلءَايَـٰتِ لَعَلَّكُم تَتَفَكَّرُونَ ﴿٢١٩﴾\\
\textamh{220.\  } & فِى ٱلدُّنيَا وَٱلءَاخِرَةِ ۗ وَيَسـَٔلُونَكَ عَنِ ٱليَتَـٰمَىٰ ۖ قُل إِصلَاحٌۭ لَّهُم خَيرٌۭ ۖ وَإِن تُخَالِطُوهُم فَإِخوَٟنُكُم ۚ وَٱللَّهُ يَعلَمُ ٱلمُفسِدَ مِنَ ٱلمُصلِحِ ۚ وَلَو شَآءَ ٱللَّهُ لَأَعنَتَكُم ۚ إِنَّ ٱللَّهَ عَزِيزٌ حَكِيمٌۭ ﴿٢٢٠﴾\\
\textamh{221.\  } & وَلَا تَنكِحُوا۟ ٱلمُشرِكَـٰتِ حَتَّىٰ يُؤمِنَّ ۚ وَلَأَمَةٌۭ مُّؤمِنَةٌ خَيرٌۭ مِّن مُّشرِكَةٍۢ وَلَو أَعجَبَتكُم ۗ وَلَا تُنكِحُوا۟ ٱلمُشرِكِينَ حَتَّىٰ يُؤمِنُوا۟ ۚ وَلَعَبدٌۭ مُّؤمِنٌ خَيرٌۭ مِّن مُّشرِكٍۢ وَلَو أَعجَبَكُم ۗ أُو۟لَـٰٓئِكَ يَدعُونَ إِلَى ٱلنَّارِ ۖ وَٱللَّهُ يَدعُوٓا۟ إِلَى ٱلجَنَّةِ وَٱلمَغفِرَةِ بِإِذنِهِۦ ۖ وَيُبَيِّنُ ءَايَـٰتِهِۦ لِلنَّاسِ لَعَلَّهُم يَتَذَكَّرُونَ ﴿٢٢١﴾\\
\textamh{222.\  } & وَيَسـَٔلُونَكَ عَنِ ٱلمَحِيضِ ۖ قُل هُوَ أَذًۭى فَٱعتَزِلُوا۟ ٱلنِّسَآءَ فِى ٱلمَحِيضِ ۖ وَلَا تَقرَبُوهُنَّ حَتَّىٰ يَطهُرنَ ۖ فَإِذَا تَطَهَّرنَ فَأتُوهُنَّ مِن حَيثُ أَمَرَكُمُ ٱللَّهُ ۚ إِنَّ ٱللَّهَ يُحِبُّ ٱلتَّوَّٰبِينَ وَيُحِبُّ ٱلمُتَطَهِّرِينَ ﴿٢٢٢﴾\\
\textamh{223.\  } & نِسَآؤُكُم حَرثٌۭ لَّكُم فَأتُوا۟ حَرثَكُم أَنَّىٰ شِئتُم ۖ وَقَدِّمُوا۟ لِأَنفُسِكُم ۚ وَٱتَّقُوا۟ ٱللَّهَ وَٱعلَمُوٓا۟ أَنَّكُم مُّلَـٰقُوهُ ۗ وَبَشِّرِ ٱلمُؤمِنِينَ ﴿٢٢٣﴾\\
\textamh{224.\  } & وَلَا تَجعَلُوا۟ ٱللَّهَ عُرضَةًۭ لِّأَيمَـٰنِكُم أَن تَبَرُّوا۟ وَتَتَّقُوا۟ وَتُصلِحُوا۟ بَينَ ٱلنَّاسِ ۗ وَٱللَّهُ سَمِيعٌ عَلِيمٌۭ ﴿٢٢٤﴾\\
\textamh{225.\  } & لَّا يُؤَاخِذُكُمُ ٱللَّهُ بِٱللَّغوِ فِىٓ أَيمَـٰنِكُم وَلَـٰكِن يُؤَاخِذُكُم بِمَا كَسَبَت قُلُوبُكُم ۗ وَٱللَّهُ غَفُورٌ حَلِيمٌۭ ﴿٢٢٥﴾\\
\textamh{226.\  } & لِّلَّذِينَ يُؤلُونَ مِن نِّسَآئِهِم تَرَبُّصُ أَربَعَةِ أَشهُرٍۢ ۖ فَإِن فَآءُو فَإِنَّ ٱللَّهَ غَفُورٌۭ رَّحِيمٌۭ ﴿٢٢٦﴾\\
\textamh{227.\  } & وَإِن عَزَمُوا۟ ٱلطَّلَـٰقَ فَإِنَّ ٱللَّهَ سَمِيعٌ عَلِيمٌۭ ﴿٢٢٧﴾\\
\textamh{228.\  } & وَٱلمُطَلَّقَـٰتُ يَتَرَبَّصنَ بِأَنفُسِهِنَّ ثَلَـٰثَةَ قُرُوٓءٍۢ ۚ وَلَا يَحِلُّ لَهُنَّ أَن يَكتُمنَ مَا خَلَقَ ٱللَّهُ فِىٓ أَرحَامِهِنَّ إِن كُنَّ يُؤمِنَّ بِٱللَّهِ وَٱليَومِ ٱلءَاخِرِ ۚ وَبُعُولَتُهُنَّ أَحَقُّ بِرَدِّهِنَّ فِى ذَٟلِكَ إِن أَرَادُوٓا۟ إِصلَـٰحًۭا ۚ وَلَهُنَّ مِثلُ ٱلَّذِى عَلَيهِنَّ بِٱلمَعرُوفِ ۚ وَلِلرِّجَالِ عَلَيهِنَّ دَرَجَةٌۭ ۗ وَٱللَّهُ عَزِيزٌ حَكِيمٌ ﴿٢٢٨﴾\\
\textamh{229.\  } & ٱلطَّلَـٰقُ مَرَّتَانِ ۖ فَإِمسَاكٌۢ بِمَعرُوفٍ أَو تَسرِيحٌۢ بِإِحسَـٰنٍۢ ۗ وَلَا يَحِلُّ لَكُم أَن تَأخُذُوا۟ مِمَّآ ءَاتَيتُمُوهُنَّ شَيـًٔا إِلَّآ أَن يَخَافَآ أَلَّا يُقِيمَا حُدُودَ ٱللَّهِ ۖ فَإِن خِفتُم أَلَّا يُقِيمَا حُدُودَ ٱللَّهِ فَلَا جُنَاحَ عَلَيهِمَا فِيمَا ٱفتَدَت بِهِۦ ۗ تِلكَ حُدُودُ ٱللَّهِ فَلَا تَعتَدُوهَا ۚ وَمَن يَتَعَدَّ حُدُودَ ٱللَّهِ فَأُو۟لَـٰٓئِكَ هُمُ ٱلظَّـٰلِمُونَ ﴿٢٢٩﴾\\
\textamh{230.\  } & فَإِن طَلَّقَهَا فَلَا تَحِلُّ لَهُۥ مِنۢ بَعدُ حَتَّىٰ تَنكِحَ زَوجًا غَيرَهُۥ ۗ فَإِن طَلَّقَهَا فَلَا جُنَاحَ عَلَيهِمَآ أَن يَتَرَاجَعَآ إِن ظَنَّآ أَن يُقِيمَا حُدُودَ ٱللَّهِ ۗ وَتِلكَ حُدُودُ ٱللَّهِ يُبَيِّنُهَا لِقَومٍۢ يَعلَمُونَ ﴿٢٣٠﴾\\
\textamh{231.\  } & وَإِذَا طَلَّقتُمُ ٱلنِّسَآءَ فَبَلَغنَ أَجَلَهُنَّ فَأَمسِكُوهُنَّ بِمَعرُوفٍ أَو سَرِّحُوهُنَّ بِمَعرُوفٍۢ ۚ وَلَا تُمسِكُوهُنَّ ضِرَارًۭا لِّتَعتَدُوا۟ ۚ وَمَن يَفعَل ذَٟلِكَ فَقَد ظَلَمَ نَفسَهُۥ ۚ وَلَا تَتَّخِذُوٓا۟ ءَايَـٰتِ ٱللَّهِ هُزُوًۭا ۚ وَٱذكُرُوا۟ نِعمَتَ ٱللَّهِ عَلَيكُم وَمَآ أَنزَلَ عَلَيكُم مِّنَ ٱلكِتَـٰبِ وَٱلحِكمَةِ يَعِظُكُم بِهِۦ ۚ وَٱتَّقُوا۟ ٱللَّهَ وَٱعلَمُوٓا۟ أَنَّ ٱللَّهَ بِكُلِّ شَىءٍ عَلِيمٌۭ ﴿٢٣١﴾\\
\textamh{232.\  } & وَإِذَا طَلَّقتُمُ ٱلنِّسَآءَ فَبَلَغنَ أَجَلَهُنَّ فَلَا تَعضُلُوهُنَّ أَن يَنكِحنَ أَزوَٟجَهُنَّ إِذَا تَرَٰضَوا۟ بَينَهُم بِٱلمَعرُوفِ ۗ ذَٟلِكَ يُوعَظُ بِهِۦ مَن كَانَ مِنكُم يُؤمِنُ بِٱللَّهِ وَٱليَومِ ٱلءَاخِرِ ۗ ذَٟلِكُم أَزكَىٰ لَكُم وَأَطهَرُ ۗ وَٱللَّهُ يَعلَمُ وَأَنتُم لَا تَعلَمُونَ ﴿٢٣٢﴾\\
\textamh{233.\  } & ۞ وَٱلوَٟلِدَٟتُ يُرضِعنَ أَولَـٰدَهُنَّ حَولَينِ كَامِلَينِ ۖ لِمَن أَرَادَ أَن يُتِمَّ ٱلرَّضَاعَةَ ۚ وَعَلَى ٱلمَولُودِ لَهُۥ رِزقُهُنَّ وَكِسوَتُهُنَّ بِٱلمَعرُوفِ ۚ لَا تُكَلَّفُ نَفسٌ إِلَّا وُسعَهَا ۚ لَا تُضَآرَّ وَٟلِدَةٌۢ بِوَلَدِهَا وَلَا مَولُودٌۭ لَّهُۥ بِوَلَدِهِۦ ۚ وَعَلَى ٱلوَارِثِ مِثلُ ذَٟلِكَ ۗ فَإِن أَرَادَا فِصَالًا عَن تَرَاضٍۢ مِّنهُمَا وَتَشَاوُرٍۢ فَلَا جُنَاحَ عَلَيهِمَا ۗ وَإِن أَرَدتُّم أَن تَستَرضِعُوٓا۟ أَولَـٰدَكُم فَلَا جُنَاحَ عَلَيكُم إِذَا سَلَّمتُم مَّآ ءَاتَيتُم بِٱلمَعرُوفِ ۗ وَٱتَّقُوا۟ ٱللَّهَ وَٱعلَمُوٓا۟ أَنَّ ٱللَّهَ بِمَا تَعمَلُونَ بَصِيرٌۭ ﴿٢٣٣﴾\\
\textamh{234.\  } & وَٱلَّذِينَ يُتَوَفَّونَ مِنكُم وَيَذَرُونَ أَزوَٟجًۭا يَتَرَبَّصنَ بِأَنفُسِهِنَّ أَربَعَةَ أَشهُرٍۢ وَعَشرًۭا ۖ فَإِذَا بَلَغنَ أَجَلَهُنَّ فَلَا جُنَاحَ عَلَيكُم فِيمَا فَعَلنَ فِىٓ أَنفُسِهِنَّ بِٱلمَعرُوفِ ۗ وَٱللَّهُ بِمَا تَعمَلُونَ خَبِيرٌۭ ﴿٢٣٤﴾\\
\textamh{235.\  } & وَلَا جُنَاحَ عَلَيكُم فِيمَا عَرَّضتُم بِهِۦ مِن خِطبَةِ ٱلنِّسَآءِ أَو أَكنَنتُم فِىٓ أَنفُسِكُم ۚ عَلِمَ ٱللَّهُ أَنَّكُم سَتَذكُرُونَهُنَّ وَلَـٰكِن لَّا تُوَاعِدُوهُنَّ سِرًّا إِلَّآ أَن تَقُولُوا۟ قَولًۭا مَّعرُوفًۭا ۚ وَلَا تَعزِمُوا۟ عُقدَةَ ٱلنِّكَاحِ حَتَّىٰ يَبلُغَ ٱلكِتَـٰبُ أَجَلَهُۥ ۚ وَٱعلَمُوٓا۟ أَنَّ ٱللَّهَ يَعلَمُ مَا فِىٓ أَنفُسِكُم فَٱحذَرُوهُ ۚ وَٱعلَمُوٓا۟ أَنَّ ٱللَّهَ غَفُورٌ حَلِيمٌۭ ﴿٢٣٥﴾\\
\textamh{236.\  } & لَّا جُنَاحَ عَلَيكُم إِن طَلَّقتُمُ ٱلنِّسَآءَ مَا لَم تَمَسُّوهُنَّ أَو تَفرِضُوا۟ لَهُنَّ فَرِيضَةًۭ ۚ وَمَتِّعُوهُنَّ عَلَى ٱلمُوسِعِ قَدَرُهُۥ وَعَلَى ٱلمُقتِرِ قَدَرُهُۥ مَتَـٰعًۢا بِٱلمَعرُوفِ ۖ حَقًّا عَلَى ٱلمُحسِنِينَ ﴿٢٣٦﴾\\
\textamh{237.\  } & وَإِن طَلَّقتُمُوهُنَّ مِن قَبلِ أَن تَمَسُّوهُنَّ وَقَد فَرَضتُم لَهُنَّ فَرِيضَةًۭ فَنِصفُ مَا فَرَضتُم إِلَّآ أَن يَعفُونَ أَو يَعفُوَا۟ ٱلَّذِى بِيَدِهِۦ عُقدَةُ ٱلنِّكَاحِ ۚ وَأَن تَعفُوٓا۟ أَقرَبُ لِلتَّقوَىٰ ۚ وَلَا تَنسَوُا۟ ٱلفَضلَ بَينَكُم ۚ إِنَّ ٱللَّهَ بِمَا تَعمَلُونَ بَصِيرٌ ﴿٢٣٧﴾\\
\textamh{238.\  } & حَـٰفِظُوا۟ عَلَى ٱلصَّلَوَٟتِ وَٱلصَّلَوٰةِ ٱلوُسطَىٰ وَقُومُوا۟ لِلَّهِ قَـٰنِتِينَ ﴿٢٣٨﴾\\
\textamh{239.\  } & فَإِن خِفتُم فَرِجَالًا أَو رُكبَانًۭا ۖ فَإِذَآ أَمِنتُم فَٱذكُرُوا۟ ٱللَّهَ كَمَا عَلَّمَكُم مَّا لَم تَكُونُوا۟ تَعلَمُونَ ﴿٢٣٩﴾\\
\textamh{240.\  } & وَٱلَّذِينَ يُتَوَفَّونَ مِنكُم وَيَذَرُونَ أَزوَٟجًۭا وَصِيَّةًۭ لِّأَزوَٟجِهِم مَّتَـٰعًا إِلَى ٱلحَولِ غَيرَ إِخرَاجٍۢ ۚ فَإِن خَرَجنَ فَلَا جُنَاحَ عَلَيكُم فِى مَا فَعَلنَ فِىٓ أَنفُسِهِنَّ مِن مَّعرُوفٍۢ ۗ وَٱللَّهُ عَزِيزٌ حَكِيمٌۭ ﴿٢٤٠﴾\\
\textamh{241.\  } & وَلِلمُطَلَّقَـٰتِ مَتَـٰعٌۢ بِٱلمَعرُوفِ ۖ حَقًّا عَلَى ٱلمُتَّقِينَ ﴿٢٤١﴾\\
\textamh{242.\  } & كَذَٟلِكَ يُبَيِّنُ ٱللَّهُ لَكُم ءَايَـٰتِهِۦ لَعَلَّكُم تَعقِلُونَ ﴿٢٤٢﴾\\
\textamh{243.\  } & ۞ أَلَم تَرَ إِلَى ٱلَّذِينَ خَرَجُوا۟ مِن دِيَـٰرِهِم وَهُم أُلُوفٌ حَذَرَ ٱلمَوتِ فَقَالَ لَهُمُ ٱللَّهُ مُوتُوا۟ ثُمَّ أَحيَـٰهُم ۚ إِنَّ ٱللَّهَ لَذُو فَضلٍ عَلَى ٱلنَّاسِ وَلَـٰكِنَّ أَكثَرَ ٱلنَّاسِ لَا يَشكُرُونَ ﴿٢٤٣﴾\\
\textamh{244.\  } & وَقَـٰتِلُوا۟ فِى سَبِيلِ ٱللَّهِ وَٱعلَمُوٓا۟ أَنَّ ٱللَّهَ سَمِيعٌ عَلِيمٌۭ ﴿٢٤٤﴾\\
\textamh{245.\  } & مَّن ذَا ٱلَّذِى يُقرِضُ ٱللَّهَ قَرضًا حَسَنًۭا فَيُضَٰعِفَهُۥ لَهُۥٓ أَضعَافًۭا كَثِيرَةًۭ ۚ وَٱللَّهُ يَقبِضُ وَيَبصُۜطُ وَإِلَيهِ تُرجَعُونَ ﴿٢٤٥﴾\\
\textamh{246.\  } & أَلَم تَرَ إِلَى ٱلمَلَإِ مِنۢ بَنِىٓ إِسرَٰٓءِيلَ مِنۢ بَعدِ مُوسَىٰٓ إِذ قَالُوا۟ لِنَبِىٍّۢ لَّهُمُ ٱبعَث لَنَا مَلِكًۭا نُّقَـٰتِل فِى سَبِيلِ ٱللَّهِ ۖ قَالَ هَل عَسَيتُم إِن كُتِبَ عَلَيكُمُ ٱلقِتَالُ أَلَّا تُقَـٰتِلُوا۟ ۖ قَالُوا۟ وَمَا لَنَآ أَلَّا نُقَـٰتِلَ فِى سَبِيلِ ٱللَّهِ وَقَد أُخرِجنَا مِن دِيَـٰرِنَا وَأَبنَآئِنَا ۖ فَلَمَّا كُتِبَ عَلَيهِمُ ٱلقِتَالُ تَوَلَّوا۟ إِلَّا قَلِيلًۭا مِّنهُم ۗ وَٱللَّهُ عَلِيمٌۢ بِٱلظَّـٰلِمِينَ ﴿٢٤٦﴾\\
\textamh{247.\  } & وَقَالَ لَهُم نَبِيُّهُم إِنَّ ٱللَّهَ قَد بَعَثَ لَكُم طَالُوتَ مَلِكًۭا ۚ قَالُوٓا۟ أَنَّىٰ يَكُونُ لَهُ ٱلمُلكُ عَلَينَا وَنَحنُ أَحَقُّ بِٱلمُلكِ مِنهُ وَلَم يُؤتَ سَعَةًۭ مِّنَ ٱلمَالِ ۚ قَالَ إِنَّ ٱللَّهَ ٱصطَفَىٰهُ عَلَيكُم وَزَادَهُۥ بَسطَةًۭ فِى ٱلعِلمِ وَٱلجِسمِ ۖ وَٱللَّهُ يُؤتِى مُلكَهُۥ مَن يَشَآءُ ۚ وَٱللَّهُ وَٟسِعٌ عَلِيمٌۭ ﴿٢٤٧﴾\\
\textamh{248.\  } & وَقَالَ لَهُم نَبِيُّهُم إِنَّ ءَايَةَ مُلكِهِۦٓ أَن يَأتِيَكُمُ ٱلتَّابُوتُ فِيهِ سَكِينَةٌۭ مِّن رَّبِّكُم وَبَقِيَّةٌۭ مِّمَّا تَرَكَ ءَالُ مُوسَىٰ وَءَالُ هَـٰرُونَ تَحمِلُهُ ٱلمَلَـٰٓئِكَةُ ۚ إِنَّ فِى ذَٟلِكَ لَءَايَةًۭ لَّكُم إِن كُنتُم مُّؤمِنِينَ ﴿٢٤٨﴾\\
\textamh{249.\  } & فَلَمَّا فَصَلَ طَالُوتُ بِٱلجُنُودِ قَالَ إِنَّ ٱللَّهَ مُبتَلِيكُم بِنَهَرٍۢ فَمَن شَرِبَ مِنهُ فَلَيسَ مِنِّى وَمَن لَّم يَطعَمهُ فَإِنَّهُۥ مِنِّىٓ إِلَّا مَنِ ٱغتَرَفَ غُرفَةًۢ بِيَدِهِۦ ۚ فَشَرِبُوا۟ مِنهُ إِلَّا قَلِيلًۭا مِّنهُم ۚ فَلَمَّا جَاوَزَهُۥ هُوَ وَٱلَّذِينَ ءَامَنُوا۟ مَعَهُۥ قَالُوا۟ لَا طَاقَةَ لَنَا ٱليَومَ بِجَالُوتَ وَجُنُودِهِۦ ۚ قَالَ ٱلَّذِينَ يَظُنُّونَ أَنَّهُم مُّلَـٰقُوا۟ ٱللَّهِ كَم مِّن فِئَةٍۢ قَلِيلَةٍ غَلَبَت فِئَةًۭ كَثِيرَةًۢ بِإِذنِ ٱللَّهِ ۗ وَٱللَّهُ مَعَ ٱلصَّـٰبِرِينَ ﴿٢٤٩﴾\\
\textamh{250.\  } & وَلَمَّا بَرَزُوا۟ لِجَالُوتَ وَجُنُودِهِۦ قَالُوا۟ رَبَّنَآ أَفرِغ عَلَينَا صَبرًۭا وَثَبِّت أَقدَامَنَا وَٱنصُرنَا عَلَى ٱلقَومِ ٱلكَـٰفِرِينَ ﴿٢٥٠﴾\\
\textamh{251.\  } & فَهَزَمُوهُم بِإِذنِ ٱللَّهِ وَقَتَلَ دَاوُۥدُ جَالُوتَ وَءَاتَىٰهُ ٱللَّهُ ٱلمُلكَ وَٱلحِكمَةَ وَعَلَّمَهُۥ مِمَّا يَشَآءُ ۗ وَلَولَا دَفعُ ٱللَّهِ ٱلنَّاسَ بَعضَهُم بِبَعضٍۢ لَّفَسَدَتِ ٱلأَرضُ وَلَـٰكِنَّ ٱللَّهَ ذُو فَضلٍ عَلَى ٱلعَـٰلَمِينَ ﴿٢٥١﴾\\
\textamh{252.\  } & تِلكَ ءَايَـٰتُ ٱللَّهِ نَتلُوهَا عَلَيكَ بِٱلحَقِّ ۚ وَإِنَّكَ لَمِنَ ٱلمُرسَلِينَ ﴿٢٥٢﴾\\
\textamh{253.\  } & ۞ تِلكَ ٱلرُّسُلُ فَضَّلنَا بَعضَهُم عَلَىٰ بَعضٍۢ ۘ مِّنهُم مَّن كَلَّمَ ٱللَّهُ ۖ وَرَفَعَ بَعضَهُم دَرَجَٰتٍۢ ۚ وَءَاتَينَا عِيسَى ٱبنَ مَريَمَ ٱلبَيِّنَـٰتِ وَأَيَّدنَـٰهُ بِرُوحِ ٱلقُدُسِ ۗ وَلَو شَآءَ ٱللَّهُ مَا ٱقتَتَلَ ٱلَّذِينَ مِنۢ بَعدِهِم مِّنۢ بَعدِ مَا جَآءَتهُمُ ٱلبَيِّنَـٰتُ وَلَـٰكِنِ ٱختَلَفُوا۟ فَمِنهُم مَّن ءَامَنَ وَمِنهُم مَّن كَفَرَ ۚ وَلَو شَآءَ ٱللَّهُ مَا ٱقتَتَلُوا۟ وَلَـٰكِنَّ ٱللَّهَ يَفعَلُ مَا يُرِيدُ ﴿٢٥٣﴾\\
\textamh{254.\  } & يَـٰٓأَيُّهَا ٱلَّذِينَ ءَامَنُوٓا۟ أَنفِقُوا۟ مِمَّا رَزَقنَـٰكُم مِّن قَبلِ أَن يَأتِىَ يَومٌۭ لَّا بَيعٌۭ فِيهِ وَلَا خُلَّةٌۭ وَلَا شَفَـٰعَةٌۭ ۗ وَٱلكَـٰفِرُونَ هُمُ ٱلظَّـٰلِمُونَ ﴿٢٥٤﴾\\
\textamh{255.\  } & ٱللَّهُ لَآ إِلَـٰهَ إِلَّا هُوَ ٱلحَىُّ ٱلقَيُّومُ ۚ لَا تَأخُذُهُۥ سِنَةٌۭ وَلَا نَومٌۭ ۚ لَّهُۥ مَا فِى ٱلسَّمَـٰوَٟتِ وَمَا فِى ٱلأَرضِ ۗ مَن ذَا ٱلَّذِى يَشفَعُ عِندَهُۥٓ إِلَّا بِإِذنِهِۦ ۚ يَعلَمُ مَا بَينَ أَيدِيهِم وَمَا خَلفَهُم ۖ وَلَا يُحِيطُونَ بِشَىءٍۢ مِّن عِلمِهِۦٓ إِلَّا بِمَا شَآءَ ۚ وَسِعَ كُرسِيُّهُ ٱلسَّمَـٰوَٟتِ وَٱلأَرضَ ۖ وَلَا يَـُٔودُهُۥ حِفظُهُمَا ۚ وَهُوَ ٱلعَلِىُّ ٱلعَظِيمُ ﴿٢٥٥﴾\\
\textamh{256.\  } & لَآ إِكرَاهَ فِى ٱلدِّينِ ۖ قَد تَّبَيَّنَ ٱلرُّشدُ مِنَ ٱلغَىِّ ۚ فَمَن يَكفُر بِٱلطَّٰغُوتِ وَيُؤمِنۢ بِٱللَّهِ فَقَدِ ٱستَمسَكَ بِٱلعُروَةِ ٱلوُثقَىٰ لَا ٱنفِصَامَ لَهَا ۗ وَٱللَّهُ سَمِيعٌ عَلِيمٌ ﴿٢٥٦﴾\\
\textamh{257.\  } & ٱللَّهُ وَلِىُّ ٱلَّذِينَ ءَامَنُوا۟ يُخرِجُهُم مِّنَ ٱلظُّلُمَـٰتِ إِلَى ٱلنُّورِ ۖ وَٱلَّذِينَ كَفَرُوٓا۟ أَولِيَآؤُهُمُ ٱلطَّٰغُوتُ يُخرِجُونَهُم مِّنَ ٱلنُّورِ إِلَى ٱلظُّلُمَـٰتِ ۗ أُو۟لَـٰٓئِكَ أَصحَـٰبُ ٱلنَّارِ ۖ هُم فِيهَا خَـٰلِدُونَ ﴿٢٥٧﴾\\
\textamh{258.\  } & أَلَم تَرَ إِلَى ٱلَّذِى حَآجَّ إِبرَٰهِۦمَ فِى رَبِّهِۦٓ أَن ءَاتَىٰهُ ٱللَّهُ ٱلمُلكَ إِذ قَالَ إِبرَٰهِۦمُ رَبِّىَ ٱلَّذِى يُحىِۦ وَيُمِيتُ قَالَ أَنَا۠ أُحىِۦ وَأُمِيتُ ۖ قَالَ إِبرَٰهِۦمُ فَإِنَّ ٱللَّهَ يَأتِى بِٱلشَّمسِ مِنَ ٱلمَشرِقِ فَأتِ بِهَا مِنَ ٱلمَغرِبِ فَبُهِتَ ٱلَّذِى كَفَرَ ۗ وَٱللَّهُ لَا يَهدِى ٱلقَومَ ٱلظَّـٰلِمِينَ ﴿٢٥٨﴾\\
\textamh{259.\  } & أَو كَٱلَّذِى مَرَّ عَلَىٰ قَريَةٍۢ وَهِىَ خَاوِيَةٌ عَلَىٰ عُرُوشِهَا قَالَ أَنَّىٰ يُحىِۦ هَـٰذِهِ ٱللَّهُ بَعدَ مَوتِهَا ۖ فَأَمَاتَهُ ٱللَّهُ مِا۟ئَةَ عَامٍۢ ثُمَّ بَعَثَهُۥ ۖ قَالَ كَم لَبِثتَ ۖ قَالَ لَبِثتُ يَومًا أَو بَعضَ يَومٍۢ ۖ قَالَ بَل لَّبِثتَ مِا۟ئَةَ عَامٍۢ فَٱنظُر إِلَىٰ طَعَامِكَ وَشَرَابِكَ لَم يَتَسَنَّه ۖ وَٱنظُر إِلَىٰ حِمَارِكَ وَلِنَجعَلَكَ ءَايَةًۭ لِّلنَّاسِ ۖ وَٱنظُر إِلَى ٱلعِظَامِ كَيفَ نُنشِزُهَا ثُمَّ نَكسُوهَا لَحمًۭا ۚ فَلَمَّا تَبَيَّنَ لَهُۥ قَالَ أَعلَمُ أَنَّ ٱللَّهَ عَلَىٰ كُلِّ شَىءٍۢ قَدِيرٌۭ ﴿٢٥٩﴾\\
\textamh{260.\  } & وَإِذ قَالَ إِبرَٰهِۦمُ رَبِّ أَرِنِى كَيفَ تُحىِ ٱلمَوتَىٰ ۖ قَالَ أَوَلَم تُؤمِن ۖ قَالَ بَلَىٰ وَلَـٰكِن لِّيَطمَئِنَّ قَلبِى ۖ قَالَ فَخُذ أَربَعَةًۭ مِّنَ ٱلطَّيرِ فَصُرهُنَّ إِلَيكَ ثُمَّ ٱجعَل عَلَىٰ كُلِّ جَبَلٍۢ مِّنهُنَّ جُزءًۭا ثُمَّ ٱدعُهُنَّ يَأتِينَكَ سَعيًۭا ۚ وَٱعلَم أَنَّ ٱللَّهَ عَزِيزٌ حَكِيمٌۭ ﴿٢٦٠﴾\\
\textamh{261.\  } & مَّثَلُ ٱلَّذِينَ يُنفِقُونَ أَموَٟلَهُم فِى سَبِيلِ ٱللَّهِ كَمَثَلِ حَبَّةٍ أَنۢبَتَت سَبعَ سَنَابِلَ فِى كُلِّ سُنۢبُلَةٍۢ مِّا۟ئَةُ حَبَّةٍۢ ۗ وَٱللَّهُ يُضَٰعِفُ لِمَن يَشَآءُ ۗ وَٱللَّهُ وَٟسِعٌ عَلِيمٌ ﴿٢٦١﴾\\
\textamh{262.\  } & ٱلَّذِينَ يُنفِقُونَ أَموَٟلَهُم فِى سَبِيلِ ٱللَّهِ ثُمَّ لَا يُتبِعُونَ مَآ أَنفَقُوا۟ مَنًّۭا وَلَآ أَذًۭى ۙ لَّهُم أَجرُهُم عِندَ رَبِّهِم وَلَا خَوفٌ عَلَيهِم وَلَا هُم يَحزَنُونَ ﴿٢٦٢﴾\\
\textamh{263.\  } & ۞ قَولٌۭ مَّعرُوفٌۭ وَمَغفِرَةٌ خَيرٌۭ مِّن صَدَقَةٍۢ يَتبَعُهَآ أَذًۭى ۗ وَٱللَّهُ غَنِىٌّ حَلِيمٌۭ ﴿٢٦٣﴾\\
\textamh{264.\  } & يَـٰٓأَيُّهَا ٱلَّذِينَ ءَامَنُوا۟ لَا تُبطِلُوا۟ صَدَقَـٰتِكُم بِٱلمَنِّ وَٱلأَذَىٰ كَٱلَّذِى يُنفِقُ مَالَهُۥ رِئَآءَ ٱلنَّاسِ وَلَا يُؤمِنُ بِٱللَّهِ وَٱليَومِ ٱلءَاخِرِ ۖ فَمَثَلُهُۥ كَمَثَلِ صَفوَانٍ عَلَيهِ تُرَابٌۭ فَأَصَابَهُۥ وَابِلٌۭ فَتَرَكَهُۥ صَلدًۭا ۖ لَّا يَقدِرُونَ عَلَىٰ شَىءٍۢ مِّمَّا كَسَبُوا۟ ۗ وَٱللَّهُ لَا يَهدِى ٱلقَومَ ٱلكَـٰفِرِينَ ﴿٢٦٤﴾\\
\textamh{265.\  } & وَمَثَلُ ٱلَّذِينَ يُنفِقُونَ أَموَٟلَهُمُ ٱبتِغَآءَ مَرضَاتِ ٱللَّهِ وَتَثبِيتًۭا مِّن أَنفُسِهِم كَمَثَلِ جَنَّةٍۭ بِرَبوَةٍ أَصَابَهَا وَابِلٌۭ فَـَٔاتَت أُكُلَهَا ضِعفَينِ فَإِن لَّم يُصِبهَا وَابِلٌۭ فَطَلٌّۭ ۗ وَٱللَّهُ بِمَا تَعمَلُونَ بَصِيرٌ ﴿٢٦٥﴾\\
\textamh{266.\  } & أَيَوَدُّ أَحَدُكُم أَن تَكُونَ لَهُۥ جَنَّةٌۭ مِّن نَّخِيلٍۢ وَأَعنَابٍۢ تَجرِى مِن تَحتِهَا ٱلأَنهَـٰرُ لَهُۥ فِيهَا مِن كُلِّ ٱلثَّمَرَٰتِ وَأَصَابَهُ ٱلكِبَرُ وَلَهُۥ ذُرِّيَّةٌۭ ضُعَفَآءُ فَأَصَابَهَآ إِعصَارٌۭ فِيهِ نَارٌۭ فَٱحتَرَقَت ۗ كَذَٟلِكَ يُبَيِّنُ ٱللَّهُ لَكُمُ ٱلءَايَـٰتِ لَعَلَّكُم تَتَفَكَّرُونَ ﴿٢٦٦﴾\\
\textamh{267.\  } & يَـٰٓأَيُّهَا ٱلَّذِينَ ءَامَنُوٓا۟ أَنفِقُوا۟ مِن طَيِّبَٰتِ مَا كَسَبتُم وَمِمَّآ أَخرَجنَا لَكُم مِّنَ ٱلأَرضِ ۖ وَلَا تَيَمَّمُوا۟ ٱلخَبِيثَ مِنهُ تُنفِقُونَ وَلَستُم بِـَٔاخِذِيهِ إِلَّآ أَن تُغمِضُوا۟ فِيهِ ۚ وَٱعلَمُوٓا۟ أَنَّ ٱللَّهَ غَنِىٌّ حَمِيدٌ ﴿٢٦٧﴾\\
\textamh{268.\  } & ٱلشَّيطَٰنُ يَعِدُكُمُ ٱلفَقرَ وَيَأمُرُكُم بِٱلفَحشَآءِ ۖ وَٱللَّهُ يَعِدُكُم مَّغفِرَةًۭ مِّنهُ وَفَضلًۭا ۗ وَٱللَّهُ وَٟسِعٌ عَلِيمٌۭ ﴿٢٦٨﴾\\
\textamh{269.\  } & يُؤتِى ٱلحِكمَةَ مَن يَشَآءُ ۚ وَمَن يُؤتَ ٱلحِكمَةَ فَقَد أُوتِىَ خَيرًۭا كَثِيرًۭا ۗ وَمَا يَذَّكَّرُ إِلَّآ أُو۟لُوا۟ ٱلأَلبَٰبِ ﴿٢٦٩﴾\\
\textamh{270.\  } & وَمَآ أَنفَقتُم مِّن نَّفَقَةٍ أَو نَذَرتُم مِّن نَّذرٍۢ فَإِنَّ ٱللَّهَ يَعلَمُهُۥ ۗ وَمَا لِلظَّـٰلِمِينَ مِن أَنصَارٍ ﴿٢٧٠﴾\\
\textamh{271.\  } & إِن تُبدُوا۟ ٱلصَّدَقَـٰتِ فَنِعِمَّا هِىَ ۖ وَإِن تُخفُوهَا وَتُؤتُوهَا ٱلفُقَرَآءَ فَهُوَ خَيرٌۭ لَّكُم ۚ وَيُكَفِّرُ عَنكُم مِّن سَيِّـَٔاتِكُم ۗ وَٱللَّهُ بِمَا تَعمَلُونَ خَبِيرٌۭ ﴿٢٧١﴾\\
\textamh{272.\  } & ۞ لَّيسَ عَلَيكَ هُدَىٰهُم وَلَـٰكِنَّ ٱللَّهَ يَهدِى مَن يَشَآءُ ۗ وَمَا تُنفِقُوا۟ مِن خَيرٍۢ فَلِأَنفُسِكُم ۚ وَمَا تُنفِقُونَ إِلَّا ٱبتِغَآءَ وَجهِ ٱللَّهِ ۚ وَمَا تُنفِقُوا۟ مِن خَيرٍۢ يُوَفَّ إِلَيكُم وَأَنتُم لَا تُظلَمُونَ ﴿٢٧٢﴾\\
\textamh{273.\  } & لِلفُقَرَآءِ ٱلَّذِينَ أُحصِرُوا۟ فِى سَبِيلِ ٱللَّهِ لَا يَستَطِيعُونَ ضَربًۭا فِى ٱلأَرضِ يَحسَبُهُمُ ٱلجَاهِلُ أَغنِيَآءَ مِنَ ٱلتَّعَفُّفِ تَعرِفُهُم بِسِيمَـٰهُم لَا يَسـَٔلُونَ ٱلنَّاسَ إِلحَافًۭا ۗ وَمَا تُنفِقُوا۟ مِن خَيرٍۢ فَإِنَّ ٱللَّهَ بِهِۦ عَلِيمٌ ﴿٢٧٣﴾\\
\textamh{274.\  } & ٱلَّذِينَ يُنفِقُونَ أَموَٟلَهُم بِٱلَّيلِ وَٱلنَّهَارِ سِرًّۭا وَعَلَانِيَةًۭ فَلَهُم أَجرُهُم عِندَ رَبِّهِم وَلَا خَوفٌ عَلَيهِم وَلَا هُم يَحزَنُونَ ﴿٢٧٤﴾\\
\textamh{275.\  } & ٱلَّذِينَ يَأكُلُونَ ٱلرِّبَوٰا۟ لَا يَقُومُونَ إِلَّا كَمَا يَقُومُ ٱلَّذِى يَتَخَبَّطُهُ ٱلشَّيطَٰنُ مِنَ ٱلمَسِّ ۚ ذَٟلِكَ بِأَنَّهُم قَالُوٓا۟ إِنَّمَا ٱلبَيعُ مِثلُ ٱلرِّبَوٰا۟ ۗ وَأَحَلَّ ٱللَّهُ ٱلبَيعَ وَحَرَّمَ ٱلرِّبَوٰا۟ ۚ فَمَن جَآءَهُۥ مَوعِظَةٌۭ مِّن رَّبِّهِۦ فَٱنتَهَىٰ فَلَهُۥ مَا سَلَفَ وَأَمرُهُۥٓ إِلَى ٱللَّهِ ۖ وَمَن عَادَ فَأُو۟لَـٰٓئِكَ أَصحَـٰبُ ٱلنَّارِ ۖ هُم فِيهَا خَـٰلِدُونَ ﴿٢٧٥﴾\\
\textamh{276.\  } & يَمحَقُ ٱللَّهُ ٱلرِّبَوٰا۟ وَيُربِى ٱلصَّدَقَـٰتِ ۗ وَٱللَّهُ لَا يُحِبُّ كُلَّ كَفَّارٍ أَثِيمٍ ﴿٢٧٦﴾\\
\textamh{277.\  } & إِنَّ ٱلَّذِينَ ءَامَنُوا۟ وَعَمِلُوا۟ ٱلصَّـٰلِحَـٰتِ وَأَقَامُوا۟ ٱلصَّلَوٰةَ وَءَاتَوُا۟ ٱلزَّكَوٰةَ لَهُم أَجرُهُم عِندَ رَبِّهِم وَلَا خَوفٌ عَلَيهِم وَلَا هُم يَحزَنُونَ ﴿٢٧٧﴾\\
\textamh{278.\  } & يَـٰٓأَيُّهَا ٱلَّذِينَ ءَامَنُوا۟ ٱتَّقُوا۟ ٱللَّهَ وَذَرُوا۟ مَا بَقِىَ مِنَ ٱلرِّبَوٰٓا۟ إِن كُنتُم مُّؤمِنِينَ ﴿٢٧٨﴾\\
\textamh{279.\  } & فَإِن لَّم تَفعَلُوا۟ فَأذَنُوا۟ بِحَربٍۢ مِّنَ ٱللَّهِ وَرَسُولِهِۦ ۖ وَإِن تُبتُم فَلَكُم رُءُوسُ أَموَٟلِكُم لَا تَظلِمُونَ وَلَا تُظلَمُونَ ﴿٢٧٩﴾\\
\textamh{280.\  } & وَإِن كَانَ ذُو عُسرَةٍۢ فَنَظِرَةٌ إِلَىٰ مَيسَرَةٍۢ ۚ وَأَن تَصَدَّقُوا۟ خَيرٌۭ لَّكُم ۖ إِن كُنتُم تَعلَمُونَ ﴿٢٨٠﴾\\
\textamh{281.\  } & وَٱتَّقُوا۟ يَومًۭا تُرجَعُونَ فِيهِ إِلَى ٱللَّهِ ۖ ثُمَّ تُوَفَّىٰ كُلُّ نَفسٍۢ مَّا كَسَبَت وَهُم لَا يُظلَمُونَ ﴿٢٨١﴾\\
\textamh{282.\  } & يَـٰٓأَيُّهَا ٱلَّذِينَ ءَامَنُوٓا۟ إِذَا تَدَايَنتُم بِدَينٍ إِلَىٰٓ أَجَلٍۢ مُّسَمًّۭى فَٱكتُبُوهُ ۚ وَليَكتُب بَّينَكُم كَاتِبٌۢ بِٱلعَدلِ ۚ وَلَا يَأبَ كَاتِبٌ أَن يَكتُبَ كَمَا عَلَّمَهُ ٱللَّهُ ۚ فَليَكتُب وَليُملِلِ ٱلَّذِى عَلَيهِ ٱلحَقُّ وَليَتَّقِ ٱللَّهَ رَبَّهُۥ وَلَا يَبخَس مِنهُ شَيـًۭٔا ۚ فَإِن كَانَ ٱلَّذِى عَلَيهِ ٱلحَقُّ سَفِيهًا أَو ضَعِيفًا أَو لَا يَستَطِيعُ أَن يُمِلَّ هُوَ فَليُملِل وَلِيُّهُۥ بِٱلعَدلِ ۚ وَٱستَشهِدُوا۟ شَهِيدَينِ مِن رِّجَالِكُم ۖ فَإِن لَّم يَكُونَا رَجُلَينِ فَرَجُلٌۭ وَٱمرَأَتَانِ مِمَّن تَرضَونَ مِنَ ٱلشُّهَدَآءِ أَن تَضِلَّ إِحدَىٰهُمَا فَتُذَكِّرَ إِحدَىٰهُمَا ٱلأُخرَىٰ ۚ وَلَا يَأبَ ٱلشُّهَدَآءُ إِذَا مَا دُعُوا۟ ۚ وَلَا تَسـَٔمُوٓا۟ أَن تَكتُبُوهُ صَغِيرًا أَو كَبِيرًا إِلَىٰٓ أَجَلِهِۦ ۚ ذَٟلِكُم أَقسَطُ عِندَ ٱللَّهِ وَأَقوَمُ لِلشَّهَـٰدَةِ وَأَدنَىٰٓ أَلَّا تَرتَابُوٓا۟ ۖ إِلَّآ أَن تَكُونَ تِجَٰرَةً حَاضِرَةًۭ تُدِيرُونَهَا بَينَكُم فَلَيسَ عَلَيكُم جُنَاحٌ أَلَّا تَكتُبُوهَا ۗ وَأَشهِدُوٓا۟ إِذَا تَبَايَعتُم ۚ وَلَا يُضَآرَّ كَاتِبٌۭ وَلَا شَهِيدٌۭ ۚ وَإِن تَفعَلُوا۟ فَإِنَّهُۥ فُسُوقٌۢ بِكُم ۗ وَٱتَّقُوا۟ ٱللَّهَ ۖ وَيُعَلِّمُكُمُ ٱللَّهُ ۗ وَٱللَّهُ بِكُلِّ شَىءٍ عَلِيمٌۭ ﴿٢٨٢﴾\\
\textamh{283.\  } & ۞ وَإِن كُنتُم عَلَىٰ سَفَرٍۢ وَلَم تَجِدُوا۟ كَاتِبًۭا فَرِهَـٰنٌۭ مَّقبُوضَةٌۭ ۖ فَإِن أَمِنَ بَعضُكُم بَعضًۭا فَليُؤَدِّ ٱلَّذِى ٱؤتُمِنَ أَمَـٰنَتَهُۥ وَليَتَّقِ ٱللَّهَ رَبَّهُۥ ۗ وَلَا تَكتُمُوا۟ ٱلشَّهَـٰدَةَ ۚ وَمَن يَكتُمهَا فَإِنَّهُۥٓ ءَاثِمٌۭ قَلبُهُۥ ۗ وَٱللَّهُ بِمَا تَعمَلُونَ عَلِيمٌۭ ﴿٢٨٣﴾\\
\textamh{284.\  } & لِّلَّهِ مَا فِى ٱلسَّمَـٰوَٟتِ وَمَا فِى ٱلأَرضِ ۗ وَإِن تُبدُوا۟ مَا فِىٓ أَنفُسِكُم أَو تُخفُوهُ يُحَاسِبكُم بِهِ ٱللَّهُ ۖ فَيَغفِرُ لِمَن يَشَآءُ وَيُعَذِّبُ مَن يَشَآءُ ۗ وَٱللَّهُ عَلَىٰ كُلِّ شَىءٍۢ قَدِيرٌ ﴿٢٨٤﴾\\
\textamh{285.\  } & ءَامَنَ ٱلرَّسُولُ بِمَآ أُنزِلَ إِلَيهِ مِن رَّبِّهِۦ وَٱلمُؤمِنُونَ ۚ كُلٌّ ءَامَنَ بِٱللَّهِ وَمَلَـٰٓئِكَتِهِۦ وَكُتُبِهِۦ وَرُسُلِهِۦ لَا نُفَرِّقُ بَينَ أَحَدٍۢ مِّن رُّسُلِهِۦ ۚ وَقَالُوا۟ سَمِعنَا وَأَطَعنَا ۖ غُفرَانَكَ رَبَّنَا وَإِلَيكَ ٱلمَصِيرُ ﴿٢٨٥﴾\\
\textamh{286.\  } & لَا يُكَلِّفُ ٱللَّهُ نَفسًا إِلَّا وُسعَهَا ۚ لَهَا مَا كَسَبَت وَعَلَيهَا مَا ٱكتَسَبَت ۗ رَبَّنَا لَا تُؤَاخِذنَآ إِن نَّسِينَآ أَو أَخطَأنَا ۚ رَبَّنَا وَلَا تَحمِل عَلَينَآ إِصرًۭا كَمَا حَمَلتَهُۥ عَلَى ٱلَّذِينَ مِن قَبلِنَا ۚ رَبَّنَا وَلَا تُحَمِّلنَا مَا لَا طَاقَةَ لَنَا بِهِۦ ۖ وَٱعفُ عَنَّا وَٱغفِر لَنَا وَٱرحَمنَآ ۚ أَنتَ مَولَىٰنَا فَٱنصُرنَا عَلَى ٱلقَومِ ٱلكَـٰفِرِينَ ﴿٢٨٦﴾\\
\end{longtable} \newpage

