%% License: BSD style (Berkley) (i.e. Put the Copyright owner's name always)
%% Writer and Copyright (to): Bewketu(Bilal) Tadilo (2016-17)
\shadowbox{\section{\LR{\textamharic{ሱራቱ አልካህፍ -}  \RL{سوره  الكهف}}}}
\begin{longtable}{%
  @{}
    p{.5\textwidth}
  @{~~~~~~~~~~~~~}||
    p{.5\textwidth}
    @{}
}
\nopagebreak
\textamh{\ \ \ \ \ \  ቢስሚላሂ አራህመኒ ራሂይም } &  بِسمِ ٱللَّهِ ٱلرَّحمَـٰنِ ٱلرَّحِيمِ\\
\textamh{1.\  } &  ٱلحَمدُ لِلَّهِ ٱلَّذِىٓ أَنزَلَ عَلَىٰ عَبدِهِ ٱلكِتَـٰبَ وَلَم يَجعَل لَّهُۥ عِوَجَا ۜ ﴿١﴾\\
\textamh{2.\  } & قَيِّمًۭا لِّيُنذِرَ بَأسًۭا شَدِيدًۭا مِّن لَّدُنهُ وَيُبَشِّرَ ٱلمُؤمِنِينَ ٱلَّذِينَ يَعمَلُونَ ٱلصَّـٰلِحَـٰتِ أَنَّ لَهُم أَجرًا حَسَنًۭا ﴿٢﴾\\
\textamh{3.\  } & مَّٰكِثِينَ فِيهِ أَبَدًۭا ﴿٣﴾\\
\textamh{4.\  } & وَيُنذِرَ ٱلَّذِينَ قَالُوا۟ ٱتَّخَذَ ٱللَّهُ وَلَدًۭا ﴿٤﴾\\
\textamh{5.\  } & مَّا لَهُم بِهِۦ مِن عِلمٍۢ وَلَا لِءَابَآئِهِم ۚ كَبُرَت كَلِمَةًۭ تَخرُجُ مِن أَفوَٟهِهِم ۚ إِن يَقُولُونَ إِلَّا كَذِبًۭا ﴿٥﴾\\
\textamh{6.\  } & فَلَعَلَّكَ بَٰخِعٌۭ نَّفسَكَ عَلَىٰٓ ءَاثَـٰرِهِم إِن لَّم يُؤمِنُوا۟ بِهَـٰذَا ٱلحَدِيثِ أَسَفًا ﴿٦﴾\\
\textamh{7.\  } & إِنَّا جَعَلنَا مَا عَلَى ٱلأَرضِ زِينَةًۭ لَّهَا لِنَبلُوَهُم أَيُّهُم أَحسَنُ عَمَلًۭا ﴿٧﴾\\
\textamh{8.\  } & وَإِنَّا لَجَٰعِلُونَ مَا عَلَيهَا صَعِيدًۭا جُرُزًا ﴿٨﴾\\
\textamh{9.\  } & أَم حَسِبتَ أَنَّ أَصحَـٰبَ ٱلكَهفِ وَٱلرَّقِيمِ كَانُوا۟ مِن ءَايَـٰتِنَا عَجَبًا ﴿٩﴾\\
\textamh{10.\  } & إِذ أَوَى ٱلفِتيَةُ إِلَى ٱلكَهفِ فَقَالُوا۟ رَبَّنَآ ءَاتِنَا مِن لَّدُنكَ رَحمَةًۭ وَهَيِّئ لَنَا مِن أَمرِنَا رَشَدًۭا ﴿١٠﴾\\
\textamh{11.\  } & فَضَرَبنَا عَلَىٰٓ ءَاذَانِهِم فِى ٱلكَهفِ سِنِينَ عَدَدًۭا ﴿١١﴾\\
\textamh{12.\  } & ثُمَّ بَعَثنَـٰهُم لِنَعلَمَ أَىُّ ٱلحِزبَينِ أَحصَىٰ لِمَا لَبِثُوٓا۟ أَمَدًۭا ﴿١٢﴾\\
\textamh{13.\  } & نَّحنُ نَقُصُّ عَلَيكَ نَبَأَهُم بِٱلحَقِّ ۚ إِنَّهُم فِتيَةٌ ءَامَنُوا۟ بِرَبِّهِم وَزِدنَـٰهُم هُدًۭى ﴿١٣﴾\\
\textamh{14.\  } & وَرَبَطنَا عَلَىٰ قُلُوبِهِم إِذ قَامُوا۟ فَقَالُوا۟ رَبُّنَا رَبُّ ٱلسَّمَـٰوَٟتِ وَٱلأَرضِ لَن نَّدعُوَا۟ مِن دُونِهِۦٓ إِلَـٰهًۭا ۖ لَّقَد قُلنَآ إِذًۭا شَطَطًا ﴿١٤﴾\\
\textamh{15.\  } & هَـٰٓؤُلَآءِ قَومُنَا ٱتَّخَذُوا۟ مِن دُونِهِۦٓ ءَالِهَةًۭ ۖ لَّولَا يَأتُونَ عَلَيهِم بِسُلطَٰنٍۭ بَيِّنٍۢ ۖ فَمَن أَظلَمُ مِمَّنِ ٱفتَرَىٰ عَلَى ٱللَّهِ كَذِبًۭا ﴿١٥﴾\\
\textamh{16.\  } & وَإِذِ ٱعتَزَلتُمُوهُم وَمَا يَعبُدُونَ إِلَّا ٱللَّهَ فَأوُۥٓا۟ إِلَى ٱلكَهفِ يَنشُر لَكُم رَبُّكُم مِّن رَّحمَتِهِۦ وَيُهَيِّئ لَكُم مِّن أَمرِكُم مِّرفَقًۭا ﴿١٦﴾\\
\textamh{17.\  } & ۞ وَتَرَى ٱلشَّمسَ إِذَا طَلَعَت تَّزَٰوَرُ عَن كَهفِهِم ذَاتَ ٱليَمِينِ وَإِذَا غَرَبَت تَّقرِضُهُم ذَاتَ ٱلشِّمَالِ وَهُم فِى فَجوَةٍۢ مِّنهُ ۚ ذَٟلِكَ مِن ءَايَـٰتِ ٱللَّهِ ۗ مَن يَهدِ ٱللَّهُ فَهُوَ ٱلمُهتَدِ ۖ وَمَن يُضلِل فَلَن تَجِدَ لَهُۥ وَلِيًّۭا مُّرشِدًۭا ﴿١٧﴾\\
\textamh{18.\  } & وَتَحسَبُهُم أَيقَاظًۭا وَهُم رُقُودٌۭ ۚ وَنُقَلِّبُهُم ذَاتَ ٱليَمِينِ وَذَاتَ ٱلشِّمَالِ ۖ وَكَلبُهُم بَٰسِطٌۭ ذِرَاعَيهِ بِٱلوَصِيدِ ۚ لَوِ ٱطَّلَعتَ عَلَيهِم لَوَلَّيتَ مِنهُم فِرَارًۭا وَلَمُلِئتَ مِنهُم رُعبًۭا ﴿١٨﴾\\
\textamh{19.\  } & وَكَذَٟلِكَ بَعَثنَـٰهُم لِيَتَسَآءَلُوا۟ بَينَهُم ۚ قَالَ قَآئِلٌۭ مِّنهُم كَم لَبِثتُم ۖ قَالُوا۟ لَبِثنَا يَومًا أَو بَعضَ يَومٍۢ ۚ قَالُوا۟ رَبُّكُم أَعلَمُ بِمَا لَبِثتُم فَٱبعَثُوٓا۟ أَحَدَكُم بِوَرِقِكُم هَـٰذِهِۦٓ إِلَى ٱلمَدِينَةِ فَليَنظُر أَيُّهَآ أَزكَىٰ طَعَامًۭا فَليَأتِكُم بِرِزقٍۢ مِّنهُ وَليَتَلَطَّف وَلَا يُشعِرَنَّ بِكُم أَحَدًا ﴿١٩﴾\\
\textamh{20.\  } & إِنَّهُم إِن يَظهَرُوا۟ عَلَيكُم يَرجُمُوكُم أَو يُعِيدُوكُم فِى مِلَّتِهِم وَلَن تُفلِحُوٓا۟ إِذًا أَبَدًۭا ﴿٢٠﴾\\
\textamh{21.\  } & وَكَذَٟلِكَ أَعثَرنَا عَلَيهِم لِيَعلَمُوٓا۟ أَنَّ وَعدَ ٱللَّهِ حَقٌّۭ وَأَنَّ ٱلسَّاعَةَ لَا رَيبَ فِيهَآ إِذ يَتَنَـٰزَعُونَ بَينَهُم أَمرَهُم ۖ فَقَالُوا۟ ٱبنُوا۟ عَلَيهِم بُنيَـٰنًۭا ۖ رَّبُّهُم أَعلَمُ بِهِم ۚ قَالَ ٱلَّذِينَ غَلَبُوا۟ عَلَىٰٓ أَمرِهِم لَنَتَّخِذَنَّ عَلَيهِم مَّسجِدًۭا ﴿٢١﴾\\
\textamh{22.\  } & سَيَقُولُونَ ثَلَـٰثَةٌۭ رَّابِعُهُم كَلبُهُم وَيَقُولُونَ خَمسَةٌۭ سَادِسُهُم كَلبُهُم رَجمًۢا بِٱلغَيبِ ۖ وَيَقُولُونَ سَبعَةٌۭ وَثَامِنُهُم كَلبُهُم ۚ قُل رَّبِّىٓ أَعلَمُ بِعِدَّتِهِم مَّا يَعلَمُهُم إِلَّا قَلِيلٌۭ ۗ فَلَا تُمَارِ فِيهِم إِلَّا مِرَآءًۭ ظَـٰهِرًۭا وَلَا تَستَفتِ فِيهِم مِّنهُم أَحَدًۭا ﴿٢٢﴾\\
\textamh{23.\  } & وَلَا تَقُولَنَّ لِشَا۟ىءٍ إِنِّى فَاعِلٌۭ ذَٟلِكَ غَدًا ﴿٢٣﴾\\
\textamh{24.\  } & إِلَّآ أَن يَشَآءَ ٱللَّهُ ۚ وَٱذكُر رَّبَّكَ إِذَا نَسِيتَ وَقُل عَسَىٰٓ أَن يَهدِيَنِ رَبِّى لِأَقرَبَ مِن هَـٰذَا رَشَدًۭا ﴿٢٤﴾\\
\textamh{25.\  } & وَلَبِثُوا۟ فِى كَهفِهِم ثَلَـٰثَ مِا۟ئَةٍۢ سِنِينَ وَٱزدَادُوا۟ تِسعًۭا ﴿٢٥﴾\\
\textamh{26.\  } & قُلِ ٱللَّهُ أَعلَمُ بِمَا لَبِثُوا۟ ۖ لَهُۥ غَيبُ ٱلسَّمَـٰوَٟتِ وَٱلأَرضِ ۖ أَبصِر بِهِۦ وَأَسمِع ۚ مَا لَهُم مِّن دُونِهِۦ مِن وَلِىٍّۢ وَلَا يُشرِكُ فِى حُكمِهِۦٓ أَحَدًۭا ﴿٢٦﴾\\
\textamh{27.\  } & وَٱتلُ مَآ أُوحِىَ إِلَيكَ مِن كِتَابِ رَبِّكَ ۖ لَا مُبَدِّلَ لِكَلِمَـٰتِهِۦ وَلَن تَجِدَ مِن دُونِهِۦ مُلتَحَدًۭا ﴿٢٧﴾\\
\textamh{28.\  } & وَٱصبِر نَفسَكَ مَعَ ٱلَّذِينَ يَدعُونَ رَبَّهُم بِٱلغَدَوٰةِ وَٱلعَشِىِّ يُرِيدُونَ وَجهَهُۥ ۖ وَلَا تَعدُ عَينَاكَ عَنهُم تُرِيدُ زِينَةَ ٱلحَيَوٰةِ ٱلدُّنيَا ۖ وَلَا تُطِع مَن أَغفَلنَا قَلبَهُۥ عَن ذِكرِنَا وَٱتَّبَعَ هَوَىٰهُ وَكَانَ أَمرُهُۥ فُرُطًۭا ﴿٢٨﴾\\
\textamh{29.\  } & وَقُلِ ٱلحَقُّ مِن رَّبِّكُم ۖ فَمَن شَآءَ فَليُؤمِن وَمَن شَآءَ فَليَكفُر ۚ إِنَّآ أَعتَدنَا لِلظَّـٰلِمِينَ نَارًا أَحَاطَ بِهِم سُرَادِقُهَا ۚ وَإِن يَستَغِيثُوا۟ يُغَاثُوا۟ بِمَآءٍۢ كَٱلمُهلِ يَشوِى ٱلوُجُوهَ ۚ بِئسَ ٱلشَّرَابُ وَسَآءَت مُرتَفَقًا ﴿٢٩﴾\\
\textamh{30.\  } & إِنَّ ٱلَّذِينَ ءَامَنُوا۟ وَعَمِلُوا۟ ٱلصَّـٰلِحَـٰتِ إِنَّا لَا نُضِيعُ أَجرَ مَن أَحسَنَ عَمَلًا ﴿٣٠﴾\\
\textamh{31.\  } & أُو۟لَـٰٓئِكَ لَهُم جَنَّـٰتُ عَدنٍۢ تَجرِى مِن تَحتِهِمُ ٱلأَنهَـٰرُ يُحَلَّونَ فِيهَا مِن أَسَاوِرَ مِن ذَهَبٍۢ وَيَلبَسُونَ ثِيَابًا خُضرًۭا مِّن سُندُسٍۢ وَإِستَبرَقٍۢ مُّتَّكِـِٔينَ فِيهَا عَلَى ٱلأَرَآئِكِ ۚ نِعمَ ٱلثَّوَابُ وَحَسُنَت مُرتَفَقًۭا ﴿٣١﴾\\
\textamh{32.\  } & ۞ وَٱضرِب لَهُم مَّثَلًۭا رَّجُلَينِ جَعَلنَا لِأَحَدِهِمَا جَنَّتَينِ مِن أَعنَـٰبٍۢ وَحَفَفنَـٰهُمَا بِنَخلٍۢ وَجَعَلنَا بَينَهُمَا زَرعًۭا ﴿٣٢﴾\\
\textamh{33.\  } & كِلتَا ٱلجَنَّتَينِ ءَاتَت أُكُلَهَا وَلَم تَظلِم مِّنهُ شَيـًۭٔا ۚ وَفَجَّرنَا خِلَـٰلَهُمَا نَهَرًۭا ﴿٣٣﴾\\
\textamh{34.\  } & وَكَانَ لَهُۥ ثَمَرٌۭ فَقَالَ لِصَـٰحِبِهِۦ وَهُوَ يُحَاوِرُهُۥٓ أَنَا۠ أَكثَرُ مِنكَ مَالًۭا وَأَعَزُّ نَفَرًۭا ﴿٣٤﴾\\
\textamh{35.\  } & وَدَخَلَ جَنَّتَهُۥ وَهُوَ ظَالِمٌۭ لِّنَفسِهِۦ قَالَ مَآ أَظُنُّ أَن تَبِيدَ هَـٰذِهِۦٓ أَبَدًۭا ﴿٣٥﴾\\
\textamh{36.\  } & وَمَآ أَظُنُّ ٱلسَّاعَةَ قَآئِمَةًۭ وَلَئِن رُّدِدتُّ إِلَىٰ رَبِّى لَأَجِدَنَّ خَيرًۭا مِّنهَا مُنقَلَبًۭا ﴿٣٦﴾\\
\textamh{37.\  } & قَالَ لَهُۥ صَاحِبُهُۥ وَهُوَ يُحَاوِرُهُۥٓ أَكَفَرتَ بِٱلَّذِى خَلَقَكَ مِن تُرَابٍۢ ثُمَّ مِن نُّطفَةٍۢ ثُمَّ سَوَّىٰكَ رَجُلًۭا ﴿٣٧﴾\\
\textamh{38.\  } & لَّٰكِنَّا۠ هُوَ ٱللَّهُ رَبِّى وَلَآ أُشرِكُ بِرَبِّىٓ أَحَدًۭا ﴿٣٨﴾\\
\textamh{39.\  } & وَلَولَآ إِذ دَخَلتَ جَنَّتَكَ قُلتَ مَا شَآءَ ٱللَّهُ لَا قُوَّةَ إِلَّا بِٱللَّهِ ۚ إِن تَرَنِ أَنَا۠ أَقَلَّ مِنكَ مَالًۭا وَوَلَدًۭا ﴿٣٩﴾\\
\textamh{40.\  } & فَعَسَىٰ رَبِّىٓ أَن يُؤتِيَنِ خَيرًۭا مِّن جَنَّتِكَ وَيُرسِلَ عَلَيهَا حُسبَانًۭا مِّنَ ٱلسَّمَآءِ فَتُصبِحَ صَعِيدًۭا زَلَقًا ﴿٤٠﴾\\
\textamh{41.\  } & أَو يُصبِحَ مَآؤُهَا غَورًۭا فَلَن تَستَطِيعَ لَهُۥ طَلَبًۭا ﴿٤١﴾\\
\textamh{42.\  } & وَأُحِيطَ بِثَمَرِهِۦ فَأَصبَحَ يُقَلِّبُ كَفَّيهِ عَلَىٰ مَآ أَنفَقَ فِيهَا وَهِىَ خَاوِيَةٌ عَلَىٰ عُرُوشِهَا وَيَقُولُ يَـٰلَيتَنِى لَم أُشرِك بِرَبِّىٓ أَحَدًۭا ﴿٤٢﴾\\
\textamh{43.\  } & وَلَم تَكُن لَّهُۥ فِئَةٌۭ يَنصُرُونَهُۥ مِن دُونِ ٱللَّهِ وَمَا كَانَ مُنتَصِرًا ﴿٤٣﴾\\
\textamh{44.\  } & هُنَالِكَ ٱلوَلَـٰيَةُ لِلَّهِ ٱلحَقِّ ۚ هُوَ خَيرٌۭ ثَوَابًۭا وَخَيرٌ عُقبًۭا ﴿٤٤﴾\\
\textamh{45.\  } & وَٱضرِب لَهُم مَّثَلَ ٱلحَيَوٰةِ ٱلدُّنيَا كَمَآءٍ أَنزَلنَـٰهُ مِنَ ٱلسَّمَآءِ فَٱختَلَطَ بِهِۦ نَبَاتُ ٱلأَرضِ فَأَصبَحَ هَشِيمًۭا تَذرُوهُ ٱلرِّيَـٰحُ ۗ وَكَانَ ٱللَّهُ عَلَىٰ كُلِّ شَىءٍۢ مُّقتَدِرًا ﴿٤٥﴾\\
\textamh{46.\  } & ٱلمَالُ وَٱلبَنُونَ زِينَةُ ٱلحَيَوٰةِ ٱلدُّنيَا ۖ وَٱلبَٰقِيَـٰتُ ٱلصَّـٰلِحَـٰتُ خَيرٌ عِندَ رَبِّكَ ثَوَابًۭا وَخَيرٌ أَمَلًۭا ﴿٤٦﴾\\
\textamh{47.\  } & وَيَومَ نُسَيِّرُ ٱلجِبَالَ وَتَرَى ٱلأَرضَ بَارِزَةًۭ وَحَشَرنَـٰهُم فَلَم نُغَادِر مِنهُم أَحَدًۭا ﴿٤٧﴾\\
\textamh{48.\  } & وَعُرِضُوا۟ عَلَىٰ رَبِّكَ صَفًّۭا لَّقَد جِئتُمُونَا كَمَا خَلَقنَـٰكُم أَوَّلَ مَرَّةٍۭ ۚ بَل زَعَمتُم أَلَّن نَّجعَلَ لَكُم مَّوعِدًۭا ﴿٤٨﴾\\
\textamh{49.\  } & وَوُضِعَ ٱلكِتَـٰبُ فَتَرَى ٱلمُجرِمِينَ مُشفِقِينَ مِمَّا فِيهِ وَيَقُولُونَ يَـٰوَيلَتَنَا مَالِ هَـٰذَا ٱلكِتَـٰبِ لَا يُغَادِرُ صَغِيرَةًۭ وَلَا كَبِيرَةً إِلَّآ أَحصَىٰهَا ۚ وَوَجَدُوا۟ مَا عَمِلُوا۟ حَاضِرًۭا ۗ وَلَا يَظلِمُ رَبُّكَ أَحَدًۭا ﴿٤٩﴾\\
\textamh{50.\  } & وَإِذ قُلنَا لِلمَلَـٰٓئِكَةِ ٱسجُدُوا۟ لِءَادَمَ فَسَجَدُوٓا۟ إِلَّآ إِبلِيسَ كَانَ مِنَ ٱلجِنِّ فَفَسَقَ عَن أَمرِ رَبِّهِۦٓ ۗ أَفَتَتَّخِذُونَهُۥ وَذُرِّيَّتَهُۥٓ أَولِيَآءَ مِن دُونِى وَهُم لَكُم عَدُوٌّۢ ۚ بِئسَ لِلظَّـٰلِمِينَ بَدَلًۭا ﴿٥٠﴾\\
\textamh{51.\  } & ۞ مَّآ أَشهَدتُّهُم خَلقَ ٱلسَّمَـٰوَٟتِ وَٱلأَرضِ وَلَا خَلقَ أَنفُسِهِم وَمَا كُنتُ مُتَّخِذَ ٱلمُضِلِّينَ عَضُدًۭا ﴿٥١﴾\\
\textamh{52.\  } & وَيَومَ يَقُولُ نَادُوا۟ شُرَكَآءِىَ ٱلَّذِينَ زَعَمتُم فَدَعَوهُم فَلَم يَستَجِيبُوا۟ لَهُم وَجَعَلنَا بَينَهُم مَّوبِقًۭا ﴿٥٢﴾\\
\textamh{53.\  } & وَرَءَا ٱلمُجرِمُونَ ٱلنَّارَ فَظَنُّوٓا۟ أَنَّهُم مُّوَاقِعُوهَا وَلَم يَجِدُوا۟ عَنهَا مَصرِفًۭا ﴿٥٣﴾\\
\textamh{54.\  } & وَلَقَد صَرَّفنَا فِى هَـٰذَا ٱلقُرءَانِ لِلنَّاسِ مِن كُلِّ مَثَلٍۢ ۚ وَكَانَ ٱلإِنسَـٰنُ أَكثَرَ شَىءٍۢ جَدَلًۭا ﴿٥٤﴾\\
\textamh{55.\  } & وَمَا مَنَعَ ٱلنَّاسَ أَن يُؤمِنُوٓا۟ إِذ جَآءَهُمُ ٱلهُدَىٰ وَيَستَغفِرُوا۟ رَبَّهُم إِلَّآ أَن تَأتِيَهُم سُنَّةُ ٱلأَوَّلِينَ أَو يَأتِيَهُمُ ٱلعَذَابُ قُبُلًۭا ﴿٥٥﴾\\
\textamh{56.\  } & وَمَا نُرسِلُ ٱلمُرسَلِينَ إِلَّا مُبَشِّرِينَ وَمُنذِرِينَ ۚ وَيُجَٰدِلُ ٱلَّذِينَ كَفَرُوا۟ بِٱلبَٰطِلِ لِيُدحِضُوا۟ بِهِ ٱلحَقَّ ۖ وَٱتَّخَذُوٓا۟ ءَايَـٰتِى وَمَآ أُنذِرُوا۟ هُزُوًۭا ﴿٥٦﴾\\
\textamh{57.\  } & وَمَن أَظلَمُ مِمَّن ذُكِّرَ بِـَٔايَـٰتِ رَبِّهِۦ فَأَعرَضَ عَنهَا وَنَسِىَ مَا قَدَّمَت يَدَاهُ ۚ إِنَّا جَعَلنَا عَلَىٰ قُلُوبِهِم أَكِنَّةً أَن يَفقَهُوهُ وَفِىٓ ءَاذَانِهِم وَقرًۭا ۖ وَإِن تَدعُهُم إِلَى ٱلهُدَىٰ فَلَن يَهتَدُوٓا۟ إِذًا أَبَدًۭا ﴿٥٧﴾\\
\textamh{58.\  } & وَرَبُّكَ ٱلغَفُورُ ذُو ٱلرَّحمَةِ ۖ لَو يُؤَاخِذُهُم بِمَا كَسَبُوا۟ لَعَجَّلَ لَهُمُ ٱلعَذَابَ ۚ بَل لَّهُم مَّوعِدٌۭ لَّن يَجِدُوا۟ مِن دُونِهِۦ مَوئِلًۭا ﴿٥٨﴾\\
\textamh{59.\  } & وَتِلكَ ٱلقُرَىٰٓ أَهلَكنَـٰهُم لَمَّا ظَلَمُوا۟ وَجَعَلنَا لِمَهلِكِهِم مَّوعِدًۭا ﴿٥٩﴾\\
\textamh{60.\  } & وَإِذ قَالَ مُوسَىٰ لِفَتَىٰهُ لَآ أَبرَحُ حَتَّىٰٓ أَبلُغَ مَجمَعَ ٱلبَحرَينِ أَو أَمضِىَ حُقُبًۭا ﴿٦٠﴾\\
\textamh{61.\  } & فَلَمَّا بَلَغَا مَجمَعَ بَينِهِمَا نَسِيَا حُوتَهُمَا فَٱتَّخَذَ سَبِيلَهُۥ فِى ٱلبَحرِ سَرَبًۭا ﴿٦١﴾\\
\textamh{62.\  } & فَلَمَّا جَاوَزَا قَالَ لِفَتَىٰهُ ءَاتِنَا غَدَآءَنَا لَقَد لَقِينَا مِن سَفَرِنَا هَـٰذَا نَصَبًۭا ﴿٦٢﴾\\
\textamh{63.\  } & قَالَ أَرَءَيتَ إِذ أَوَينَآ إِلَى ٱلصَّخرَةِ فَإِنِّى نَسِيتُ ٱلحُوتَ وَمَآ أَنسَىٰنِيهُ إِلَّا ٱلشَّيطَٰنُ أَن أَذكُرَهُۥ ۚ وَٱتَّخَذَ سَبِيلَهُۥ فِى ٱلبَحرِ عَجَبًۭا ﴿٦٣﴾\\
\textamh{64.\  } & قَالَ ذَٟلِكَ مَا كُنَّا نَبغِ ۚ فَٱرتَدَّا عَلَىٰٓ ءَاثَارِهِمَا قَصَصًۭا ﴿٦٤﴾\\
\textamh{65.\  } & فَوَجَدَا عَبدًۭا مِّن عِبَادِنَآ ءَاتَينَـٰهُ رَحمَةًۭ مِّن عِندِنَا وَعَلَّمنَـٰهُ مِن لَّدُنَّا عِلمًۭا ﴿٦٥﴾\\
\textamh{66.\  } & قَالَ لَهُۥ مُوسَىٰ هَل أَتَّبِعُكَ عَلَىٰٓ أَن تُعَلِّمَنِ مِمَّا عُلِّمتَ رُشدًۭا ﴿٦٦﴾\\
\textamh{67.\  } & قَالَ إِنَّكَ لَن تَستَطِيعَ مَعِىَ صَبرًۭا ﴿٦٧﴾\\
\textamh{68.\  } & وَكَيفَ تَصبِرُ عَلَىٰ مَا لَم تُحِط بِهِۦ خُبرًۭا ﴿٦٨﴾\\
\textamh{69.\  } & قَالَ سَتَجِدُنِىٓ إِن شَآءَ ٱللَّهُ صَابِرًۭا وَلَآ أَعصِى لَكَ أَمرًۭا ﴿٦٩﴾\\
\textamh{70.\  } & قَالَ فَإِنِ ٱتَّبَعتَنِى فَلَا تَسـَٔلنِى عَن شَىءٍ حَتَّىٰٓ أُحدِثَ لَكَ مِنهُ ذِكرًۭا ﴿٧٠﴾\\
\textamh{71.\  } & فَٱنطَلَقَا حَتَّىٰٓ إِذَا رَكِبَا فِى ٱلسَّفِينَةِ خَرَقَهَا ۖ قَالَ أَخَرَقتَهَا لِتُغرِقَ أَهلَهَا لَقَد جِئتَ شَيـًٔا إِمرًۭا ﴿٧١﴾\\
\textamh{72.\  } & قَالَ أَلَم أَقُل إِنَّكَ لَن تَستَطِيعَ مَعِىَ صَبرًۭا ﴿٧٢﴾\\
\textamh{73.\  } & قَالَ لَا تُؤَاخِذنِى بِمَا نَسِيتُ وَلَا تُرهِقنِى مِن أَمرِى عُسرًۭا ﴿٧٣﴾\\
\textamh{74.\  } & فَٱنطَلَقَا حَتَّىٰٓ إِذَا لَقِيَا غُلَـٰمًۭا فَقَتَلَهُۥ قَالَ أَقَتَلتَ نَفسًۭا زَكِيَّةًۢ بِغَيرِ نَفسٍۢ لَّقَد جِئتَ شَيـًۭٔا نُّكرًۭا ﴿٧٤﴾\\
\textamh{75.\  } & ۞ قَالَ أَلَم أَقُل لَّكَ إِنَّكَ لَن تَستَطِيعَ مَعِىَ صَبرًۭا ﴿٧٥﴾\\
\textamh{76.\  } & قَالَ إِن سَأَلتُكَ عَن شَىءٍۭ بَعدَهَا فَلَا تُصَـٰحِبنِى ۖ قَد بَلَغتَ مِن لَّدُنِّى عُذرًۭا ﴿٧٦﴾\\
\textamh{77.\  } & فَٱنطَلَقَا حَتَّىٰٓ إِذَآ أَتَيَآ أَهلَ قَريَةٍ ٱستَطعَمَآ أَهلَهَا فَأَبَوا۟ أَن يُضَيِّفُوهُمَا فَوَجَدَا فِيهَا جِدَارًۭا يُرِيدُ أَن يَنقَضَّ فَأَقَامَهُۥ ۖ قَالَ لَو شِئتَ لَتَّخَذتَ عَلَيهِ أَجرًۭا ﴿٧٧﴾\\
\textamh{78.\  } & قَالَ هَـٰذَا فِرَاقُ بَينِى وَبَينِكَ ۚ سَأُنَبِّئُكَ بِتَأوِيلِ مَا لَم تَستَطِع عَّلَيهِ صَبرًا ﴿٧٨﴾\\
\textamh{79.\  } & أَمَّا ٱلسَّفِينَةُ فَكَانَت لِمَسَـٰكِينَ يَعمَلُونَ فِى ٱلبَحرِ فَأَرَدتُّ أَن أَعِيبَهَا وَكَانَ وَرَآءَهُم مَّلِكٌۭ يَأخُذُ كُلَّ سَفِينَةٍ غَصبًۭا ﴿٧٩﴾\\
\textamh{80.\  } & وَأَمَّا ٱلغُلَـٰمُ فَكَانَ أَبَوَاهُ مُؤمِنَينِ فَخَشِينَآ أَن يُرهِقَهُمَا طُغيَـٰنًۭا وَكُفرًۭا ﴿٨٠﴾\\
\textamh{81.\  } & فَأَرَدنَآ أَن يُبدِلَهُمَا رَبُّهُمَا خَيرًۭا مِّنهُ زَكَوٰةًۭ وَأَقرَبَ رُحمًۭا ﴿٨١﴾\\
\textamh{82.\  } & وَأَمَّا ٱلجِدَارُ فَكَانَ لِغُلَـٰمَينِ يَتِيمَينِ فِى ٱلمَدِينَةِ وَكَانَ تَحتَهُۥ كَنزٌۭ لَّهُمَا وَكَانَ أَبُوهُمَا صَـٰلِحًۭا فَأَرَادَ رَبُّكَ أَن يَبلُغَآ أَشُدَّهُمَا وَيَستَخرِجَا كَنزَهُمَا رَحمَةًۭ مِّن رَّبِّكَ ۚ وَمَا فَعَلتُهُۥ عَن أَمرِى ۚ ذَٟلِكَ تَأوِيلُ مَا لَم تَسطِع عَّلَيهِ صَبرًۭا ﴿٨٢﴾\\
\textamh{83.\  } & وَيَسـَٔلُونَكَ عَن ذِى ٱلقَرنَينِ ۖ قُل سَأَتلُوا۟ عَلَيكُم مِّنهُ ذِكرًا ﴿٨٣﴾\\
\textamh{84.\  } & إِنَّا مَكَّنَّا لَهُۥ فِى ٱلأَرضِ وَءَاتَينَـٰهُ مِن كُلِّ شَىءٍۢ سَبَبًۭا ﴿٨٤﴾\\
\textamh{85.\  } & فَأَتبَعَ سَبَبًا ﴿٨٥﴾\\
\textamh{86.\  } & حَتَّىٰٓ إِذَا بَلَغَ مَغرِبَ ٱلشَّمسِ وَجَدَهَا تَغرُبُ فِى عَينٍ حَمِئَةٍۢ وَوَجَدَ عِندَهَا قَومًۭا ۗ قُلنَا يَـٰذَا ٱلقَرنَينِ إِمَّآ أَن تُعَذِّبَ وَإِمَّآ أَن تَتَّخِذَ فِيهِم حُسنًۭا ﴿٨٦﴾\\
\textamh{87.\  } & قَالَ أَمَّا مَن ظَلَمَ فَسَوفَ نُعَذِّبُهُۥ ثُمَّ يُرَدُّ إِلَىٰ رَبِّهِۦ فَيُعَذِّبُهُۥ عَذَابًۭا نُّكرًۭا ﴿٨٧﴾\\
\textamh{88.\  } & وَأَمَّا مَن ءَامَنَ وَعَمِلَ صَـٰلِحًۭا فَلَهُۥ جَزَآءً ٱلحُسنَىٰ ۖ وَسَنَقُولُ لَهُۥ مِن أَمرِنَا يُسرًۭا ﴿٨٨﴾\\
\textamh{89.\  } & ثُمَّ أَتبَعَ سَبَبًا ﴿٨٩﴾\\
\textamh{90.\  } & حَتَّىٰٓ إِذَا بَلَغَ مَطلِعَ ٱلشَّمسِ وَجَدَهَا تَطلُعُ عَلَىٰ قَومٍۢ لَّم نَجعَل لَّهُم مِّن دُونِهَا سِترًۭا ﴿٩٠﴾\\
\textamh{91.\  } & كَذَٟلِكَ وَقَد أَحَطنَا بِمَا لَدَيهِ خُبرًۭا ﴿٩١﴾\\
\textamh{92.\  } & ثُمَّ أَتبَعَ سَبَبًا ﴿٩٢﴾\\
\textamh{93.\  } & حَتَّىٰٓ إِذَا بَلَغَ بَينَ ٱلسَّدَّينِ وَجَدَ مِن دُونِهِمَا قَومًۭا لَّا يَكَادُونَ يَفقَهُونَ قَولًۭا ﴿٩٣﴾\\
\textamh{94.\  } & قَالُوا۟ يَـٰذَا ٱلقَرنَينِ إِنَّ يَأجُوجَ وَمَأجُوجَ مُفسِدُونَ فِى ٱلأَرضِ فَهَل نَجعَلُ لَكَ خَرجًا عَلَىٰٓ أَن تَجعَلَ بَينَنَا وَبَينَهُم سَدًّۭا ﴿٩٤﴾\\
\textamh{95.\  } & قَالَ مَا مَكَّنِّى فِيهِ رَبِّى خَيرٌۭ فَأَعِينُونِى بِقُوَّةٍ أَجعَل بَينَكُم وَبَينَهُم رَدمًا ﴿٩٥﴾\\
\textamh{96.\  } & ءَاتُونِى زُبَرَ ٱلحَدِيدِ ۖ حَتَّىٰٓ إِذَا سَاوَىٰ بَينَ ٱلصَّدَفَينِ قَالَ ٱنفُخُوا۟ ۖ حَتَّىٰٓ إِذَا جَعَلَهُۥ نَارًۭا قَالَ ءَاتُونِىٓ أُفرِغ عَلَيهِ قِطرًۭا ﴿٩٦﴾\\
\textamh{97.\  } & فَمَا ٱسطَٰعُوٓا۟ أَن يَظهَرُوهُ وَمَا ٱستَطَٰعُوا۟ لَهُۥ نَقبًۭا ﴿٩٧﴾\\
\textamh{98.\  } & قَالَ هَـٰذَا رَحمَةٌۭ مِّن رَّبِّى ۖ فَإِذَا جَآءَ وَعدُ رَبِّى جَعَلَهُۥ دَكَّآءَ ۖ وَكَانَ وَعدُ رَبِّى حَقًّۭا ﴿٩٨﴾\\
\textamh{99.\  } & ۞ وَتَرَكنَا بَعضَهُم يَومَئِذٍۢ يَمُوجُ فِى بَعضٍۢ ۖ وَنُفِخَ فِى ٱلصُّورِ فَجَمَعنَـٰهُم جَمعًۭا ﴿٩٩﴾\\
\textamh{100.\  } & وَعَرَضنَا جَهَنَّمَ يَومَئِذٍۢ لِّلكَـٰفِرِينَ عَرضًا ﴿١٠٠﴾\\
\textamh{101.\  } & ٱلَّذِينَ كَانَت أَعيُنُهُم فِى غِطَآءٍ عَن ذِكرِى وَكَانُوا۟ لَا يَستَطِيعُونَ سَمعًا ﴿١٠١﴾\\
\textamh{102.\  } & أَفَحَسِبَ ٱلَّذِينَ كَفَرُوٓا۟ أَن يَتَّخِذُوا۟ عِبَادِى مِن دُونِىٓ أَولِيَآءَ ۚ إِنَّآ أَعتَدنَا جَهَنَّمَ لِلكَـٰفِرِينَ نُزُلًۭا ﴿١٠٢﴾\\
\textamh{103.\  } & قُل هَل نُنَبِّئُكُم بِٱلأَخسَرِينَ أَعمَـٰلًا ﴿١٠٣﴾\\
\textamh{104.\  } & ٱلَّذِينَ ضَلَّ سَعيُهُم فِى ٱلحَيَوٰةِ ٱلدُّنيَا وَهُم يَحسَبُونَ أَنَّهُم يُحسِنُونَ صُنعًا ﴿١٠٤﴾\\
\textamh{105.\  } & أُو۟لَـٰٓئِكَ ٱلَّذِينَ كَفَرُوا۟ بِـَٔايَـٰتِ رَبِّهِم وَلِقَآئِهِۦ فَحَبِطَت أَعمَـٰلُهُم فَلَا نُقِيمُ لَهُم يَومَ ٱلقِيَـٰمَةِ وَزنًۭا ﴿١٠٥﴾\\
\textamh{106.\  } & ذَٟلِكَ جَزَآؤُهُم جَهَنَّمُ بِمَا كَفَرُوا۟ وَٱتَّخَذُوٓا۟ ءَايَـٰتِى وَرُسُلِى هُزُوًا ﴿١٠٦﴾\\
\textamh{107.\  } & إِنَّ ٱلَّذِينَ ءَامَنُوا۟ وَعَمِلُوا۟ ٱلصَّـٰلِحَـٰتِ كَانَت لَهُم جَنَّـٰتُ ٱلفِردَوسِ نُزُلًا ﴿١٠٧﴾\\
\textamh{108.\  } & خَـٰلِدِينَ فِيهَا لَا يَبغُونَ عَنهَا حِوَلًۭا ﴿١٠٨﴾\\
\textamh{109.\  } & قُل لَّو كَانَ ٱلبَحرُ مِدَادًۭا لِّكَلِمَـٰتِ رَبِّى لَنَفِدَ ٱلبَحرُ قَبلَ أَن تَنفَدَ كَلِمَـٰتُ رَبِّى وَلَو جِئنَا بِمِثلِهِۦ مَدَدًۭا ﴿١٠٩﴾\\
\textamh{110.\  } & قُل إِنَّمَآ أَنَا۠ بَشَرٌۭ مِّثلُكُم يُوحَىٰٓ إِلَىَّ أَنَّمَآ إِلَـٰهُكُم إِلَـٰهٌۭ وَٟحِدٌۭ ۖ فَمَن كَانَ يَرجُوا۟ لِقَآءَ رَبِّهِۦ فَليَعمَل عَمَلًۭا صَـٰلِحًۭا وَلَا يُشرِك بِعِبَادَةِ رَبِّهِۦٓ أَحَدًۢا ﴿١١٠﴾\\
\end{longtable} \newpage
