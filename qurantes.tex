%% License: BSD style (Berkley) (i.e. Put the Copyright owner}s name always)
%% Writer and Copyright (to): Bewketu(Bilal) Tadilo (2016-17)
\documentclass[11pt,a4paper,oneside,arabic]{l3doc}%, ,fleqn oneside]
\usepackage[bottom=2cm, top=2cm, right=2cm, text={7in,10in}, centering]{geometry}
\usepackage{graphicx}
\usepackage{fontspec}

\usepackage{longtable}
\usepackage{tabu}
\usepackage{arabicnumbers}
\usepackage[RTLdocument]{bidi}

\newfontfamily\arabicfont[Script=Arabic, Numbers=Arabic, Scale = 1.5]{mequran}%KFGQPC Uthmanic Script HAFS}% Taha Naskh}
\newfontfamily\amharicfont[Script = Amharic, Scale = 1.1]{Abyssinica SIL}
\setmainfont[Script=Arabic, Scale=1.5, Numbers=UpperCase, Language=arabic]{mequran}
\newcommand{\unicodefont}[1]{{\fontfamily {TeX Gyre Pagella}\selectfont#1}\arabicfont}
%\SetAllahWithAlif

\newcommand{\textamh}[1]{\noindent\raggedright\LR{\amharicfont #1}}
\newcommand{\textarabb}[1]{\arabicfont #1}
\pagestyle{fancy}
\renewcommand{\footrule}{\color{black}7pt}
\renewcommand{\footrulewidth}{0.6pt}
\renewcommand{\footrule}{{\color{gray}%
\vskip-\footruleskip\vskip-\footrulewidth
\hrule width\textwidth  height\footrulewidth\vskip\footruleskip}}
\fancypagestyle{plain}{
\fancyhf{}
\noindent
\fancyhead[LO,RE]{
\thisfancyput(3.1in,-4.5in){%
\setlength{\unitlength}{1in}
\fancyoval(7.7,9.9)}
%\put(3.25in, -4.4in){\line(0,1){0.8\paperheight}}
%\put(3.25in, -4.2in){\line(0,1){0.8\paperheight}}
}

%\fancyfoot[RE,LO]{\put(.8\textwidth,.2\textheight){\large\thepage}}
\fancyfoot[LO]{\large\unicodefont{\thepage}\vskip-1.1cm\hskip-0.6cm}
\fancyfoot[RE]{\vskip-1.1cm\hskip-0.6cm\large\unicodefont{\thepage}}
%\DeclareFixedFont{\Oarb}{LAE}{OmegaArabicBold}{m}{n}{8mm}
\begin{document}
%\textit{\unicodefont{\secnumdepth}}
%\textit{\unicodefont{\thepage}}
%\newbox\mybox
{
%  \parindent 0pt
  \null
  \colorlet{mintgreen}{green!50!black!50}
\def\nodeshadowed[#1]#2;{\node[scale=2,above,#1]{\global\setbox\mybox=\hbox{#2}\copy\mybox};
      \node[scale=2,above,#1,yscale=-1,scope fading=south,opacity=0.4]{\box\mybox};}

  \thispagestyle{empty}
  \vskip 3cm
  \vfill
  \hfil
  \begin{tikzpicture}[overlay]
    \coordinate (front) at (0,0);
    \coordinate (horizon) at (0,.31\paperheight);
    \coordinate (bottom) at (0,-.6\paperheight);
    \coordinate (sky) at (0,.57\paperheight);
    \coordinate (left) at (-.1\paperwidth,0);
    \coordinate (right) at (0.98\paperwidth,0);

    \shade [bottom color=blue!30!black!10,top color=blue!30!black!50]
      ([yshift=-5mm]horizon -|  left) rectangle (sky -| right);
    \shade [bottom color=black!70!green!25,top color=black!70!green!10]
      (front -| left) -- (horizon -| left)
      decorate [decoration=random steps] { -- (horizon -| right) }
      -- (front -| right) -- cycle;
    \shade [top color=black!70!green!25,bottom color=black!25]
      ([yshift=-5mm-1pt]front -| left) rectangle ([yshift=1pt]front -| right);
    \fill [black!25] (bottom -| left) rectangle ([yshift=-5mm]front -| right);

%    \def\nodeshadowed[#1]#2;
%{\node[scale=2,above,#1]{\global\setbox\mybox=\hbox{#2}\copy\mybox};
%      \node[scale=2,above,#1,yscale=-1,scope fading=south,opacity=0.4]{\box\mybox};};
	 	 \nodeshadowed[at={(1.7cm,0.8cm)}]{\tikz \draw[green!20!black, rotate=90]
    [l-system={rule set={F -> FF-[-F+F]+[+F-F]}, axiom=F, order=4,
      step=2pt, randomize step percent=50, angle=30, randomize angle percent=5}]
 lindenmayer system;};
\foreach \j in {2cm,16cm}
	 \nodeshadowed[at={(\j,1cm)}]{\tikz \draw[green!20!black, rotate=90]
    [l-system={rule set={F -> FF-[-F+F]+[+F-F]}, axiom=F, order=4,
      step=2pt, randomize step percent=50, angle=30, randomize angle percent=5}]
 lindenmayer system;};



    \nodeshadowed [at={(10,6.8)}] {\Huge \textcolor{mintgreen}{كريم ٱلقرءان}};
    \nodeshadowed [at={(6,10  )},yslant=0.05] {\textamharic{\Huge\textcolor{orange}{ቁርኣን}}};
    %\nodeshadowed [at={( 0,5.3)}] {\huge \textcolor{mintgreen}{\&}};
    \nodeshadowed [at={(12,10  )},yslant=-0.05] {\textamharic{\Huge\textcolor{orange}{ከሪም}}};
    \nodeshadowed [at={(9,1.5)}] {\large \textamharic{\textcolor{brown}{የቁርኣን ከሪም የአማርኛ ትርጉም}}};
    \nodeshadowed [at={( 9,-6  )}] {\textamharic{በ((x-በዉቀቱ)\large ቢላል ታድሎ)}\huge تادلو بلال};
               \foreach \i in {0.5,0.6,...,2}
      \fill [white,decoration=Koch snowflake,opacity=.9]
            [shift=(horizon),shift={(rand*11,rnd*7)},scale=\i]
            [double copy shadow={opacity=0.2,shadow xshift=0pt,shadow
              yshift=3*\i pt,fill=white,draw=none}]
        decorate {
          decorate {
            decorate {
              (0,0) -- ++(60:1) -- ++(-60:1) -- cycle
            }
          }
        };

  \end{tikzpicture}
\vfill
\vbox{}
\clearpage
}


%\Hijritoday[0] 
%\textamh{\today-} 
%\pagenumbering{}
\tableofcontents{}
%\%%%%%%%%%%%%%%%%%%%%%%%%
%TODO: please substitute alif which is broken by      ـٰ
%%%%%%%%%%%%%%%%%%%%%%
\cleardoublepage
\pagestyle{plain}  

{\arabicfont\section{\RL{{\LR{{\amharicfont ሱራቱ አልፈቲሃ - }} \RL{سوره  الفاتحة}}}}}
\noindent\begin{longtable}{%
  @{}
  p{0.5\textwidth}
  @{~~~~~~~~~~~~}
	||
  p{0.5\textwidth}
    @{}
}
\textamh{1.\ ቢስሚላሂ አራህመኒ ራሂይም   } & \textarabb{ بِسْمِ ٱللَّهِ الرَّحْمَـٰنِ الرَّحِيمِ﴿١﴾}\\
\textamh{2.\ (ኣልሃምዱሊላሂ) ምስጋና ሁሉ ለኣላህ የአለሚን (የሰዎች፥ ጅኖች፥ ያለ ነገር ሁሉ) ጌታ   } & \textarabb{ ٱلْحَمْدُ لِلَّهِ رَبِّ ٱلْعَـٰلَمِينَ﴿٢﴾ }\\
\textamh{3.\ ከሁሉም በላይ ሰጪ፥ ከሁሉም በላይ ምህረተኛው   } & \textarabb{ٱلرَّحْمَـٰنِ ٱلرَّحِيمِ﴿٣﴾  } \\
\textamh{4.\ የዛች ቀን (የፍርድ ቀን) ብቸኛ ባለቤት   } &   \textarabb{مَـٰلِكِ يَوْمِ ٱلدِّينِ ﴿٤﴾ }  \\
\textamh{5.\ አንተን ብቻ እናመልካለን፤ አንተን ብቻ እርዳታ እንጠይቃለን   } & \textarabb{ إِيَّاكَ نَعْبُدُ وَإِيَّاكَ نَسْتَعِينُ ﴿٥﴾  } \\
\textamh{6.\ ምራነ በቀጥተኛው (በትክክለኛው)  መንገድ   } & \textarabb{ٱهْدِنَا ٱلصِّرَٟطَ ٱلْمُسْتَقِيمَ ﴿٦﴾}  \\
\textamh{7.\ የአንተን ፀጋ ያደረግክላቸውን (ሰዎች)  መንገድ፥ የአንተን ቁጣ እንዳተርፉት (እንደይሁዶች) ሳይሆን ፥እንደሳቱትም (እንደክርስቲያኖች) ሳይሆን } &  \textarabb{ صِرَٟطَ ٱلَّذِينَ أَنْعَمْتَ عَلَيْهِمْ غَيْرِ ٱلْمَغْضُوبِ عَلَيْهِمْ وَلَا ٱلضَّآلِّينَ ﴿٧﴾ }
\end{longtable}
%ــٰ
%\shadowbox{\begin{minipage}{1\columnwidth}
%\end{minipage}}
{\arabicfont\section{{\textRL{\amharicfont ሱራቱ አልበቀራ -} }  سوره  البقرة }}
\begin{longtabu}{
  @{}
    X[4,l,p]
  @{~~~~~}||
    X[4,r,p]
    %p{.5\textwidth}
    @{}
}
\nopagebreak
\textamh{ቢስሚላሂ አራህመኒ ራሂይም   } &  بِسْمِ ٱللَّهِ ٱلرَّحْمَـٰنِ ٱلرَّحِيمِ     \\
\textamh{1.\ አሊፍ ላም ሚም (የፊደላቱን ትርጉም ኣላህ ብቻ ያዉቃል)  } &  الٓمٓ ﴿١﴾  \\
\textamh{2.\ ይሄ ነው መጽሃፉ፥ምንም ጥርጥር የሌለበት፤ አምላክን ለሚፈሩ መሪ የሆነ።   } &  ذَٟلِكَ ٱلْكِتَـٰبُ لَا رَيْبَ ۛ فِيهِ ۛ هُدًۭى لِّلْمُتَّقِينَ ﴿٢﴾  \\
\textamh{3.\ በማይታየው(ጋይብ) የሚያምኑ፥ሳላት የሚቆሙ የሰጠናቸዉን (የረዘቅናቸውን) የሚሰጡ   } & ٱلَّذِينَ يُؤْمِنُونَ بِٱلْغَيْبِ وَيُقِيمُونَ ٱلصَّلَوٰةَ وَمِمَّا رَزَقْنَـٰهُمْ يُنفِقُونَ ﴿٣﴾  \\
\textamh{4.\ ለአንተ በወርደው (ኦ! ሙሐመድ(ሠአወሰ)) (በዚህ ቁርአን) የሚያምኑ፤ ደግሞ ከአንተ በፊት በወረደላቸው (ተውራት፥ ወንጌል) እና በሰማያዊ ህይወት (አኪራ) ያለ ምንም ጥርጥር የሚያምኑ   } &  وَٱلَّذِينَ يُؤْمِنُونَ بِمَآ أُنزِلَ إِلَيْكَ وَمَآ أُنزِلَ مِن قَبْلِكَ وَبِٱلْءَاخِرَةِ هُمْ يُوقِنُونَ ﴿٤﴾  \\
\textamh{5.\ እነሱ ናቸው ከአምላካቸው ምሬት ያገኙ (የተመሩ) እነሱም ናቸው (በስኬት) አላፊዎች   } &  أُو۟لَـٰٓئِكَ عَلَىٰ هُدًۭى مِّن رَّبِّهِمْ ۖ وَأُو۟لَـٰٓئِكَ هُمُ ٱلْمُفْلِحُونَ ﴿٥﴾   \\
\textamh{6.\ በእውነት ለእነዚያ ለማይምኑት (ካፊሮች) (ኦ! ሙሐመድ(ሠአወሰ))ብታስጠነቅቃቸውም ባታስጠነቅቃቸውም አንድ ነው፤ አያምኑም።   } &  إِنَّ ٱلَّذِينَ كَفَرُوا۟ سَوَآءٌ عَلَيْهِمْ ءَأَنذَرْتَهُمْ أَمْ لَمْ تُنذِرْهُمْ لَا يُؤْمِنُونَ ﴿٦﴾  \\
\textamh{7.\ ኣላህ ልባቸዉን አትሞታል መስሚያቸውንም እንደዚያው፥ ማያቸው ላይ ግርዶሽ አለ፤ ለነሱ ታላቅ ቅጣት ይጠብቃቸዋል።   } &  خَتَمَ ٱللَّهُ عَلَىٰ قُلُوبِهِمْ وَعَلَىٰ سَمْعِهِمْ ۖ وَعَلَىٰٓ أَبْصَـٰرِهِمْ غِشَـٰوَةٌۭ ۖ وَلَهُمْ عَذَابٌ عَظِيمٌۭ ﴿٧﴾  \\
\textamh{8.\ ከሰዎች መካከል ደግሞ በኣላህ እና በፍርድ ቀን (የውሚ አኪራ) እናምናለን  የሚሉ አሉ፤ ግን አማኞች አይደሉም።   } &  وَمِنَ ٱلنَّاسِ مَن يَقُولُ ءَامَنَّا بِٱللَّهِ وَبِٱلْيَوْمِ ٱلْءَاخِرِ وَمَا هُم بِمُؤْمِنِينَ ﴿٨﴾  \\
\textamh{9.\ ኣላህንና አማኞችን ሊያጭበረብሩ (ያስባሉ)፤  ከራሳቸው በቀር ማንንም አያጭበረብሩም፤ ግን አያውቁትም።   } &  يُخَـٰدِعُونَ ٱللَّهَ وَٱلَّذِينَ ءَامَنُوا۟ وَمَا يَخْدَعُونَ إِلَّآ أَنفُسَهُمْ وَمَا يَشْعُرُونَ ﴿٩﴾  \\
\textamh{10.\ ልባቸው ዉስጥ በሽታ አለ (የጥርጣሬና የንፍቀት) ኣላህም በሽታቸውን ጨምሮበታል፤ አሰቃቂ ስቃይ ለነሱ ይሆናል ሃሰት ሲናገሩ ስለቆዩ } & 
\  فِى قُلُوبِهِم مَّرَضٌۭ فَزَادَهُمُ ٱللَّهُ مَرَضًۭا ۖ وَلَهُمْ عَذَابٌ أَلِيمٌۢ بِمَا كَانُوا۟ يَكْذِبُونَ ﴿١٠﴾  \\
\textamh{11.\ \enquote{ምድር (መሬት) ላይ አትበጥብጡ} ሲባሉ፥\enquote{እኛ እኮ ሰላም ፈጣሪዎች ነን} ይላሉ   } &  وَإِذَا قِيلَ لَهُمْ لَا تُفْسِدُوا۟ فِى ٱلْأَرْضِ قَالُوٓا۟ إِنَّمَا نَحْنُ مُصْلِحُونَ ﴿١١﴾  \\
\textamh{12.\ በእዉነት! ራሳቸው ናቸው በጥባጮቹ ግን አያዉቁትም።   } &  أَلَآ إِنَّهُمْ هُمُ ٱلْمُفْسِدُونَ وَلَٟكِن لَّا يَشْعُرُونَ ﴿١٢﴾ \\
\textamh{13.\ \enquote{እመኑ ልክ እንደአማኞቹ ሰዎች} ሲባሉ፥\enquote{ሞኞቹ እንዳመኑት እንመን እንዴ?} አሉ። በእዉነት! እነሱው ናቸው ሞኞቹ ግን አያውቁትም።   } &  وَإِذَا قِيلَ لَهُمْ ءَامِنُوا۟ كَمَآ ءَامَنَ ٱلنَّاسُ قَالُوٓا۟ أَنُؤْمِنُ كَمَآ ءَامَنَ ٱلسُّفَهَآءُ ۗ أَلَآ إِنَّهُمْ هُمُ ٱلسُّفَهَآءُ وَلَٟكِن لَّا يَعْلَمُونَ ﴿١٣﴾   \\
\textamh{14.\ አማኞችን ሲያገኙ\enquote{እናምናለን} ይላሉ፤ ነገር ግን ከሰይጣኖቻቸው (ሌሎች መናፍቃን) ጋር ብቻቸውን ሲሆኑ\enquote{በእውነት ከናንት ጋር ነን፤ ስናሾፍ ነው የነበር} ይላሉ።   } &  وَإِذَا لَقُوا۟ ٱلَّذِينَ ءَامَنُوا۟ قَالُوٓا۟ ءَامَنَّا وَإِذَا خَلَوْا۟ إِلَىٰ شَيَـٰطِينِهِمْ قَالُوٓا۟ إِنَّا مَعَكُمْ إِنَّمَا نَحْنُ مُسْتَهْزِءُونَ ﴿١٤﴾ \\
\textamh{15.\ ኣላህ ራሱ ያላግጥባቸዋል፥ እንዲቅበዘበዙ መጥፎ ስራቸዉን ያበዘላቸዋል።  } &  ٱللَّهُ يَسْتَهْزِئُ بِهِمْ وَيَمُدُّهُمْ فِى طُغْيَـٰنِهِمْ يَعْمَهُونَ ﴿١٥﴾\\ 
\textamh{16.\ እነዚህ ናቸው ምሬት (መመራትን) ባለመመራት የገዙት፤ ንግዳቸውም ትርፍ አልባ ሁኖ ቀረ። ሳይመሩ ቀሩ።   } &  أُو۟لَٟٓئِكَ ٱلَّذِينَ ٱشْتَرَوُا۟ ٱلضَّلَٟلَةَ بِٱلْهُدَىٰ فَمَا رَبِحَت تِّجَٟرَتُهُمْ وَمَا كَانُوا۟ مُهْتَدِينَ ﴿١﴾ \\

\textamh{17.\ ምሳሌቸው ልክ እሳት እንዳቃጠለ ሰው ነው፤ ነዶ ብረሃን ሲሆንለት ኣላህ ብረሃናቸዉን ወስዶ ጨለማ ዉስጥ ከተታቸው። ማየት  አይችሉም።
\ } &   مَثَلُهُمْ كَمَثَلِ ٱلَّذِى ٱسْتَوْقَدَ نَارًۭا فَلَمَّآ أَضَآءَتْ مَا حَوْلَهُۥ ذَهَبَ ٱللَّهُ بِنُورِهِمْ وَتَرَكَهُمْ فِى ظُلُمَـٰتٍۢ لَّا يُبْصِرُونَ ﴿١٧﴾\\ 
\textamh{18.\ ደንቆሮ፥ ዲዳ፥ እና እዉር ናቸው፤  አይመለሱም።    } &   صُمٌّۢ بُكْمٌ عُمْىٌۭ فَهُمْ لَا يَرْجِعُونَ ﴿١٨﴾\\
\textamh{19.\ ወይም ደግሞ ልክ እንደ ድቅድቅ ደምና  ዉስጡ ጨለማ፥ ነጎድጓድ (ረአድ)፥በርቅ  (ብልጭታ)ጣታቸዉን ጆሯቸው ዉስጥ  ይከታሉ ከበርቁ ድምጽ የሞት ፍርሃት የተነሳ።  ኣላህ ግን የማይምኑትን አጥሮ ይይዛል።   } &   أَوْ كَصَيِّبٍۢ مِّنَ ٱلسَّمَآءِ فِيهِ ظُلُمَـٰتٌۭ وَرَعْدٌۭ وَبَرْقٌۭ يَجْعَلُونَ أَصَٟبِعَهُمْ فِىٓ ءَاذَانِهِم مِّنَ ٱلصَّوَٟعِقِ حَذَرَ ٱلْمَوْتِ ۚ وَٱللَّهُ مُحِيطٌۢ بِٱلْكَٟفِرِينَ ﴿١٩﴾\\
\textamh{20.\ ብልጭታዉ ማያቸዉን ይወስዳል፥ ሲበራ በዚያ ይሄዳሉ፥ ጨለማ ሲሆን ደግሞ ይቆማሉ፤ ኣላህ ቢፈቅድ ኑሮ መስሚያቸዉንና ማያቸውን ይወስድ ነበር። በእርግጠኛንት ኣላህ ሁሉን ማድረግ ይችላል።   } &  يَكَادُ ٱلْبَرْقُ يَخْطَفُ أَبْصَٟرَهُمْ ۖ كُلَّمَآ أَضَآءَ لَهُم مَّشَوْا۟ فِيهِ وَإِذَآ أَظْلَمَ عَلَيْهِمْ قَامُوا۟ ۚ وَلَوْ شَآءَ ٱللَّهُ لَذَهَبَ بِسَمْعِهِمْ وَأَبْصَٟرِهِمْ ۚ إِنَّ ٱللَّهَ عَلَىٰ كُلِّ شَىْءٍۢ قَدِيرٌۭ ﴿٢٠﴾\\
\textamh{21.\ ኦ! ሰዎች ሆይ፥ አምላካችሁን አምልኩ እናንተንም ሆነ ከናንተ በፊት የነበሩትን የፈጠረ እናንተም ሙታቁን (አምለክ ተገዥ/ፈሪ) እንድትሆኑ።   } &  يَـٰٓأَيُّهَا ٱلنَّاسُ ٱعْبُدُوا۟ رَبَّكُمُ ٱلَّذِى خَلَقَكُمْ وَٱلَّذِينَ مِن قَبْلِكُمْ لَعَلَّكُمْ تَتَّقُونَ ﴿٢١﴾\\
\textamh{22.\ መሬትን (ምድርን) (እንደፍራሽ) ማረፊያ ሰማይን መከለያ ያደረገላችሁ እናም  ከሰማይ ዉሃ አወረደ፥ በዚያም አዝእርትና ፍራፍሬ አበቀለላችሁ ለናንተ ሪዝቅ የሚሆን። ስለዚህ ለኣላህ ሌላ እኩያ አታድርጉ፤ እያወቃችሁ ሳል (እሱ ብቻ መመለክ እንዳለበት)   } &  ٱلَّذِى جَعَلَ لَكُمُ ٱلْأَرْضَ فِرَٟشًۭا وَٱلسَّمَآءَ بِنَآءًۭ وَأَنزَلَ مِنَ ٱلسَّمَآءِ مَآءًۭ فَأَخْرَجَ بِهِۦ مِنَ ٱلثَّمَرَٟتِ رِزْقًۭا لَّكُمْ ۖ فَلَا تَجْعَلُوا۟ لِلَّهِ أَندَادًۭا وَأَنتُمْ تَعْلَمُونَ ﴿٢٢﴾\\
\textamh{23.\ ለባሪያችን (ሙሐመድ(ሠአወሰ)) ባወርደነው (ቁረአን) ጥርጣሬ ካላችሁ (እናነተ ፓጋን አረቦችና አይሁዶች) እስኪ በሉ አንድ እንዲህ ያለ ምእራፍ (ሱራ) አምጡ (ፍጠሩ) እና ከኣላህ በቀር ምስክሮቻችሁን (ረዳቶቻችሁን) ጥሩ ፤እዉነተኛ ከሆናችሁ   } &  وَإِن كُنتُمْ فِى رَيْبٍۢ مِّمَّا نَزَّلْنَا عَلَىٰ عَبْدِنَا فَأْتُوا۟ بِسُورَةٍۢ مِّن مِّثْلِهِۦ وَٱدْعُوا۟ شُهَدَآءَكُم مِّن دُونِ ٱللَّهِ إِن كُنتُمْ صَٟدِقِينَ ﴿٢٣﴾\\
\textamh{24.\ ካላደረጋችሁ ግን ደግሞም አታደርጉትም  ማቀጣጠያውና ነዳጁ ሰዉና ድንጋይ የሆኑበትን እሳት ፍሩ (ጀሃነም)፤ ለከሃዲዎች (ለማያምኑት) የተዘጋጀ።   } &  فَإِن لَّمْ تَفْعَلُوا۟ وَلَن تَفْعَلُوا۟ فَٱتَّقُوا۟ ٱلنَّارَ ٱلَّتِى وَقُودُهَا ٱلنَّاسُ وَٱلْحِجَارَةُ ۖ أُعِدَّتْ لِلْكَٟفِرِينَ ﴿٢٤﴾\\
\textamh{25.\ አማኞች ሁነው ጥሩ ስራ ለሚሰሩ አብስር(ሩ) ለነሱ ገነት (ጀነት)፥ በስራቸዉ ወንዞች የሚፈሱበት፥ ሁሌ ከዚያ ፍራፍሬ ሲሰጡ\enquote{እንደዚህ አይነት  በፊት ተሰጥቶናል} ይላሉ (ያስታዉሳሉ)እናም  በአምሳያ ይሰጣቸዋል (አንድ አይነት ግን ጣእሙ  የተለያየ)፤ እዚያም ጠሃራ (ንጹህ) የሆኑ ሚስቶች ይኖሯቸዋል፤ ለዘላለሙ ይቀመጣሉ።   } &  وَبَشِّرِ ٱلَّذِينَ ءَامَنُوا۟ وَعَمِلُوا۟ ٱلصَّٟلِحَٟتِ أَنَّ لَهُمْ جَنَّٟتٍۢ تَجْرِى مِن تَحْتِهَا ٱلْأَنْهَـٰرُ ۖ كُلَّمَا رُزِقُوا۟ مِنْهَا مِن ثَمَرَةٍۢ رِّزْقًۭا ۙ قَالُوا۟ هَـٰذَا ٱلَّذِى رُزِقْنَا مِن قَبْلُ ۖ وَأُتُوا۟ بِهِۦ مُتَشَٟبِهًۭا ۖ وَلَهُمْ فِيهَآ أَزْوَٟجٌۭ مُّطَهَّرَةٌۭ ۖ وَهُمْ فِيهَا خَـٰلِدُونَ ﴿٢٥﴾ ۞\\
\textamh{26.\ በእዉነት ኣላህ ምሳሌ (በትንሿም) በትንኝ ወይም ከሷም ባነሰ ወይም በተለቀ ለማቅረብ አያፍርም፤ ለሚያምኑት እዉነቱ (ሀቁ) ከአምላካቸው  እንደሆነ ያዉቃሉ፤ የማየምኑት ግን\enquote{ኣላህ በዚህ ምሳሌ ምን አስቦ (ማለቱ) ነው?} ይላሉ።  በዚያ ግን ብዙዎችን ያስታል፥ ብዙዎችንም ይመራል የሚያስተው ፋሲቁን (የማይገዙለትን፥ የሚያምጹትን) ነው።   } &   إِنَّ ٱللَّهَ لَا يَسْتَحْىِۦٓ أَن يَضْرِبَ مَثَلًۭا مَّا بَعُوضَةًۭ فَمَا فَوْقَهَا ۚ فَأَمَّا ٱلَّذِينَ ءَامَنُوا۟ فَيَعْلَمُونَ أَنَّهُ ٱلْحَقُّ مِن رَّبِّهِمْ ۖ وَأَمَّا ٱلَّذِينَ كَفَرُوا۟ فَيَقُولُونَ مَاذَآ أَرَادَ ٱللَّهُ بِهَـٰذَا مَثَلًۭا ۘ يُضِلُّ بِهِۦ كَثِيرًۭا وَيَهْدِى بِهِۦ كَثِيرًۭا ۚ وَمَا يُضِلُّ بِهِۦٓ إِلَّا ٱلْفَٟسِقِينَ ﴿٢٦﴾\\
\textamh{27.\ የኣላህን ዉል ስምምነት ከገቡ በኋላ የሚበጥሱ፥ እዲደረግ ያዘዘዉን የሚያጣሙ (የሚያፈርሱ) እና ምድር (መሬት) ላይ የሚበጠብጡ፥እነሱ ናቸው ካሲሩን (የሚከስሩ)   } &  ٱلَّذِينَ يَنقُضُونَ عَهْدَ ٱللَّهِ مِنۢ بَعْدِ مِيثَٟقِهِۦ وَيَقْطَعُونَ مَآ أَمَرَ ٱللَّهُ بِهِۦٓ أَن يُوصَلَ وَيُفْسِدُونَ فِى ٱلْأَرْضِ ۚ أُو۟لَٟٓئِكَ هُمُ ٱلْخَـٰسِرُونَ ﴿٢٧﴾\\
\textamh{28.\ እንዴት በኣላህ አታምኑም? ሙት እንደነበራችሁ  እያያችሁ ህይወት ሰጣችሁ። ከዚያም ሞትን ይሰጣችኋል፥ ከዚያም ደግሞ ህይወት ይስጣችኋል (ያስነሳችኋል የትንሳኤ ቀን፥ የፍርድ ቀን) ከዚያም ወደሱ ትመለሳላችሁ።   } &  كَيْفَ تَكْفُرُونَ بِٱللَّهِ وَكُنتُمْ أَمْوَٟتًۭا فَأَحْيَـٰكُمْ ۖ ثُمَّ يُمِيتُكُمْ ثُمَّ يُحْيِيكُمْ ثُمَّ إِلَيْهِ تُرْجَعُونَ ﴿٢٨﴾\\
\textamh{29.\ እሱ እኮ ነው ምድር ላይ ያለዉን ሁሉ ለእናንተ የፈጠረው። ከዚያም ከፍ ብሎ (ኢስትወ) ወደ ሰማይ ሰባት ሰማያት አደረጋቸው እና የሁሉ  ነገር አዋቂ ነው   } &  هُوَ ٱلَّذِى خَلَقَ لَكُم مَّا فِى ٱلْأَرْضِ جَمِيعًۭا ثُمَّ ٱسْتَوَىٰٓ إِلَى ٱلسَّمَآءِ فَسَوَّىٰهُنَّ سَبْعَ سَمَـٰوَٟتٍۢ ۚ وَهُوَ بِكُلِّ شَىْءٍ عَلِيمٌۭ ﴿٢٩﴾\\
\textamh{30.\ አምላክህ ለመላኢክት (እንዲህ) አላቸው:\enquote{በእዉነት (ሰዉን) ትውልድ በትውልድ ምድር  ላይ ላስቀምጥ ነው}።  (እንዲህ) አሉ:\enquote{የሚበጠብጥና ደምን የሚያፈስ ታስቀምጣለህን? እኛ ስባሃትክንና  ምስጋናህንን (እያደርግን) እና እየቀደስነህ} (ኣላህ) አለ:\enquote{እናንተ የማተውቁትን አዉቃለሁ}   } &  وَإِذْ قَالَ رَبُّكَ لِلْمَلَٟٓئِكَةِ إِنِّى جَاعِلٌۭ فِى ٱلْأَرْضِ خَلِيفَةًۭ ۖ قَالُوٓا۟ أَتَجْعَلُ فِيهَا مَن يُفْسِدُ فِيهَا وَيَسْفِكُ ٱلدِّمَآءَ وَنَحْنُ نُسَبِّحُ بِحَمْدِكَ وَنُقَدِّسُ لَكَ ۖ قَالَ إِنِّىٓ أَعْلَمُ مَا لَا تَعْلَمُونَ ﴿٣٠﴾\\
\textamh{31.\ እና አደምን (አዳም) ሁሉን ስም (የሁሉን ነገር) አስተማረው፤ ከዚያም ለመላኢክት (ሁሉን) አሳየና\enquote{በሉ የነዚህን ስም ካወቃችሁ ንገሩኝ እዉነተኛ ከሆናችሁ} አላቸው።   } &  وَعَلَّمَ ءَادَمَ ٱلْأَسْمَآءَ كُلَّهَا ثُمَّ عَرَضَهُمْ عَلَى ٱلْمَلَٟٓئِكَةِ فَقَالَ أَنۢبِـُٔونِى بِأَسْمَآءِ هَـٰٓؤُلَآءِ إِن كُنتُمْ صَٟدِقِينَ ﴿٣١﴾\\

\textamh{32.\ (እነሱም) አሉ:\enquote{ስብሃት ለአንተ ይሁን፥ አንተ ካስተመርከነ ዉጭ ሌላ እዉቀት የለነም፥ አንተ ነህ ሁሉን አወቂ፥ ሁሉን መርማሪ (ጥበበኛ) ነህ} } &  قَالُوا۟ سُبْحَٟنَكَ لَا عِلْمَ لَنَآ إِلَّا مَا عَلَّمْتَنَآ ۖ إِنَّكَ أَنتَ ٱلْعَلِيمُ ٱلْحَكِيمُ ﴿٣٢﴾\\
\textamh{33.\ (ኣላህ) አለ:\enquote{ያኣ አደም (አዳም)! ስማቸዉን ንገራቸው}፥ (አደምም) ስማቸዉን ከነገራቸው  በኋላ (ኣላህ) አለ:\enquote{የማይታየዉን በሰማይና በምድር ዉስጥ ያለዉን አውቀዋለሁ፥ የደብቀችሁትንም  የምትገልጹትንም አዉቀዋለሁ አላልኳችሁንም?}   } &  قَالَ يَـٰٓـَٔادَمُ أَنۢبِئْهُم بِأَسْمَآئِهِمْ ۖ فَلَمَّآ أَنۢبَأَهُم بِأَسْمَآئِهِمْ قَالَ أَلَمْ أَقُل لَّكُمْ إِنِّىٓ أَعْلَمُ غَيْبَ ٱلسَّمَـٰوَٟتِ وَٱلْأَرْضِ وَأَعْلَمُ مَا تُبْدُونَ وَمَا كُنتُمْ تَكْتُمُونَ ﴿٣٣﴾\\
\textamh{34.\ ለመላኢክት\enquote{ለአዳም ስገዱ} አልናቸው እነሱም ሰገዱ ከኢብሊስ (ሰይጣን) በስተቀር እሱ ተቃወመ እና ራሱን ከፍ አደረገ (ኮራ) እናም ከካህዲዎች (ካፊሮች) ሆነ (ኣላህን የማይታዘዝ)።   } &  وَإِذْ قُلْنَا لِلْمَلَٟٓئِكَةِ ٱسْجُدُوا۟ لِءَادَمَ فَسَجَدُوٓا۟ إِلَّآ إِبْلِيسَ أَبَىٰ وَٱسْتَكْبَرَ وَكَانَ مِنَ ٱلْكَٟفِرِينَ ﴿٣٤﴾\\
\textamh{35.\ እና አልነ:\enquote{ያኣ አደሙ! (አንተ አዳም) አንተና ሚስትህ ገነት (ጀነት) ዉስጥ  ተቀመጡ፤ ብሉ በነጻነት የፈለገችሁትንና ያማረችሁን ነገር በሙሉ፤ ነገር ግን ከዚች ዛፍ አትቅረቡ ከመጥፎ ሰሪዎች (ዛሊሙን) መካከል ትሆናላችሁ።}   } &  وَقُلْنَا يَـٰٓـَٔادَمُ ٱسْكُنْ أَنتَ وَزَوْجُكَ ٱلْجَنَّةَ وَكُلَا مِنْهَا رَغَدًا حَيْثُ شِئْتُمَا وَلَا تَقْرَبَا هَـٰذِهِ ٱلشَّجَرَةَ فَتَكُونَا مِنَ ٱلظَّٟلِمِينَ ﴿٣٥﴾\\
\textamh{36.\ ከዚያም ሸይጣን (ሰይጣን) ሸተት አደረጋቸው (አሳሳታቸው) ከነበሩበት አስወጣቸው። አልናቸው (  ኣላህ):\enquote{ዉረዱ (ዉጡ)፥ ሁላችሁ፥ እርስበራሰችሁ ጠላት ሁናችሁ። ምድር መኖሪያችሁ ይሆናል ለጊዜዉም መደሰቻ}   } &  فَأَزَلَّهُمَا ٱلشَّيْطَٟنُ عَنْهَا فَأَخْرَجَهُمَا مِمَّا كَانَا فِيهِ ۖ وَقُلْنَا ٱهْبِطُوا۟ بَعْضُكُمْ لِبَعْضٍ عَدُوٌّۭ ۖ وَلَكُمْ فِى ٱلْأَرْضِ مُسْتَقَرٌّۭ وَمَتَـٰعٌ إِلَىٰ حِينٍۢ ﴿٣٦﴾\\
\textamh{37.\ ከዚያም አዳም (አደም) ከአምላኩ ድምጽ ሰማ፤ ይቅርም አለው። በእዉነት እሱ ብቻ ነው ይቅር ባይ፤ ከሁሉም በላይ ምህረተኛው።   } &  فَتَلَقَّىٰٓ ءَادَمُ مِن رَّبِّهِۦ كَلِمَـٰتٍۢ فَتَابَ عَلَيْهِ ۚ إِنَّهُۥ هُوَ ٱلتَّوَّابُ ٱلرَّحِيمُ ﴿٣٧﴾\\
\textamh{38.\ አልነ ( ኣላህ):\enquote{ሁላችሁም ከዚህ ቦታ ዉረዱ (ዉጡ)፥ ከዚያም ከ እኔ ምሬት (መመሪያ) ሲመጣለችሁ፥ የእኔን መመሪያ የሚከተል፥ ከነሱ ላይ ፍራሀት አይኖርም አያዙኑምም}   } &   قُلْنَا ٱهْبِطُوا۟ مِنْهَا جَمِيعًۭا ۖ فَإِمَّا يَأْتِيَنَّكُم مِّنِّى هُدًۭى فَمَن تَبِعَ هُدَاىَ فَلَا خَوْفٌ عَلَيْهِمْ وَلَا هُمْ يَحْزَنُونَ ﴿٣٨﴾\\
\textamh{39.\ ነገር ግን የሚክዱት (የማይምኑት) እና  አያትችን (ምልክታችን፥ ጥቅሳችን፥ ማስረጃችን) የማይቀበሉ፥ እነሱ የእሳቱ ነዋሪዎች ናቸው፥ ለዘላለም ይኖሩበታል።   } &  وَٱلَّذِينَ كَفَرُوا۟ وَكَذَّبُوا۟ بِـَٔايَـٰتِنَآ أُو۟لَٟٓئِكَ أَصْحَٟبُ ٱلنَّارِ ۖ هُمْ فِيهَا خَـٰلِدُونَ ﴿٣٩﴾ \\
\textamh{40.\ እናንት የእስራኤል ልጆች! ለእናንተ የደረግኩትን አስታዉሱ፥ እናንተም ቃል ኪዳኔን አክብሩ እኔም ኪዳናችሁን እንዳሟላላችሁ (እንዳከብርላችሁ) ከኔ በቀር ማንንም አትፍሩ።  } &  يَـٰبَنِىٓ إِسْرَٟٓءِيلَ ٱذْكُرُوا۟ نِعْمَتِىَ ٱلَّتِىٓ أَنْعَمْتُ عَلَيْكُمْ وَأَوْفُوا۟ بِعَهْدِىٓ أُوفِ بِعَهْدِكُمْ وَإِيَّٟىَ فَٱرْهَبُونِ ﴿٤٠﴾\\ 
\textamh{41.\ ባወርደኩት (በዚህ ቁርአን) እመኑ፥ እናንተ ያለዉን (ተውራት፥ ወንጌል) የሚያረጋግጥላችሁ፤  ከካሀዲዎች የመጀመሪያ አትሁኑ፤ አያቴን  (ተውራት፥ ወንጌልን፥ ምልክቴን፥ ጥቅሶቼን) በትንሽ  ዋጋ አትቸርችሩ፤ ፍሩኝ እኔን ብቻ ፍሩ   } &  وَءَامِنُوا۟ بِمَآ أَنزَلْتُ مُصَدِّقًۭا لِّمَا مَعَكُمْ وَلَا تَكُونُوٓا۟ أَوَّلَ كَافِرٍۭ بِهِۦ ۖ وَلَا تَشْتَرُوا۟ بِـَٔايَـٰتِى ثَمَنًۭا قَلِيلًۭا وَإِيَّٟىَ فَٱتَّقُونِ ﴿٤١﴾\\
\textamh{42.\ ሀቁን (እዉነቱን) በሐሰት አታልብሱ እዉነቱንም አትደብቁ እናንተ እያወቃችሁ (ሙሐመድ(ሠአወሰ) የኣላህ መልክተኛ መሆኑን)   } &  وَلَا تَلْبِسُوا۟ ٱلْحَقَّ بِٱلْبَٟطِلِ وَتَكْتُمُوا۟ ٱلْحَقَّ وَأَنتُمْ تَعْلَمُونَ ﴿٤٢﴾\\
\textamh{43.\ ሳላት ቁሙ፥ ዘካት ክፈሉ፥ ኢርከ (ጎንበስ ብላችሁ ለኣላህ) አር-ራኪኡን (ስገዱ)   } &  وَأَقِيمُوا۟ ٱلصَّلَوٰةَ وَءَاتُوا۟ ٱلزَّكَوٰةَ وَٱرْكَعُوا۟ مَعَ ٱلرَّٟكِعِينَ ﴿٤٣﴾ ۞\\
\textamh{44.\ ሰዉን የጽድቅ ስራ እንዲሰሩ (ለኣላህ እንዲገዙ) ታዛላችሁ ራሳችሁ ማድረጉን ረስታችሁ፥ መጽሃፉን እያነበባችሁ? አቅል የላችሁም (አታስቡም) ወይ?   } &   أَتَأْمُرُونَ ٱلنَّاسَ بِٱلْبِرِّ وَتَنسَوْنَ أَنفُسَكُمْ وَأَنتُمْ تَتْلُونَ ٱلْكِتَـٰبَ ۚ أَفَلَا تَعْقِلُونَ ﴿٤٤﴾\\
\textamh{45.\ በትእግስትና በሳለት (ጸሎት) እርዳታ ፈልጉ፤ በእዉነት ከባድ (ፈተና -ከቢር) ነው ለአል-ኻሺሁኡን (እዉነተኛ የኣላህ  አማኞች) በስተቀር   } &   وَٱسْتَعِينُوا۟ بِٱلصَّبْرِ وَٱلصَّلَوٰةِ ۚ وَإِنَّهَا لَكَبِيرَةٌ إِلَّا عَلَى ٱلْخَـٰشِعِينَ ﴿٤٥﴾\\
\textamh{46.\ እነዚህ ናቸው አምላካቸዉን በእርግ- ጠኝነት እንደሚጋናኙ የሚያውቁ፤ ወደሱም ይመለሳሉ።   } &  ٱلَّذِينَ يَظُنُّونَ أَنَّهُم مُّلَٟقُوا۟ رَبِّهِمْ وَأَنَّهُمْ إِلَيْهِ رَٟجِعُونَ ﴿٤٦﴾\\
\textamh{47.\ እናንት የእስራኤል ልጆች! ለእናንተ የደረግኩትን አስታዉሱ፥ ከአላሚን አስበልጬ እንደመረጥኳችሁ   } &  يَـٰبَنِىٓ إِسْرَٟٓءِيلَ ٱذْكُرُوا۟ نِعْمَتِىَ ٱلَّتِىٓ أَنْعَمْتُ عَلَيْكُمْ وَأَنِّى فَضَّلْتُكُمْ عَلَى ٱلْعَٟلَمِينَ ﴿٤٧﴾\\
\textamh{48.\ አንድ ቀን ግን ፍሩ (የፍርድ ቀን) አንዱ ሌላው የማያወጣበት፥ ወይንም ምልድጃ የማይቀበልበት ወይንም ካሳ ክፍያ የማይቀበሉበት ወይንም የማይረዱበት   } &  وَٱتَّقُوا۟ يَوْمًۭا لَّا تَجْزِى نَفْسٌ عَن نَّفْسٍۢ شَيْـًۭٔا وَلَا يُقْبَلُ مِنْهَا شَفَٟعَةٌۭ وَلَا يُؤْخَذُ مِنْهَا عَدْلٌۭ وَلَا هُمْ يُنصَرُونَ ﴿٤٨﴾\\
\textamh{49.\ ከፈርኦን ሰዎች አወጣናችሁ፥ በከባድ  ቅጣት ሲቀጧችሁ፥ ልጆቻችሁን እየገደሉ ሴቶቻችሁን እያቆዩ፥ እዚያ ከአምላካችሁ  ከባድ ፈተና ነበር   } &  وَإِذْ نَجَّيْنَـٰكُم مِّنْ ءَالِ فِرْعَوْنَ يَسُومُونَكُمْ سُوٓءَ ٱلْعَذَابِ يُذَبِّحُونَ أَبْنَآءَكُمْ وَيَسْتَحْيُونَ نِسَآءَكُمْ ۚ وَفِى ذَٟلِكُم بَلَآءٌۭ مِّن رَّبِّكُمْ عَظِيمٌۭ ﴿٤٩﴾\\
\textamh{50.\ ባህሩን ከፍለን እናንተን አድነን የፊራኡን (የፈርኦንን) ሰዎች አይናችሁ እያየ  አሰመጥናቸው    } &  وَإِذْ فَرَقْنَا بِكُمُ ٱلْبَحْرَ فَأَنجَيْنَـٰكُمْ وَأَغْرَقْنَآ ءَالَ فِرْعَوْنَ وَأَنتُمْ تَنظُرُونَ ﴿٥٠﴾\\
\textamh{51.\ ለአረባ ለሊት ሙሳን (ሙሴን) ስናደርግለት  (ለብቻው)፥ (በሌለበት) ጥጃዉን  (እንደአምላክ) ለራሳችሁ አደረጋችሁ እናንተም ዛሊሙን(ጣኦት አምላኪ፥ ጥፋተኞች) ሆናችሁ።   } &  وَإِذْ وَٟعَدْنَا مُوسَىٰٓ أَرْبَعِينَ لَيْلَةًۭ ثُمَّ ٱتَّخَذْتُمُ ٱلْعِجْلَ مِنۢ بَعْدِهِۦ وَأَنتُمْ ظَٟلِمُونَ ﴿٥١﴾\\
\textamh{52.\ ከዚያም በኋላ ይቅር አለናችሁ እንድታመሰግኑ   } &  ثُمَّ عَفَوْنَا عَنكُم مِّنۢ بَعْدِ ذَٟلِكَ لَعَلَّكُمْ تَشْكُرُونَ ﴿٥٢﴾\\
\textamh{53.\ ለሙሳም መጽሃፍና መፍረጃ (እዉነቱን ከሐሰት) ሰጠነው በዚያ በትክክል መመራት እንድትችሉ።   } &  وَإِذْ ءَاتَيْنَا مُوسَى ٱلْكِتَـٰبَ وَٱلْفُرْقَانَ لَعَلَّكُمْ تَهْتَدُونَ ﴿٥٣﴾\\
\textamh{54.\ ሙሳም ወደ ሰዎቹ አለ:\enquote{ሰዎቼ ሆይ!፥ በእዉነት ራሳችሁን በድላችኋል ጥጃዉን  በማምለክ። ወደ አምላክችሁ ንስሃ ግቡ፥ ራሳች- ሁን (ያጠፉትን) ግደሉ፥ ያ በአምላካችሁ ዘነድ ጥሩ ይሆንላችኋል} (ኣላህም) ንስሀችሁን ተቀበለ። በእዉነት እሱ ብቻ ነው ንስሀ ተቀበይ፥ ከሁሉም በላይ  ምህረተኛው   } &  وَإِذْ قَالَ مُوسَىٰ لِقَوْمِهِۦ يَـٰقَوْمِ إِنَّكُمْ ظَلَمْتُمْ أَنفُسَكُم بِٱتِّخَاذِكُمُ ٱلْعِجْلَ فَتُوبُوٓا۟ إِلَىٰ بَارِئِكُمْ فَٱقْتُلُوٓا۟ أَنفُسَكُمْ ذَٟلِكُمْ خَيْرٌۭ لَّكُمْ عِندَ بَارِئِكُمْ فَتَابَ عَلَيْكُمْ ۚ إِنَّهُۥ هُوَ ٱلتَّوَّابُ ٱلرَّحِيمُ ﴿٥٤﴾\\
\textamh{55.\ እናንተም ሙሳን:\enquote{ያኣ ሙሳ (ኦ ሙሳ)! ኣላህን ካላየን ምንም አናምንህም} አላችሁ። ወዲያዉም መብረቅ መጥቶ አይናችሁ እያየ ያዛችሁ።   } &  وَإِذْ قُلْتُمْ يَـٰمُوسَىٰ لَن نُّؤْمِنَ لَكَ حَتَّىٰ نَرَى ٱللَّهَ جَهْرَةًۭ فَأَخَذَتْكُمُ ٱلصَّٟعِقَةُ وَأَنتُمْ تَنظُرُونَ ﴿٥٥﴾\\
\textamh{56.\ ከዚያም አስነሳናችሁ (ህይወት ሰጠናችሁ) ከሞታችሁ በኋላ፥ አመስጋኝ እንድትሆኑ   } &  ثُمَّ بَعَثْنَـٰكُم مِّنۢ بَعْدِ مَوْتِكُمْ لَعَلَّكُمْ تَشْكُرُونَ ﴿٥٦﴾\\
\textamh{57.\ በደመና ጋረድናችሁ፥ ከሰማይም መናና  ሰልዋ አወርድንላችሁ፤\enquote{ብሉ የሰጠናችሁን (ያወርደነዉን) ጥሩና የተፈቀደ (ሃላል) ምግብ} (ግን አማጹ)። እኛን አልበደሉነም ነገር ግን ራሳቸዉን ነው የበደሉ።   } &  وَظَلَّلْنَا عَلَيْكُمُ ٱلْغَمَامَ وَأَنزَلْنَا عَلَيْكُمُ ٱلْمَنَّ وَٱلسَّلْوَىٰ ۖ كُلُوا۟ مِن طَيِّبَٟتِ مَا رَزَقْنَـٰكُمْ ۖ وَمَا ظَلَمُونَا وَلَٟكِن كَانُوٓا۟ أَنفُسَهُمْ يَظْلِمُونَ ﴿٥٧﴾\\
\textamh{58.\ አልን (ኣላህ) :\enquote{እዚህ ከተማ ግቡ  (እየሩሳሌም) እና ብሉ እንደፈለጋችሁ በደስታ (ያማራችሁን)ከፈልገችሁበት ቦታ ግቡ በአክብሮት (በሱጀደ፥ በአክብሮት ጎንበስ ብላችሁ)} እናም (እንዲህ) በሉ:\enquote{ይቅር በለነ} ሀጢያታችሁን ይቅር እንላችኋለን ጥሩ የሚ- ሰሩትን እንጨምርላቸዋለን።   } &  وَإِذْ قُلْنَا ٱدْخُلُوا۟ هَـٰذِهِ ٱلْقَرْيَةَ فَكُلُوا۟ مِنْهَا حَيْثُ شِئْتُمْ رَغَدًۭا وَٱدْخُلُوا۟ ٱلْبَابَ سُجَّدًۭا وَقُولُوا۟ حِطَّةٌۭ نَّغْفِرْ لَكُمْ خَطَٟيَـٰكُمْ ۚ وَسَنَزِيدُ ٱلْمُحْسِنِينَ ﴿٥٨﴾\\
\textamh{59.\ ነገር ግን መጥፎ ሰሪዎቹ የተነገራቸውን ቃል በሌላ ቀየሩት፤ ከነዚህ ዛሊሞች (መጥፎ ሰሪዎች) ላይ ሪጅዘን (ቅጣት) ከሰማይ አወርድንባቸው በኣላህ ትእዛዝ ላይ ስላመጹ   } &  فَبَدَّلَ ٱلَّذِينَ ظَلَمُوا۟ قَوْلًا غَيْرَ ٱلَّذِى قِيلَ لَهُمْ فَأَنزَلْنَا عَلَى ٱلَّذِينَ ظَلَمُوا۟ رِجْزًۭا مِّنَ ٱلسَّمَآءِ بِمَا كَانُوا۟ يَفْسُقُونَ ﴿٥٩﴾ ۞\\
\textamh{60.\ ሙሳ ለሰዎቹ ውሃ ሲጠይቅ፤ አልን (ኣላህ):\enquote{አለቱን በበትርህ ምታው}። ከዚያም አስራ ሁለት ምንጮች ፈሰሱ። ሁሉም (ነገድ) የየራሱን ውሃ መጠጫ ቦታ የዉቁ ነበር።\enquote{ብሉ ጠጡ ኣላህ የሰጣችሁን፤ በደል አትስሩ መሬት (ምድር) ላይ እየበጠበጣችሁ።} } &  وَإِذِ ٱسْتَسْقَىٰ مُوسَىٰ لِقَوْمِهِۦ فَقُلْنَا ٱضْرِب بِّعَصَاكَ ٱلْحَجَرَ ۖ فَٱنفَجَرَتْ مِنْهُ ٱثْنَتَا عَشْرَةَ عَيْنًۭا ۖ قَدْ عَلِمَ كُلُّ أُنَاسٍۢ مَّشْرَبَهُمْ ۖ كُلُوا۟ وَٱشْرَبُوا۟ مِن رِّزْقِ ٱللَّهِ وَلَا تَعْثَوْا۟ فِى ٱلْأَرْضِ مُفْسِدِينَ ﴿٦٠﴾\\
\textamh{61.\ እናንተም (እንዲህ) አላችሁ:\enquote{ያኣ ሙሳ! (ኦ! ሙሴ) አንድ አይነት  ምግብ ብቻ አልተቻለንም። ስለዚህ አምላክህን ምድር የሚያበቅለዉን ስጠን ብለህ ጠይቅልን፥ ባቄላዉን ኮከምበር(?)፥ ፉም (ነጭ ሽንኩርት ወይም ስንዴ)፥ ምስሩን ቀይ ሽንኩርቱን}። አለ\enquote{ጥሩ የሆነዉን  ከዚያ ባነሰ ትለዉጣላችሁ? ሂዱ ዉረዱ (ዉጡ ከዚህ) ወደ አንዱ ከተማና የፈለጋችሁትን ታገኛላችሁ} ሀፍረሀትና ስቃይ ተከናነቡ፥ ራሳቸው ላይ የኣላህን ቁጣ አመጡ። ያም የሆነው የኣላህን አያት (ጥቅሶችን፥ ማስረጃዎችን፥ ምልክቶችን ተአምራቱን) እየካዱ ስለነበርና ነቢያቱን በሃሰት ሲገሉ ስለኖሩ ነው። ያም የሆነው ስለማይገዙና (ትእዛዝ ስለማያከብሩ) ከማይገባ በላይ ተላላፊዎች ስለነበሩ ነው።   } &  وَإِذْ قُلْتُمْ يَـٰمُوسَىٰ لَن نَّصْبِرَ عَلَىٰ طَعَامٍۢ وَٟحِدٍۢ فَٱدْعُ لَنَا رَبَّكَ يُخْرِجْ لَنَا مِمَّا تُنۢبِتُ ٱلْأَرْضُ مِنۢ بَقْلِهَا وَقِثَّآئِهَا وَفُومِهَا وَعَدَسِهَا وَبَصَلِهَا ۖ قَالَ أَتَسْتَبْدِلُونَ ٱلَّذِى هُوَ أَدْنَىٰ بِٱلَّذِى هُوَ خَيْرٌ ۚ ٱهْبِطُوا۟ مِصْرًۭا فَإِنَّ لَكُم مَّا سَأَلْتُمْ ۗ وَضُرِبَتْ عَلَيْهِمُ ٱلذِّلَّةُ وَٱلْمَسْكَنَةُ وَبَآءُو بِغَضَبٍۢ مِّنَ ٱللَّهِ ۗ ذَٟلِكَ بِأَنَّهُمْ كَانُوا۟ يَكْفُرُونَ بِـَٔايَـٰتِ ٱللَّهِ وَيَقْتُلُونَ ٱلنَّبِيِّۦنَ بِغَيْرِ ٱلْحَقِّ ۗ ذَٟلِكَ بِمَا عَصَوا۟ وَّكَانُوا۟ يَعْتَدُونَ ﴿٦١﴾\\
\textamh{62.\ በእውነት የሚያምኑትና አይሁዶች፥ ናሳራዎች  (ክርስቲያኖች)፥ ሳቢያኖች ማንኛዉም በኣላህና በመጨረሻዉ ቀን የሚያምን እና ጥሩ ስራ  የሚሰራ፥ እነሱ ክፍያቸዉን ከአምላካቸው ያገኛሉ፤ እነሱም ላይ ፍርሃት አይኖርም  አያዝኑምም።   } &   إِنَّ ٱلَّذِينَ ءَامَنُوا۟ وَٱلَّذِينَ هَادُوا۟ وَٱلنَّصَٟرَىٰ وَٱلصَّٟبِـِٔينَ مَنْ ءَامَنَ بِٱللَّهِ وَٱلْيَوْمِ ٱلْءَاخِرِ وَعَمِلَ صَٟلِحًۭا فَلَهُمْ أَجْرُهُمْ عِندَ رَبِّهِمْ وَلَا خَوْفٌ عَلَيْهِمْ وَلَا هُمْ يَحْزَنُونَ ﴿٦٢﴾\\
\textamh{63.\ (ኦ! የእስራእል ልጆች) ቃል ኪዳናችሁን ገብተን ተራራዉን ከናንተ በላይ አድርገን\enquote{ይህን የሰጠናችሁን ጠበቅ አድርጋችሁያዙ፥ ዉስጡ ያለዉን አስታውሱ  በዚያም አል-ሙታቁን (ፈሪሃ-ኣላህ ያለው)  ትሆናላችሁ}  } &  وَإِذْ أَخَذْنَا مِيثَٟقَكُمْ وَرَفَعْنَا فَوْقَكُمُ ٱلطُّورَ خُذُوا۟ مَآ ءَاتَيْنَـٰكُم بِقُوَّةٍۢ وَٱذْكُرُوا۟ مَا فِيهِ لَعَلَّكُمْ تَتَّقُونَ ﴿٦٣﴾\\
\textamh{64.\ ከዚያም (ራሳችሁ) ዘወር አላችሁ። የኣላህ ጸጋና ምህረት እናንተ ላይ ባይሆን ኑሮ ከከሳሪዎች መካከል ትሆኑ ነበር   } &  ثُمَّ تَوَلَّيْتُم مِّنۢ بَعْدِ ذَٟلِكَ ۖ فَلَوْلَا فَضْلُ ٱللَّهِ عَلَيْكُمْ وَرَحْمَتُهُۥ لَكُنتُم مِّنَ ٱلْخَـٰسِرِينَ ﴿٦٤﴾\\
\textamh{65.\ እናም ታዉቃላችሁ ከናንተ መካከል ሰንበትን የተላለፉትን፤ እኛም አልናችዉ\enquote{ሁኑ ዝንጆሮዎች፥ የረከሰና የተጣለ}   } &  وَلَقَدْ عَلِمْتُمُ ٱلَّذِينَ ٱعْتَدَوْا۟ مِنكُمْ فِى ٱلسَّبْتِ فَقُلْنَا لَهُمْ كُونُوا۟ قِرَدَةً خَـٰسِـِٔينَ ﴿٦٥﴾\\
\textamh{66.\ ይህንም ቅጣት ምሳሌ አደርገነው ለነሱም ከነሱም በኋላ ለመጡት ትዉልዶች እና  ለአል-ሙታቁን (ፈሪሃ-ኣላህ ላላቸው) ትምህርት።   } &  فَجَعَلْنَـٰهَا نَكَٟلًۭا لِّمَا بَيْنَ يَدَيْهَا وَمَا خَلْفَهَا وَمَوْعِظَةًۭ لِّلْمُتَّقِينَ ﴿٦٦﴾\\
\textamh{67.\ ሙሳም (አላቸው):\enquote{በእዉነት፥ ኣላህ አንድ ላም ታርዱለት ዘንድ ያዛችኋል}። እነሱም አሉ:\enquote{ታላግጥብናለህ እንዴ?}። እሱም አለ:\enquote{በኣላህ እከለላለሁ ከጅሎች መካከል እንዳልሆን}   } &  وَإِذْ قَالَ مُوسَىٰ لِقَوْمِهِۦٓ إِنَّ ٱللَّهَ يَأْمُرُكُمْ أَن تَذْبَحُوا۟ بَقَرَةًۭ ۖ قَالُوٓا۟ أَتَتَّخِذُنَا هُزُوًۭا ۖ قَالَ أَعُوذُ بِٱللَّهِ أَنْ أَكُونَ مِنَ ٱلْجَٟهِلِينَ ﴿٦٧﴾\\
\textamh{68.\ እነሱም አሉ:\enquote{አምላክህን ጠይቅልን ምን እንደሆነ በትክክል እንዲገልጽልን} እሱም አለ:\enquote{(አምላክ) እንዲህ ይላል፥  በእዉነት፥ ያላረጀች ወይም  ትንሽም ያልሆነች፥ ነገር ግን በሁለቱ መካከል የሆነች። በሉ የታዘዛችሁትን አድርጉ።}   } &  قَالُوا۟ ٱدْعُ لَنَا رَبَّكَ يُبَيِّن لَّنَا مَا هِىَ ۚ قَالَ إِنَّهُۥ يَقُولُ إِنَّهَا بَقَرَةٌۭ لَّا فَارِضٌۭ وَلَا بِكْرٌ عَوَانٌۢ بَيْنَ ذَٟلِكَ ۖ فَٱفْعَلُوا۟ مَا تُؤْمَرُونَ ﴿٦٨﴾\\
\textamh{69.\ እነሱም አሉ:\enquote{አምላክህን ጠይቅልን ቀለሟ ምን እንደሆነ እንዲገልጽልን} እሱም አለ:\enquote{(አምላክ) እንዲህ ይላል፥ ቢጫ ላም፥ ቀለሟ ቦግ ያለ፥ ለሚያያት የሚስደስት}   } &  قَالُوا۟ ٱدْعُ لَنَا رَبَّكَ يُبَيِّن لَّنَا مَا لَوْنُهَا ۚ قَالَ إِنَّهُۥ يَقُولُ إِنَّهَا بَقَرَةٌۭ صَفْرَآءُ فَاقِعٌۭ لَّوْنُهَا تَسُرُّ ٱلنَّٟظِرِينَ ﴿٦٩﴾\\
\textamh{70.\ እነሱም አሉ:\enquote{አምላክህን ጠይቅልን ምን እንደሆነ በትክክል እንዲገልጽልን። ለእኛ ሁሉም ላሞች አንድ አይነት ናቸው፤ እናም በእርግጠኝነት፥ ኣላህ ከፈቀደ፥ እኛ እንመራለን}    } &  قَالُوا۟ ٱدْعُ لَنَا رَبَّكَ يُبَيِّن لَّنَا مَا هِىَ إِنَّ ٱلْبَقَرَ تَشَٟبَهَ عَلَيْنَا وَإِنَّآ إِن شَآءَ ٱللَّهُ لَمُهْتَدُونَ ﴿٧٠﴾\\
\textamh{71.\ እሱም (ሙሳ) አለ:\enquote{(አምላክ) እንዲህ ይላል፥ መሬት ለማረስ ወይንም ሜዳ ዉሃ ለማጠጣት ያልሰለጠነ፥ ጤነኛ ቀለሙም ቦግ ካለ (ደማቅ?) ከቢጫ ሌላ ያልሆነ።} እነሱም አሉ:\enquote{አሁን እዉነቱን አመጣህልን}። እናም አረዱ ግን ላለማድረግ ተቃርበው ነበር።   } &  قَالَ إِنَّهُۥ يَقُولُ إِنَّهَا بَقَرَةٌۭ لَّا ذَلُولٌۭ تُثِيرُ ٱلْأَرْضَ وَلَا تَسْقِى ٱلْحَرْثَ مُسَلَّمَةٌۭ لَّا شِيَةَ فِيهَا ۚ قَالُوا۟ ٱلْـَٟٔنَ جِئْتَ بِٱلْحَقِّ ۚ فَذَبَحُوهَا وَمَا كَادُوا۟ يَفْعَلُونَ ﴿٧١﴾\\
\textamh{72.\ እናም ሰው ገደላችሁ እርስበራሰችሁ ማን እንዳደረገው ስትወነጃጀሉ፤ ነገር ግን ኣላህ አወጣው ስትደበቁት የነበረዉን   } &  وَإِذْ قَتَلْتُمْ نَفْسًۭا فَٱدَّٟرَْٟٔتُمْ فِيهَا ۖ وَٱللَّهُ مُخْرِجٌۭ مَّا كُنتُمْ تَكْتُمُونَ ﴿٧٢﴾\\
\textamh{73.\ እናም አልነ:\enquote{(የሞተዉን ሰው በላሟ) በቁራጭ ምቱት}። ስለዚህ ኣላህ የሞተዉን ያስነሳል እና አያቱን (ማረጋገጫ፥ ጥቅሶች) ያሳያል በዚያ እንዲገባችሁ።   } &   فَقُلْنَا ٱضْرِبُوهُ بِبَعْضِهَا ۚ كَذَٟلِكَ يُحْىِ ٱللَّهُ ٱلْمَوْتَىٰ وَيُرِيكُمْ ءَايَـٰتِهِۦ لَعَلَّكُمْ تَعْقِلُونَ ﴿٧٣﴾\\
\textamh{74.\ ከዚያም በኋላ ልባችሁ ደንደነ፥ አለት ሆኑ ከዚያም የከፋ ድንዳኔ። ከአለቶች እንኳን ውሃ ያሚወጣባቸው አሉ፥ አንዳዶችም ሲሰነጠቁ ዉሃ ይፈሳል፥ ከነሱም መካከል በኣላህ ፍርሃት የሚወድቁ አሉ። እና ኣላህ የምታደርጉትን የማያዉቅ አይደለም።   } &   ثُمَّ قَسَتْ قُلُوبُكُم مِّنۢ بَعْدِ ذَٟلِكَ فَهِىَ كَٱلْحِجَارَةِ أَوْ أَشَدُّ قَسْوَةًۭ ۚ وَإِنَّ مِنَ ٱلْحِجَارَةِ لَمَا يَتَفَجَّرُ مِنْهُ ٱلْأَنْهَـٰرُ ۚ وَإِنَّ مِنْهَا لَمَا يَشَّقَّقُ فَيَخْرُجُ مِنْهُ ٱلْمَآءُ ۚ وَإِنَّ مِنْهَا لَمَا يَهْبِطُ مِنْ خَشْيَةِ ٱللَّهِ ۗ وَمَا ٱللَّهُ بِغَٟفِلٍ عَمَّا تَعْمَلُونَ ﴿٧٤﴾ ۞\\
\textamh{75.\ እናንተ (አማኞች) በሃይማኖታችሁ ያምናሉ (ይሁዶችን) ብላችሁ ታስባላችሁ፥ የኣላህን ቃል (ተውራት(ቶራህ)) ሲሰሙ ኑረው ነገር ግን በራሳቸው እያወቁ ከገባቸው በኋላ እየቀይሩት አልነበር።    } &   أَفَتَطْمَعُونَ أَن يُؤْمِنُوا۟ لَكُمْ وَقَدْ كَانَ فَرِيقٌۭ مِّنْهُمْ يَسْمَعُونَ كَلَٟمَ ٱللَّهِ ثُمَّ يُحَرِّفُونَهُۥ مِنۢ بَعْدِ مَا عَقَلُوهُ وَهُمْ يَعْلَمُونَ ﴿٧٥﴾\\
\textamh{76.\ አማኞችን ሲያገኙ (ይሁዶች)\enquote{እናምናለን} ይላሉ ብቻቸውን እርስበርስ ሲገናኙ\enquote{እናንተ (ይሁዶች) ለነሱ (ለሙስሊሞች) ኣላህ የገለጸላችሁን (ይሁዶችን፥ ስለነብዩ ሙሐመድ (ሠአወሰ) ባህሪይ ተውራት (ቶራህ) ዉስጥ የተጻፈ ገለጻ) ትነገሯቸዋላችሁን} እናነተ (ይሁዶች) አእምሮ የላችሁም ወይ?   } &  وَإِذَا لَقُوا۟ ٱلَّذِينَ ءَامَنُوا۟ قَالُوٓا۟ ءَامَنَّا وَإِذَا خَلَا بَعْضُهُمْ إِلَىٰ بَعْضٍۢ قَالُوٓا۟ أَتُحَدِّثُونَهُم بِمَا فَتَحَ ٱللَّهُ عَلَيْكُمْ لِيُحَآجُّوكُم بِهِۦ عِندَ رَبِّكُمْ ۚ أَفَلَا تَعْقِلُونَ ﴿٧٦﴾\\
\textamh{77.\ ኣላህ የሚገልጹትንና የሚደብቁትን እንደሚያውቅ አያውቁምን?   } &   أَوَلَا يَعْلَمُونَ أَنَّ ٱللَّهَ يَعْلَمُ مَا يُسِرُّونَ وَمَا يُعْلِنُونَ ﴿٧٧﴾\\
\textamh{78.\ ከነሱ መካከል ደግሞ ያልተማሩ (ፊደል ያልቆጠሩ) አሉ፥ መጽሐፉን የማይውቁ፥ ሀሰት የሆነ ምኞትን ያምናሉ፤ ሌላ ሳይሆን የሚያደርጉት መገመት ብቻ።   } &  وَمِنْهُمْ أُمِّيُّونَ لَا يَعْلَمُونَ ٱلْكِتَـٰبَ إِلَّآ أَمَانِىَّ وَإِنْ هُمْ إِلَّا يَظُنُّونَ ﴿٧٨﴾\\
\textamh{79.\ ወዮለቸው በራሳቸው እጅ መጽሐፉን ጽፈው ከዚያም\enquote{ይሄ ከኣላህ ነው} የሚሉ በትንሽ ዋጋ ለመቸርቸር! ወዮ እጃቸው ለጻፈው ነገር፥ ወዮ በዚያም ለሚያገኙት፤    } &  فَوَيْلٌۭ لِّلَّذِينَ يَكْتُبُونَ ٱلْكِتَـٰبَ بِأَيْدِيهِمْ ثُمَّ يَقُولُونَ هَـٰذَا مِنْ عِندِ ٱللَّهِ لِيَشْتَرُوا۟ بِهِۦ ثَمَنًۭا قَلِيلًۭا ۖ فَوَيْلٌۭ لَّهُم مِّمَّا كَتَبَتْ أَيْدِيهِمْ وَوَيْلٌۭ لَّهُم مِّمَّا يَكْسِبُونَ ﴿٧٩﴾\\
\textamh{80.\ እናም ይላሉ (ይሁዶች):\enquote{እሳቱ (ጀሀነም) ከተወሰኑ ቀናት በቀር አይነካንም}። (እንዲህ) በል (ኦ ሙሐመድ (ሠአወሰ):\enquote{ከኣላህ ዉል አላችሁ ወይ፥ ኣላህ ዉሉን እንዳይሰብር? ወይስ ስለኣላህ የማታዉቁትን ትላላችሁ?}   } &  وَقَالُوا۟ لَن تَمَسَّنَا ٱلنَّارُ إِلَّآ أَيَّامًۭا مَّعْدُودَةًۭ ۚ قُلْ أَتَّخَذْتُمْ عِندَ ٱللَّهِ عَهْدًۭا فَلَن يُخْلِفَ ٱللَّهُ عَهْدَهُۥٓ ۖ أَمْ تَقُولُونَ عَلَى ٱللَّهِ مَا لَا تَعْلَمُونَ ﴿٨٠﴾\\
\textamh{81.\ አዎ! ማንም መጥፎ ስራውን ያገኘና ሀጢያቱ የከበበዉ፥ እነሱ የእሳቱ ነዋሪዎች ናቸው፤ እዛም ለዘላለም ይኖራሉ   } &  بَلَىٰ مَن كَسَبَ سَيِّئَةًۭ وَأَحَٟطَتْ بِهِۦ خَطِيٓـَٔتُهُۥ فَأُو۟لَٟٓئِكَ أَصْحَٟبُ ٱلنَّارِ ۖ هُمْ فِيهَا خَـٰلِدُونَ ﴿٨١﴾\\
\textamh{82.\ የሚያምኑና ጥሩ ስራ የሚሰሩ፥ እነሱ የገነት ነዋሪዎች ናቸው፥ እዛም ለዘላልም ይኖሩበታል   } &  وَٱلَّذِينَ ءَامَنُوا۟ وَعَمِلُوا۟ ٱلصَّٟلِحَٟتِ أُو۟لَٟٓئِكَ أَصْحَٟبُ ٱلْجَنَّةِ ۖ هُمْ فِيهَا خَـٰلِدُونَ ﴿٨٢﴾\\
\textamh{83.\ ከእስራእል ልጆች ጋር ቃል ኪዳን ስንገባ: ከኣላህ በቀር ማንንም አታምልኩ፥ ለወላጆቻችሁ ታዛዥና (አሳቢ) ጥሩ ሰሪ ሁኑ፥ ለዘመዶች፥ ለወላጅ አልባዎች ለማሳኪን (ለድሆች) እና ጥሩ የሆነ ለሰዎች ተናገሩ፥ ሳላት ቁሞ፥ ዘካት ክፈሉ። ከዚያም ወደኋለ ሸተት አላችሁ ትንሽ ቁጥር ካላቸው በቀር፥ እናንተም ወደ ኋለ ባዮች (ዘወር ባዮች) ናችሁ።   } &  وَإِذْ أَخَذْنَا مِيثَٟقَ بَنِىٓ إِسْرَٟٓءِيلَ لَا تَعْبُدُونَ إِلَّا ٱللَّهَ وَبِٱلْوَٟلِدَيْنِ إِحْسَانًۭا وَذِى ٱلْقُرْبَىٰ وَٱلْيَتَـٰمَىٰ وَٱلْمَسَٟكِينِ وَقُولُوا۟ لِلنَّاسِ حُسْنًۭا وَأَقِيمُوا۟ ٱلصَّلَوٰةَ وَءَاتُوا۟ ٱلزَّكَوٰةَ ثُمَّ تَوَلَّيْتُمْ إِلَّا قَلِيلًۭا مِّنكُمْ وَأَنتُم مُّعْرِضُونَ ﴿٨٣﴾\\
\textamh{84.\ ከናንተ ጋር ቃል ኪዳን ስንገባ: የራሳችሁን ሰዎች ደም አታፍስሱ፥ ደግሞም ከመኖሪያቸው አታስወጧቸው። ከዚያም ዉሉን ወሰዳችሁ (ተቀበላችሁ) ራሳችሁ እየመሰከራችሁ።   } &  وَإِذْ أَخَذْنَا مِيثَٟقَكُمْ لَا تَسْفِكُونَ دِمَآءَكُمْ وَلَا تُخْرِجُونَ أَنفُسَكُم مِّن دِيَـٰرِكُمْ ثُمَّ أَقْرَرْتُمْ وَأَنتُمْ تَشْهَدُونَ ﴿٨٤﴾\\
\textamh{85.\ ከዚያም በኋላ እናንተው ናችሁ እርስበርስ የተገዳደላችሁ፥ ከናንተ መካከል ያሉትንም ከቤታቸው ያስወጣችሁ፥ (ጠላቶቻቸዉን) የረዳችሁ፥ በሀጢያትና በመተላለፍ። ወደ እናንተ ምርኮኞች ሁነው ሲመጡ፥ ዋጋ (የማስፈቻ) ትከፍላላችሁ፥ ነገር ግን እነሱን ማስወጣት ክልክል ነበር። ስለዚህ አንዱን የመጽሃፍ ክፍል እያመናችሁ ሌላኛዉን ትክዳላችሁ? ምንድነው ታዲያ እንዲህ ለሚያደርግ ክፍያው በዚህ አለም ዉርዴት፥ የትንሳኤ ቀን ደግሞ ክፉ የሆነ ስቃይ ካለበት መመደብ። እና ኣላህ የምታደርጉትን የማያዉቅ አይደለም።   } &   ثُمَّ أَنتُمْ هَـٰٓؤُلَآءِ تَقْتُلُونَ أَنفُسَكُمْ وَتُخْرِجُونَ فَرِيقًۭا مِّنكُم مِّن دِيَـٰرِهِمْ تَظَٟهَرُونَ عَلَيْهِم بِٱلْإِثْمِ وَٱلْعُدْوَٟنِ وَإِن يَأْتُوكُمْ أُسَٟرَىٰ تُفَٟدُوهُمْ وَهُوَ مُحَرَّمٌ عَلَيْكُمْ إِخْرَاجُهُمْ ۚ أَفَتُؤْمِنُونَ بِبَعْضِ ٱلْكِتَـٰبِ وَتَكْفُرُونَ بِبَعْضٍۢ ۚ فَمَا جَزَآءُ مَن يَفْعَلُ ذَٟلِكَ مِنكُمْ إِلَّا خِزْىٌۭ فِى ٱلْحَيَوٰةِ ٱلدُّنْيَا ۖ وَيَوْمَ ٱلْقِيَـٰمَةِ يُرَدُّونَ إِلَىٰٓ أَشَدِّ ٱلْعَذَابِ ۗ وَمَا ٱللَّهُ بِغَٟفِلٍ عَمَّا تَعْمَلُونَ ﴿٨٥﴾\\
\textamh{86.\ እነዚህ ናቸው የዚህን አለም በሰማያዊ (በሚቀጥለው አለም) (በአኪራ) የነገዱ። ቅጣቸው አይቀለልም ደግሞም እርዳታ አይኖራቸውም፤   } &  أُو۟لَٟٓئِكَ ٱلَّذِينَ ٱشْتَرَوُا۟ ٱلْحَيَوٰةَ ٱلدُّنْيَا بِٱلْءَاخِرَةِ ۖ فَلَا يُخَفَّفُ عَنْهُمُ ٱلْعَذَابُ وَلَا هُمْ يُنصَرُونَ ﴿٨٦﴾\\
\textamh{87.\ ለሙሳ (ሙሴ) መጽሃፍ ሰጠነው እናም ተከታታይ መልእክተኞች አስከተልነ። ለኢሳ (የሱስ)፥ የማሪያም ልጅ፥ ግልጽ ምልክት ሰጠነው፥ በመንፈስ ቅዱስ (ጂብሪል (ገብርኤል)) ረዳነው። እናንተ የማትፈልጉት መልእክተኛ ሲመጣላችሁ ኮራችሁ? አንዳንዶችን ካዳችሁ፥ አንዳዶችንም ገደላችሁ።   } &  وَلَقَدْ ءَاتَيْنَا مُوسَى ٱلْكِتَـٰبَ وَقَفَّيْنَا مِنۢ بَعْدِهِۦ بِٱلرُّسُلِ ۖ وَءَاتَيْنَا عِيسَى ٱبْنَ مَرْيَمَ ٱلْبَيِّنَـٰتِ وَأَيَّدْنَـٰهُ بِرُوحِ ٱلْقُدُسِ ۗ أَفَكُلَّمَا جَآءَكُمْ رَسُولٌۢ بِمَا لَا تَهْوَىٰٓ أَنفُسُكُمُ ٱسْتَكْبَرْتُمْ فَفَرِيقًۭا كَذَّبْتُمْ وَفَرِيقًۭا تَقْتُلُونَ ﴿٨٧﴾\\
\textamh{88.\ እነሱም አሉ (ሰዎች)\enquote{ልባችን የታሸገ (የኣላህን ቀል ከማወቅ) ነው።} አይደለም፥ ኣላህ ስለክህደታቸው ረግሞኣቸዋል፥ ከትንሽ በታች ነው የሚያምኑ፤   } &  وَقَالُوا۟ قُلُوبُنَا غُلْفٌۢ ۚ بَل لَّعَنَهُمُ ٱللَّهُ بِكُفْرِهِمْ فَقَلِيلًۭا مَّا يُؤْمِنُونَ ﴿٨٨﴾\\
\textamh{89.\ ከኣላህ መጽሐፍ (ይሄ ቁርአን) ሲመጣላቸው ከነሱ ያለዉን የሚያረጋግጥ (ተውራት፥ ወንጌል)፥ ምንም እንኳ በፊት ኣላህን ቢጠይቁም (ሙሐመድ (ሠአወሰ) እንዲመጣ) ከሃዲዎችን (የማያምኑትን) ለማሸነፍ፥ ከዚያ የሚያዉቁት ነገር ወደነሱ ሲመጣ፥ ካዱ። ስለዚህ የኣላህ እርግማን ከከሀዲዎች ላይ ይሁን።   } &  وَلَمَّا جَآءَهُمْ كِتَـٰبٌۭ مِّنْ عِندِ ٱللَّهِ مُصَدِّقٌۭ لِّمَا مَعَهُمْ وَكَانُوا۟ مِن قَبْلُ يَسْتَفْتِحُونَ عَلَى ٱلَّذِينَ كَفَرُوا۟ فَلَمَّا جَآءَهُم مَّا عَرَفُوا۟ كَفَرُوا۟ بِهِۦ ۚ فَلَعْنَةُ ٱللَّهِ عَلَى ٱلْكَٟفِرِينَ ﴿٨٩﴾\\
\end{longtabu} 


\end{document}
