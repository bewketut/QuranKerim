%% License: BSD style (Berkley) (i.e. Put the Copyright owner's name always)
%% Writer and Copyright (to): Bewketu(Bilal) Tadilo (2016-17)
\shadowbox{\section{\LR{\textamharic{ሱራቱ ቃፍ -}  \RL{سوره  ق}}}}

  
    
  
    
    

\nopagebreak
  بِسمِ ٱللَّهِ ٱلرَّحمَـٰنِ ٱلرَّحِيمِ
  قٓ ۚ وَٱلقُرءَانِ ٱلمَجِيدِ ﴿١﴾
 بَل عَجِبُوٓا۟ أَن جَآءَهُم مُّنذِرٌۭ مِّنهُم فَقَالَ ٱلكَـٰفِرُونَ هَـٰذَا شَىءٌ عَجِيبٌ ﴿٢﴾
 أَءِذَا مِتنَا وَكُنَّا تُرَابًۭا ۖ ذَٟلِكَ رَجعٌۢ بَعِيدٌۭ ﴿٣﴾
 قَد عَلِمنَا مَا تَنقُصُ ٱلأَرضُ مِنهُم ۖ وَعِندَنَا كِتَـٰبٌ حَفِيظٌۢ ﴿٤﴾
 بَل كَذَّبُوا۟ بِٱلحَقِّ لَمَّا جَآءَهُم فَهُم فِىٓ أَمرٍۢ مَّرِيجٍ ﴿٥﴾
 أَفَلَم يَنظُرُوٓا۟ إِلَى ٱلسَّمَآءِ فَوقَهُم كَيفَ بَنَينَـٰهَا وَزَيَّنَّـٰهَا وَمَا لَهَا مِن فُرُوجٍۢ ﴿٦﴾
 وَٱلأَرضَ مَدَدنَـٰهَا وَأَلقَينَا فِيهَا رَوَٟسِىَ وَأَنۢبَتنَا فِيهَا مِن كُلِّ زَوجٍۭ بَهِيجٍۢ ﴿٧﴾
 تَبصِرَةًۭ وَذِكرَىٰ لِكُلِّ عَبدٍۢ مُّنِيبٍۢ ﴿٨﴾
 وَنَزَّلنَا مِنَ ٱلسَّمَآءِ مَآءًۭ مُّبَٰرَكًۭا فَأَنۢبَتنَا بِهِۦ جَنَّـٰتٍۢ وَحَبَّ ٱلحَصِيدِ ﴿٩﴾
 وَٱلنَّخلَ بَاسِقَـٰتٍۢ لَّهَا طَلعٌۭ نَّضِيدٌۭ ﴿١٠﴾
 رِّزقًۭا لِّلعِبَادِ ۖ وَأَحيَينَا بِهِۦ بَلدَةًۭ مَّيتًۭا ۚ كَذَٟلِكَ ٱلخُرُوجُ ﴿١١﴾
 كَذَّبَت قَبلَهُم قَومُ نُوحٍۢ وَأَصحَـٰبُ ٱلرَّسِّ وَثَمُودُ ﴿١٢﴾
 وَعَادٌۭ وَفِرعَونُ وَإِخوَٟنُ لُوطٍۢ ﴿١٣﴾
 وَأَصحَـٰبُ ٱلأَيكَةِ وَقَومُ تُبَّعٍۢ ۚ كُلٌّۭ كَذَّبَ ٱلرُّسُلَ فَحَقَّ وَعِيدِ ﴿١٤﴾
 أَفَعَيِينَا بِٱلخَلقِ ٱلأَوَّلِ ۚ بَل هُم فِى لَبسٍۢ مِّن خَلقٍۢ جَدِيدٍۢ ﴿١٥﴾
 وَلَقَد خَلَقنَا ٱلإِنسَـٰنَ وَنَعلَمُ مَا تُوَسوِسُ بِهِۦ نَفسُهُۥ ۖ وَنَحنُ أَقرَبُ إِلَيهِ مِن حَبلِ ٱلوَرِيدِ ﴿١٦﴾
 إِذ يَتَلَقَّى ٱلمُتَلَقِّيَانِ عَنِ ٱليَمِينِ وَعَنِ ٱلشِّمَالِ قَعِيدٌۭ ﴿١٧﴾
 مَّا يَلفِظُ مِن قَولٍ إِلَّا لَدَيهِ رَقِيبٌ عَتِيدٌۭ ﴿١٨﴾
 وَجَآءَت سَكرَةُ ٱلمَوتِ بِٱلحَقِّ ۖ ذَٟلِكَ مَا كُنتَ مِنهُ تَحِيدُ ﴿١٩﴾
 وَنُفِخَ فِى ٱلصُّورِ ۚ ذَٟلِكَ يَومُ ٱلوَعِيدِ ﴿٢٠﴾
 وَجَآءَت كُلُّ نَفسٍۢ مَّعَهَا سَآئِقٌۭ وَشَهِيدٌۭ ﴿٢١﴾
 لَّقَد كُنتَ فِى غَفلَةٍۢ مِّن هَـٰذَا فَكَشَفنَا عَنكَ غِطَآءَكَ فَبَصَرُكَ ٱليَومَ حَدِيدٌۭ ﴿٢٢﴾
 وَقَالَ قَرِينُهُۥ هَـٰذَا مَا لَدَىَّ عَتِيدٌ ﴿٢٣﴾
 أَلقِيَا فِى جَهَنَّمَ كُلَّ كَفَّارٍ عَنِيدٍۢ ﴿٢٤﴾
 مَّنَّاعٍۢ لِّلخَيرِ مُعتَدٍۢ مُّرِيبٍ ﴿٢٥﴾
 ٱلَّذِى جَعَلَ مَعَ ٱللَّهِ إِلَـٰهًا ءَاخَرَ فَأَلقِيَاهُ فِى ٱلعَذَابِ ٱلشَّدِيدِ ﴿٢٦﴾
 ۞ قَالَ قَرِينُهُۥ رَبَّنَا مَآ أَطغَيتُهُۥ وَلَـٰكِن كَانَ فِى ضَلَـٰلٍۭ بَعِيدٍۢ ﴿٢٧﴾
 قَالَ لَا تَختَصِمُوا۟ لَدَىَّ وَقَد قَدَّمتُ إِلَيكُم بِٱلوَعِيدِ ﴿٢٨﴾
 مَا يُبَدَّلُ ٱلقَولُ لَدَىَّ وَمَآ أَنَا۠ بِظَلَّٰمٍۢ لِّلعَبِيدِ ﴿٢٩﴾
 يَومَ نَقُولُ لِجَهَنَّمَ هَلِ ٱمتَلَأتِ وَتَقُولُ هَل مِن مَّزِيدٍۢ ﴿٣٠﴾
 وَأُزلِفَتِ ٱلجَنَّةُ لِلمُتَّقِينَ غَيرَ بَعِيدٍ ﴿٣١﴾
 هَـٰذَا مَا تُوعَدُونَ لِكُلِّ أَوَّابٍ حَفِيظٍۢ ﴿٣٢﴾
 مَّن خَشِىَ ٱلرَّحمَـٰنَ بِٱلغَيبِ وَجَآءَ بِقَلبٍۢ مُّنِيبٍ ﴿٣٣﴾
 ٱدخُلُوهَا بِسَلَـٰمٍۢ ۖ ذَٟلِكَ يَومُ ٱلخُلُودِ ﴿٣٤﴾
 لَهُم مَّا يَشَآءُونَ فِيهَا وَلَدَينَا مَزِيدٌۭ ﴿٣٥﴾
 وَكَم أَهلَكنَا قَبلَهُم مِّن قَرنٍ هُم أَشَدُّ مِنهُم بَطشًۭا فَنَقَّبُوا۟ فِى ٱلبِلَـٰدِ هَل مِن مَّحِيصٍ ﴿٣٦﴾
 إِنَّ فِى ذَٟلِكَ لَذِكرَىٰ لِمَن كَانَ لَهُۥ قَلبٌ أَو أَلقَى ٱلسَّمعَ وَهُوَ شَهِيدٌۭ ﴿٣٧﴾
 وَلَقَد خَلَقنَا ٱلسَّمَـٰوَٟتِ وَٱلأَرضَ وَمَا بَينَهُمَا فِى سِتَّةِ أَيَّامٍۢ وَمَا مَسَّنَا مِن لُّغُوبٍۢ ﴿٣٨﴾
 فَٱصبِر عَلَىٰ مَا يَقُولُونَ وَسَبِّح بِحَمدِ رَبِّكَ قَبلَ طُلُوعِ ٱلشَّمسِ وَقَبلَ ٱلغُرُوبِ ﴿٣٩﴾
 وَمِنَ ٱلَّيلِ فَسَبِّحهُ وَأَدبَٰرَ ٱلسُّجُودِ ﴿٤٠﴾
 وَٱستَمِع يَومَ يُنَادِ ٱلمُنَادِ مِن مَّكَانٍۢ قَرِيبٍۢ ﴿٤١﴾
 يَومَ يَسمَعُونَ ٱلصَّيحَةَ بِٱلحَقِّ ۚ ذَٟلِكَ يَومُ ٱلخُرُوجِ ﴿٤٢﴾
 إِنَّا نَحنُ نُحىِۦ وَنُمِيتُ وَإِلَينَا ٱلمَصِيرُ ﴿٤٣﴾
 يَومَ تَشَقَّقُ ٱلأَرضُ عَنهُم سِرَاعًۭا ۚ ذَٟلِكَ حَشرٌ عَلَينَا يَسِيرٌۭ ﴿٤٤﴾
 نَّحنُ أَعلَمُ بِمَا يَقُولُونَ ۖ وَمَآ أَنتَ عَلَيهِم بِجَبَّارٍۢ ۖ فَذَكِّر بِٱلقُرءَانِ مَن يَخَافُ وَعِيدِ ﴿٤٥﴾
 
