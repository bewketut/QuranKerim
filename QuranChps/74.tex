%% License: BSD style (Berkley) (i.e. Put the Copyright owner's name always)
%% Writer and Copyright (to): Bewketu(Bilal) Tadilo (2016-17)
\shadowbox{\section{\LR{\textamharic{ሱራቱ አልሙደቲር -}  \RL{سوره  المدثر}}}}

  
    
  
    
    

\nopagebreak
  بِسمِ ٱللَّهِ ٱلرَّحمَـٰنِ ٱلرَّحِيمِ
  يَـٰٓأَيُّهَا ٱلمُدَّثِّرُ ﴿١﴾
 قُم فَأَنذِر ﴿٢﴾
 وَرَبَّكَ فَكَبِّر ﴿٣﴾
 وَثِيَابَكَ فَطَهِّر ﴿٤﴾
 وَٱلرُّجزَ فَٱهجُر ﴿٥﴾
 وَلَا تَمنُن تَستَكثِرُ ﴿٦﴾
 وَلِرَبِّكَ فَٱصبِر ﴿٧﴾
 فَإِذَا نُقِرَ فِى ٱلنَّاقُورِ ﴿٨﴾
 فَذَٟلِكَ يَومَئِذٍۢ يَومٌ عَسِيرٌ ﴿٩﴾
 عَلَى ٱلكَـٰفِرِينَ غَيرُ يَسِيرٍۢ ﴿١٠﴾
 ذَرنِى وَمَن خَلَقتُ وَحِيدًۭا ﴿١١﴾
 وَجَعَلتُ لَهُۥ مَالًۭا مَّمدُودًۭا ﴿١٢﴾
 وَبَنِينَ شُهُودًۭا ﴿١٣﴾
 وَمَهَّدتُّ لَهُۥ تَمهِيدًۭا ﴿١٤﴾
 ثُمَّ يَطمَعُ أَن أَزِيدَ ﴿١٥﴾
 كَلَّآ ۖ إِنَّهُۥ كَانَ لِءَايَـٰتِنَا عَنِيدًۭا ﴿١٦﴾
 سَأُرهِقُهُۥ صَعُودًا ﴿١٧﴾
 إِنَّهُۥ فَكَّرَ وَقَدَّرَ ﴿١٨﴾
 فَقُتِلَ كَيفَ قَدَّرَ ﴿١٩﴾
 ثُمَّ قُتِلَ كَيفَ قَدَّرَ ﴿٢٠﴾
 ثُمَّ نَظَرَ ﴿٢١﴾
 ثُمَّ عَبَسَ وَبَسَرَ ﴿٢٢﴾
 ثُمَّ أَدبَرَ وَٱستَكبَرَ ﴿٢٣﴾
 فَقَالَ إِن هَـٰذَآ إِلَّا سِحرٌۭ يُؤثَرُ ﴿٢٤﴾
 إِن هَـٰذَآ إِلَّا قَولُ ٱلبَشَرِ ﴿٢٥﴾
 سَأُصلِيهِ سَقَرَ ﴿٢٦﴾
 وَمَآ أَدرَىٰكَ مَا سَقَرُ ﴿٢٧﴾
 لَا تُبقِى وَلَا تَذَرُ ﴿٢٨﴾
 لَوَّاحَةٌۭ لِّلبَشَرِ ﴿٢٩﴾
 عَلَيهَا تِسعَةَ عَشَرَ ﴿٣٠﴾
 وَمَا جَعَلنَآ أَصحَـٰبَ ٱلنَّارِ إِلَّا مَلَـٰٓئِكَةًۭ ۙ وَمَا جَعَلنَا عِدَّتَهُم إِلَّا فِتنَةًۭ لِّلَّذِينَ كَفَرُوا۟ لِيَستَيقِنَ ٱلَّذِينَ أُوتُوا۟ ٱلكِتَـٰبَ وَيَزدَادَ ٱلَّذِينَ ءَامَنُوٓا۟ إِيمَـٰنًۭا ۙ وَلَا يَرتَابَ ٱلَّذِينَ أُوتُوا۟ ٱلكِتَـٰبَ وَٱلمُؤمِنُونَ ۙ وَلِيَقُولَ ٱلَّذِينَ فِى قُلُوبِهِم مَّرَضٌۭ وَٱلكَـٰفِرُونَ مَاذَآ أَرَادَ ٱللَّهُ بِهَـٰذَا مَثَلًۭا ۚ كَذَٟلِكَ يُضِلُّ ٱللَّهُ مَن يَشَآءُ وَيَهدِى مَن يَشَآءُ ۚ وَمَا يَعلَمُ جُنُودَ رَبِّكَ إِلَّا هُوَ ۚ وَمَا هِىَ إِلَّا ذِكرَىٰ لِلبَشَرِ ﴿٣١﴾
 كَلَّا وَٱلقَمَرِ ﴿٣٢﴾
 وَٱلَّيلِ إِذ أَدبَرَ ﴿٣٣﴾
 وَٱلصُّبحِ إِذَآ أَسفَرَ ﴿٣٤﴾
 إِنَّهَا لَإِحدَى ٱلكُبَرِ ﴿٣٥﴾
 نَذِيرًۭا لِّلبَشَرِ ﴿٣٦﴾
 لِمَن شَآءَ مِنكُم أَن يَتَقَدَّمَ أَو يَتَأَخَّرَ ﴿٣٧﴾
 كُلُّ نَفسٍۭ بِمَا كَسَبَت رَهِينَةٌ ﴿٣٨﴾
 إِلَّآ أَصحَـٰبَ ٱليَمِينِ ﴿٣٩﴾
 فِى جَنَّـٰتٍۢ يَتَسَآءَلُونَ ﴿٤٠﴾
 عَنِ ٱلمُجرِمِينَ ﴿٤١﴾
 مَا سَلَكَكُم فِى سَقَرَ ﴿٤٢﴾
 قَالُوا۟ لَم نَكُ مِنَ ٱلمُصَلِّينَ ﴿٤٣﴾
 وَلَم نَكُ نُطعِمُ ٱلمِسكِينَ ﴿٤٤﴾
 وَكُنَّا نَخُوضُ مَعَ ٱلخَآئِضِينَ ﴿٤٥﴾
 وَكُنَّا نُكَذِّبُ بِيَومِ ٱلدِّينِ ﴿٤٦﴾
 حَتَّىٰٓ أَتَىٰنَا ٱليَقِينُ ﴿٤٧﴾
 فَمَا تَنفَعُهُم شَفَـٰعَةُ ٱلشَّـٰفِعِينَ ﴿٤٨﴾
 فَمَا لَهُم عَنِ ٱلتَّذكِرَةِ مُعرِضِينَ ﴿٤٩﴾
 كَأَنَّهُم حُمُرٌۭ مُّستَنفِرَةٌۭ ﴿٥٠﴾
 فَرَّت مِن قَسوَرَةٍۭ ﴿٥١﴾
 بَل يُرِيدُ كُلُّ ٱمرِئٍۢ مِّنهُم أَن يُؤتَىٰ صُحُفًۭا مُّنَشَّرَةًۭ ﴿٥٢﴾
 كَلَّا ۖ بَل لَّا يَخَافُونَ ٱلءَاخِرَةَ ﴿٥٣﴾
 كَلَّآ إِنَّهُۥ تَذكِرَةٌۭ ﴿٥٤﴾
 فَمَن شَآءَ ذَكَرَهُۥ ﴿٥٥﴾
 وَمَا يَذكُرُونَ إِلَّآ أَن يَشَآءَ ٱللَّهُ ۚ هُوَ أَهلُ ٱلتَّقوَىٰ وَأَهلُ ٱلمَغفِرَةِ ﴿٥٦﴾
 
