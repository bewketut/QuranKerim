%% License: BSD style (Berkley) (i.e. Put the Copyright owner's name always)
%% Writer and Copyright (to): Bewketu(Bilal) Tadilo (2016-17)
\shadowbox{\section{\LR{\textamharic{ሱራቱ አልሙጀዲላ -}  \RL{سوره  المجادلة}}}}

  
    
  
    
    

\nopagebreak
  بِسمِ ٱللَّهِ ٱلرَّحمَـٰنِ ٱلرَّحِيمِ
  قَد سَمِعَ ٱللَّهُ قَولَ ٱلَّتِى تُجَٰدِلُكَ فِى زَوجِهَا وَتَشتَكِىٓ إِلَى ٱللَّهِ وَٱللَّهُ يَسمَعُ تَحَاوُرَكُمَآ ۚ إِنَّ ٱللَّهَ سَمِيعٌۢ بَصِيرٌ ﴿١﴾
 ٱلَّذِينَ يُظَـٰهِرُونَ مِنكُم مِّن نِّسَآئِهِم مَّا هُنَّ أُمَّهَـٰتِهِم ۖ إِن أُمَّهَـٰتُهُم إِلَّا ٱلَّٰٓـِٔى وَلَدنَهُم ۚ وَإِنَّهُم لَيَقُولُونَ مُنكَرًۭا مِّنَ ٱلقَولِ وَزُورًۭا ۚ وَإِنَّ ٱللَّهَ لَعَفُوٌّ غَفُورٌۭ ﴿٢﴾
 وَٱلَّذِينَ يُظَـٰهِرُونَ مِن نِّسَآئِهِم ثُمَّ يَعُودُونَ لِمَا قَالُوا۟ فَتَحرِيرُ رَقَبَةٍۢ مِّن قَبلِ أَن يَتَمَآسَّا ۚ ذَٟلِكُم تُوعَظُونَ بِهِۦ ۚ وَٱللَّهُ بِمَا تَعمَلُونَ خَبِيرٌۭ ﴿٣﴾
 فَمَن لَّم يَجِد فَصِيَامُ شَهرَينِ مُتَتَابِعَينِ مِن قَبلِ أَن يَتَمَآسَّا ۖ فَمَن لَّم يَستَطِع فَإِطعَامُ سِتِّينَ مِسكِينًۭا ۚ ذَٟلِكَ لِتُؤمِنُوا۟ بِٱللَّهِ وَرَسُولِهِۦ ۚ وَتِلكَ حُدُودُ ٱللَّهِ ۗ وَلِلكَـٰفِرِينَ عَذَابٌ أَلِيمٌ ﴿٤﴾
 إِنَّ ٱلَّذِينَ يُحَآدُّونَ ٱللَّهَ وَرَسُولَهُۥ كُبِتُوا۟ كَمَا كُبِتَ ٱلَّذِينَ مِن قَبلِهِم ۚ وَقَد أَنزَلنَآ ءَايَـٰتٍۭ بَيِّنَـٰتٍۢ ۚ وَلِلكَـٰفِرِينَ عَذَابٌۭ مُّهِينٌۭ ﴿٥﴾
 يَومَ يَبعَثُهُمُ ٱللَّهُ جَمِيعًۭا فَيُنَبِّئُهُم بِمَا عَمِلُوٓا۟ ۚ أَحصَىٰهُ ٱللَّهُ وَنَسُوهُ ۚ وَٱللَّهُ عَلَىٰ كُلِّ شَىءٍۢ شَهِيدٌ ﴿٦﴾
 أَلَم تَرَ أَنَّ ٱللَّهَ يَعلَمُ مَا فِى ٱلسَّمَـٰوَٟتِ وَمَا فِى ٱلأَرضِ ۖ مَا يَكُونُ مِن نَّجوَىٰ ثَلَـٰثَةٍ إِلَّا هُوَ رَابِعُهُم وَلَا خَمسَةٍ إِلَّا هُوَ سَادِسُهُم وَلَآ أَدنَىٰ مِن ذَٟلِكَ وَلَآ أَكثَرَ إِلَّا هُوَ مَعَهُم أَينَ مَا كَانُوا۟ ۖ ثُمَّ يُنَبِّئُهُم بِمَا عَمِلُوا۟ يَومَ ٱلقِيَـٰمَةِ ۚ إِنَّ ٱللَّهَ بِكُلِّ شَىءٍ عَلِيمٌ ﴿٧﴾
 أَلَم تَرَ إِلَى ٱلَّذِينَ نُهُوا۟ عَنِ ٱلنَّجوَىٰ ثُمَّ يَعُودُونَ لِمَا نُهُوا۟ عَنهُ وَيَتَنَـٰجَونَ بِٱلإِثمِ وَٱلعُدوَٟنِ وَمَعصِيَتِ ٱلرَّسُولِ وَإِذَا جَآءُوكَ حَيَّوكَ بِمَا لَم يُحَيِّكَ بِهِ ٱللَّهُ وَيَقُولُونَ فِىٓ أَنفُسِهِم لَولَا يُعَذِّبُنَا ٱللَّهُ بِمَا نَقُولُ ۚ حَسبُهُم جَهَنَّمُ يَصلَونَهَا ۖ فَبِئسَ ٱلمَصِيرُ ﴿٨﴾
 يَـٰٓأَيُّهَا ٱلَّذِينَ ءَامَنُوٓا۟ إِذَا تَنَـٰجَيتُم فَلَا تَتَنَـٰجَوا۟ بِٱلإِثمِ وَٱلعُدوَٟنِ وَمَعصِيَتِ ٱلرَّسُولِ وَتَنَـٰجَوا۟ بِٱلبِرِّ وَٱلتَّقوَىٰ ۖ وَٱتَّقُوا۟ ٱللَّهَ ٱلَّذِىٓ إِلَيهِ تُحشَرُونَ ﴿٩﴾
 إِنَّمَا ٱلنَّجوَىٰ مِنَ ٱلشَّيطَٰنِ لِيَحزُنَ ٱلَّذِينَ ءَامَنُوا۟ وَلَيسَ بِضَآرِّهِم شَيـًٔا إِلَّا بِإِذنِ ٱللَّهِ ۚ وَعَلَى ٱللَّهِ فَليَتَوَكَّلِ ٱلمُؤمِنُونَ ﴿١٠﴾
 يَـٰٓأَيُّهَا ٱلَّذِينَ ءَامَنُوٓا۟ إِذَا قِيلَ لَكُم تَفَسَّحُوا۟ فِى ٱلمَجَٰلِسِ فَٱفسَحُوا۟ يَفسَحِ ٱللَّهُ لَكُم ۖ وَإِذَا قِيلَ ٱنشُزُوا۟ فَٱنشُزُوا۟ يَرفَعِ ٱللَّهُ ٱلَّذِينَ ءَامَنُوا۟ مِنكُم وَٱلَّذِينَ أُوتُوا۟ ٱلعِلمَ دَرَجَٰتٍۢ ۚ وَٱللَّهُ بِمَا تَعمَلُونَ خَبِيرٌۭ ﴿١١﴾
 يَـٰٓأَيُّهَا ٱلَّذِينَ ءَامَنُوٓا۟ إِذَا نَـٰجَيتُمُ ٱلرَّسُولَ فَقَدِّمُوا۟ بَينَ يَدَى نَجوَىٰكُم صَدَقَةًۭ ۚ ذَٟلِكَ خَيرٌۭ لَّكُم وَأَطهَرُ ۚ فَإِن لَّم تَجِدُوا۟ فَإِنَّ ٱللَّهَ غَفُورٌۭ رَّحِيمٌ ﴿١٢﴾
 ءَأَشفَقتُم أَن تُقَدِّمُوا۟ بَينَ يَدَى نَجوَىٰكُم صَدَقَـٰتٍۢ ۚ فَإِذ لَم تَفعَلُوا۟ وَتَابَ ٱللَّهُ عَلَيكُم فَأَقِيمُوا۟ ٱلصَّلَوٰةَ وَءَاتُوا۟ ٱلزَّكَوٰةَ وَأَطِيعُوا۟ ٱللَّهَ وَرَسُولَهُۥ ۚ وَٱللَّهُ خَبِيرٌۢ بِمَا تَعمَلُونَ ﴿١٣﴾
 ۞ أَلَم تَرَ إِلَى ٱلَّذِينَ تَوَلَّوا۟ قَومًا غَضِبَ ٱللَّهُ عَلَيهِم مَّا هُم مِّنكُم وَلَا مِنهُم وَيَحلِفُونَ عَلَى ٱلكَذِبِ وَهُم يَعلَمُونَ ﴿١٤﴾
 أَعَدَّ ٱللَّهُ لَهُم عَذَابًۭا شَدِيدًا ۖ إِنَّهُم سَآءَ مَا كَانُوا۟ يَعمَلُونَ ﴿١٥﴾
 ٱتَّخَذُوٓا۟ أَيمَـٰنَهُم جُنَّةًۭ فَصَدُّوا۟ عَن سَبِيلِ ٱللَّهِ فَلَهُم عَذَابٌۭ مُّهِينٌۭ ﴿١٦﴾
 لَّن تُغنِىَ عَنهُم أَموَٟلُهُم وَلَآ أَولَـٰدُهُم مِّنَ ٱللَّهِ شَيـًٔا ۚ أُو۟لَـٰٓئِكَ أَصحَـٰبُ ٱلنَّارِ ۖ هُم فِيهَا خَـٰلِدُونَ ﴿١٧﴾
 يَومَ يَبعَثُهُمُ ٱللَّهُ جَمِيعًۭا فَيَحلِفُونَ لَهُۥ كَمَا يَحلِفُونَ لَكُم ۖ وَيَحسَبُونَ أَنَّهُم عَلَىٰ شَىءٍ ۚ أَلَآ إِنَّهُم هُمُ ٱلكَـٰذِبُونَ ﴿١٨﴾
 ٱستَحوَذَ عَلَيهِمُ ٱلشَّيطَٰنُ فَأَنسَىٰهُم ذِكرَ ٱللَّهِ ۚ أُو۟لَـٰٓئِكَ حِزبُ ٱلشَّيطَٰنِ ۚ أَلَآ إِنَّ حِزبَ ٱلشَّيطَٰنِ هُمُ ٱلخَـٰسِرُونَ ﴿١٩﴾
 إِنَّ ٱلَّذِينَ يُحَآدُّونَ ٱللَّهَ وَرَسُولَهُۥٓ أُو۟لَـٰٓئِكَ فِى ٱلأَذَلِّينَ ﴿٢٠﴾
 كَتَبَ ٱللَّهُ لَأَغلِبَنَّ أَنَا۠ وَرُسُلِىٓ ۚ إِنَّ ٱللَّهَ قَوِىٌّ عَزِيزٌۭ ﴿٢١﴾
 لَّا تَجِدُ قَومًۭا يُؤمِنُونَ بِٱللَّهِ وَٱليَومِ ٱلءَاخِرِ يُوَآدُّونَ مَن حَآدَّ ٱللَّهَ وَرَسُولَهُۥ وَلَو كَانُوٓا۟ ءَابَآءَهُم أَو أَبنَآءَهُم أَو إِخوَٟنَهُم أَو عَشِيرَتَهُم ۚ أُو۟لَـٰٓئِكَ كَتَبَ فِى قُلُوبِهِمُ ٱلإِيمَـٰنَ وَأَيَّدَهُم بِرُوحٍۢ مِّنهُ ۖ وَيُدخِلُهُم جَنَّـٰتٍۢ تَجرِى مِن تَحتِهَا ٱلأَنهَـٰرُ خَـٰلِدِينَ فِيهَا ۚ رَضِىَ ٱللَّهُ عَنهُم وَرَضُوا۟ عَنهُ ۚ أُو۟لَـٰٓئِكَ حِزبُ ٱللَّهِ ۚ أَلَآ إِنَّ حِزبَ ٱللَّهِ هُمُ ٱلمُفلِحُونَ ﴿٢٢﴾
 
