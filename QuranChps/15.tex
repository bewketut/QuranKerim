%% License: BSD style (Berkley) (i.e. Put the Copyright owner's name always)
%% Writer and Copyright (to): Bewketu(Bilal) Tadilo (2016-17)
\shadowbox{\section{\LR{\textamharic{ሱራቱ አልሂጅር -}  \RL{سوره  الحجر}}}}

  
    
  
    
    

\nopagebreak
  بِسمِ ٱللَّهِ ٱلرَّحمَـٰنِ ٱلرَّحِيمِ
  الٓر ۚ تِلكَ ءَايَـٰتُ ٱلكِتَـٰبِ وَقُرءَانٍۢ مُّبِينٍۢ ﴿١﴾
 رُّبَمَا يَوَدُّ ٱلَّذِينَ كَفَرُوا۟ لَو كَانُوا۟ مُسلِمِينَ ﴿٢﴾
 ذَرهُم يَأكُلُوا۟ وَيَتَمَتَّعُوا۟ وَيُلهِهِمُ ٱلأَمَلُ ۖ فَسَوفَ يَعلَمُونَ ﴿٣﴾
 وَمَآ أَهلَكنَا مِن قَريَةٍ إِلَّا وَلَهَا كِتَابٌۭ مَّعلُومٌۭ ﴿٤﴾
 مَّا تَسبِقُ مِن أُمَّةٍ أَجَلَهَا وَمَا يَستَـٔخِرُونَ ﴿٥﴾
 وَقَالُوا۟ يَـٰٓأَيُّهَا ٱلَّذِى نُزِّلَ عَلَيهِ ٱلذِّكرُ إِنَّكَ لَمَجنُونٌۭ ﴿٦﴾
 لَّو مَا تَأتِينَا بِٱلمَلَـٰٓئِكَةِ إِن كُنتَ مِنَ ٱلصَّـٰدِقِينَ ﴿٧﴾
 مَا نُنَزِّلُ ٱلمَلَـٰٓئِكَةَ إِلَّا بِٱلحَقِّ وَمَا كَانُوٓا۟ إِذًۭا مُّنظَرِينَ ﴿٨﴾
 إِنَّا نَحنُ نَزَّلنَا ٱلذِّكرَ وَإِنَّا لَهُۥ لَحَـٰفِظُونَ ﴿٩﴾
 وَلَقَد أَرسَلنَا مِن قَبلِكَ فِى شِيَعِ ٱلأَوَّلِينَ ﴿١٠﴾
 وَمَا يَأتِيهِم مِّن رَّسُولٍ إِلَّا كَانُوا۟ بِهِۦ يَستَهزِءُونَ ﴿١١﴾
 كَذَٟلِكَ نَسلُكُهُۥ فِى قُلُوبِ ٱلمُجرِمِينَ ﴿١٢﴾
 لَا يُؤمِنُونَ بِهِۦ ۖ وَقَد خَلَت سُنَّةُ ٱلأَوَّلِينَ ﴿١٣﴾
 وَلَو فَتَحنَا عَلَيهِم بَابًۭا مِّنَ ٱلسَّمَآءِ فَظَلُّوا۟ فِيهِ يَعرُجُونَ ﴿١٤﴾
 لَقَالُوٓا۟ إِنَّمَا سُكِّرَت أَبصَـٰرُنَا بَل نَحنُ قَومٌۭ مَّسحُورُونَ ﴿١٥﴾
 وَلَقَد جَعَلنَا فِى ٱلسَّمَآءِ بُرُوجًۭا وَزَيَّنَّـٰهَا لِلنَّـٰظِرِينَ ﴿١٦﴾
 وَحَفِظنَـٰهَا مِن كُلِّ شَيطَٰنٍۢ رَّجِيمٍ ﴿١٧﴾
 إِلَّا مَنِ ٱستَرَقَ ٱلسَّمعَ فَأَتبَعَهُۥ شِهَابٌۭ مُّبِينٌۭ ﴿١٨﴾
 وَٱلأَرضَ مَدَدنَـٰهَا وَأَلقَينَا فِيهَا رَوَٟسِىَ وَأَنۢبَتنَا فِيهَا مِن كُلِّ شَىءٍۢ مَّوزُونٍۢ ﴿١٩﴾
 وَجَعَلنَا لَكُم فِيهَا مَعَـٰيِشَ وَمَن لَّستُم لَهُۥ بِرَٰزِقِينَ ﴿٢٠﴾
 وَإِن مِّن شَىءٍ إِلَّا عِندَنَا خَزَآئِنُهُۥ وَمَا نُنَزِّلُهُۥٓ إِلَّا بِقَدَرٍۢ مَّعلُومٍۢ ﴿٢١﴾
 وَأَرسَلنَا ٱلرِّيَـٰحَ لَوَٟقِحَ فَأَنزَلنَا مِنَ ٱلسَّمَآءِ مَآءًۭ فَأَسقَينَـٰكُمُوهُ وَمَآ أَنتُم لَهُۥ بِخَـٰزِنِينَ ﴿٢٢﴾
 وَإِنَّا لَنَحنُ نُحىِۦ وَنُمِيتُ وَنَحنُ ٱلوَٟرِثُونَ ﴿٢٣﴾
 وَلَقَد عَلِمنَا ٱلمُستَقدِمِينَ مِنكُم وَلَقَد عَلِمنَا ٱلمُستَـٔخِرِينَ ﴿٢٤﴾
 وَإِنَّ رَبَّكَ هُوَ يَحشُرُهُم ۚ إِنَّهُۥ حَكِيمٌ عَلِيمٌۭ ﴿٢٥﴾
 وَلَقَد خَلَقنَا ٱلإِنسَـٰنَ مِن صَلصَـٰلٍۢ مِّن حَمَإٍۢ مَّسنُونٍۢ ﴿٢٦﴾
 وَٱلجَآنَّ خَلَقنَـٰهُ مِن قَبلُ مِن نَّارِ ٱلسَّمُومِ ﴿٢٧﴾
 وَإِذ قَالَ رَبُّكَ لِلمَلَـٰٓئِكَةِ إِنِّى خَـٰلِقٌۢ بَشَرًۭا مِّن صَلصَـٰلٍۢ مِّن حَمَإٍۢ مَّسنُونٍۢ ﴿٢٨﴾
 فَإِذَا سَوَّيتُهُۥ وَنَفَختُ فِيهِ مِن رُّوحِى فَقَعُوا۟ لَهُۥ سَـٰجِدِينَ ﴿٢٩﴾
 فَسَجَدَ ٱلمَلَـٰٓئِكَةُ كُلُّهُم أَجمَعُونَ ﴿٣٠﴾
 إِلَّآ إِبلِيسَ أَبَىٰٓ أَن يَكُونَ مَعَ ٱلسَّٰجِدِينَ ﴿٣١﴾
 قَالَ يَـٰٓإِبلِيسُ مَا لَكَ أَلَّا تَكُونَ مَعَ ٱلسَّٰجِدِينَ ﴿٣٢﴾
 قَالَ لَم أَكُن لِّأَسجُدَ لِبَشَرٍ خَلَقتَهُۥ مِن صَلصَـٰلٍۢ مِّن حَمَإٍۢ مَّسنُونٍۢ ﴿٣٣﴾
 قَالَ فَٱخرُج مِنهَا فَإِنَّكَ رَجِيمٌۭ ﴿٣٤﴾
 وَإِنَّ عَلَيكَ ٱللَّعنَةَ إِلَىٰ يَومِ ٱلدِّينِ ﴿٣٥﴾
 قَالَ رَبِّ فَأَنظِرنِىٓ إِلَىٰ يَومِ يُبعَثُونَ ﴿٣٦﴾
 قَالَ فَإِنَّكَ مِنَ ٱلمُنظَرِينَ ﴿٣٧﴾
 إِلَىٰ يَومِ ٱلوَقتِ ٱلمَعلُومِ ﴿٣٨﴾
 قَالَ رَبِّ بِمَآ أَغوَيتَنِى لَأُزَيِّنَنَّ لَهُم فِى ٱلأَرضِ وَلَأُغوِيَنَّهُم أَجمَعِينَ ﴿٣٩﴾
 إِلَّا عِبَادَكَ مِنهُمُ ٱلمُخلَصِينَ ﴿٤٠﴾
 قَالَ هَـٰذَا صِرَٰطٌ عَلَىَّ مُستَقِيمٌ ﴿٤١﴾
 إِنَّ عِبَادِى لَيسَ لَكَ عَلَيهِم سُلطَٰنٌ إِلَّا مَنِ ٱتَّبَعَكَ مِنَ ٱلغَاوِينَ ﴿٤٢﴾
 وَإِنَّ جَهَنَّمَ لَمَوعِدُهُم أَجمَعِينَ ﴿٤٣﴾
 لَهَا سَبعَةُ أَبوَٟبٍۢ لِّكُلِّ بَابٍۢ مِّنهُم جُزءٌۭ مَّقسُومٌ ﴿٤٤﴾
 إِنَّ ٱلمُتَّقِينَ فِى جَنَّـٰتٍۢ وَعُيُونٍ ﴿٤٥﴾
 ٱدخُلُوهَا بِسَلَـٰمٍ ءَامِنِينَ ﴿٤٦﴾
 وَنَزَعنَا مَا فِى صُدُورِهِم مِّن غِلٍّ إِخوَٟنًا عَلَىٰ سُرُرٍۢ مُّتَقَـٰبِلِينَ ﴿٤٧﴾
 لَا يَمَسُّهُم فِيهَا نَصَبٌۭ وَمَا هُم مِّنهَا بِمُخرَجِينَ ﴿٤٨﴾
 ۞ نَبِّئ عِبَادِىٓ أَنِّىٓ أَنَا ٱلغَفُورُ ٱلرَّحِيمُ ﴿٤٩﴾
 وَأَنَّ عَذَابِى هُوَ ٱلعَذَابُ ٱلأَلِيمُ ﴿٥٠﴾
 وَنَبِّئهُم عَن ضَيفِ إِبرَٰهِيمَ ﴿٥١﴾
 إِذ دَخَلُوا۟ عَلَيهِ فَقَالُوا۟ سَلَـٰمًۭا قَالَ إِنَّا مِنكُم وَجِلُونَ ﴿٥٢﴾
 قَالُوا۟ لَا تَوجَل إِنَّا نُبَشِّرُكَ بِغُلَـٰمٍ عَلِيمٍۢ ﴿٥٣﴾
 قَالَ أَبَشَّرتُمُونِى عَلَىٰٓ أَن مَّسَّنِىَ ٱلكِبَرُ فَبِمَ تُبَشِّرُونَ ﴿٥٤﴾
 قَالُوا۟ بَشَّرنَـٰكَ بِٱلحَقِّ فَلَا تَكُن مِّنَ ٱلقَـٰنِطِينَ ﴿٥٥﴾
 قَالَ وَمَن يَقنَطُ مِن رَّحمَةِ رَبِّهِۦٓ إِلَّا ٱلضَّآلُّونَ ﴿٥٦﴾
 قَالَ فَمَا خَطبُكُم أَيُّهَا ٱلمُرسَلُونَ ﴿٥٧﴾
 قَالُوٓا۟ إِنَّآ أُرسِلنَآ إِلَىٰ قَومٍۢ مُّجرِمِينَ ﴿٥٨﴾
 إِلَّآ ءَالَ لُوطٍ إِنَّا لَمُنَجُّوهُم أَجمَعِينَ ﴿٥٩﴾
 إِلَّا ٱمرَأَتَهُۥ قَدَّرنَآ ۙ إِنَّهَا لَمِنَ ٱلغَٰبِرِينَ ﴿٦٠﴾
 فَلَمَّا جَآءَ ءَالَ لُوطٍ ٱلمُرسَلُونَ ﴿٦١﴾
 قَالَ إِنَّكُم قَومٌۭ مُّنكَرُونَ ﴿٦٢﴾
 قَالُوا۟ بَل جِئنَـٰكَ بِمَا كَانُوا۟ فِيهِ يَمتَرُونَ ﴿٦٣﴾
 وَأَتَينَـٰكَ بِٱلحَقِّ وَإِنَّا لَصَـٰدِقُونَ ﴿٦٤﴾
 فَأَسرِ بِأَهلِكَ بِقِطعٍۢ مِّنَ ٱلَّيلِ وَٱتَّبِع أَدبَٰرَهُم وَلَا يَلتَفِت مِنكُم أَحَدٌۭ وَٱمضُوا۟ حَيثُ تُؤمَرُونَ ﴿٦٥﴾
 وَقَضَينَآ إِلَيهِ ذَٟلِكَ ٱلأَمرَ أَنَّ دَابِرَ هَـٰٓؤُلَآءِ مَقطُوعٌۭ مُّصبِحِينَ ﴿٦٦﴾
 وَجَآءَ أَهلُ ٱلمَدِينَةِ يَستَبشِرُونَ ﴿٦٧﴾
 قَالَ إِنَّ هَـٰٓؤُلَآءِ ضَيفِى فَلَا تَفضَحُونِ ﴿٦٨﴾
 وَٱتَّقُوا۟ ٱللَّهَ وَلَا تُخزُونِ ﴿٦٩﴾
 قَالُوٓا۟ أَوَلَم نَنهَكَ عَنِ ٱلعَـٰلَمِينَ ﴿٧٠﴾
 قَالَ هَـٰٓؤُلَآءِ بَنَاتِىٓ إِن كُنتُم فَـٰعِلِينَ ﴿٧١﴾
 لَعَمرُكَ إِنَّهُم لَفِى سَكرَتِهِم يَعمَهُونَ ﴿٧٢﴾
 فَأَخَذَتهُمُ ٱلصَّيحَةُ مُشرِقِينَ ﴿٧٣﴾
 فَجَعَلنَا عَـٰلِيَهَا سَافِلَهَا وَأَمطَرنَا عَلَيهِم حِجَارَةًۭ مِّن سِجِّيلٍ ﴿٧٤﴾
 إِنَّ فِى ذَٟلِكَ لَءَايَـٰتٍۢ لِّلمُتَوَسِّمِينَ ﴿٧٥﴾
 وَإِنَّهَا لَبِسَبِيلٍۢ مُّقِيمٍ ﴿٧٦﴾
 إِنَّ فِى ذَٟلِكَ لَءَايَةًۭ لِّلمُؤمِنِينَ ﴿٧٧﴾
 وَإِن كَانَ أَصحَـٰبُ ٱلأَيكَةِ لَظَـٰلِمِينَ ﴿٧٨﴾
 فَٱنتَقَمنَا مِنهُم وَإِنَّهُمَا لَبِإِمَامٍۢ مُّبِينٍۢ ﴿٧٩﴾
 وَلَقَد كَذَّبَ أَصحَـٰبُ ٱلحِجرِ ٱلمُرسَلِينَ ﴿٨٠﴾
 وَءَاتَينَـٰهُم ءَايَـٰتِنَا فَكَانُوا۟ عَنهَا مُعرِضِينَ ﴿٨١﴾
 وَكَانُوا۟ يَنحِتُونَ مِنَ ٱلجِبَالِ بُيُوتًا ءَامِنِينَ ﴿٨٢﴾
 فَأَخَذَتهُمُ ٱلصَّيحَةُ مُصبِحِينَ ﴿٨٣﴾
 فَمَآ أَغنَىٰ عَنهُم مَّا كَانُوا۟ يَكسِبُونَ ﴿٨٤﴾
 وَمَا خَلَقنَا ٱلسَّمَـٰوَٟتِ وَٱلأَرضَ وَمَا بَينَهُمَآ إِلَّا بِٱلحَقِّ ۗ وَإِنَّ ٱلسَّاعَةَ لَءَاتِيَةٌۭ ۖ فَٱصفَحِ ٱلصَّفحَ ٱلجَمِيلَ ﴿٨٥﴾
 إِنَّ رَبَّكَ هُوَ ٱلخَلَّٰقُ ٱلعَلِيمُ ﴿٨٦﴾
 وَلَقَد ءَاتَينَـٰكَ سَبعًۭا مِّنَ ٱلمَثَانِى وَٱلقُرءَانَ ٱلعَظِيمَ ﴿٨٧﴾
 لَا تَمُدَّنَّ عَينَيكَ إِلَىٰ مَا مَتَّعنَا بِهِۦٓ أَزوَٟجًۭا مِّنهُم وَلَا تَحزَن عَلَيهِم وَٱخفِض جَنَاحَكَ لِلمُؤمِنِينَ ﴿٨٨﴾
 وَقُل إِنِّىٓ أَنَا ٱلنَّذِيرُ ٱلمُبِينُ ﴿٨٩﴾
 كَمَآ أَنزَلنَا عَلَى ٱلمُقتَسِمِينَ ﴿٩٠﴾
 ٱلَّذِينَ جَعَلُوا۟ ٱلقُرءَانَ عِضِينَ ﴿٩١﴾
 فَوَرَبِّكَ لَنَسـَٔلَنَّهُم أَجمَعِينَ ﴿٩٢﴾
 عَمَّا كَانُوا۟ يَعمَلُونَ ﴿٩٣﴾
 فَٱصدَع بِمَا تُؤمَرُ وَأَعرِض عَنِ ٱلمُشرِكِينَ ﴿٩٤﴾
 إِنَّا كَفَينَـٰكَ ٱلمُستَهزِءِينَ ﴿٩٥﴾
 ٱلَّذِينَ يَجعَلُونَ مَعَ ٱللَّهِ إِلَـٰهًا ءَاخَرَ ۚ فَسَوفَ يَعلَمُونَ ﴿٩٦﴾
 وَلَقَد نَعلَمُ أَنَّكَ يَضِيقُ صَدرُكَ بِمَا يَقُولُونَ ﴿٩٧﴾
 فَسَبِّح بِحَمدِ رَبِّكَ وَكُن مِّنَ ٱلسَّٰجِدِينَ ﴿٩٨﴾
 وَٱعبُد رَبَّكَ حَتَّىٰ يَأتِيَكَ ٱليَقِينُ ﴿٩٩﴾
 
