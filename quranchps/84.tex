%% License: BSD style (Berkley) (i.e. Put the Copyright owner's name always)
%% Writer and Copyright (to): Bewketu(Bilal) Tadilo (2016-17)
\begin{center}\section{ሱራቱ አልአነሺቃቅ -  \textarabic{سوره  الإنشقاق}}\end{center}
\begin{longtable}{%
  @{}
    p{.5\textwidth}
  @{~~~}
    p{.5\textwidth}
    @{}
}
ቢስሚላሂ አራህመኒ ራሂይም &  \mytextarabic{بِسْمِ ٱللَّهِ ٱلرَّحْمَـٰنِ ٱلرَّحِيمِ}\\
1.\  & \mytextarabic{ إِذَا ٱلسَّمَآءُ ٱنشَقَّتْ ﴿١﴾}\\
2.\  & \mytextarabic{وَأَذِنَتْ لِرَبِّهَا وَحُقَّتْ ﴿٢﴾}\\
3.\  & \mytextarabic{وَإِذَا ٱلْأَرْضُ مُدَّتْ ﴿٣﴾}\\
4.\  & \mytextarabic{وَأَلْقَتْ مَا فِيهَا وَتَخَلَّتْ ﴿٤﴾}\\
5.\  & \mytextarabic{وَأَذِنَتْ لِرَبِّهَا وَحُقَّتْ ﴿٥﴾}\\
6.\  & \mytextarabic{يَـٰٓأَيُّهَا ٱلْإِنسَـٰنُ إِنَّكَ كَادِحٌ إِلَىٰ رَبِّكَ كَدْحًۭا فَمُلَـٰقِيهِ ﴿٦﴾}\\
7.\  & \mytextarabic{فَأَمَّا مَنْ أُوتِىَ كِتَـٰبَهُۥ بِيَمِينِهِۦ ﴿٧﴾}\\
8.\  & \mytextarabic{فَسَوْفَ يُحَاسَبُ حِسَابًۭا يَسِيرًۭا ﴿٨﴾}\\
9.\  & \mytextarabic{وَيَنقَلِبُ إِلَىٰٓ أَهْلِهِۦ مَسْرُورًۭا ﴿٩﴾}\\
10.\  & \mytextarabic{وَأَمَّا مَنْ أُوتِىَ كِتَـٰبَهُۥ وَرَآءَ ظَهْرِهِۦ ﴿١٠﴾}\\
11.\  & \mytextarabic{فَسَوْفَ يَدْعُوا۟ ثُبُورًۭا ﴿١١﴾}\\
12.\  & \mytextarabic{وَيَصْلَىٰ سَعِيرًا ﴿١٢﴾}\\
13.\  & \mytextarabic{إِنَّهُۥ كَانَ فِىٓ أَهْلِهِۦ مَسْرُورًا ﴿١٣﴾}\\
14.\  & \mytextarabic{إِنَّهُۥ ظَنَّ أَن لَّن يَحُورَ ﴿١٤﴾}\\
15.\  & \mytextarabic{بَلَىٰٓ إِنَّ رَبَّهُۥ كَانَ بِهِۦ بَصِيرًۭا ﴿١٥﴾}\\
16.\  & \mytextarabic{فَلَآ أُقْسِمُ بِٱلشَّفَقِ ﴿١٦﴾}\\
17.\  & \mytextarabic{وَٱلَّيْلِ وَمَا وَسَقَ ﴿١٧﴾}\\
18.\  & \mytextarabic{وَٱلْقَمَرِ إِذَا ٱتَّسَقَ ﴿١٨﴾}\\
19.\  & \mytextarabic{لَتَرْكَبُنَّ طَبَقًا عَن طَبَقٍۢ ﴿١٩﴾}\\
20.\  & \mytextarabic{فَمَا لَهُمْ لَا يُؤْمِنُونَ ﴿٢٠﴾}\\
21.\  & \mytextarabic{وَإِذَا قُرِئَ عَلَيْهِمُ ٱلْقُرْءَانُ لَا يَسْجُدُونَ ۩ ﴿٢١﴾}\\
22.\  & \mytextarabic{بَلِ ٱلَّذِينَ كَفَرُوا۟ يُكَذِّبُونَ ﴿٢٢﴾}\\
23.\  & \mytextarabic{وَٱللَّهُ أَعْلَمُ بِمَا يُوعُونَ ﴿٢٣﴾}\\
24.\  & \mytextarabic{فَبَشِّرْهُم بِعَذَابٍ أَلِيمٍ ﴿٢٤﴾}\\
25.\  & \mytextarabic{إِلَّا ٱلَّذِينَ ءَامَنُوا۟ وَعَمِلُوا۟ ٱلصَّـٰلِحَـٰتِ لَهُمْ أَجْرٌ غَيْرُ مَمْنُونٍۭ ﴿٢٥﴾}\\
\end{longtable}
\clearpage