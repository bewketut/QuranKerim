%% License: BSD style (Berkley) (i.e. Put the Copyright owner's name always)
%% Writer and Copyright (to): Bewketu(Bilal) Tadilo (2016-17)
\begin{center}\section{ሱራቱ አልዙኽሩፍ -  \textarabic{سوره  الزخرف}}\end{center}
\begin{longtable}{%
  @{}
    p{.5\textwidth}
  @{~~~}
    p{.5\textwidth}
    @{}
}
ቢስሚላሂ አራህመኒ ራሂይም &  \mytextarabic{بِسْمِ ٱللَّهِ ٱلرَّحْمَـٰنِ ٱلرَّحِيمِ}\\
1.\  & \mytextarabic{ حمٓ ﴿١﴾}\\
2.\  & \mytextarabic{وَٱلْكِتَـٰبِ ٱلْمُبِينِ ﴿٢﴾}\\
3.\  & \mytextarabic{إِنَّا جَعَلْنَـٰهُ قُرْءَٰنًا عَرَبِيًّۭا لَّعَلَّكُمْ تَعْقِلُونَ ﴿٣﴾}\\
4.\  & \mytextarabic{وَإِنَّهُۥ فِىٓ أُمِّ ٱلْكِتَـٰبِ لَدَيْنَا لَعَلِىٌّ حَكِيمٌ ﴿٤﴾}\\
5.\  & \mytextarabic{أَفَنَضْرِبُ عَنكُمُ ٱلذِّكْرَ صَفْحًا أَن كُنتُمْ قَوْمًۭا مُّسْرِفِينَ ﴿٥﴾}\\
6.\  & \mytextarabic{وَكَمْ أَرْسَلْنَا مِن نَّبِىٍّۢ فِى ٱلْأَوَّلِينَ ﴿٦﴾}\\
7.\  & \mytextarabic{وَمَا يَأْتِيهِم مِّن نَّبِىٍّ إِلَّا كَانُوا۟ بِهِۦ يَسْتَهْزِءُونَ ﴿٧﴾}\\
8.\  & \mytextarabic{فَأَهْلَكْنَآ أَشَدَّ مِنْهُم بَطْشًۭا وَمَضَىٰ مَثَلُ ٱلْأَوَّلِينَ ﴿٨﴾}\\
9.\  & \mytextarabic{وَلَئِن سَأَلْتَهُم مَّنْ خَلَقَ ٱلسَّمَـٰوَٟتِ وَٱلْأَرْضَ لَيَقُولُنَّ خَلَقَهُنَّ ٱلْعَزِيزُ ٱلْعَلِيمُ ﴿٩﴾}\\
10.\  & \mytextarabic{ٱلَّذِى جَعَلَ لَكُمُ ٱلْأَرْضَ مَهْدًۭا وَجَعَلَ لَكُمْ فِيهَا سُبُلًۭا لَّعَلَّكُمْ تَهْتَدُونَ ﴿١٠﴾}\\
11.\  & \mytextarabic{وَٱلَّذِى نَزَّلَ مِنَ ٱلسَّمَآءِ مَآءًۢ بِقَدَرٍۢ فَأَنشَرْنَا بِهِۦ بَلْدَةًۭ مَّيْتًۭا ۚ كَذَٟلِكَ تُخْرَجُونَ ﴿١١﴾}\\
12.\  & \mytextarabic{وَٱلَّذِى خَلَقَ ٱلْأَزْوَٟجَ كُلَّهَا وَجَعَلَ لَكُم مِّنَ ٱلْفُلْكِ وَٱلْأَنْعَـٰمِ مَا تَرْكَبُونَ ﴿١٢﴾}\\
13.\  & \mytextarabic{لِتَسْتَوُۥا۟ عَلَىٰ ظُهُورِهِۦ ثُمَّ تَذْكُرُوا۟ نِعْمَةَ رَبِّكُمْ إِذَا ٱسْتَوَيْتُمْ عَلَيْهِ وَتَقُولُوا۟ سُبْحَـٰنَ ٱلَّذِى سَخَّرَ لَنَا هَـٰذَا وَمَا كُنَّا لَهُۥ مُقْرِنِينَ ﴿١٣﴾}\\
14.\  & \mytextarabic{وَإِنَّآ إِلَىٰ رَبِّنَا لَمُنقَلِبُونَ ﴿١٤﴾}\\
15.\  & \mytextarabic{وَجَعَلُوا۟ لَهُۥ مِنْ عِبَادِهِۦ جُزْءًا ۚ إِنَّ ٱلْإِنسَـٰنَ لَكَفُورٌۭ مُّبِينٌ ﴿١٥﴾}\\
16.\  & \mytextarabic{أَمِ ٱتَّخَذَ مِمَّا يَخْلُقُ بَنَاتٍۢ وَأَصْفَىٰكُم بِٱلْبَنِينَ ﴿١٦﴾}\\
17.\  & \mytextarabic{وَإِذَا بُشِّرَ أَحَدُهُم بِمَا ضَرَبَ لِلرَّحْمَـٰنِ مَثَلًۭا ظَلَّ وَجْهُهُۥ مُسْوَدًّۭا وَهُوَ كَظِيمٌ ﴿١٧﴾}\\
18.\  & \mytextarabic{أَوَمَن يُنَشَّؤُا۟ فِى ٱلْحِلْيَةِ وَهُوَ فِى ٱلْخِصَامِ غَيْرُ مُبِينٍۢ ﴿١٨﴾}\\
19.\  & \mytextarabic{وَجَعَلُوا۟ ٱلْمَلَـٰٓئِكَةَ ٱلَّذِينَ هُمْ عِبَٰدُ ٱلرَّحْمَـٰنِ إِنَـٰثًا ۚ أَشَهِدُوا۟ خَلْقَهُمْ ۚ سَتُكْتَبُ شَهَـٰدَتُهُمْ وَيُسْـَٔلُونَ ﴿١٩﴾}\\
20.\  & \mytextarabic{وَقَالُوا۟ لَوْ شَآءَ ٱلرَّحْمَـٰنُ مَا عَبَدْنَـٰهُم ۗ مَّا لَهُم بِذَٟلِكَ مِنْ عِلْمٍ ۖ إِنْ هُمْ إِلَّا يَخْرُصُونَ ﴿٢٠﴾}\\
21.\  & \mytextarabic{أَمْ ءَاتَيْنَـٰهُمْ كِتَـٰبًۭا مِّن قَبْلِهِۦ فَهُم بِهِۦ مُسْتَمْسِكُونَ ﴿٢١﴾}\\
22.\  & \mytextarabic{بَلْ قَالُوٓا۟ إِنَّا وَجَدْنَآ ءَابَآءَنَا عَلَىٰٓ أُمَّةٍۢ وَإِنَّا عَلَىٰٓ ءَاثَـٰرِهِم مُّهْتَدُونَ ﴿٢٢﴾}\\
23.\  & \mytextarabic{وَكَذَٟلِكَ مَآ أَرْسَلْنَا مِن قَبْلِكَ فِى قَرْيَةٍۢ مِّن نَّذِيرٍ إِلَّا قَالَ مُتْرَفُوهَآ إِنَّا وَجَدْنَآ ءَابَآءَنَا عَلَىٰٓ أُمَّةٍۢ وَإِنَّا عَلَىٰٓ ءَاثَـٰرِهِم مُّقْتَدُونَ ﴿٢٣﴾}\\
24.\  & \mytextarabic{۞ قَـٰلَ أَوَلَوْ جِئْتُكُم بِأَهْدَىٰ مِمَّا وَجَدتُّمْ عَلَيْهِ ءَابَآءَكُمْ ۖ قَالُوٓا۟ إِنَّا بِمَآ أُرْسِلْتُم بِهِۦ كَـٰفِرُونَ ﴿٢٤﴾}\\
25.\  & \mytextarabic{فَٱنتَقَمْنَا مِنْهُمْ ۖ فَٱنظُرْ كَيْفَ كَانَ عَـٰقِبَةُ ٱلْمُكَذِّبِينَ ﴿٢٥﴾}\\
26.\  & \mytextarabic{وَإِذْ قَالَ إِبْرَٰهِيمُ لِأَبِيهِ وَقَوْمِهِۦٓ إِنَّنِى بَرَآءٌۭ مِّمَّا تَعْبُدُونَ ﴿٢٦﴾}\\
27.\  & \mytextarabic{إِلَّا ٱلَّذِى فَطَرَنِى فَإِنَّهُۥ سَيَهْدِينِ ﴿٢٧﴾}\\
28.\  & \mytextarabic{وَجَعَلَهَا كَلِمَةًۢ بَاقِيَةًۭ فِى عَقِبِهِۦ لَعَلَّهُمْ يَرْجِعُونَ ﴿٢٨﴾}\\
29.\  & \mytextarabic{بَلْ مَتَّعْتُ هَـٰٓؤُلَآءِ وَءَابَآءَهُمْ حَتَّىٰ جَآءَهُمُ ٱلْحَقُّ وَرَسُولٌۭ مُّبِينٌۭ ﴿٢٩﴾}\\
30.\  & \mytextarabic{وَلَمَّا جَآءَهُمُ ٱلْحَقُّ قَالُوا۟ هَـٰذَا سِحْرٌۭ وَإِنَّا بِهِۦ كَـٰفِرُونَ ﴿٣٠﴾}\\
31.\  & \mytextarabic{وَقَالُوا۟ لَوْلَا نُزِّلَ هَـٰذَا ٱلْقُرْءَانُ عَلَىٰ رَجُلٍۢ مِّنَ ٱلْقَرْيَتَيْنِ عَظِيمٍ ﴿٣١﴾}\\
32.\  & \mytextarabic{أَهُمْ يَقْسِمُونَ رَحْمَتَ رَبِّكَ ۚ نَحْنُ قَسَمْنَا بَيْنَهُم مَّعِيشَتَهُمْ فِى ٱلْحَيَوٰةِ ٱلدُّنْيَا ۚ وَرَفَعْنَا بَعْضَهُمْ فَوْقَ بَعْضٍۢ دَرَجَٰتٍۢ لِّيَتَّخِذَ بَعْضُهُم بَعْضًۭا سُخْرِيًّۭا ۗ وَرَحْمَتُ رَبِّكَ خَيْرٌۭ مِّمَّا يَجْمَعُونَ ﴿٣٢﴾}\\
33.\  & \mytextarabic{وَلَوْلَآ أَن يَكُونَ ٱلنَّاسُ أُمَّةًۭ وَٟحِدَةًۭ لَّجَعَلْنَا لِمَن يَكْفُرُ بِٱلرَّحْمَـٰنِ لِبُيُوتِهِمْ سُقُفًۭا مِّن فِضَّةٍۢ وَمَعَارِجَ عَلَيْهَا يَظْهَرُونَ ﴿٣٣﴾}\\
34.\  & \mytextarabic{وَلِبُيُوتِهِمْ أَبْوَٟبًۭا وَسُرُرًا عَلَيْهَا يَتَّكِـُٔونَ ﴿٣٤﴾}\\
35.\  & \mytextarabic{وَزُخْرُفًۭا ۚ وَإِن كُلُّ ذَٟلِكَ لَمَّا مَتَـٰعُ ٱلْحَيَوٰةِ ٱلدُّنْيَا ۚ وَٱلْءَاخِرَةُ عِندَ رَبِّكَ لِلْمُتَّقِينَ ﴿٣٥﴾}\\
36.\  & \mytextarabic{وَمَن يَعْشُ عَن ذِكْرِ ٱلرَّحْمَـٰنِ نُقَيِّضْ لَهُۥ شَيْطَٰنًۭا فَهُوَ لَهُۥ قَرِينٌۭ ﴿٣٦﴾}\\
37.\  & \mytextarabic{وَإِنَّهُمْ لَيَصُدُّونَهُمْ عَنِ ٱلسَّبِيلِ وَيَحْسَبُونَ أَنَّهُم مُّهْتَدُونَ ﴿٣٧﴾}\\
38.\  & \mytextarabic{حَتَّىٰٓ إِذَا جَآءَنَا قَالَ يَـٰلَيْتَ بَيْنِى وَبَيْنَكَ بُعْدَ ٱلْمَشْرِقَيْنِ فَبِئْسَ ٱلْقَرِينُ ﴿٣٨﴾}\\
39.\  & \mytextarabic{وَلَن يَنفَعَكُمُ ٱلْيَوْمَ إِذ ظَّلَمْتُمْ أَنَّكُمْ فِى ٱلْعَذَابِ مُشْتَرِكُونَ ﴿٣٩﴾}\\
40.\  & \mytextarabic{أَفَأَنتَ تُسْمِعُ ٱلصُّمَّ أَوْ تَهْدِى ٱلْعُمْىَ وَمَن كَانَ فِى ضَلَـٰلٍۢ مُّبِينٍۢ ﴿٤٠﴾}\\
41.\  & \mytextarabic{فَإِمَّا نَذْهَبَنَّ بِكَ فَإِنَّا مِنْهُم مُّنتَقِمُونَ ﴿٤١﴾}\\
42.\  & \mytextarabic{أَوْ نُرِيَنَّكَ ٱلَّذِى وَعَدْنَـٰهُمْ فَإِنَّا عَلَيْهِم مُّقْتَدِرُونَ ﴿٤٢﴾}\\
43.\  & \mytextarabic{فَٱسْتَمْسِكْ بِٱلَّذِىٓ أُوحِىَ إِلَيْكَ ۖ إِنَّكَ عَلَىٰ صِرَٰطٍۢ مُّسْتَقِيمٍۢ ﴿٤٣﴾}\\
44.\  & \mytextarabic{وَإِنَّهُۥ لَذِكْرٌۭ لَّكَ وَلِقَوْمِكَ ۖ وَسَوْفَ تُسْـَٔلُونَ ﴿٤٤﴾}\\
45.\  & \mytextarabic{وَسْـَٔلْ مَنْ أَرْسَلْنَا مِن قَبْلِكَ مِن رُّسُلِنَآ أَجَعَلْنَا مِن دُونِ ٱلرَّحْمَـٰنِ ءَالِهَةًۭ يُعْبَدُونَ ﴿٤٥﴾}\\
46.\  & \mytextarabic{وَلَقَدْ أَرْسَلْنَا مُوسَىٰ بِـَٔايَـٰتِنَآ إِلَىٰ فِرْعَوْنَ وَمَلَإِي۟هِۦ فَقَالَ إِنِّى رَسُولُ رَبِّ ٱلْعَـٰلَمِينَ ﴿٤٦﴾}\\
47.\  & \mytextarabic{فَلَمَّا جَآءَهُم بِـَٔايَـٰتِنَآ إِذَا هُم مِّنْهَا يَضْحَكُونَ ﴿٤٧﴾}\\
48.\  & \mytextarabic{وَمَا نُرِيهِم مِّنْ ءَايَةٍ إِلَّا هِىَ أَكْبَرُ مِنْ أُخْتِهَا ۖ وَأَخَذْنَـٰهُم بِٱلْعَذَابِ لَعَلَّهُمْ يَرْجِعُونَ ﴿٤٨﴾}\\
49.\  & \mytextarabic{وَقَالُوا۟ يَـٰٓأَيُّهَ ٱلسَّاحِرُ ٱدْعُ لَنَا رَبَّكَ بِمَا عَهِدَ عِندَكَ إِنَّنَا لَمُهْتَدُونَ ﴿٤٩﴾}\\
50.\  & \mytextarabic{فَلَمَّا كَشَفْنَا عَنْهُمُ ٱلْعَذَابَ إِذَا هُمْ يَنكُثُونَ ﴿٥٠﴾}\\
51.\  & \mytextarabic{وَنَادَىٰ فِرْعَوْنُ فِى قَوْمِهِۦ قَالَ يَـٰقَوْمِ أَلَيْسَ لِى مُلْكُ مِصْرَ وَهَـٰذِهِ ٱلْأَنْهَـٰرُ تَجْرِى مِن تَحْتِىٓ ۖ أَفَلَا تُبْصِرُونَ ﴿٥١﴾}\\
52.\  & \mytextarabic{أَمْ أَنَا۠ خَيْرٌۭ مِّنْ هَـٰذَا ٱلَّذِى هُوَ مَهِينٌۭ وَلَا يَكَادُ يُبِينُ ﴿٥٢﴾}\\
53.\  & \mytextarabic{فَلَوْلَآ أُلْقِىَ عَلَيْهِ أَسْوِرَةٌۭ مِّن ذَهَبٍ أَوْ جَآءَ مَعَهُ ٱلْمَلَـٰٓئِكَةُ مُقْتَرِنِينَ ﴿٥٣﴾}\\
54.\  & \mytextarabic{فَٱسْتَخَفَّ قَوْمَهُۥ فَأَطَاعُوهُ ۚ إِنَّهُمْ كَانُوا۟ قَوْمًۭا فَـٰسِقِينَ ﴿٥٤﴾}\\
55.\  & \mytextarabic{فَلَمَّآ ءَاسَفُونَا ٱنتَقَمْنَا مِنْهُمْ فَأَغْرَقْنَـٰهُمْ أَجْمَعِينَ ﴿٥٥﴾}\\
56.\  & \mytextarabic{فَجَعَلْنَـٰهُمْ سَلَفًۭا وَمَثَلًۭا لِّلْءَاخِرِينَ ﴿٥٦﴾}\\
57.\  & \mytextarabic{۞ وَلَمَّا ضُرِبَ ٱبْنُ مَرْيَمَ مَثَلًا إِذَا قَوْمُكَ مِنْهُ يَصِدُّونَ ﴿٥٧﴾}\\
58.\  & \mytextarabic{وَقَالُوٓا۟ ءَأَٰلِهَتُنَا خَيْرٌ أَمْ هُوَ ۚ مَا ضَرَبُوهُ لَكَ إِلَّا جَدَلًۢا ۚ بَلْ هُمْ قَوْمٌ خَصِمُونَ ﴿٥٨﴾}\\
59.\  & \mytextarabic{إِنْ هُوَ إِلَّا عَبْدٌ أَنْعَمْنَا عَلَيْهِ وَجَعَلْنَـٰهُ مَثَلًۭا لِّبَنِىٓ إِسْرَٰٓءِيلَ ﴿٥٩﴾}\\
60.\  & \mytextarabic{وَلَوْ نَشَآءُ لَجَعَلْنَا مِنكُم مَّلَـٰٓئِكَةًۭ فِى ٱلْأَرْضِ يَخْلُفُونَ ﴿٦٠﴾}\\
61.\  & \mytextarabic{وَإِنَّهُۥ لَعِلْمٌۭ لِّلسَّاعَةِ فَلَا تَمْتَرُنَّ بِهَا وَٱتَّبِعُونِ ۚ هَـٰذَا صِرَٰطٌۭ مُّسْتَقِيمٌۭ ﴿٦١﴾}\\
62.\  & \mytextarabic{وَلَا يَصُدَّنَّكُمُ ٱلشَّيْطَٰنُ ۖ إِنَّهُۥ لَكُمْ عَدُوٌّۭ مُّبِينٌۭ ﴿٦٢﴾}\\
63.\  & \mytextarabic{وَلَمَّا جَآءَ عِيسَىٰ بِٱلْبَيِّنَـٰتِ قَالَ قَدْ جِئْتُكُم بِٱلْحِكْمَةِ وَلِأُبَيِّنَ لَكُم بَعْضَ ٱلَّذِى تَخْتَلِفُونَ فِيهِ ۖ فَٱتَّقُوا۟ ٱللَّهَ وَأَطِيعُونِ ﴿٦٣﴾}\\
64.\  & \mytextarabic{إِنَّ ٱللَّهَ هُوَ رَبِّى وَرَبُّكُمْ فَٱعْبُدُوهُ ۚ هَـٰذَا صِرَٰطٌۭ مُّسْتَقِيمٌۭ ﴿٦٤﴾}\\
65.\  & \mytextarabic{فَٱخْتَلَفَ ٱلْأَحْزَابُ مِنۢ بَيْنِهِمْ ۖ فَوَيْلٌۭ لِّلَّذِينَ ظَلَمُوا۟ مِنْ عَذَابِ يَوْمٍ أَلِيمٍ ﴿٦٥﴾}\\
66.\  & \mytextarabic{هَلْ يَنظُرُونَ إِلَّا ٱلسَّاعَةَ أَن تَأْتِيَهُم بَغْتَةًۭ وَهُمْ لَا يَشْعُرُونَ ﴿٦٦﴾}\\
67.\  & \mytextarabic{ٱلْأَخِلَّآءُ يَوْمَئِذٍۭ بَعْضُهُمْ لِبَعْضٍ عَدُوٌّ إِلَّا ٱلْمُتَّقِينَ ﴿٦٧﴾}\\
68.\  & \mytextarabic{يَـٰعِبَادِ لَا خَوْفٌ عَلَيْكُمُ ٱلْيَوْمَ وَلَآ أَنتُمْ تَحْزَنُونَ ﴿٦٨﴾}\\
69.\  & \mytextarabic{ٱلَّذِينَ ءَامَنُوا۟ بِـَٔايَـٰتِنَا وَكَانُوا۟ مُسْلِمِينَ ﴿٦٩﴾}\\
70.\  & \mytextarabic{ٱدْخُلُوا۟ ٱلْجَنَّةَ أَنتُمْ وَأَزْوَٟجُكُمْ تُحْبَرُونَ ﴿٧٠﴾}\\
71.\  & \mytextarabic{يُطَافُ عَلَيْهِم بِصِحَافٍۢ مِّن ذَهَبٍۢ وَأَكْوَابٍۢ ۖ وَفِيهَا مَا تَشْتَهِيهِ ٱلْأَنفُسُ وَتَلَذُّ ٱلْأَعْيُنُ ۖ وَأَنتُمْ فِيهَا خَـٰلِدُونَ ﴿٧١﴾}\\
72.\  & \mytextarabic{وَتِلْكَ ٱلْجَنَّةُ ٱلَّتِىٓ أُورِثْتُمُوهَا بِمَا كُنتُمْ تَعْمَلُونَ ﴿٧٢﴾}\\
73.\  & \mytextarabic{لَكُمْ فِيهَا فَـٰكِهَةٌۭ كَثِيرَةٌۭ مِّنْهَا تَأْكُلُونَ ﴿٧٣﴾}\\
74.\  & \mytextarabic{إِنَّ ٱلْمُجْرِمِينَ فِى عَذَابِ جَهَنَّمَ خَـٰلِدُونَ ﴿٧٤﴾}\\
75.\  & \mytextarabic{لَا يُفَتَّرُ عَنْهُمْ وَهُمْ فِيهِ مُبْلِسُونَ ﴿٧٥﴾}\\
76.\  & \mytextarabic{وَمَا ظَلَمْنَـٰهُمْ وَلَـٰكِن كَانُوا۟ هُمُ ٱلظَّـٰلِمِينَ ﴿٧٦﴾}\\
77.\  & \mytextarabic{وَنَادَوْا۟ يَـٰمَـٰلِكُ لِيَقْضِ عَلَيْنَا رَبُّكَ ۖ قَالَ إِنَّكُم مَّٰكِثُونَ ﴿٧٧﴾}\\
78.\  & \mytextarabic{لَقَدْ جِئْنَـٰكُم بِٱلْحَقِّ وَلَـٰكِنَّ أَكْثَرَكُمْ لِلْحَقِّ كَـٰرِهُونَ ﴿٧٨﴾}\\
79.\  & \mytextarabic{أَمْ أَبْرَمُوٓا۟ أَمْرًۭا فَإِنَّا مُبْرِمُونَ ﴿٧٩﴾}\\
80.\  & \mytextarabic{أَمْ يَحْسَبُونَ أَنَّا لَا نَسْمَعُ سِرَّهُمْ وَنَجْوَىٰهُم ۚ بَلَىٰ وَرُسُلُنَا لَدَيْهِمْ يَكْتُبُونَ ﴿٨٠﴾}\\
81.\  & \mytextarabic{قُلْ إِن كَانَ لِلرَّحْمَـٰنِ وَلَدٌۭ فَأَنَا۠ أَوَّلُ ٱلْعَـٰبِدِينَ ﴿٨١﴾}\\
82.\  & \mytextarabic{سُبْحَـٰنَ رَبِّ ٱلسَّمَـٰوَٟتِ وَٱلْأَرْضِ رَبِّ ٱلْعَرْشِ عَمَّا يَصِفُونَ ﴿٨٢﴾}\\
83.\  & \mytextarabic{فَذَرْهُمْ يَخُوضُوا۟ وَيَلْعَبُوا۟ حَتَّىٰ يُلَـٰقُوا۟ يَوْمَهُمُ ٱلَّذِى يُوعَدُونَ ﴿٨٣﴾}\\
84.\  & \mytextarabic{وَهُوَ ٱلَّذِى فِى ٱلسَّمَآءِ إِلَـٰهٌۭ وَفِى ٱلْأَرْضِ إِلَـٰهٌۭ ۚ وَهُوَ ٱلْحَكِيمُ ٱلْعَلِيمُ ﴿٨٤﴾}\\
85.\  & \mytextarabic{وَتَبَارَكَ ٱلَّذِى لَهُۥ مُلْكُ ٱلسَّمَـٰوَٟتِ وَٱلْأَرْضِ وَمَا بَيْنَهُمَا وَعِندَهُۥ عِلْمُ ٱلسَّاعَةِ وَإِلَيْهِ تُرْجَعُونَ ﴿٨٥﴾}\\
86.\  & \mytextarabic{وَلَا يَمْلِكُ ٱلَّذِينَ يَدْعُونَ مِن دُونِهِ ٱلشَّفَـٰعَةَ إِلَّا مَن شَهِدَ بِٱلْحَقِّ وَهُمْ يَعْلَمُونَ ﴿٨٦﴾}\\
87.\  & \mytextarabic{وَلَئِن سَأَلْتَهُم مَّنْ خَلَقَهُمْ لَيَقُولُنَّ ٱللَّهُ ۖ فَأَنَّىٰ يُؤْفَكُونَ ﴿٨٧﴾}\\
88.\  & \mytextarabic{وَقِيلِهِۦ يَـٰرَبِّ إِنَّ هَـٰٓؤُلَآءِ قَوْمٌۭ لَّا يُؤْمِنُونَ ﴿٨٨﴾}\\
89.\  & \mytextarabic{فَٱصْفَحْ عَنْهُمْ وَقُلْ سَلَـٰمٌۭ ۚ فَسَوْفَ يَعْلَمُونَ ﴿٨٩﴾}\\
\end{longtable}
\clearpage