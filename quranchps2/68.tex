%% License: BSD style (Berkley) (i.e. Put the Copyright owner's name always)
%% Writer and Copyright (to): Bewketu(Bilal) Tadilo (2016-17)
\begin{center}\section{\LR{\textamhsec{ሱራቱ አልቀለም -}  \textarabic{سوره  القلم}}}\end{center}
\begin{longtable}{%
  @{}
    p{.5\textwidth}
  @{~~~}
    p{.5\textwidth}
    @{}
}
\textamh{ቢስሚላሂ አራህመኒ ራሂይም } &  \mytextarabic{بِسْمِ ٱللَّهِ ٱلرَّحْمَـٰنِ ٱلرَّحِيمِ}\\
\textamh{1.\  } & \mytextarabic{ نٓ ۚ وَٱلْقَلَمِ وَمَا يَسْطُرُونَ ﴿١﴾}\\
\textamh{2.\  } & \mytextarabic{مَآ أَنتَ بِنِعْمَةِ رَبِّكَ بِمَجْنُونٍۢ ﴿٢﴾}\\
\textamh{3.\  } & \mytextarabic{وَإِنَّ لَكَ لَأَجْرًا غَيْرَ مَمْنُونٍۢ ﴿٣﴾}\\
\textamh{4.\  } & \mytextarabic{وَإِنَّكَ لَعَلَىٰ خُلُقٍ عَظِيمٍۢ ﴿٤﴾}\\
\textamh{5.\  } & \mytextarabic{فَسَتُبْصِرُ وَيُبْصِرُونَ ﴿٥﴾}\\
\textamh{6.\  } & \mytextarabic{بِأَييِّكُمُ ٱلْمَفْتُونُ ﴿٦﴾}\\
\textamh{7.\  } & \mytextarabic{إِنَّ رَبَّكَ هُوَ أَعْلَمُ بِمَن ضَلَّ عَن سَبِيلِهِۦ وَهُوَ أَعْلَمُ بِٱلْمُهْتَدِينَ ﴿٧﴾}\\
\textamh{8.\  } & \mytextarabic{فَلَا تُطِعِ ٱلْمُكَذِّبِينَ ﴿٨﴾}\\
\textamh{9.\  } & \mytextarabic{وَدُّوا۟ لَوْ تُدْهِنُ فَيُدْهِنُونَ ﴿٩﴾}\\
\textamh{10.\  } & \mytextarabic{وَلَا تُطِعْ كُلَّ حَلَّافٍۢ مَّهِينٍ ﴿١٠﴾}\\
\textamh{11.\  } & \mytextarabic{هَمَّازٍۢ مَّشَّآءٍۭ بِنَمِيمٍۢ ﴿١١﴾}\\
\textamh{12.\  } & \mytextarabic{مَّنَّاعٍۢ لِّلْخَيْرِ مُعْتَدٍ أَثِيمٍ ﴿١٢﴾}\\
\textamh{13.\  } & \mytextarabic{عُتُلٍّۭ بَعْدَ ذَٟلِكَ زَنِيمٍ ﴿١٣﴾}\\
\textamh{14.\  } & \mytextarabic{أَن كَانَ ذَا مَالٍۢ وَبَنِينَ ﴿١٤﴾}\\
\textamh{15.\  } & \mytextarabic{إِذَا تُتْلَىٰ عَلَيْهِ ءَايَـٰتُنَا قَالَ أَسَـٰطِيرُ ٱلْأَوَّلِينَ ﴿١٥﴾}\\
\textamh{16.\  } & \mytextarabic{سَنَسِمُهُۥ عَلَى ٱلْخُرْطُومِ ﴿١٦﴾}\\
\textamh{17.\  } & \mytextarabic{إِنَّا بَلَوْنَـٰهُمْ كَمَا بَلَوْنَآ أَصْحَـٰبَ ٱلْجَنَّةِ إِذْ أَقْسَمُوا۟ لَيَصْرِمُنَّهَا مُصْبِحِينَ ﴿١٧﴾}\\
\textamh{18.\  } & \mytextarabic{وَلَا يَسْتَثْنُونَ ﴿١٨﴾}\\
\textamh{19.\  } & \mytextarabic{فَطَافَ عَلَيْهَا طَآئِفٌۭ مِّن رَّبِّكَ وَهُمْ نَآئِمُونَ ﴿١٩﴾}\\
\textamh{20.\  } & \mytextarabic{فَأَصْبَحَتْ كَٱلصَّرِيمِ ﴿٢٠﴾}\\
\textamh{21.\  } & \mytextarabic{فَتَنَادَوْا۟ مُصْبِحِينَ ﴿٢١﴾}\\
\textamh{22.\  } & \mytextarabic{أَنِ ٱغْدُوا۟ عَلَىٰ حَرْثِكُمْ إِن كُنتُمْ صَـٰرِمِينَ ﴿٢٢﴾}\\
\textamh{23.\  } & \mytextarabic{فَٱنطَلَقُوا۟ وَهُمْ يَتَخَـٰفَتُونَ ﴿٢٣﴾}\\
\textamh{24.\  } & \mytextarabic{أَن لَّا يَدْخُلَنَّهَا ٱلْيَوْمَ عَلَيْكُم مِّسْكِينٌۭ ﴿٢٤﴾}\\
\textamh{25.\  } & \mytextarabic{وَغَدَوْا۟ عَلَىٰ حَرْدٍۢ قَـٰدِرِينَ ﴿٢٥﴾}\\
\textamh{26.\  } & \mytextarabic{فَلَمَّا رَأَوْهَا قَالُوٓا۟ إِنَّا لَضَآلُّونَ ﴿٢٦﴾}\\
\textamh{27.\  } & \mytextarabic{بَلْ نَحْنُ مَحْرُومُونَ ﴿٢٧﴾}\\
\textamh{28.\  } & \mytextarabic{قَالَ أَوْسَطُهُمْ أَلَمْ أَقُل لَّكُمْ لَوْلَا تُسَبِّحُونَ ﴿٢٨﴾}\\
\textamh{29.\  } & \mytextarabic{قَالُوا۟ سُبْحَـٰنَ رَبِّنَآ إِنَّا كُنَّا ظَـٰلِمِينَ ﴿٢٩﴾}\\
\textamh{30.\  } & \mytextarabic{فَأَقْبَلَ بَعْضُهُمْ عَلَىٰ بَعْضٍۢ يَتَلَـٰوَمُونَ ﴿٣٠﴾}\\
\textamh{31.\  } & \mytextarabic{قَالُوا۟ يَـٰوَيْلَنَآ إِنَّا كُنَّا طَٰغِينَ ﴿٣١﴾}\\
\textamh{32.\  } & \mytextarabic{عَسَىٰ رَبُّنَآ أَن يُبْدِلَنَا خَيْرًۭا مِّنْهَآ إِنَّآ إِلَىٰ رَبِّنَا رَٰغِبُونَ ﴿٣٢﴾}\\
\textamh{33.\  } & \mytextarabic{كَذَٟلِكَ ٱلْعَذَابُ ۖ وَلَعَذَابُ ٱلْءَاخِرَةِ أَكْبَرُ ۚ لَوْ كَانُوا۟ يَعْلَمُونَ ﴿٣٣﴾}\\
\textamh{34.\  } & \mytextarabic{إِنَّ لِلْمُتَّقِينَ عِندَ رَبِّهِمْ جَنَّـٰتِ ٱلنَّعِيمِ ﴿٣٤﴾}\\
\textamh{35.\  } & \mytextarabic{أَفَنَجْعَلُ ٱلْمُسْلِمِينَ كَٱلْمُجْرِمِينَ ﴿٣٥﴾}\\
\textamh{36.\  } & \mytextarabic{مَا لَكُمْ كَيْفَ تَحْكُمُونَ ﴿٣٦﴾}\\
\textamh{37.\  } & \mytextarabic{أَمْ لَكُمْ كِتَـٰبٌۭ فِيهِ تَدْرُسُونَ ﴿٣٧﴾}\\
\textamh{38.\  } & \mytextarabic{إِنَّ لَكُمْ فِيهِ لَمَا تَخَيَّرُونَ ﴿٣٨﴾}\\
\textamh{39.\  } & \mytextarabic{أَمْ لَكُمْ أَيْمَـٰنٌ عَلَيْنَا بَٰلِغَةٌ إِلَىٰ يَوْمِ ٱلْقِيَـٰمَةِ ۙ إِنَّ لَكُمْ لَمَا تَحْكُمُونَ ﴿٣٩﴾}\\
\textamh{40.\  } & \mytextarabic{سَلْهُمْ أَيُّهُم بِذَٟلِكَ زَعِيمٌ ﴿٤٠﴾}\\
\textamh{41.\  } & \mytextarabic{أَمْ لَهُمْ شُرَكَآءُ فَلْيَأْتُوا۟ بِشُرَكَآئِهِمْ إِن كَانُوا۟ صَـٰدِقِينَ ﴿٤١﴾}\\
\textamh{42.\  } & \mytextarabic{يَوْمَ يُكْشَفُ عَن سَاقٍۢ وَيُدْعَوْنَ إِلَى ٱلسُّجُودِ فَلَا يَسْتَطِيعُونَ ﴿٤٢﴾}\\
\textamh{43.\  } & \mytextarabic{خَـٰشِعَةً أَبْصَـٰرُهُمْ تَرْهَقُهُمْ ذِلَّةٌۭ ۖ وَقَدْ كَانُوا۟ يُدْعَوْنَ إِلَى ٱلسُّجُودِ وَهُمْ سَـٰلِمُونَ ﴿٤٣﴾}\\
\textamh{44.\  } & \mytextarabic{فَذَرْنِى وَمَن يُكَذِّبُ بِهَـٰذَا ٱلْحَدِيثِ ۖ سَنَسْتَدْرِجُهُم مِّنْ حَيْثُ لَا يَعْلَمُونَ ﴿٤٤﴾}\\
\textamh{45.\  } & \mytextarabic{وَأُمْلِى لَهُمْ ۚ إِنَّ كَيْدِى مَتِينٌ ﴿٤٥﴾}\\
\textamh{46.\  } & \mytextarabic{أَمْ تَسْـَٔلُهُمْ أَجْرًۭا فَهُم مِّن مَّغْرَمٍۢ مُّثْقَلُونَ ﴿٤٦﴾}\\
\textamh{47.\  } & \mytextarabic{أَمْ عِندَهُمُ ٱلْغَيْبُ فَهُمْ يَكْتُبُونَ ﴿٤٧﴾}\\
\textamh{48.\  } & \mytextarabic{فَٱصْبِرْ لِحُكْمِ رَبِّكَ وَلَا تَكُن كَصَاحِبِ ٱلْحُوتِ إِذْ نَادَىٰ وَهُوَ مَكْظُومٌۭ ﴿٤٨﴾}\\
\textamh{49.\  } & \mytextarabic{لَّوْلَآ أَن تَدَٟرَكَهُۥ نِعْمَةٌۭ مِّن رَّبِّهِۦ لَنُبِذَ بِٱلْعَرَآءِ وَهُوَ مَذْمُومٌۭ ﴿٤٩﴾}\\
\textamh{50.\  } & \mytextarabic{فَٱجْتَبَٰهُ رَبُّهُۥ فَجَعَلَهُۥ مِنَ ٱلصَّـٰلِحِينَ ﴿٥٠﴾}\\
\textamh{51.\  } & \mytextarabic{وَإِن يَكَادُ ٱلَّذِينَ كَفَرُوا۟ لَيُزْلِقُونَكَ بِأَبْصَـٰرِهِمْ لَمَّا سَمِعُوا۟ ٱلذِّكْرَ وَيَقُولُونَ إِنَّهُۥ لَمَجْنُونٌۭ ﴿٥١﴾}\\
\textamh{52.\  } & \mytextarabic{وَمَا هُوَ إِلَّا ذِكْرٌۭ لِّلْعَـٰلَمِينَ ﴿٥٢﴾}\\
\end{longtable}
\clearpage