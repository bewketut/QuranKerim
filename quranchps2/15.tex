%% License: BSD style (Berkley) (i.e. Put the Copyright owner's name always)
%% Writer and Copyright (to): Bewketu(Bilal) Tadilo (2016-17)
\centering\section{\LR{\textamharic{ሱራቱ አልሂጅር -}  \RL{سوره  الحجر}}}
\begin{longtable}{%
  @{}
    p{.5\textwidth}
  @{~~~~~~~~~~~~}
    p{.5\textwidth}
    @{}
}
\nopagebreak
\textamh{ቢስሚላሂ አራህመኒ ራሂይም } &  بِسْمِ ٱللَّهِ ٱلرَّحْمَـٰنِ ٱلرَّحِيمِ\\
\textamh{1.\  } &  الٓر ۚ تِلْكَ ءَايَـٰتُ ٱلْكِتَـٰبِ وَقُرْءَانٍۢ مُّبِينٍۢ ﴿١﴾\\
\textamh{2.\  } & رُّبَمَا يَوَدُّ ٱلَّذِينَ كَفَرُوا۟ لَوْ كَانُوا۟ مُسْلِمِينَ ﴿٢﴾\\
\textamh{3.\  } & ذَرْهُمْ يَأْكُلُوا۟ وَيَتَمَتَّعُوا۟ وَيُلْهِهِمُ ٱلْأَمَلُ ۖ فَسَوْفَ يَعْلَمُونَ ﴿٣﴾\\
\textamh{4.\  } & وَمَآ أَهْلَكْنَا مِن قَرْيَةٍ إِلَّا وَلَهَا كِتَابٌۭ مَّعْلُومٌۭ ﴿٤﴾\\
\textamh{5.\  } & مَّا تَسْبِقُ مِنْ أُمَّةٍ أَجَلَهَا وَمَا يَسْتَـْٔخِرُونَ ﴿٥﴾\\
\textamh{6.\  } & وَقَالُوا۟ يَـٰٓأَيُّهَا ٱلَّذِى نُزِّلَ عَلَيْهِ ٱلذِّكْرُ إِنَّكَ لَمَجْنُونٌۭ ﴿٦﴾\\
\textamh{7.\  } & لَّوْ مَا تَأْتِينَا بِٱلْمَلَـٰٓئِكَةِ إِن كُنتَ مِنَ ٱلصَّـٰدِقِينَ ﴿٧﴾\\
\textamh{8.\  } & مَا نُنَزِّلُ ٱلْمَلَـٰٓئِكَةَ إِلَّا بِٱلْحَقِّ وَمَا كَانُوٓا۟ إِذًۭا مُّنظَرِينَ ﴿٨﴾\\
\textamh{9.\  } & إِنَّا نَحْنُ نَزَّلْنَا ٱلذِّكْرَ وَإِنَّا لَهُۥ لَحَـٰفِظُونَ ﴿٩﴾\\
\textamh{10.\  } & وَلَقَدْ أَرْسَلْنَا مِن قَبْلِكَ فِى شِيَعِ ٱلْأَوَّلِينَ ﴿١٠﴾\\
\textamh{11.\  } & وَمَا يَأْتِيهِم مِّن رَّسُولٍ إِلَّا كَانُوا۟ بِهِۦ يَسْتَهْزِءُونَ ﴿١١﴾\\
\textamh{12.\  } & كَذَٟلِكَ نَسْلُكُهُۥ فِى قُلُوبِ ٱلْمُجْرِمِينَ ﴿١٢﴾\\
\textamh{13.\  } & لَا يُؤْمِنُونَ بِهِۦ ۖ وَقَدْ خَلَتْ سُنَّةُ ٱلْأَوَّلِينَ ﴿١٣﴾\\
\textamh{14.\  } & وَلَوْ فَتَحْنَا عَلَيْهِم بَابًۭا مِّنَ ٱلسَّمَآءِ فَظَلُّوا۟ فِيهِ يَعْرُجُونَ ﴿١٤﴾\\
\textamh{15.\  } & لَقَالُوٓا۟ إِنَّمَا سُكِّرَتْ أَبْصَـٰرُنَا بَلْ نَحْنُ قَوْمٌۭ مَّسْحُورُونَ ﴿١٥﴾\\
\textamh{16.\  } & وَلَقَدْ جَعَلْنَا فِى ٱلسَّمَآءِ بُرُوجًۭا وَزَيَّنَّـٰهَا لِلنَّـٰظِرِينَ ﴿١٦﴾\\
\textamh{17.\  } & وَحَفِظْنَـٰهَا مِن كُلِّ شَيْطَٰنٍۢ رَّجِيمٍ ﴿١٧﴾\\
\textamh{18.\  } & إِلَّا مَنِ ٱسْتَرَقَ ٱلسَّمْعَ فَأَتْبَعَهُۥ شِهَابٌۭ مُّبِينٌۭ ﴿١٨﴾\\
\textamh{19.\  } & وَٱلْأَرْضَ مَدَدْنَـٰهَا وَأَلْقَيْنَا فِيهَا رَوَٟسِىَ وَأَنۢبَتْنَا فِيهَا مِن كُلِّ شَىْءٍۢ مَّوْزُونٍۢ ﴿١٩﴾\\
\textamh{20.\  } & وَجَعَلْنَا لَكُمْ فِيهَا مَعَـٰيِشَ وَمَن لَّسْتُمْ لَهُۥ بِرَٰزِقِينَ ﴿٢٠﴾\\
\textamh{21.\  } & وَإِن مِّن شَىْءٍ إِلَّا عِندَنَا خَزَآئِنُهُۥ وَمَا نُنَزِّلُهُۥٓ إِلَّا بِقَدَرٍۢ مَّعْلُومٍۢ ﴿٢١﴾\\
\textamh{22.\  } & وَأَرْسَلْنَا ٱلرِّيَـٰحَ لَوَٟقِحَ فَأَنزَلْنَا مِنَ ٱلسَّمَآءِ مَآءًۭ فَأَسْقَيْنَـٰكُمُوهُ وَمَآ أَنتُمْ لَهُۥ بِخَـٰزِنِينَ ﴿٢٢﴾\\
\textamh{23.\  } & وَإِنَّا لَنَحْنُ نُحْىِۦ وَنُمِيتُ وَنَحْنُ ٱلْوَٟرِثُونَ ﴿٢٣﴾\\
\textamh{24.\  } & وَلَقَدْ عَلِمْنَا ٱلْمُسْتَقْدِمِينَ مِنكُمْ وَلَقَدْ عَلِمْنَا ٱلْمُسْتَـْٔخِرِينَ ﴿٢٤﴾\\
\textamh{25.\  } & وَإِنَّ رَبَّكَ هُوَ يَحْشُرُهُمْ ۚ إِنَّهُۥ حَكِيمٌ عَلِيمٌۭ ﴿٢٥﴾\\
\textamh{26.\  } & وَلَقَدْ خَلَقْنَا ٱلْإِنسَـٰنَ مِن صَلْصَـٰلٍۢ مِّنْ حَمَإٍۢ مَّسْنُونٍۢ ﴿٢٦﴾\\
\textamh{27.\  } & وَٱلْجَآنَّ خَلَقْنَـٰهُ مِن قَبْلُ مِن نَّارِ ٱلسَّمُومِ ﴿٢٧﴾\\
\textamh{28.\  } & وَإِذْ قَالَ رَبُّكَ لِلْمَلَـٰٓئِكَةِ إِنِّى خَـٰلِقٌۢ بَشَرًۭا مِّن صَلْصَـٰلٍۢ مِّنْ حَمَإٍۢ مَّسْنُونٍۢ ﴿٢٨﴾\\
\textamh{29.\  } & فَإِذَا سَوَّيْتُهُۥ وَنَفَخْتُ فِيهِ مِن رُّوحِى فَقَعُوا۟ لَهُۥ سَـٰجِدِينَ ﴿٢٩﴾\\
\textamh{30.\  } & فَسَجَدَ ٱلْمَلَـٰٓئِكَةُ كُلُّهُمْ أَجْمَعُونَ ﴿٣٠﴾\\
\textamh{31.\  } & إِلَّآ إِبْلِيسَ أَبَىٰٓ أَن يَكُونَ مَعَ ٱلسَّٰجِدِينَ ﴿٣١﴾\\
\textamh{32.\  } & قَالَ يَـٰٓإِبْلِيسُ مَا لَكَ أَلَّا تَكُونَ مَعَ ٱلسَّٰجِدِينَ ﴿٣٢﴾\\
\textamh{33.\  } & قَالَ لَمْ أَكُن لِّأَسْجُدَ لِبَشَرٍ خَلَقْتَهُۥ مِن صَلْصَـٰلٍۢ مِّنْ حَمَإٍۢ مَّسْنُونٍۢ ﴿٣٣﴾\\
\textamh{34.\  } & قَالَ فَٱخْرُجْ مِنْهَا فَإِنَّكَ رَجِيمٌۭ ﴿٣٤﴾\\
\textamh{35.\  } & وَإِنَّ عَلَيْكَ ٱللَّعْنَةَ إِلَىٰ يَوْمِ ٱلدِّينِ ﴿٣٥﴾\\
\textamh{36.\  } & قَالَ رَبِّ فَأَنظِرْنِىٓ إِلَىٰ يَوْمِ يُبْعَثُونَ ﴿٣٦﴾\\
\textamh{37.\  } & قَالَ فَإِنَّكَ مِنَ ٱلْمُنظَرِينَ ﴿٣٧﴾\\
\textamh{38.\  } & إِلَىٰ يَوْمِ ٱلْوَقْتِ ٱلْمَعْلُومِ ﴿٣٨﴾\\
\textamh{39.\  } & قَالَ رَبِّ بِمَآ أَغْوَيْتَنِى لَأُزَيِّنَنَّ لَهُمْ فِى ٱلْأَرْضِ وَلَأُغْوِيَنَّهُمْ أَجْمَعِينَ ﴿٣٩﴾\\
\textamh{40.\  } & إِلَّا عِبَادَكَ مِنْهُمُ ٱلْمُخْلَصِينَ ﴿٤٠﴾\\
\textamh{41.\  } & قَالَ هَـٰذَا صِرَٰطٌ عَلَىَّ مُسْتَقِيمٌ ﴿٤١﴾\\
\textamh{42.\  } & إِنَّ عِبَادِى لَيْسَ لَكَ عَلَيْهِمْ سُلْطَٰنٌ إِلَّا مَنِ ٱتَّبَعَكَ مِنَ ٱلْغَاوِينَ ﴿٤٢﴾\\
\textamh{43.\  } & وَإِنَّ جَهَنَّمَ لَمَوْعِدُهُمْ أَجْمَعِينَ ﴿٤٣﴾\\
\textamh{44.\  } & لَهَا سَبْعَةُ أَبْوَٟبٍۢ لِّكُلِّ بَابٍۢ مِّنْهُمْ جُزْءٌۭ مَّقْسُومٌ ﴿٤٤﴾\\
\textamh{45.\  } & إِنَّ ٱلْمُتَّقِينَ فِى جَنَّـٰتٍۢ وَعُيُونٍ ﴿٤٥﴾\\
\textamh{46.\  } & ٱدْخُلُوهَا بِسَلَـٰمٍ ءَامِنِينَ ﴿٤٦﴾\\
\textamh{47.\  } & وَنَزَعْنَا مَا فِى صُدُورِهِم مِّنْ غِلٍّ إِخْوَٟنًا عَلَىٰ سُرُرٍۢ مُّتَقَـٰبِلِينَ ﴿٤٧﴾\\
\textamh{48.\  } & لَا يَمَسُّهُمْ فِيهَا نَصَبٌۭ وَمَا هُم مِّنْهَا بِمُخْرَجِينَ ﴿٤٨﴾\\
\textamh{49.\  } & ۞ نَبِّئْ عِبَادِىٓ أَنِّىٓ أَنَا ٱلْغَفُورُ ٱلرَّحِيمُ ﴿٤٩﴾\\
\textamh{50.\  } & وَأَنَّ عَذَابِى هُوَ ٱلْعَذَابُ ٱلْأَلِيمُ ﴿٥٠﴾\\
\textamh{51.\  } & وَنَبِّئْهُمْ عَن ضَيْفِ إِبْرَٰهِيمَ ﴿٥١﴾\\
\textamh{52.\  } & إِذْ دَخَلُوا۟ عَلَيْهِ فَقَالُوا۟ سَلَـٰمًۭا قَالَ إِنَّا مِنكُمْ وَجِلُونَ ﴿٥٢﴾\\
\textamh{53.\  } & قَالُوا۟ لَا تَوْجَلْ إِنَّا نُبَشِّرُكَ بِغُلَـٰمٍ عَلِيمٍۢ ﴿٥٣﴾\\
\textamh{54.\  } & قَالَ أَبَشَّرْتُمُونِى عَلَىٰٓ أَن مَّسَّنِىَ ٱلْكِبَرُ فَبِمَ تُبَشِّرُونَ ﴿٥٤﴾\\
\textamh{55.\  } & قَالُوا۟ بَشَّرْنَـٰكَ بِٱلْحَقِّ فَلَا تَكُن مِّنَ ٱلْقَـٰنِطِينَ ﴿٥٥﴾\\
\textamh{56.\  } & قَالَ وَمَن يَقْنَطُ مِن رَّحْمَةِ رَبِّهِۦٓ إِلَّا ٱلضَّآلُّونَ ﴿٥٦﴾\\
\textamh{57.\  } & قَالَ فَمَا خَطْبُكُمْ أَيُّهَا ٱلْمُرْسَلُونَ ﴿٥٧﴾\\
\textamh{58.\  } & قَالُوٓا۟ إِنَّآ أُرْسِلْنَآ إِلَىٰ قَوْمٍۢ مُّجْرِمِينَ ﴿٥٨﴾\\
\textamh{59.\  } & إِلَّآ ءَالَ لُوطٍ إِنَّا لَمُنَجُّوهُمْ أَجْمَعِينَ ﴿٥٩﴾\\
\textamh{60.\  } & إِلَّا ٱمْرَأَتَهُۥ قَدَّرْنَآ ۙ إِنَّهَا لَمِنَ ٱلْغَٰبِرِينَ ﴿٦٠﴾\\
\textamh{61.\  } & فَلَمَّا جَآءَ ءَالَ لُوطٍ ٱلْمُرْسَلُونَ ﴿٦١﴾\\
\textamh{62.\  } & قَالَ إِنَّكُمْ قَوْمٌۭ مُّنكَرُونَ ﴿٦٢﴾\\
\textamh{63.\  } & قَالُوا۟ بَلْ جِئْنَـٰكَ بِمَا كَانُوا۟ فِيهِ يَمْتَرُونَ ﴿٦٣﴾\\
\textamh{64.\  } & وَأَتَيْنَـٰكَ بِٱلْحَقِّ وَإِنَّا لَصَـٰدِقُونَ ﴿٦٤﴾\\
\textamh{65.\  } & فَأَسْرِ بِأَهْلِكَ بِقِطْعٍۢ مِّنَ ٱلَّيْلِ وَٱتَّبِعْ أَدْبَٰرَهُمْ وَلَا يَلْتَفِتْ مِنكُمْ أَحَدٌۭ وَٱمْضُوا۟ حَيْثُ تُؤْمَرُونَ ﴿٦٥﴾\\
\textamh{66.\  } & وَقَضَيْنَآ إِلَيْهِ ذَٟلِكَ ٱلْأَمْرَ أَنَّ دَابِرَ هَـٰٓؤُلَآءِ مَقْطُوعٌۭ مُّصْبِحِينَ ﴿٦٦﴾\\
\textamh{67.\  } & وَجَآءَ أَهْلُ ٱلْمَدِينَةِ يَسْتَبْشِرُونَ ﴿٦٧﴾\\
\textamh{68.\  } & قَالَ إِنَّ هَـٰٓؤُلَآءِ ضَيْفِى فَلَا تَفْضَحُونِ ﴿٦٨﴾\\
\textamh{69.\  } & وَٱتَّقُوا۟ ٱللَّهَ وَلَا تُخْزُونِ ﴿٦٩﴾\\
\textamh{70.\  } & قَالُوٓا۟ أَوَلَمْ نَنْهَكَ عَنِ ٱلْعَـٰلَمِينَ ﴿٧٠﴾\\
\textamh{71.\  } & قَالَ هَـٰٓؤُلَآءِ بَنَاتِىٓ إِن كُنتُمْ فَـٰعِلِينَ ﴿٧١﴾\\
\textamh{72.\  } & لَعَمْرُكَ إِنَّهُمْ لَفِى سَكْرَتِهِمْ يَعْمَهُونَ ﴿٧٢﴾\\
\textamh{73.\  } & فَأَخَذَتْهُمُ ٱلصَّيْحَةُ مُشْرِقِينَ ﴿٧٣﴾\\
\textamh{74.\  } & فَجَعَلْنَا عَـٰلِيَهَا سَافِلَهَا وَأَمْطَرْنَا عَلَيْهِمْ حِجَارَةًۭ مِّن سِجِّيلٍ ﴿٧٤﴾\\
\textamh{75.\  } & إِنَّ فِى ذَٟلِكَ لَءَايَـٰتٍۢ لِّلْمُتَوَسِّمِينَ ﴿٧٥﴾\\
\textamh{76.\  } & وَإِنَّهَا لَبِسَبِيلٍۢ مُّقِيمٍ ﴿٧٦﴾\\
\textamh{77.\  } & إِنَّ فِى ذَٟلِكَ لَءَايَةًۭ لِّلْمُؤْمِنِينَ ﴿٧٧﴾\\
\textamh{78.\  } & وَإِن كَانَ أَصْحَـٰبُ ٱلْأَيْكَةِ لَظَـٰلِمِينَ ﴿٧٨﴾\\
\textamh{79.\  } & فَٱنتَقَمْنَا مِنْهُمْ وَإِنَّهُمَا لَبِإِمَامٍۢ مُّبِينٍۢ ﴿٧٩﴾\\
\textamh{80.\  } & وَلَقَدْ كَذَّبَ أَصْحَـٰبُ ٱلْحِجْرِ ٱلْمُرْسَلِينَ ﴿٨٠﴾\\
\textamh{81.\  } & وَءَاتَيْنَـٰهُمْ ءَايَـٰتِنَا فَكَانُوا۟ عَنْهَا مُعْرِضِينَ ﴿٨١﴾\\
\textamh{82.\  } & وَكَانُوا۟ يَنْحِتُونَ مِنَ ٱلْجِبَالِ بُيُوتًا ءَامِنِينَ ﴿٨٢﴾\\
\textamh{83.\  } & فَأَخَذَتْهُمُ ٱلصَّيْحَةُ مُصْبِحِينَ ﴿٨٣﴾\\
\textamh{84.\  } & فَمَآ أَغْنَىٰ عَنْهُم مَّا كَانُوا۟ يَكْسِبُونَ ﴿٨٤﴾\\
\textamh{85.\  } & وَمَا خَلَقْنَا ٱلسَّمَـٰوَٟتِ وَٱلْأَرْضَ وَمَا بَيْنَهُمَآ إِلَّا بِٱلْحَقِّ ۗ وَإِنَّ ٱلسَّاعَةَ لَءَاتِيَةٌۭ ۖ فَٱصْفَحِ ٱلصَّفْحَ ٱلْجَمِيلَ ﴿٨٥﴾\\
\textamh{86.\  } & إِنَّ رَبَّكَ هُوَ ٱلْخَلَّٰقُ ٱلْعَلِيمُ ﴿٨٦﴾\\
\textamh{87.\  } & وَلَقَدْ ءَاتَيْنَـٰكَ سَبْعًۭا مِّنَ ٱلْمَثَانِى وَٱلْقُرْءَانَ ٱلْعَظِيمَ ﴿٨٧﴾\\
\textamh{88.\  } & لَا تَمُدَّنَّ عَيْنَيْكَ إِلَىٰ مَا مَتَّعْنَا بِهِۦٓ أَزْوَٟجًۭا مِّنْهُمْ وَلَا تَحْزَنْ عَلَيْهِمْ وَٱخْفِضْ جَنَاحَكَ لِلْمُؤْمِنِينَ ﴿٨٨﴾\\
\textamh{89.\  } & وَقُلْ إِنِّىٓ أَنَا ٱلنَّذِيرُ ٱلْمُبِينُ ﴿٨٩﴾\\
\textamh{90.\  } & كَمَآ أَنزَلْنَا عَلَى ٱلْمُقْتَسِمِينَ ﴿٩٠﴾\\
\textamh{91.\  } & ٱلَّذِينَ جَعَلُوا۟ ٱلْقُرْءَانَ عِضِينَ ﴿٩١﴾\\
\textamh{92.\  } & فَوَرَبِّكَ لَنَسْـَٔلَنَّهُمْ أَجْمَعِينَ ﴿٩٢﴾\\
\textamh{93.\  } & عَمَّا كَانُوا۟ يَعْمَلُونَ ﴿٩٣﴾\\
\textamh{94.\  } & فَٱصْدَعْ بِمَا تُؤْمَرُ وَأَعْرِضْ عَنِ ٱلْمُشْرِكِينَ ﴿٩٤﴾\\
\textamh{95.\  } & إِنَّا كَفَيْنَـٰكَ ٱلْمُسْتَهْزِءِينَ ﴿٩٥﴾\\
\textamh{96.\  } & ٱلَّذِينَ يَجْعَلُونَ مَعَ ٱللَّهِ إِلَـٰهًا ءَاخَرَ ۚ فَسَوْفَ يَعْلَمُونَ ﴿٩٦﴾\\
\textamh{97.\  } & وَلَقَدْ نَعْلَمُ أَنَّكَ يَضِيقُ صَدْرُكَ بِمَا يَقُولُونَ ﴿٩٧﴾\\
\textamh{98.\  } & فَسَبِّحْ بِحَمْدِ رَبِّكَ وَكُن مِّنَ ٱلسَّٰجِدِينَ ﴿٩٨﴾\\
\textamh{99.\  } & وَٱعْبُدْ رَبَّكَ حَتَّىٰ يَأْتِيَكَ ٱلْيَقِينُ ﴿٩٩﴾\\
\end{longtable}
\clearpage