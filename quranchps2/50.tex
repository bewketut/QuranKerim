%% License: BSD style (Berkley) (i.e. Put the Copyright owner's name always)
%% Writer and Copyright (to): Bewketu(Bilal) Tadilo (2016-17)
\begin{center}\section{\LR{\textamhsec{ሱራቱ ቃፍ -}  \textarabic{سوره  ق}}}\end{center}
\begin{longtable}{%
  @{}
    p{.5\textwidth}
  @{~~~}
    p{.5\textwidth}
    @{}
}
\textamh{ቢስሚላሂ አራህመኒ ራሂይም } &  \mytextarabic{بِسْمِ ٱللَّهِ ٱلرَّحْمَـٰنِ ٱلرَّحِيمِ}\\
\textamh{1.\  } & \mytextarabic{ قٓ ۚ وَٱلْقُرْءَانِ ٱلْمَجِيدِ ﴿١﴾}\\
\textamh{2.\  } & \mytextarabic{بَلْ عَجِبُوٓا۟ أَن جَآءَهُم مُّنذِرٌۭ مِّنْهُمْ فَقَالَ ٱلْكَـٰفِرُونَ هَـٰذَا شَىْءٌ عَجِيبٌ ﴿٢﴾}\\
\textamh{3.\  } & \mytextarabic{أَءِذَا مِتْنَا وَكُنَّا تُرَابًۭا ۖ ذَٟلِكَ رَجْعٌۢ بَعِيدٌۭ ﴿٣﴾}\\
\textamh{4.\  } & \mytextarabic{قَدْ عَلِمْنَا مَا تَنقُصُ ٱلْأَرْضُ مِنْهُمْ ۖ وَعِندَنَا كِتَـٰبٌ حَفِيظٌۢ ﴿٤﴾}\\
\textamh{5.\  } & \mytextarabic{بَلْ كَذَّبُوا۟ بِٱلْحَقِّ لَمَّا جَآءَهُمْ فَهُمْ فِىٓ أَمْرٍۢ مَّرِيجٍ ﴿٥﴾}\\
\textamh{6.\  } & \mytextarabic{أَفَلَمْ يَنظُرُوٓا۟ إِلَى ٱلسَّمَآءِ فَوْقَهُمْ كَيْفَ بَنَيْنَـٰهَا وَزَيَّنَّـٰهَا وَمَا لَهَا مِن فُرُوجٍۢ ﴿٦﴾}\\
\textamh{7.\  } & \mytextarabic{وَٱلْأَرْضَ مَدَدْنَـٰهَا وَأَلْقَيْنَا فِيهَا رَوَٟسِىَ وَأَنۢبَتْنَا فِيهَا مِن كُلِّ زَوْجٍۭ بَهِيجٍۢ ﴿٧﴾}\\
\textamh{8.\  } & \mytextarabic{تَبْصِرَةًۭ وَذِكْرَىٰ لِكُلِّ عَبْدٍۢ مُّنِيبٍۢ ﴿٨﴾}\\
\textamh{9.\  } & \mytextarabic{وَنَزَّلْنَا مِنَ ٱلسَّمَآءِ مَآءًۭ مُّبَٰرَكًۭا فَأَنۢبَتْنَا بِهِۦ جَنَّـٰتٍۢ وَحَبَّ ٱلْحَصِيدِ ﴿٩﴾}\\
\textamh{10.\  } & \mytextarabic{وَٱلنَّخْلَ بَاسِقَـٰتٍۢ لَّهَا طَلْعٌۭ نَّضِيدٌۭ ﴿١٠﴾}\\
\textamh{11.\  } & \mytextarabic{رِّزْقًۭا لِّلْعِبَادِ ۖ وَأَحْيَيْنَا بِهِۦ بَلْدَةًۭ مَّيْتًۭا ۚ كَذَٟلِكَ ٱلْخُرُوجُ ﴿١١﴾}\\
\textamh{12.\  } & \mytextarabic{كَذَّبَتْ قَبْلَهُمْ قَوْمُ نُوحٍۢ وَأَصْحَـٰبُ ٱلرَّسِّ وَثَمُودُ ﴿١٢﴾}\\
\textamh{13.\  } & \mytextarabic{وَعَادٌۭ وَفِرْعَوْنُ وَإِخْوَٟنُ لُوطٍۢ ﴿١٣﴾}\\
\textamh{14.\  } & \mytextarabic{وَأَصْحَـٰبُ ٱلْأَيْكَةِ وَقَوْمُ تُبَّعٍۢ ۚ كُلٌّۭ كَذَّبَ ٱلرُّسُلَ فَحَقَّ وَعِيدِ ﴿١٤﴾}\\
\textamh{15.\  } & \mytextarabic{أَفَعَيِينَا بِٱلْخَلْقِ ٱلْأَوَّلِ ۚ بَلْ هُمْ فِى لَبْسٍۢ مِّنْ خَلْقٍۢ جَدِيدٍۢ ﴿١٥﴾}\\
\textamh{16.\  } & \mytextarabic{وَلَقَدْ خَلَقْنَا ٱلْإِنسَـٰنَ وَنَعْلَمُ مَا تُوَسْوِسُ بِهِۦ نَفْسُهُۥ ۖ وَنَحْنُ أَقْرَبُ إِلَيْهِ مِنْ حَبْلِ ٱلْوَرِيدِ ﴿١٦﴾}\\
\textamh{17.\  } & \mytextarabic{إِذْ يَتَلَقَّى ٱلْمُتَلَقِّيَانِ عَنِ ٱلْيَمِينِ وَعَنِ ٱلشِّمَالِ قَعِيدٌۭ ﴿١٧﴾}\\
\textamh{18.\  } & \mytextarabic{مَّا يَلْفِظُ مِن قَوْلٍ إِلَّا لَدَيْهِ رَقِيبٌ عَتِيدٌۭ ﴿١٨﴾}\\
\textamh{19.\  } & \mytextarabic{وَجَآءَتْ سَكْرَةُ ٱلْمَوْتِ بِٱلْحَقِّ ۖ ذَٟلِكَ مَا كُنتَ مِنْهُ تَحِيدُ ﴿١٩﴾}\\
\textamh{20.\  } & \mytextarabic{وَنُفِخَ فِى ٱلصُّورِ ۚ ذَٟلِكَ يَوْمُ ٱلْوَعِيدِ ﴿٢٠﴾}\\
\textamh{21.\  } & \mytextarabic{وَجَآءَتْ كُلُّ نَفْسٍۢ مَّعَهَا سَآئِقٌۭ وَشَهِيدٌۭ ﴿٢١﴾}\\
\textamh{22.\  } & \mytextarabic{لَّقَدْ كُنتَ فِى غَفْلَةٍۢ مِّنْ هَـٰذَا فَكَشَفْنَا عَنكَ غِطَآءَكَ فَبَصَرُكَ ٱلْيَوْمَ حَدِيدٌۭ ﴿٢٢﴾}\\
\textamh{23.\  } & \mytextarabic{وَقَالَ قَرِينُهُۥ هَـٰذَا مَا لَدَىَّ عَتِيدٌ ﴿٢٣﴾}\\
\textamh{24.\  } & \mytextarabic{أَلْقِيَا فِى جَهَنَّمَ كُلَّ كَفَّارٍ عَنِيدٍۢ ﴿٢٤﴾}\\
\textamh{25.\  } & \mytextarabic{مَّنَّاعٍۢ لِّلْخَيْرِ مُعْتَدٍۢ مُّرِيبٍ ﴿٢٥﴾}\\
\textamh{26.\  } & \mytextarabic{ٱلَّذِى جَعَلَ مَعَ ٱللَّهِ إِلَـٰهًا ءَاخَرَ فَأَلْقِيَاهُ فِى ٱلْعَذَابِ ٱلشَّدِيدِ ﴿٢٦﴾}\\
\textamh{27.\  } & \mytextarabic{۞ قَالَ قَرِينُهُۥ رَبَّنَا مَآ أَطْغَيْتُهُۥ وَلَـٰكِن كَانَ فِى ضَلَـٰلٍۭ بَعِيدٍۢ ﴿٢٧﴾}\\
\textamh{28.\  } & \mytextarabic{قَالَ لَا تَخْتَصِمُوا۟ لَدَىَّ وَقَدْ قَدَّمْتُ إِلَيْكُم بِٱلْوَعِيدِ ﴿٢٨﴾}\\
\textamh{29.\  } & \mytextarabic{مَا يُبَدَّلُ ٱلْقَوْلُ لَدَىَّ وَمَآ أَنَا۠ بِظَلَّٰمٍۢ لِّلْعَبِيدِ ﴿٢٩﴾}\\
\textamh{30.\  } & \mytextarabic{يَوْمَ نَقُولُ لِجَهَنَّمَ هَلِ ٱمْتَلَأْتِ وَتَقُولُ هَلْ مِن مَّزِيدٍۢ ﴿٣٠﴾}\\
\textamh{31.\  } & \mytextarabic{وَأُزْلِفَتِ ٱلْجَنَّةُ لِلْمُتَّقِينَ غَيْرَ بَعِيدٍ ﴿٣١﴾}\\
\textamh{32.\  } & \mytextarabic{هَـٰذَا مَا تُوعَدُونَ لِكُلِّ أَوَّابٍ حَفِيظٍۢ ﴿٣٢﴾}\\
\textamh{33.\  } & \mytextarabic{مَّنْ خَشِىَ ٱلرَّحْمَـٰنَ بِٱلْغَيْبِ وَجَآءَ بِقَلْبٍۢ مُّنِيبٍ ﴿٣٣﴾}\\
\textamh{34.\  } & \mytextarabic{ٱدْخُلُوهَا بِسَلَـٰمٍۢ ۖ ذَٟلِكَ يَوْمُ ٱلْخُلُودِ ﴿٣٤﴾}\\
\textamh{35.\  } & \mytextarabic{لَهُم مَّا يَشَآءُونَ فِيهَا وَلَدَيْنَا مَزِيدٌۭ ﴿٣٥﴾}\\
\textamh{36.\  } & \mytextarabic{وَكَمْ أَهْلَكْنَا قَبْلَهُم مِّن قَرْنٍ هُمْ أَشَدُّ مِنْهُم بَطْشًۭا فَنَقَّبُوا۟ فِى ٱلْبِلَـٰدِ هَلْ مِن مَّحِيصٍ ﴿٣٦﴾}\\
\textamh{37.\  } & \mytextarabic{إِنَّ فِى ذَٟلِكَ لَذِكْرَىٰ لِمَن كَانَ لَهُۥ قَلْبٌ أَوْ أَلْقَى ٱلسَّمْعَ وَهُوَ شَهِيدٌۭ ﴿٣٧﴾}\\
\textamh{38.\  } & \mytextarabic{وَلَقَدْ خَلَقْنَا ٱلسَّمَـٰوَٟتِ وَٱلْأَرْضَ وَمَا بَيْنَهُمَا فِى سِتَّةِ أَيَّامٍۢ وَمَا مَسَّنَا مِن لُّغُوبٍۢ ﴿٣٨﴾}\\
\textamh{39.\  } & \mytextarabic{فَٱصْبِرْ عَلَىٰ مَا يَقُولُونَ وَسَبِّحْ بِحَمْدِ رَبِّكَ قَبْلَ طُلُوعِ ٱلشَّمْسِ وَقَبْلَ ٱلْغُرُوبِ ﴿٣٩﴾}\\
\textamh{40.\  } & \mytextarabic{وَمِنَ ٱلَّيْلِ فَسَبِّحْهُ وَأَدْبَٰرَ ٱلسُّجُودِ ﴿٤٠﴾}\\
\textamh{41.\  } & \mytextarabic{وَٱسْتَمِعْ يَوْمَ يُنَادِ ٱلْمُنَادِ مِن مَّكَانٍۢ قَرِيبٍۢ ﴿٤١﴾}\\
\textamh{42.\  } & \mytextarabic{يَوْمَ يَسْمَعُونَ ٱلصَّيْحَةَ بِٱلْحَقِّ ۚ ذَٟلِكَ يَوْمُ ٱلْخُرُوجِ ﴿٤٢﴾}\\
\textamh{43.\  } & \mytextarabic{إِنَّا نَحْنُ نُحْىِۦ وَنُمِيتُ وَإِلَيْنَا ٱلْمَصِيرُ ﴿٤٣﴾}\\
\textamh{44.\  } & \mytextarabic{يَوْمَ تَشَقَّقُ ٱلْأَرْضُ عَنْهُمْ سِرَاعًۭا ۚ ذَٟلِكَ حَشْرٌ عَلَيْنَا يَسِيرٌۭ ﴿٤٤﴾}\\
\textamh{45.\  } & \mytextarabic{نَّحْنُ أَعْلَمُ بِمَا يَقُولُونَ ۖ وَمَآ أَنتَ عَلَيْهِم بِجَبَّارٍۢ ۖ فَذَكِّرْ بِٱلْقُرْءَانِ مَن يَخَافُ وَعِيدِ ﴿٤٥﴾}\\
\end{longtable}
\clearpage