%% License: BSD style (Berkley) (i.e. Put the Copyright owner's name always)
%% Writer and Copyright (to): Bewketu(Bilal) Tadilo (2016-17)
\begin{center}\section{\LR{\textamhsec{ሱራቱ አልሀቃ -}  \textarabic{سوره  الحاقة}}}\end{center}
\begin{longtable}{%
  @{}
    p{.5\textwidth}
  @{~~~}
    p{.5\textwidth}
    @{}
}
\textamh{ቢስሚላሂ አራህመኒ ራሂይም } &  \mytextarabic{بِسْمِ ٱللَّهِ ٱلرَّحْمَـٰنِ ٱلرَّحِيمِ}\\
\textamh{1.\  } & \mytextarabic{ ٱلْحَآقَّةُ ﴿١﴾}\\
\textamh{2.\  } & \mytextarabic{مَا ٱلْحَآقَّةُ ﴿٢﴾}\\
\textamh{3.\  } & \mytextarabic{وَمَآ أَدْرَىٰكَ مَا ٱلْحَآقَّةُ ﴿٣﴾}\\
\textamh{4.\  } & \mytextarabic{كَذَّبَتْ ثَمُودُ وَعَادٌۢ بِٱلْقَارِعَةِ ﴿٤﴾}\\
\textamh{5.\  } & \mytextarabic{فَأَمَّا ثَمُودُ فَأُهْلِكُوا۟ بِٱلطَّاغِيَةِ ﴿٥﴾}\\
\textamh{6.\  } & \mytextarabic{وَأَمَّا عَادٌۭ فَأُهْلِكُوا۟ بِرِيحٍۢ صَرْصَرٍ عَاتِيَةٍۢ ﴿٦﴾}\\
\textamh{7.\  } & \mytextarabic{سَخَّرَهَا عَلَيْهِمْ سَبْعَ لَيَالٍۢ وَثَمَـٰنِيَةَ أَيَّامٍ حُسُومًۭا فَتَرَى ٱلْقَوْمَ فِيهَا صَرْعَىٰ كَأَنَّهُمْ أَعْجَازُ نَخْلٍ خَاوِيَةٍۢ ﴿٧﴾}\\
\textamh{8.\  } & \mytextarabic{فَهَلْ تَرَىٰ لَهُم مِّنۢ بَاقِيَةٍۢ ﴿٨﴾}\\
\textamh{9.\  } & \mytextarabic{وَجَآءَ فِرْعَوْنُ وَمَن قَبْلَهُۥ وَٱلْمُؤْتَفِكَـٰتُ بِٱلْخَاطِئَةِ ﴿٩﴾}\\
\textamh{10.\  } & \mytextarabic{فَعَصَوْا۟ رَسُولَ رَبِّهِمْ فَأَخَذَهُمْ أَخْذَةًۭ رَّابِيَةً ﴿١٠﴾}\\
\textamh{11.\  } & \mytextarabic{إِنَّا لَمَّا طَغَا ٱلْمَآءُ حَمَلْنَـٰكُمْ فِى ٱلْجَارِيَةِ ﴿١١﴾}\\
\textamh{12.\  } & \mytextarabic{لِنَجْعَلَهَا لَكُمْ تَذْكِرَةًۭ وَتَعِيَهَآ أُذُنٌۭ وَٟعِيَةٌۭ ﴿١٢﴾}\\
\textamh{13.\  } & \mytextarabic{فَإِذَا نُفِخَ فِى ٱلصُّورِ نَفْخَةٌۭ وَٟحِدَةٌۭ ﴿١٣﴾}\\
\textamh{14.\  } & \mytextarabic{وَحُمِلَتِ ٱلْأَرْضُ وَٱلْجِبَالُ فَدُكَّتَا دَكَّةًۭ وَٟحِدَةًۭ ﴿١٤﴾}\\
\textamh{15.\  } & \mytextarabic{فَيَوْمَئِذٍۢ وَقَعَتِ ٱلْوَاقِعَةُ ﴿١٥﴾}\\
\textamh{16.\  } & \mytextarabic{وَٱنشَقَّتِ ٱلسَّمَآءُ فَهِىَ يَوْمَئِذٍۢ وَاهِيَةٌۭ ﴿١٦﴾}\\
\textamh{17.\  } & \mytextarabic{وَٱلْمَلَكُ عَلَىٰٓ أَرْجَآئِهَا ۚ وَيَحْمِلُ عَرْشَ رَبِّكَ فَوْقَهُمْ يَوْمَئِذٍۢ ثَمَـٰنِيَةٌۭ ﴿١٧﴾}\\
\textamh{18.\  } & \mytextarabic{يَوْمَئِذٍۢ تُعْرَضُونَ لَا تَخْفَىٰ مِنكُمْ خَافِيَةٌۭ ﴿١٨﴾}\\
\textamh{19.\  } & \mytextarabic{فَأَمَّا مَنْ أُوتِىَ كِتَـٰبَهُۥ بِيَمِينِهِۦ فَيَقُولُ هَآؤُمُ ٱقْرَءُوا۟ كِتَـٰبِيَهْ ﴿١٩﴾}\\
\textamh{20.\  } & \mytextarabic{إِنِّى ظَنَنتُ أَنِّى مُلَـٰقٍ حِسَابِيَهْ ﴿٢٠﴾}\\
\textamh{21.\  } & \mytextarabic{فَهُوَ فِى عِيشَةٍۢ رَّاضِيَةٍۢ ﴿٢١﴾}\\
\textamh{22.\  } & \mytextarabic{فِى جَنَّةٍ عَالِيَةٍۢ ﴿٢٢﴾}\\
\textamh{23.\  } & \mytextarabic{قُطُوفُهَا دَانِيَةٌۭ ﴿٢٣﴾}\\
\textamh{24.\  } & \mytextarabic{كُلُوا۟ وَٱشْرَبُوا۟ هَنِيٓـًٔۢا بِمَآ أَسْلَفْتُمْ فِى ٱلْأَيَّامِ ٱلْخَالِيَةِ ﴿٢٤﴾}\\
\textamh{25.\  } & \mytextarabic{وَأَمَّا مَنْ أُوتِىَ كِتَـٰبَهُۥ بِشِمَالِهِۦ فَيَقُولُ يَـٰلَيْتَنِى لَمْ أُوتَ كِتَـٰبِيَهْ ﴿٢٥﴾}\\
\textamh{26.\  } & \mytextarabic{وَلَمْ أَدْرِ مَا حِسَابِيَهْ ﴿٢٦﴾}\\
\textamh{27.\  } & \mytextarabic{يَـٰلَيْتَهَا كَانَتِ ٱلْقَاضِيَةَ ﴿٢٧﴾}\\
\textamh{28.\  } & \mytextarabic{مَآ أَغْنَىٰ عَنِّى مَالِيَهْ ۜ ﴿٢٨﴾}\\
\textamh{29.\  } & \mytextarabic{هَلَكَ عَنِّى سُلْطَٰنِيَهْ ﴿٢٩﴾}\\
\textamh{30.\  } & \mytextarabic{خُذُوهُ فَغُلُّوهُ ﴿٣٠﴾}\\
\textamh{31.\  } & \mytextarabic{ثُمَّ ٱلْجَحِيمَ صَلُّوهُ ﴿٣١﴾}\\
\textamh{32.\  } & \mytextarabic{ثُمَّ فِى سِلْسِلَةٍۢ ذَرْعُهَا سَبْعُونَ ذِرَاعًۭا فَٱسْلُكُوهُ ﴿٣٢﴾}\\
\textamh{33.\  } & \mytextarabic{إِنَّهُۥ كَانَ لَا يُؤْمِنُ بِٱللَّهِ ٱلْعَظِيمِ ﴿٣٣﴾}\\
\textamh{34.\  } & \mytextarabic{وَلَا يَحُضُّ عَلَىٰ طَعَامِ ٱلْمِسْكِينِ ﴿٣٤﴾}\\
\textamh{35.\  } & \mytextarabic{فَلَيْسَ لَهُ ٱلْيَوْمَ هَـٰهُنَا حَمِيمٌۭ ﴿٣٥﴾}\\
\textamh{36.\  } & \mytextarabic{وَلَا طَعَامٌ إِلَّا مِنْ غِسْلِينٍۢ ﴿٣٦﴾}\\
\textamh{37.\  } & \mytextarabic{لَّا يَأْكُلُهُۥٓ إِلَّا ٱلْخَـٰطِـُٔونَ ﴿٣٧﴾}\\
\textamh{38.\  } & \mytextarabic{فَلَآ أُقْسِمُ بِمَا تُبْصِرُونَ ﴿٣٨﴾}\\
\textamh{39.\  } & \mytextarabic{وَمَا لَا تُبْصِرُونَ ﴿٣٩﴾}\\
\textamh{40.\  } & \mytextarabic{إِنَّهُۥ لَقَوْلُ رَسُولٍۢ كَرِيمٍۢ ﴿٤٠﴾}\\
\textamh{41.\  } & \mytextarabic{وَمَا هُوَ بِقَوْلِ شَاعِرٍۢ ۚ قَلِيلًۭا مَّا تُؤْمِنُونَ ﴿٤١﴾}\\
\textamh{42.\  } & \mytextarabic{وَلَا بِقَوْلِ كَاهِنٍۢ ۚ قَلِيلًۭا مَّا تَذَكَّرُونَ ﴿٤٢﴾}\\
\textamh{43.\  } & \mytextarabic{تَنزِيلٌۭ مِّن رَّبِّ ٱلْعَـٰلَمِينَ ﴿٤٣﴾}\\
\textamh{44.\  } & \mytextarabic{وَلَوْ تَقَوَّلَ عَلَيْنَا بَعْضَ ٱلْأَقَاوِيلِ ﴿٤٤﴾}\\
\textamh{45.\  } & \mytextarabic{لَأَخَذْنَا مِنْهُ بِٱلْيَمِينِ ﴿٤٥﴾}\\
\textamh{46.\  } & \mytextarabic{ثُمَّ لَقَطَعْنَا مِنْهُ ٱلْوَتِينَ ﴿٤٦﴾}\\
\textamh{47.\  } & \mytextarabic{فَمَا مِنكُم مِّنْ أَحَدٍ عَنْهُ حَـٰجِزِينَ ﴿٤٧﴾}\\
\textamh{48.\  } & \mytextarabic{وَإِنَّهُۥ لَتَذْكِرَةٌۭ لِّلْمُتَّقِينَ ﴿٤٨﴾}\\
\textamh{49.\  } & \mytextarabic{وَإِنَّا لَنَعْلَمُ أَنَّ مِنكُم مُّكَذِّبِينَ ﴿٤٩﴾}\\
\textamh{50.\  } & \mytextarabic{وَإِنَّهُۥ لَحَسْرَةٌ عَلَى ٱلْكَـٰفِرِينَ ﴿٥٠﴾}\\
\textamh{51.\  } & \mytextarabic{وَإِنَّهُۥ لَحَقُّ ٱلْيَقِينِ ﴿٥١﴾}\\
\textamh{52.\  } & \mytextarabic{فَسَبِّحْ بِٱسْمِ رَبِّكَ ٱلْعَظِيمِ ﴿٥٢﴾}\\
\end{longtable}
\clearpage